% Options for packages loaded elsewhere
% Options for packages loaded elsewhere
\PassOptionsToPackage{unicode}{hyperref}
\PassOptionsToPackage{hyphens}{url}
\PassOptionsToPackage{dvipsnames,svgnames,x11names}{xcolor}
%
\documentclass[
  letterpaper,
  DIV=11,
  numbers=noendperiod]{scrreprt}
\usepackage{xcolor}
\usepackage{amsmath,amssymb}
\setcounter{secnumdepth}{-\maxdimen} % remove section numbering
\usepackage{iftex}
\ifPDFTeX
  \usepackage[T1]{fontenc}
  \usepackage[utf8]{inputenc}
  \usepackage{textcomp} % provide euro and other symbols
\else % if luatex or xetex
  \usepackage{unicode-math} % this also loads fontspec
  \defaultfontfeatures{Scale=MatchLowercase}
  \defaultfontfeatures[\rmfamily]{Ligatures=TeX,Scale=1}
\fi
\usepackage{lmodern}
\ifPDFTeX\else
  % xetex/luatex font selection
\fi
% Use upquote if available, for straight quotes in verbatim environments
\IfFileExists{upquote.sty}{\usepackage{upquote}}{}
\IfFileExists{microtype.sty}{% use microtype if available
  \usepackage[]{microtype}
  \UseMicrotypeSet[protrusion]{basicmath} % disable protrusion for tt fonts
}{}
\makeatletter
\@ifundefined{KOMAClassName}{% if non-KOMA class
  \IfFileExists{parskip.sty}{%
    \usepackage{parskip}
  }{% else
    \setlength{\parindent}{0pt}
    \setlength{\parskip}{6pt plus 2pt minus 1pt}}
}{% if KOMA class
  \KOMAoptions{parskip=half}}
\makeatother
% Make \paragraph and \subparagraph free-standing
\makeatletter
\ifx\paragraph\undefined\else
  \let\oldparagraph\paragraph
  \renewcommand{\paragraph}{
    \@ifstar
      \xxxParagraphStar
      \xxxParagraphNoStar
  }
  \newcommand{\xxxParagraphStar}[1]{\oldparagraph*{#1}\mbox{}}
  \newcommand{\xxxParagraphNoStar}[1]{\oldparagraph{#1}\mbox{}}
\fi
\ifx\subparagraph\undefined\else
  \let\oldsubparagraph\subparagraph
  \renewcommand{\subparagraph}{
    \@ifstar
      \xxxSubParagraphStar
      \xxxSubParagraphNoStar
  }
  \newcommand{\xxxSubParagraphStar}[1]{\oldsubparagraph*{#1}\mbox{}}
  \newcommand{\xxxSubParagraphNoStar}[1]{\oldsubparagraph{#1}\mbox{}}
\fi
\makeatother

\usepackage{color}
\usepackage{fancyvrb}
\newcommand{\VerbBar}{|}
\newcommand{\VERB}{\Verb[commandchars=\\\{\}]}
\DefineVerbatimEnvironment{Highlighting}{Verbatim}{commandchars=\\\{\}}
% Add ',fontsize=\small' for more characters per line
\usepackage{framed}
\definecolor{shadecolor}{RGB}{241,243,245}
\newenvironment{Shaded}{\begin{snugshade}}{\end{snugshade}}
\newcommand{\AlertTok}[1]{\textcolor[rgb]{0.68,0.00,0.00}{#1}}
\newcommand{\AnnotationTok}[1]{\textcolor[rgb]{0.37,0.37,0.37}{#1}}
\newcommand{\AttributeTok}[1]{\textcolor[rgb]{0.40,0.45,0.13}{#1}}
\newcommand{\BaseNTok}[1]{\textcolor[rgb]{0.68,0.00,0.00}{#1}}
\newcommand{\BuiltInTok}[1]{\textcolor[rgb]{0.00,0.23,0.31}{#1}}
\newcommand{\CharTok}[1]{\textcolor[rgb]{0.13,0.47,0.30}{#1}}
\newcommand{\CommentTok}[1]{\textcolor[rgb]{0.37,0.37,0.37}{#1}}
\newcommand{\CommentVarTok}[1]{\textcolor[rgb]{0.37,0.37,0.37}{\textit{#1}}}
\newcommand{\ConstantTok}[1]{\textcolor[rgb]{0.56,0.35,0.01}{#1}}
\newcommand{\ControlFlowTok}[1]{\textcolor[rgb]{0.00,0.23,0.31}{\textbf{#1}}}
\newcommand{\DataTypeTok}[1]{\textcolor[rgb]{0.68,0.00,0.00}{#1}}
\newcommand{\DecValTok}[1]{\textcolor[rgb]{0.68,0.00,0.00}{#1}}
\newcommand{\DocumentationTok}[1]{\textcolor[rgb]{0.37,0.37,0.37}{\textit{#1}}}
\newcommand{\ErrorTok}[1]{\textcolor[rgb]{0.68,0.00,0.00}{#1}}
\newcommand{\ExtensionTok}[1]{\textcolor[rgb]{0.00,0.23,0.31}{#1}}
\newcommand{\FloatTok}[1]{\textcolor[rgb]{0.68,0.00,0.00}{#1}}
\newcommand{\FunctionTok}[1]{\textcolor[rgb]{0.28,0.35,0.67}{#1}}
\newcommand{\ImportTok}[1]{\textcolor[rgb]{0.00,0.46,0.62}{#1}}
\newcommand{\InformationTok}[1]{\textcolor[rgb]{0.37,0.37,0.37}{#1}}
\newcommand{\KeywordTok}[1]{\textcolor[rgb]{0.00,0.23,0.31}{\textbf{#1}}}
\newcommand{\NormalTok}[1]{\textcolor[rgb]{0.00,0.23,0.31}{#1}}
\newcommand{\OperatorTok}[1]{\textcolor[rgb]{0.37,0.37,0.37}{#1}}
\newcommand{\OtherTok}[1]{\textcolor[rgb]{0.00,0.23,0.31}{#1}}
\newcommand{\PreprocessorTok}[1]{\textcolor[rgb]{0.68,0.00,0.00}{#1}}
\newcommand{\RegionMarkerTok}[1]{\textcolor[rgb]{0.00,0.23,0.31}{#1}}
\newcommand{\SpecialCharTok}[1]{\textcolor[rgb]{0.37,0.37,0.37}{#1}}
\newcommand{\SpecialStringTok}[1]{\textcolor[rgb]{0.13,0.47,0.30}{#1}}
\newcommand{\StringTok}[1]{\textcolor[rgb]{0.13,0.47,0.30}{#1}}
\newcommand{\VariableTok}[1]{\textcolor[rgb]{0.07,0.07,0.07}{#1}}
\newcommand{\VerbatimStringTok}[1]{\textcolor[rgb]{0.13,0.47,0.30}{#1}}
\newcommand{\WarningTok}[1]{\textcolor[rgb]{0.37,0.37,0.37}{\textit{#1}}}

\usepackage{longtable,booktabs,array}
\usepackage{calc} % for calculating minipage widths
% Correct order of tables after \paragraph or \subparagraph
\usepackage{etoolbox}
\makeatletter
\patchcmd\longtable{\par}{\if@noskipsec\mbox{}\fi\par}{}{}
\makeatother
% Allow footnotes in longtable head/foot
\IfFileExists{footnotehyper.sty}{\usepackage{footnotehyper}}{\usepackage{footnote}}
\makesavenoteenv{longtable}
\usepackage{graphicx}
\makeatletter
\newsavebox\pandoc@box
\newcommand*\pandocbounded[1]{% scales image to fit in text height/width
  \sbox\pandoc@box{#1}%
  \Gscale@div\@tempa{\textheight}{\dimexpr\ht\pandoc@box+\dp\pandoc@box\relax}%
  \Gscale@div\@tempb{\linewidth}{\wd\pandoc@box}%
  \ifdim\@tempb\p@<\@tempa\p@\let\@tempa\@tempb\fi% select the smaller of both
  \ifdim\@tempa\p@<\p@\scalebox{\@tempa}{\usebox\pandoc@box}%
  \else\usebox{\pandoc@box}%
  \fi%
}
% Set default figure placement to htbp
\def\fps@figure{htbp}
\makeatother





\setlength{\emergencystretch}{3em} % prevent overfull lines

\providecommand{\tightlist}{%
  \setlength{\itemsep}{0pt}\setlength{\parskip}{0pt}}



 


\KOMAoption{captions}{tableheading}
\makeatletter
\@ifpackageloaded{bookmark}{}{\usepackage{bookmark}}
\makeatother
\makeatletter
\@ifpackageloaded{caption}{}{\usepackage{caption}}
\AtBeginDocument{%
\ifdefined\contentsname
  \renewcommand*\contentsname{Table of contents}
\else
  \newcommand\contentsname{Table of contents}
\fi
\ifdefined\listfigurename
  \renewcommand*\listfigurename{List of Figures}
\else
  \newcommand\listfigurename{List of Figures}
\fi
\ifdefined\listtablename
  \renewcommand*\listtablename{List of Tables}
\else
  \newcommand\listtablename{List of Tables}
\fi
\ifdefined\figurename
  \renewcommand*\figurename{Figure}
\else
  \newcommand\figurename{Figure}
\fi
\ifdefined\tablename
  \renewcommand*\tablename{Table}
\else
  \newcommand\tablename{Table}
\fi
}
\@ifpackageloaded{float}{}{\usepackage{float}}
\floatstyle{ruled}
\@ifundefined{c@chapter}{\newfloat{codelisting}{h}{lop}}{\newfloat{codelisting}{h}{lop}[chapter]}
\floatname{codelisting}{Listing}
\newcommand*\listoflistings{\listof{codelisting}{List of Listings}}
\makeatother
\makeatletter
\makeatother
\makeatletter
\@ifpackageloaded{caption}{}{\usepackage{caption}}
\@ifpackageloaded{subcaption}{}{\usepackage{subcaption}}
\makeatother
\usepackage{bookmark}
\IfFileExists{xurl.sty}{\usepackage{xurl}}{} % add URL line breaks if available
\urlstyle{same}
\hypersetup{
  pdftitle={The Little Book of C},
  pdfauthor={Duc-Tam Nguyen},
  colorlinks=true,
  linkcolor={blue},
  filecolor={Maroon},
  citecolor={Blue},
  urlcolor={Blue},
  pdfcreator={LaTeX via pandoc}}


\title{The Little Book of C}
\usepackage{etoolbox}
\makeatletter
\providecommand{\subtitle}[1]{% add subtitle to \maketitle
  \apptocmd{\@title}{\par {\large #1 \par}}{}{}
}
\makeatother
\subtitle{Version 0.1.0}
\author{Duc-Tam Nguyen}
\date{2025-09-06}
\begin{document}
\maketitle

\renewcommand*\contentsname{Table of contents}
{
\hypersetup{linkcolor=}
\setcounter{tocdepth}{2}
\tableofcontents
}

\bookmarksetup{startatroot}

\chapter{Part I. First Steps}\label{part-i.-first-steps}

\section{Chapter 1. Getting started}\label{chapter-1.-getting-started}

\subsection{1.1 What is C?}\label{what-is-c}

C is a programming language. But before we go deeper, let's step back:
what is a programming language?

A programming language is a way for humans to give precise, step-by-step
instructions to a computer. Just as we use English, Vietnamese, or
Japanese to talk to each other, we use languages like C, Python, or Java
to ``talk'' to a computer.

C was created in the early 1970s at Bell Labs by Dennis Ritchie. It was
designed to write the UNIX operating system - and it became so
successful that many modern languages (C++, Java, Rust, Go, even parts
of Python) are deeply influenced by it.

Even today, more than fifty years later, C remains everywhere:

\begin{itemize}
\tightlist
\item
  The operating system inside your phone and laptop has thousands of
  lines of C code.
\item
  Device drivers - the programs that let your computer talk to printers,
  cameras, or Wi-Fi cards - are usually written in C.
\item
  Embedded systems, like the computer inside your microwave or car, are
  often powered by C.
\end{itemize}

C is sometimes called a ``low-level high-level language'':

\begin{itemize}
\tightlist
\item
  It's -low-level- because it lets you control memory and hardware
  directly, almost like assembly.
\item
  It's -high-level- because it still uses readable words and structures
  that humans can understand, unlike raw 1s and 0s.
\end{itemize}

Think of C as the middle ground: close enough to the machine to be
powerful, but high-level enough to be practical for humans.

\subsubsection{A First Glimpse}\label{a-first-glimpse}

Here's a very small piece of C code. Don't worry if it looks mysterious
- we'll explain it step by step in the next sections:

\begin{Shaded}
\begin{Highlighting}[]
\PreprocessorTok{\#include }\ImportTok{\textless{}stdio.h\textgreater{}}

\DataTypeTok{int}\NormalTok{ main}\OperatorTok{(}\DataTypeTok{void}\OperatorTok{)} \OperatorTok{\{}
\NormalTok{    printf}\OperatorTok{(}\StringTok{"Hello, world!}\SpecialCharTok{\textbackslash{}n}\StringTok{"}\OperatorTok{);}
    \ControlFlowTok{return} \DecValTok{0}\OperatorTok{;}
\OperatorTok{\}}
\end{Highlighting}
\end{Shaded}

This tiny program does one thing: it prints the words
\texttt{Hello,\ world!} on the screen.

Even though this is just a few lines, it already shows you the essence
of programming: you write instructions, the computer follows them, and
you get a result.

\subsubsection{Why Learn C First?}\label{why-learn-c-first}

\begin{enumerate}
\def\labelenumi{\arabic{enumi}.}
\tightlist
\item
  C has a small set of features. You can learn the whole language core
  in this little book.
\item
  Once you learn C, you'll find Java, C++, Go, and others easier to
  understand.
\item
  C gives you direct access to memory, which is rare in modern
  languages. This helps you understand how computers really work.
\item
  C has been around for over 50 years, and it isn't going away anytime
  soon. Learning it is like learning the grammar of computing.
\end{enumerate}

\subsubsection{Why It Matters}\label{why-it-matters}

Learning C is like learning to drive a manual car before moving to an
automatic: it teaches you what's happening under the hood. Even if you
later use higher-level languages, your knowledge of C will give you
confidence and deeper insight into performance, efficiency, and
problem-solving.

\subsubsection{Exercises}\label{exercises}

\begin{enumerate}
\def\labelenumi{\arabic{enumi}.}
\tightlist
\item
  Look up (online or in books) one example of a system you use every day
  that was written in C.
\item
  Explain in your own words the difference between a programming
  language and a human language.
\item
  If C is over 50 years old, why do you think it's still widely used
  today?
\end{enumerate}

\subsection{1.2 Hello, World!}\label{hello-world}

Every journey in C programming begins with a simple tradition: writing a
program that prints ``Hello, world!'' to the screen.

This may look small, but it teaches you the essentials of how a C
program is structured.

\subsubsection{The Full Program}\label{the-full-program}

Here is the complete code:

\begin{Shaded}
\begin{Highlighting}[]
\PreprocessorTok{\#include }\ImportTok{\textless{}stdio.h\textgreater{}}

\DataTypeTok{int}\NormalTok{ main}\OperatorTok{(}\DataTypeTok{void}\OperatorTok{)} \OperatorTok{\{}
\NormalTok{    printf}\OperatorTok{(}\StringTok{"Hello, world!}\SpecialCharTok{\textbackslash{}n}\StringTok{"}\OperatorTok{);}
    \ControlFlowTok{return} \DecValTok{0}\OperatorTok{;}
\OperatorTok{\}}
\end{Highlighting}
\end{Shaded}

When you run this program, it shows:

\begin{Shaded}
\begin{Highlighting}[]
\ExtensionTok{Hello,}\NormalTok{ world!}
\end{Highlighting}
\end{Shaded}

\subsubsection{Breaking It Down}\label{breaking-it-down}

Let's go through this step by step:

\begin{enumerate}
\def\labelenumi{\arabic{enumi}.}
\item
  \texttt{\#include\ \textless{}stdio.h\textgreater{}}

  \begin{itemize}
  \tightlist
  \item
    This tells the compiler: \emph{``Before you build my program, please
    include the Standard Input/Output library.''}
  \item
    The file \texttt{stdio.h} contains the definition of
    \texttt{printf}, which we'll use to display text.
  \item
    Without this line, the compiler wouldn't know what \texttt{printf}
    means.
  \end{itemize}
\item
  \texttt{int\ main(void)}

  \begin{itemize}
  \tightlist
  \item
    Every C program starts execution at a special function called
    \texttt{main}.
  \item
    \texttt{int} means that \texttt{main} will return an integer number
    to the operating system when it finishes.
  \item
    \texttt{(void)} means \emph{``main takes no arguments.''} Later
    we'll see other forms that can take inputs.
  \end{itemize}
\item
  \texttt{\{\ ...\ \}}

  \begin{itemize}
  \tightlist
  \item
    The curly braces mark the start \texttt{\{} and end \texttt{\}} of a
    block of code.
  \item
    Everything inside belongs to the \texttt{main} function.
  \end{itemize}
\item
  \texttt{printf("Hello,\ world!\textbackslash{}n");}

  \begin{itemize}
  \tightlist
  \item
    \texttt{printf} is a function that prints text.
  \item
    The text we want to print is inside quotation marks \texttt{"..."}.
  \item
    The \texttt{\textbackslash{}n} is a newline character: it moves the
    cursor to the next line after printing. Without it, the next output
    would continue on the same line.
  \item
    The semicolon \texttt{;} ends the instruction - in C, every
    statement must end with \texttt{;}.
  \end{itemize}
\item
  \texttt{return\ 0;}

  \begin{itemize}
  \tightlist
  \item
    This line tells the operating system: -``The program finished
    successfully.''-
  \item
    A return value of \texttt{0} usually means success. Other numbers
    can indicate errors, which you'll learn about later.
  \end{itemize}
\end{enumerate}

\subsubsection{How It Fits Together}\label{how-it-fits-together}

\begin{itemize}
\tightlist
\item
  The program starts at \texttt{main}.
\item
  It runs each line inside the braces.
\item
  It prints \texttt{Hello,\ world!} with a newline.
\item
  It exits and returns \texttt{0} to the system.
\end{itemize}

That's the life cycle of your very first program.

\subsubsection{Why It Matters}\label{why-it-matters-1}

This tiny program shows you many key ideas that repeat throughout C:

\begin{itemize}
\tightlist
\item
  Libraries (\texttt{\#include}) give you tools you didn't write
  yourself.
\item
  Functions (\texttt{main}, \texttt{printf}) are the building blocks of
  programs.
\item
  Statements (ending with \texttt{;}) are instructions the computer
  follows.
\item
  Return values (\texttt{return\ 0}) communicate success or failure.
\end{itemize}

Once you understand ``Hello, world!'', you can build more complex
programs with confidence.

\subsubsection{Exercises}\label{exercises-1}

\begin{enumerate}
\def\labelenumi{\arabic{enumi}.}
\item
  Change the message in the program to print your name instead of
  ``Hello, world!''.
\item
  Remove the \texttt{\textbackslash{}n} and see what happens when you
  run the program.
\item
  Try printing two lines by calling \texttt{printf} twice:

\begin{Shaded}
\begin{Highlighting}[]
\NormalTok{printf}\OperatorTok{(}\StringTok{"First line}\SpecialCharTok{\textbackslash{}n}\StringTok{"}\OperatorTok{);}
\NormalTok{printf}\OperatorTok{(}\StringTok{"Second line}\SpecialCharTok{\textbackslash{}n}\StringTok{"}\OperatorTok{);}
\end{Highlighting}
\end{Shaded}
\end{enumerate}

\subsection{1.3 Compiling and Running a
Program}\label{compiling-and-running-a-program}

Writing a program is only the first step. To make the computer
understand it, we need to translate our C code into something the
machine can run. This translation process is called compilation.

\subsubsection{From Source Code to
Program}\label{from-source-code-to-program}

When you write C code in a text file (for example, \texttt{hello.c}),
you are writing source code - human-readable instructions. But your
computer only understands machine code - long sequences of 0s and 1s.

The compiler's job is to take your source code and produce a program
that your computer can execute.

The process usually looks like this:

\begin{enumerate}
\def\labelenumi{\arabic{enumi}.}
\tightlist
\item
  Write code → You save your C code in a file like \texttt{hello.c}.
\item
  Compile → The compiler checks your code and translates it into an
  executable program.
\item
  Run → You execute the compiled program, and the computer follows your
  instructions.
\end{enumerate}

\subsubsection{Using a Compiler}\label{using-a-compiler}

The most common compiler today is GCC (GNU Compiler Collection). Another
popular one is Clang. Both support the modern C23 standard.

Let's say your program is saved in a file called \texttt{hello.c}.

To compile it with GCC:

\begin{Shaded}
\begin{Highlighting}[]
\FunctionTok{gcc}\NormalTok{ hello.c }\AttributeTok{{-}o}\NormalTok{ hello}
\end{Highlighting}
\end{Shaded}

\begin{itemize}
\tightlist
\item
  \texttt{gcc} is the compiler command.
\item
  \texttt{hello.c} is your source code.
\item
  \texttt{-o\ hello} tells the compiler to create an output program
  named \texttt{hello}.
\end{itemize}

After this command, you will see a new file named \texttt{hello}.

\subsubsection{Running the Program}\label{running-the-program}

On Linux or macOS, you can run it like this:

\begin{Shaded}
\begin{Highlighting}[]
\ExtensionTok{./hello}
\end{Highlighting}
\end{Shaded}

On Windows, you might just type:

\begin{Shaded}
\begin{Highlighting}[]
\ExtensionTok{hello.exe}
\end{Highlighting}
\end{Shaded}

And you should see:

\begin{Shaded}
\begin{Highlighting}[]
\ExtensionTok{Hello,}\NormalTok{ world!}
\end{Highlighting}
\end{Shaded}

\subsubsection{Common Mistakes}\label{common-mistakes}

When compiling, beginners often see errors. Here are some typical ones:

\begin{itemize}
\item
  Missing semicolon:

\begin{Shaded}
\begin{Highlighting}[]
\ExtensionTok{error:}\NormalTok{ expected ‘}\KeywordTok{;}\ExtensionTok{’}\NormalTok{ before ‘return’}
\end{Highlighting}
\end{Shaded}

  → In C, every statement must end with \texttt{;}.
\item
  Misspelled function name:

\begin{Shaded}
\begin{Highlighting}[]
\ExtensionTok{error:}\NormalTok{ implicit declaration of function ‘print’}
\end{Highlighting}
\end{Shaded}

  → You probably wrote \texttt{print} instead of \texttt{printf}.
\item
  Missing quotes:

\begin{Shaded}
\begin{Highlighting}[]
\ExtensionTok{error:}\NormalTok{ missing terminating }\StringTok{" character}
\end{Highlighting}
\end{Shaded}

  → Strings must always be inside \texttt{"\ "}.
\end{itemize}

\subsubsection{Why It Matters}\label{why-it-matters-2}

Understanding compilation is crucial because:

\begin{itemize}
\tightlist
\item
  It shows you that C is different from interpreted languages (like
  Python).
\item
  It teaches you to think in two steps: write → compile → run.
\item
  It prepares you for reading error messages - an essential skill for
  every programmer.
\end{itemize}

\subsubsection{Exercises}\label{exercises-2}

\begin{enumerate}
\def\labelenumi{\arabic{enumi}.}
\item
  Save the ``Hello, world!'' program in a file called \texttt{hello.c},
  compile it with GCC or Clang, and run it.
\item
  Introduce a mistake (like removing a semicolon) and see what error
  message the compiler gives you.
\item
  Try compiling with a different output name, e.g.:

\begin{Shaded}
\begin{Highlighting}[]
\FunctionTok{gcc}\NormalTok{ hello.c }\AttributeTok{{-}o}\NormalTok{ myprogram}
\end{Highlighting}
\end{Shaded}

  Then run \texttt{./myprogram}.
\end{enumerate}

\subsection{1.4 Editing, Saving, and
Errors}\label{editing-saving-and-errors}

Now that you know how to compile and run a C program, let's talk about
something every programmer does all the time: editing and fixing errors.
Writing code isn't just about typing it once perfectly. In fact, most of
programming is a cycle of edit → compile → fix errors → run → repeat.

\subsubsection{Editing and Saving Code}\label{editing-and-saving-code}

C programs are just text files. You can edit them with any text editor.
Some common choices:

\begin{itemize}
\tightlist
\item
  Linux/macOS: VS Code, Sublime Text, nano, vim
\item
  Windows: VS Code, Notepad++, or even the built-in Notepad (though more
  advanced editors are recommended)
\end{itemize}

When you save your file, make sure it has the \texttt{.c} extension. For
example:

\begin{Shaded}
\begin{Highlighting}[]
\ExtensionTok{hello.c}
\end{Highlighting}
\end{Shaded}

The compiler looks at that file extension to know it's C source code.

\subsubsection{The Reality of Errors}\label{the-reality-of-errors}

It's normal for your first compilation to show errors. Don't be
discouraged - errors are part of programming. Even professionals see
them daily.

For example, if you forget a semicolon:

\begin{Shaded}
\begin{Highlighting}[]
\PreprocessorTok{\#include }\ImportTok{\textless{}stdio.h\textgreater{}}

\DataTypeTok{int}\NormalTok{ main}\OperatorTok{(}\DataTypeTok{void}\OperatorTok{)} \OperatorTok{\{}
\NormalTok{    printf}\OperatorTok{(}\StringTok{"Hello, world!}\SpecialCharTok{\textbackslash{}n}\StringTok{"}\OperatorTok{)}
    \ControlFlowTok{return} \DecValTok{0}\OperatorTok{;}
\OperatorTok{\}}
\end{Highlighting}
\end{Shaded}

When compiled, GCC might say:

\begin{Shaded}
\begin{Highlighting}[]
\ExtensionTok{hello.c:}\NormalTok{ In function ‘main’:}
\ExtensionTok{hello.c:4:5:}\NormalTok{ error: expected ‘}\KeywordTok{;}\ExtensionTok{’}\NormalTok{ before ‘return’}
\end{Highlighting}
\end{Shaded}

This looks intimidating, but let's break it down:

\begin{itemize}
\tightlist
\item
  \texttt{hello.c:4:5} → File \texttt{hello.c}, line 4, character 5
\item
  \texttt{error:\ expected\ ‘;’\ before\ ‘return’} → The compiler wanted
  a \texttt{;} but found \texttt{return} instead
\end{itemize}

Once you add the missing semicolon, the program compiles.

\subsubsection{Warnings vs Errors}\label{warnings-vs-errors}

Compilers give two kinds of feedback:

\begin{itemize}
\tightlist
\item
  Errors → Your program cannot run until you fix them.
\item
  Warnings → Your program will run, but the compiler is alerting you to
  something suspicious.
\end{itemize}

Example warning:

\begin{Shaded}
\begin{Highlighting}[]
\DataTypeTok{int}\NormalTok{ main}\OperatorTok{(}\DataTypeTok{void}\OperatorTok{)} \OperatorTok{\{}
    \DataTypeTok{int}\NormalTok{ x}\OperatorTok{;}
\NormalTok{    printf}\OperatorTok{(}\StringTok{"}\SpecialCharTok{\%d\textbackslash{}n}\StringTok{"}\OperatorTok{,}\NormalTok{ x}\OperatorTok{);}
    \ControlFlowTok{return} \DecValTok{0}\OperatorTok{;}
\OperatorTok{\}}
\end{Highlighting}
\end{Shaded}

The compiler may warn:

\begin{Shaded}
\begin{Highlighting}[]
\ExtensionTok{warning:}\NormalTok{ ‘x’ is used uninitialized in this function}
\end{Highlighting}
\end{Shaded}

This means you're trying to print \texttt{x} before giving it a value -
dangerous behavior. The program may run, but the result will be
unpredictable.

\subsubsection{Debugging Mindset}\label{debugging-mindset}

When you see an error:

\begin{enumerate}
\def\labelenumi{\arabic{enumi}.}
\tightlist
\item
  Read it carefully - The compiler usually tells you the line number.
\item
  Don't panic - Errors are normal.
\item
  Fix one at a time - Often, one small mistake (like a missing
  semicolon) can cause multiple errors.
\item
  Recompile after each fix - Don't try to fix everything at once without
  testing.
\end{enumerate}

\subsubsection{Why It Matters}\label{why-it-matters-3}

Learning how to read and respond to error messages is a key skill.
Beginners who treat errors as enemies get frustrated. Experts see errors
as helpful feedback. The compiler is your friend: it points out problems
before your program misbehaves.

\subsubsection{Exercises}\label{exercises-3}

\begin{enumerate}
\def\labelenumi{\arabic{enumi}.}
\tightlist
\item
  Write a simple program with a deliberate mistake (for example,
  misspell \texttt{printf} as \texttt{prntf}). Compile it, and read the
  error message.
\item
  Remove a closing brace \texttt{\}} and see what kind of error appears.
\item
  Create a program with both a warning and an error. Can you fix both?
\end{enumerate}

\subsection{1.5 Why C Still Matters}\label{why-c-still-matters}

At this point, you might be wondering: \emph{C is over fifty years old.
Why should I learn it today?} After all, there are newer languages like
Python, JavaScript, Rust, and Go. The answer is simple: C is still at
the heart of computing.

\subsubsection{C Is Everywhere}\label{c-is-everywhere}

\begin{itemize}
\tightlist
\item
  The Linux kernel, Windows core, and macOS internals are mostly written
  in C.
\item
  The tiny chips inside your car, microwave, TV, or washing machine
  often run code written in C.
\item
  MySQL, SQLite, Git, and even parts of Python itself are implemented in
  C.
\item
  Many performance-critical parts of game engines rely on C for speed.
\end{itemize}

Learning C means understanding the language that underpins much of
modern software.

\subsubsection{Foundation for Other
Languages}\label{foundation-for-other-languages}

Many modern languages borrow C's syntax and concepts:

\begin{itemize}
\tightlist
\item
  C++ extends C with classes and object-oriented features.
\item
  Java, C\#, Go, Rust all use C-like curly braces, operators, and
  control flow.
\item
  Python itself is implemented in C, and its extension modules often use
  C for performance.
\end{itemize}

When you learn C, you build a foundation that transfers to other
languages.

\subsubsection{Performance and Control}\label{performance-and-control}

C gives you direct access to memory. Unlike Python or Java, where memory
is managed for you, in C you decide:

\begin{itemize}
\tightlist
\item
  Where data lives
\item
  How much memory is used
\item
  When memory is freed
\end{itemize}

This makes C extremely fast - and also teaches you how computers
actually work under the hood.

\subsubsection{Longevity and Stability}\label{longevity-and-stability}

Languages come and go, but C endures. The C23 standard is the latest in
a long line of updates that keep C modern while preserving
compatibility. Code written decades ago in C can often still compile
today.

That's rare in the world of technology.

\subsubsection{A Mindset Shift}\label{a-mindset-shift}

Learning C isn't just about syntax. It changes how you think:

\begin{itemize}
\tightlist
\item
  You learn to be precise.
\item
  You become comfortable with errors and debugging.
\item
  You start thinking about efficiency, not just correctness.
\end{itemize}

C teaches you programming at its most essential.

\subsubsection{Why It Matters}\label{why-it-matters-4}

Even if you later move on to Python, JavaScript, or Rust, learning C
gives you a mental model that makes you a stronger programmer. It's like
learning arithmetic before using a calculator: once you understand the
basics, you can appreciate and use advanced tools better.

\subsubsection{Exercises}\label{exercises-4}

\begin{enumerate}
\def\labelenumi{\arabic{enumi}.}
\tightlist
\item
  Look up three widely used software projects that are written in C.
\item
  Ask yourself: if you had to write a program for a device with very
  little memory (like a smartwatch or a sensor), why might C be a good
  choice?
\item
  Compare the age of C (1972) with another technology you use every day.
  Why do you think C has lasted so long?
\end{enumerate}

\section{Chapter 2. Variables and
Types}\label{chapter-2.-variables-and-types}

\subsection{2.1 Numbers and Text}\label{numbers-and-text}

So far, our programs only printed fixed text like
\texttt{"Hello,\ world!"}. That's nice, but real programs need to work
with data - numbers, words, and symbols that can change each time the
program runs. In C, we represent data using variables.

\subsubsection{Variables: Boxes for
Data}\label{variables-boxes-for-data}

Think of a variable as a box with a label on it:

\begin{itemize}
\tightlist
\item
  The box holds some value (like a number or letter).
\item
  The label (the variable's name) tells us how to refer to it later.
\end{itemize}

In C, you must tell the computer what kind of data the box holds. This
is called the type of the variable.

\subsubsection{Numbers}\label{numbers}

C supports several kinds of numbers:

\begin{itemize}
\tightlist
\item
  Integers: whole numbers like \texttt{-5}, \texttt{0}, \texttt{42}
\item
  Floating-point numbers: numbers with decimal points like
  \texttt{3.14}, \texttt{-0.5}, \texttt{2.71828}
\end{itemize}

Here's a simple program that uses both:

\begin{Shaded}
\begin{Highlighting}[]
\PreprocessorTok{\#include }\ImportTok{\textless{}stdio.h\textgreater{}}

\DataTypeTok{int}\NormalTok{ main}\OperatorTok{(}\DataTypeTok{void}\OperatorTok{)} \OperatorTok{\{}
    \DataTypeTok{int}\NormalTok{ age }\OperatorTok{=} \DecValTok{20}\OperatorTok{;}            \CommentTok{// an integer}
    \DataTypeTok{float}\NormalTok{ height }\OperatorTok{=} \FloatTok{1.75}\OperatorTok{;}     \CommentTok{// a floating{-}point number}

\NormalTok{    printf}\OperatorTok{(}\StringTok{"Age: }\SpecialCharTok{\%d\textbackslash{}n}\StringTok{"}\OperatorTok{,}\NormalTok{ age}\OperatorTok{);}
\NormalTok{    printf}\OperatorTok{(}\StringTok{"Height: }\SpecialCharTok{\%.2f}\StringTok{ meters}\SpecialCharTok{\textbackslash{}n}\StringTok{"}\OperatorTok{,}\NormalTok{ height}\OperatorTok{);}

    \ControlFlowTok{return} \DecValTok{0}\OperatorTok{;}
\OperatorTok{\}}
\end{Highlighting}
\end{Shaded}

Output:

\begin{verbatim}
Age: 20
Height: 1.75 meters
\end{verbatim}

Notice:

\begin{itemize}
\tightlist
\item
  \texttt{\%d} tells \texttt{printf} to print an integer.
\item
  \texttt{\%f} tells \texttt{printf} to print a floating-point number.
\item
  \texttt{\%.2f} means ``print with 2 digits after the decimal point.''
\end{itemize}

\subsubsection{Text (Characters)}\label{text-characters}

C represents text as characters, written in single quotes:
\texttt{\textquotesingle{}A\textquotesingle{}},
\texttt{\textquotesingle{}z\textquotesingle{}},
\texttt{\textquotesingle{}?\textquotesingle{}}. Characters are stored as
small integer codes (ASCII codes).

Example:

\begin{Shaded}
\begin{Highlighting}[]
\PreprocessorTok{\#include }\ImportTok{\textless{}stdio.h\textgreater{}}

\DataTypeTok{int}\NormalTok{ main}\OperatorTok{(}\DataTypeTok{void}\OperatorTok{)} \OperatorTok{\{}
    \DataTypeTok{char}\NormalTok{ grade }\OperatorTok{=} \CharTok{\textquotesingle{}A\textquotesingle{}}\OperatorTok{;}   \CommentTok{// a single character}

\NormalTok{    printf}\OperatorTok{(}\StringTok{"Your grade is }\SpecialCharTok{\%c\textbackslash{}n}\StringTok{"}\OperatorTok{,}\NormalTok{ grade}\OperatorTok{);}

    \ControlFlowTok{return} \DecValTok{0}\OperatorTok{;}
\OperatorTok{\}}
\end{Highlighting}
\end{Shaded}

Output:

\begin{verbatim}
Your grade is A
\end{verbatim}

The \texttt{\%c} format specifier prints a single character.

\subsubsection{Text (Strings)}\label{text-strings}

A string is a sequence of characters, like \texttt{"Hello"} or
\texttt{"C\ programming"}. In C, strings are written inside double
quotes \texttt{"..."} and stored as arrays of characters (we'll learn
more in Chapter 6).

For now, you can print them directly:

\begin{Shaded}
\begin{Highlighting}[]
\PreprocessorTok{\#include }\ImportTok{\textless{}stdio.h\textgreater{}}

\DataTypeTok{int}\NormalTok{ main}\OperatorTok{(}\DataTypeTok{void}\OperatorTok{)} \OperatorTok{\{}
    \DataTypeTok{char}\NormalTok{ name}\OperatorTok{[]} \OperatorTok{=} \StringTok{"Alice"}\OperatorTok{;}

\NormalTok{    printf}\OperatorTok{(}\StringTok{"Hello, }\SpecialCharTok{\%s}\StringTok{!}\SpecialCharTok{\textbackslash{}n}\StringTok{"}\OperatorTok{,}\NormalTok{ name}\OperatorTok{);}

    \ControlFlowTok{return} \DecValTok{0}\OperatorTok{;}
\OperatorTok{\}}
\end{Highlighting}
\end{Shaded}

Output:

\begin{verbatim}
Hello, Alice!
\end{verbatim}

The \texttt{\%s} format specifier prints a string.

\subsubsection{Why It Matters}\label{why-it-matters-5}

\begin{itemize}
\tightlist
\item
  Variables let your program handle different inputs instead of fixed
  text.
\item
  Types ensure the computer knows how to store and work with the data.
\item
  By learning numbers and text, you're starting to write programs that
  can interact with the world, not just display the same message every
  time.
\end{itemize}

\subsubsection{Exercises}\label{exercises-5}

\begin{enumerate}
\def\labelenumi{\arabic{enumi}.}
\item
  Write a program that declares an integer \texttt{year} and prints:

\begin{Shaded}
\begin{Highlighting}[]
\ExtensionTok{The}\NormalTok{ year is 2025}
\end{Highlighting}
\end{Shaded}

  using \texttt{printf}.
\item
  Modify the program to include your favorite decimal number (like your
  height or a constant such as π). Print it with 3 decimal places.
\item
  Create a program that stores your first initial as a \texttt{char} and
  prints:

\begin{Shaded}
\begin{Highlighting}[]
\ExtensionTok{My}\NormalTok{ initial is X}
\end{Highlighting}
\end{Shaded}

  replacing \texttt{X} with your character.
\end{enumerate}

\subsection{2.2 Declaring Variables}\label{declaring-variables}

In the previous section, we used variables like
\texttt{int\ age\ =\ 20;} and \texttt{float\ height\ =\ 1.75;}. But what
does it mean to declare a variable in C?

\subsubsection{The Anatomy of a
Declaration}\label{the-anatomy-of-a-declaration}

A variable declaration in C has two parts:

\begin{verbatim}
type name;
\end{verbatim}

\begin{itemize}
\tightlist
\item
  type → what kind of data the variable will hold (e.g., \texttt{int},
  \texttt{float}, \texttt{char})
\item
  name → the label you give the variable, so you can use it later
\end{itemize}

Example:

\begin{Shaded}
\begin{Highlighting}[]
\DataTypeTok{int}\NormalTok{ score}\OperatorTok{;}
\end{Highlighting}
\end{Shaded}

This means: ``Make a box named \texttt{score} that can hold an
integer.'' Right now, the contents of the box are undefined (whatever
random value happened to be in memory).

\subsubsection{Initialization}\label{initialization}

You can also give a variable an initial value when declaring it:

\begin{Shaded}
\begin{Highlighting}[]
\DataTypeTok{int}\NormalTok{ score }\OperatorTok{=} \DecValTok{100}\OperatorTok{;}
\end{Highlighting}
\end{Shaded}

This both creates the variable and sets its value. This is called
initialization.

If you don't initialize, C does not set it to zero automatically. Using
an uninitialized variable is a common beginner mistake.

\subsubsection{Multiple Declarations}\label{multiple-declarations}

You can declare several variables of the same type on one line:

\begin{Shaded}
\begin{Highlighting}[]
\DataTypeTok{int}\NormalTok{ x }\OperatorTok{=} \DecValTok{1}\OperatorTok{,}\NormalTok{ y }\OperatorTok{=} \DecValTok{2}\OperatorTok{,}\NormalTok{ z }\OperatorTok{=} \DecValTok{3}\OperatorTok{;}
\end{Highlighting}
\end{Shaded}

This creates three integers: \texttt{x}, \texttt{y}, and \texttt{z}. For
readability, beginners are encouraged to declare each variable on its
own line.

\subsubsection{Naming Rules}\label{naming-rules}

Variable names in C:

\begin{itemize}
\tightlist
\item
  Must start with a letter or underscore (\texttt{\_})
\item
  Can contain letters, digits, and underscores
\item
  Are case-sensitive (\texttt{Age} and \texttt{age} are different)
\item
  Cannot be a keyword (e.g., \texttt{int}, \texttt{return},
  \texttt{for})
\end{itemize}

Good names make your code clearer. Compare:

\begin{Shaded}
\begin{Highlighting}[]
\DataTypeTok{int}\NormalTok{ x}\OperatorTok{;}   \CommentTok{// unclear}
\DataTypeTok{int}\NormalTok{ age}\OperatorTok{;} \CommentTok{// clear}
\end{Highlighting}
\end{Shaded}

\subsubsection{A Full Example}\label{a-full-example}

Here's a small program that shows variable declaration, initialization,
and usage:

\begin{Shaded}
\begin{Highlighting}[]
\PreprocessorTok{\#include }\ImportTok{\textless{}stdio.h\textgreater{}}

\DataTypeTok{int}\NormalTok{ main}\OperatorTok{(}\DataTypeTok{void}\OperatorTok{)} \OperatorTok{\{}
    \DataTypeTok{int}\NormalTok{ year }\OperatorTok{=} \DecValTok{2025}\OperatorTok{;}          \CommentTok{// integer}
    \DataTypeTok{float}\NormalTok{ pi }\OperatorTok{=} \FloatTok{3.14159}\OperatorTok{;}       \CommentTok{// floating{-}point number}
    \DataTypeTok{char}\NormalTok{ initial }\OperatorTok{=} \CharTok{\textquotesingle{}N\textquotesingle{}}\OperatorTok{;}       \CommentTok{// character}
    \DataTypeTok{int}\NormalTok{ a }\OperatorTok{=} \DecValTok{5}\OperatorTok{,}\NormalTok{ b }\OperatorTok{=} \DecValTok{10}\OperatorTok{;}        \CommentTok{// multiple integers}

\NormalTok{    printf}\OperatorTok{(}\StringTok{"Year: }\SpecialCharTok{\%d\textbackslash{}n}\StringTok{"}\OperatorTok{,}\NormalTok{ year}\OperatorTok{);}
\NormalTok{    printf}\OperatorTok{(}\StringTok{"Value of pi: }\SpecialCharTok{\%.2f\textbackslash{}n}\StringTok{"}\OperatorTok{,}\NormalTok{ pi}\OperatorTok{);}
\NormalTok{    printf}\OperatorTok{(}\StringTok{"My initial: }\SpecialCharTok{\%c\textbackslash{}n}\StringTok{"}\OperatorTok{,}\NormalTok{ initial}\OperatorTok{);}
\NormalTok{    printf}\OperatorTok{(}\StringTok{"a = }\SpecialCharTok{\%d}\StringTok{, b = }\SpecialCharTok{\%d\textbackslash{}n}\StringTok{"}\OperatorTok{,}\NormalTok{ a}\OperatorTok{,}\NormalTok{ b}\OperatorTok{);}

    \ControlFlowTok{return} \DecValTok{0}\OperatorTok{;}
\OperatorTok{\}}
\end{Highlighting}
\end{Shaded}

Output:

\begin{verbatim}
Year: 2025
Value of pi: 3.14
My initial: N
a = 5, b = 10
\end{verbatim}

\subsubsection{Why It Matters}\label{why-it-matters-6}

\begin{itemize}
\tightlist
\item
  Declaring variables tells the computer what kind of information to
  expect.
\item
  Initializing ensures your variables start with predictable values.
\item
  Good naming makes your code readable and easier to maintain.
\end{itemize}

Without proper declarations, your program won't compile, or worse - it
will run with unpredictable results.

\subsubsection{Exercises}\label{exercises-6}

\begin{enumerate}
\def\labelenumi{\arabic{enumi}.}
\tightlist
\item
  Write a program that declares an integer \texttt{day}, a float
  \texttt{temperature}, and a char \texttt{grade}. Initialize them and
  print their values.
\item
  Create two integer variables, \texttt{x} and \texttt{y}, and print
  their sum.
\item
  Try declaring a variable without initializing it. Print it - what
  happens? Why is this dangerous?
\end{enumerate}

\subsection{2.3 Basic Types in C23}\label{basic-types-in-c23}

Every variable in C must have a type. The type tells the computer:

\begin{itemize}
\tightlist
\item
  How much memory to reserve
\item
  How to interpret the bits stored inside
\item
  What operations are allowed
\end{itemize}

C23 keeps the same familiar types from older standards but also adds
refinements for safety and clarity.

\subsubsection{Integers}\label{integers}

\begin{itemize}
\tightlist
\item
  \texttt{int} → most common integer type (size depends on the system,
  often 32 bits)
\item
  \texttt{short}, \texttt{long}, \texttt{long\ long} → smaller or larger
  ranges
\item
  \texttt{unsigned} versions → only non-negative values
\end{itemize}

Examples:

\begin{Shaded}
\begin{Highlighting}[]
\DataTypeTok{int}\NormalTok{ age }\OperatorTok{=} \DecValTok{20}\OperatorTok{;}
\DataTypeTok{unsigned} \DataTypeTok{int}\NormalTok{ population }\OperatorTok{=} \DecValTok{8000000}\OperatorTok{;}
\DataTypeTok{long} \DataTypeTok{long}\NormalTok{ stars }\OperatorTok{=} \DecValTok{10000000000}\BuiltInTok{LL}\OperatorTok{;}
\end{Highlighting}
\end{Shaded}

\subsubsection{Floating-Point Numbers}\label{floating-point-numbers}

For decimals and scientific values:

\begin{itemize}
\tightlist
\item
  \texttt{float} → typically 32-bit
\item
  \texttt{double} → 64-bit, more precise
\item
  \texttt{long\ double} → even more precision (implementation-dependent)
\end{itemize}

Examples:

\begin{Shaded}
\begin{Highlighting}[]
\DataTypeTok{float}\NormalTok{ pi }\OperatorTok{=} \FloatTok{3.14}\BuiltInTok{f}\OperatorTok{;}
\DataTypeTok{double}\NormalTok{ e }\OperatorTok{=} \FloatTok{2.718281828}\OperatorTok{;}
\end{Highlighting}
\end{Shaded}

\subsubsection{Characters}\label{characters}

\begin{itemize}
\tightlist
\item
  \texttt{char} → stores a single character, like
  \texttt{\textquotesingle{}A\textquotesingle{}} or
  \texttt{\textquotesingle{}z\textquotesingle{}}
\item
  Stored as an integer code (usually ASCII or UTF-8 in modern systems)
\end{itemize}

Example:

\begin{Shaded}
\begin{Highlighting}[]
\DataTypeTok{char}\NormalTok{ letter }\OperatorTok{=} \CharTok{\textquotesingle{}C\textquotesingle{}}\OperatorTok{;}
\end{Highlighting}
\end{Shaded}

\subsubsection{Booleans (C23 and C99+)}\label{booleans-c23-and-c99}

C did not originally have a true boolean type. Now you can use:

\begin{Shaded}
\begin{Highlighting}[]
\PreprocessorTok{\#include }\ImportTok{\textless{}stdbool.h\textgreater{}}

\DataTypeTok{bool}\NormalTok{ is\_happy }\OperatorTok{=} \KeywordTok{true}\OperatorTok{;}
\DataTypeTok{bool}\NormalTok{ is\_sad }\OperatorTok{=} \KeywordTok{false}\OperatorTok{;}
\end{Highlighting}
\end{Shaded}

\subsubsection{Special Types in C23}\label{special-types-in-c23}

C23 introduces safer and clearer usage:

\begin{itemize}
\tightlist
\item
  \texttt{nullptr}: a special constant to represent ``no pointer''
  (instead of older \texttt{NULL}).
\item
  \texttt{char8\_t}: for explicit UTF-8 characters, making Unicode
  handling clearer.
\end{itemize}

Example:

\begin{Shaded}
\begin{Highlighting}[]
\PreprocessorTok{\#include }\ImportTok{\textless{}stddef.h\textgreater{}}\PreprocessorTok{   }\CommentTok{// for nullptr}
\PreprocessorTok{\#include }\ImportTok{\textless{}uchar.h\textgreater{}}\PreprocessorTok{    }\CommentTok{// for char8\_t}

\DataTypeTok{int} \OperatorTok{{-}}\NormalTok{p }\OperatorTok{=} \KeywordTok{nullptr}\OperatorTok{;}     \CommentTok{// safe null pointer}
\NormalTok{char8\_t smiley }\OperatorTok{=} \CharTok{u8\textquotesingle{}☺\textquotesingle{}}\OperatorTok{;}  \CommentTok{// UTF{-}8 character literal}
\end{Highlighting}
\end{Shaded}

\subsubsection{A Full Example}\label{a-full-example-1}

This program demonstrates several basic types side by side:

\begin{Shaded}
\begin{Highlighting}[]
\PreprocessorTok{\#include }\ImportTok{\textless{}stdio.h\textgreater{}}
\PreprocessorTok{\#include }\ImportTok{\textless{}stdbool.h\textgreater{}}
\PreprocessorTok{\#include }\ImportTok{\textless{}stddef.h\textgreater{}}
\PreprocessorTok{\#include }\ImportTok{\textless{}uchar.h\textgreater{}}

\DataTypeTok{int}\NormalTok{ main}\OperatorTok{(}\DataTypeTok{void}\OperatorTok{)} \OperatorTok{\{}
    \DataTypeTok{int}\NormalTok{ year }\OperatorTok{=} \DecValTok{2025}\OperatorTok{;}
    \DataTypeTok{unsigned} \DataTypeTok{int}\NormalTok{ population }\OperatorTok{=} \DecValTok{1000000}\OperatorTok{;}
    \DataTypeTok{float}\NormalTok{ pi }\OperatorTok{=} \FloatTok{3.14159}\BuiltInTok{f}\OperatorTok{;}
    \DataTypeTok{double}\NormalTok{ e }\OperatorTok{=} \FloatTok{2.718281828}\OperatorTok{;}
    \DataTypeTok{char}\NormalTok{ grade }\OperatorTok{=} \CharTok{\textquotesingle{}A\textquotesingle{}}\OperatorTok{;}
    \DataTypeTok{bool}\NormalTok{ is\_student }\OperatorTok{=} \KeywordTok{true}\OperatorTok{;}
    \DataTypeTok{int} \OperatorTok{{-}}\NormalTok{ptr }\OperatorTok{=} \KeywordTok{nullptr}\OperatorTok{;}
\NormalTok{    char8\_t smiley }\OperatorTok{=} \CharTok{u8\textquotesingle{}☺\textquotesingle{}}\OperatorTok{;}

\NormalTok{    printf}\OperatorTok{(}\StringTok{"Year: }\SpecialCharTok{\%d\textbackslash{}n}\StringTok{"}\OperatorTok{,}\NormalTok{ year}\OperatorTok{);}
\NormalTok{    printf}\OperatorTok{(}\StringTok{"Population: }\SpecialCharTok{\%u\textbackslash{}n}\StringTok{"}\OperatorTok{,}\NormalTok{ population}\OperatorTok{);}
\NormalTok{    printf}\OperatorTok{(}\StringTok{"Pi: }\SpecialCharTok{\%.2f\textbackslash{}n}\StringTok{"}\OperatorTok{,}\NormalTok{ pi}\OperatorTok{);}
\NormalTok{    printf}\OperatorTok{(}\StringTok{"Euler\textquotesingle{}s number: }\SpecialCharTok{\%.5f\textbackslash{}n}\StringTok{"}\OperatorTok{,}\NormalTok{ e}\OperatorTok{);}
\NormalTok{    printf}\OperatorTok{(}\StringTok{"Grade: }\SpecialCharTok{\%c\textbackslash{}n}\StringTok{"}\OperatorTok{,}\NormalTok{ grade}\OperatorTok{);}
\NormalTok{    printf}\OperatorTok{(}\StringTok{"Is student: }\SpecialCharTok{\%s\textbackslash{}n}\StringTok{"}\OperatorTok{,}\NormalTok{ is\_student }\OperatorTok{?} \StringTok{"true"} \OperatorTok{:} \StringTok{"false"}\OperatorTok{);}
\NormalTok{    printf}\OperatorTok{(}\StringTok{"Pointer: }\SpecialCharTok{\%p\textbackslash{}n}\StringTok{"}\OperatorTok{,} \OperatorTok{(}\DataTypeTok{void}\OperatorTok{{-})}\NormalTok{ptr}\OperatorTok{);}
\NormalTok{    printf}\OperatorTok{(}\StringTok{"Smiley (UTF{-}8 code): U+}\SpecialCharTok{\%04X\textbackslash{}n}\StringTok{"}\OperatorTok{,}\NormalTok{ smiley}\OperatorTok{);}

    \ControlFlowTok{return} \DecValTok{0}\OperatorTok{;}
\OperatorTok{\}}
\end{Highlighting}
\end{Shaded}

Sample Output:

\begin{verbatim}
Year: 2025
Population: 1000000
Pi: 3.14
Euler's number: 2.71828
Grade: A
Is student: true
Pointer: (nil)
Smiley (UTF-8 code): U+263A
\end{verbatim}

\subsubsection{Why It Matters}\label{why-it-matters-7}

Understanding types is crucial because:

\begin{itemize}
\tightlist
\item
  They define the shape of your data.
\item
  Choosing the right type avoids wasted memory or incorrect results.
\item
  Modern C23 types (\texttt{bool}, \texttt{nullptr}, \texttt{char8\_t})
  make code safer and clearer.
\end{itemize}

\subsubsection{Exercises}\label{exercises-7}

\begin{enumerate}
\def\labelenumi{\arabic{enumi}.}
\tightlist
\item
  Declare one variable of each type (\texttt{int}, \texttt{float},
  \texttt{double}, \texttt{char}, \texttt{bool}). Print them using
  \texttt{printf}.
\item
  Try printing an \texttt{unsigned\ int} with \texttt{\%d}. What
  happens? Why is it wrong?
\item
  Write a program that stores the value of π in both \texttt{float} and
  \texttt{double}, then prints both with 10 decimal places. Compare the
  results.
\end{enumerate}

\subsection{\texorpdfstring{2.4 Constants (\texttt{const} and
\texttt{constexpr})}{2.4 Constants (const and constexpr)}}\label{constants-const-and-constexpr}

So far, we've seen variables like:

\begin{Shaded}
\begin{Highlighting}[]
\DataTypeTok{int}\NormalTok{ age }\OperatorTok{=} \DecValTok{20}\OperatorTok{;}
\DataTypeTok{float}\NormalTok{ pi }\OperatorTok{=} \FloatTok{3.14159}\BuiltInTok{f}\OperatorTok{;}
\end{Highlighting}
\end{Shaded}

But sometimes, we want values that should never change once set. These
are called constants.

Constants make your programs safer, clearer, and easier to maintain.

\subsubsection{\texorpdfstring{\texttt{const}}{const}}\label{const}

The \texttt{const} keyword tells the compiler: -``This variable's value
cannot be changed.''-

Example:

\begin{Shaded}
\begin{Highlighting}[]
\DataTypeTok{const} \DataTypeTok{int}\NormalTok{ days\_in\_week }\OperatorTok{=} \DecValTok{7}\OperatorTok{;}
\end{Highlighting}
\end{Shaded}

If you try to assign a new value later:

\begin{Shaded}
\begin{Highlighting}[]
\NormalTok{days\_in\_week }\OperatorTok{=} \DecValTok{8}\OperatorTok{;}   \CommentTok{// ❌ Error: assignment of read{-}only variable}
\end{Highlighting}
\end{Shaded}

\subsubsection{\texorpdfstring{\texttt{constexpr} (C23 and
C11+)}{constexpr (C23 and C11+)}}\label{constexpr-c23-and-c11}

The \texttt{constexpr} keyword ensures that a value is calculated at
compile time, not at runtime. It's like telling the compiler: -``Work
this out in advance.''-

Example:

\begin{Shaded}
\begin{Highlighting}[]
\KeywordTok{constexpr} \DataTypeTok{int}\NormalTok{ minutes\_in\_day }\OperatorTok{=} \DecValTok{24} \OperatorTok{{-}} \DecValTok{60}\OperatorTok{;}
\end{Highlighting}
\end{Shaded}

This guarantees that \texttt{minutes\_in\_day} is a constant known
before the program even runs.

\subsubsection{\texorpdfstring{Difference Between \texttt{const} and
\texttt{constexpr}}{Difference Between const and constexpr}}\label{difference-between-const-and-constexpr}

\begin{itemize}
\tightlist
\item
  \texttt{const} → value cannot be changed after the program starts
\item
  \texttt{constexpr} → value must be known before the program starts (at
  compile time)
\end{itemize}

Every \texttt{constexpr} is also \texttt{const}, but not every
\texttt{const} is \texttt{constexpr}.

\subsubsection{A Full Example}\label{a-full-example-2}

Here's a program that shows both \texttt{const} and \texttt{constexpr}
in action:

\begin{Shaded}
\begin{Highlighting}[]
\PreprocessorTok{\#include }\ImportTok{\textless{}stdio.h\textgreater{}}

\DataTypeTok{int}\NormalTok{ main}\OperatorTok{(}\DataTypeTok{void}\OperatorTok{)} \OperatorTok{\{}
    \DataTypeTok{const} \DataTypeTok{float}\NormalTok{ pi }\OperatorTok{=} \FloatTok{3.14159}\BuiltInTok{f}\OperatorTok{;}              \CommentTok{// constant value}
    \KeywordTok{constexpr} \DataTypeTok{int}\NormalTok{ seconds\_in\_hour }\OperatorTok{=} \DecValTok{60} \OperatorTok{{-}} \DecValTok{60}\OperatorTok{;} \CommentTok{// compile{-}time constant}
    \DataTypeTok{int}\NormalTok{ radius }\OperatorTok{=} \DecValTok{5}\OperatorTok{;}

    \DataTypeTok{float}\NormalTok{ circumference }\OperatorTok{=} \DecValTok{2} \OperatorTok{{-}}\NormalTok{ pi }\OperatorTok{{-}}\NormalTok{ radius}\OperatorTok{;}  \CommentTok{// uses const}
    \DataTypeTok{int}\NormalTok{ hours }\OperatorTok{=} \DecValTok{2}\OperatorTok{;}
    \DataTypeTok{int}\NormalTok{ seconds }\OperatorTok{=}\NormalTok{ hours }\OperatorTok{{-}}\NormalTok{ seconds\_in\_hour}\OperatorTok{;}  \CommentTok{// uses constexpr}

\NormalTok{    printf}\OperatorTok{(}\StringTok{"Radius: }\SpecialCharTok{\%d\textbackslash{}n}\StringTok{"}\OperatorTok{,}\NormalTok{ radius}\OperatorTok{);}
\NormalTok{    printf}\OperatorTok{(}\StringTok{"Circumference: }\SpecialCharTok{\%.2f\textbackslash{}n}\StringTok{"}\OperatorTok{,}\NormalTok{ circumference}\OperatorTok{);}
\NormalTok{    printf}\OperatorTok{(}\StringTok{"}\SpecialCharTok{\%d}\StringTok{ hours = }\SpecialCharTok{\%d}\StringTok{ seconds}\SpecialCharTok{\textbackslash{}n}\StringTok{"}\OperatorTok{,}\NormalTok{ hours}\OperatorTok{,}\NormalTok{ seconds}\OperatorTok{);}

    \ControlFlowTok{return} \DecValTok{0}\OperatorTok{;}
\OperatorTok{\}}
\end{Highlighting}
\end{Shaded}

Output:

\begin{verbatim}
Radius: 5
Circumference: 31.42
2 hours = 7200 seconds
\end{verbatim}

\subsubsection{Why It Matters}\label{why-it-matters-8}

\begin{itemize}
\tightlist
\item
  Safety → prevents accidental changes to values that should stay fixed.
\item
  Clarity → communicates intent (``this never changes'').
\item
  Performance → \texttt{constexpr} lets the compiler compute values
  ahead of time.
\end{itemize}

Good C programs rely heavily on constants for configuration and clarity.

\subsubsection{Exercises}\label{exercises-8}

\begin{enumerate}
\def\labelenumi{\arabic{enumi}.}
\tightlist
\item
  Declare a \texttt{const\ int} for the number of days in a year and
  print it.
\item
  Write a program using \texttt{constexpr} to calculate the number of
  minutes in a week.
\item
  Try declaring \texttt{const\ int\ x\ =\ 10;} and then reassigning
  \texttt{x\ =\ 20;}. What happens?
\end{enumerate}

\subsection{\texorpdfstring{2.5 Input and Output with \texttt{scanf} and
\texttt{printf}}{2.5 Input and Output with scanf and printf}}\label{input-and-output-with-scanf-and-printf}

So far, our programs only showed fixed values using \texttt{printf}. But
real programs need to interact with the user:

\begin{itemize}
\tightlist
\item
  Input → let the user type values
\item
  Output → display results
\end{itemize}

In C, the two main functions for this are:

\begin{itemize}
\tightlist
\item
  \texttt{printf} → prints output to the screen
\item
  \texttt{scanf} → reads input from the user
\end{itemize}

Both are defined in the \texttt{stdio.h} library.

\subsubsection{\texorpdfstring{Using
\texttt{printf}}{Using printf}}\label{using-printf}

You've already seen examples like:

\begin{Shaded}
\begin{Highlighting}[]
\NormalTok{printf}\OperatorTok{(}\StringTok{"Hello, world!}\SpecialCharTok{\textbackslash{}n}\StringTok{"}\OperatorTok{);}
\end{Highlighting}
\end{Shaded}

The key idea is format specifiers:

\begin{itemize}
\tightlist
\item
  \texttt{\%d} → integer
\item
  \texttt{\%f} → floating-point number
\item
  \texttt{\%c} → single character
\item
  \texttt{\%s} → string
\end{itemize}

Example:

\begin{Shaded}
\begin{Highlighting}[]
\DataTypeTok{int}\NormalTok{ age }\OperatorTok{=} \DecValTok{20}\OperatorTok{;}
\NormalTok{printf}\OperatorTok{(}\StringTok{"I am }\SpecialCharTok{\%d}\StringTok{ years old.}\SpecialCharTok{\textbackslash{}n}\StringTok{"}\OperatorTok{,}\NormalTok{ age}\OperatorTok{);}
\end{Highlighting}
\end{Shaded}

\subsubsection{\texorpdfstring{Using
\texttt{scanf}}{Using scanf}}\label{using-scanf}

The \texttt{scanf} function reads user input.

Syntax:

\begin{Shaded}
\begin{Highlighting}[]
\NormalTok{scanf}\OperatorTok{(}\StringTok{"format"}\OperatorTok{,} \OperatorTok{\&}\NormalTok{variable}\OperatorTok{);}
\end{Highlighting}
\end{Shaded}

\begin{itemize}
\tightlist
\item
  \texttt{"format"} tells \texttt{scanf} what kind of data to expect
\item
  \texttt{\&variable} gives the address of the variable (so
  \texttt{scanf} can write into it)
\end{itemize}

Example:

\begin{Shaded}
\begin{Highlighting}[]
\DataTypeTok{int}\NormalTok{ age}\OperatorTok{;}
\NormalTok{scanf}\OperatorTok{(}\StringTok{"}\SpecialCharTok{\%d}\StringTok{"}\OperatorTok{,} \OperatorTok{\&}\NormalTok{age}\OperatorTok{);}
\end{Highlighting}
\end{Shaded}

This reads an integer from the keyboard and stores it in \texttt{age}.

\subsubsection{A Full Example}\label{a-full-example-3}

Here's a complete program that combines input and output:

\begin{Shaded}
\begin{Highlighting}[]
\PreprocessorTok{\#include }\ImportTok{\textless{}stdio.h\textgreater{}}

\DataTypeTok{int}\NormalTok{ main}\OperatorTok{(}\DataTypeTok{void}\OperatorTok{)} \OperatorTok{\{}
    \DataTypeTok{int}\NormalTok{ age}\OperatorTok{;}
    \DataTypeTok{float}\NormalTok{ height}\OperatorTok{;}
    \DataTypeTok{char}\NormalTok{ initial}\OperatorTok{;}
    \DataTypeTok{char}\NormalTok{ name}\OperatorTok{[}\DecValTok{20}\OperatorTok{];}   \CommentTok{// enough space for up to 19 characters + \textquotesingle{}\textbackslash{}0\textquotesingle{}}

\NormalTok{    printf}\OperatorTok{(}\StringTok{"Enter your age: "}\OperatorTok{);}
\NormalTok{    scanf}\OperatorTok{(}\StringTok{"}\SpecialCharTok{\%d}\StringTok{"}\OperatorTok{,} \OperatorTok{\&}\NormalTok{age}\OperatorTok{);}

\NormalTok{    printf}\OperatorTok{(}\StringTok{"Enter your height in meters: "}\OperatorTok{);}
\NormalTok{    scanf}\OperatorTok{(}\StringTok{"}\SpecialCharTok{\%f}\StringTok{"}\OperatorTok{,} \OperatorTok{\&}\NormalTok{height}\OperatorTok{);}

\NormalTok{    printf}\OperatorTok{(}\StringTok{"Enter your first initial: "}\OperatorTok{);}
\NormalTok{    scanf}\OperatorTok{(}\StringTok{" }\SpecialCharTok{\%c}\StringTok{"}\OperatorTok{,} \OperatorTok{\&}\NormalTok{initial}\OperatorTok{);}   \CommentTok{// notice the space before \%c to skip whitespace}

\NormalTok{    printf}\OperatorTok{(}\StringTok{"Enter your name: "}\OperatorTok{);}
\NormalTok{    scanf}\OperatorTok{(}\StringTok{"}\SpecialCharTok{\%19s}\StringTok{"}\OperatorTok{,}\NormalTok{ name}\OperatorTok{);}      \CommentTok{// read string (up to 19 chars)}

\NormalTok{    printf}\OperatorTok{(}\StringTok{"}\SpecialCharTok{\textbackslash{}n}\StringTok{{-}{-}{-} Profile {-}{-}{-}}\SpecialCharTok{\textbackslash{}n}\StringTok{"}\OperatorTok{);}
\NormalTok{    printf}\OperatorTok{(}\StringTok{"Name: }\SpecialCharTok{\%s\textbackslash{}n}\StringTok{"}\OperatorTok{,}\NormalTok{ name}\OperatorTok{);}
\NormalTok{    printf}\OperatorTok{(}\StringTok{"Initial: }\SpecialCharTok{\%c\textbackslash{}n}\StringTok{"}\OperatorTok{,}\NormalTok{ initial}\OperatorTok{);}
\NormalTok{    printf}\OperatorTok{(}\StringTok{"Age: }\SpecialCharTok{\%d}\StringTok{ years}\SpecialCharTok{\textbackslash{}n}\StringTok{"}\OperatorTok{,}\NormalTok{ age}\OperatorTok{);}
\NormalTok{    printf}\OperatorTok{(}\StringTok{"Height: }\SpecialCharTok{\%.2f}\StringTok{ m}\SpecialCharTok{\textbackslash{}n}\StringTok{"}\OperatorTok{,}\NormalTok{ height}\OperatorTok{);}

    \ControlFlowTok{return} \DecValTok{0}\OperatorTok{;}
\OperatorTok{\}}
\end{Highlighting}
\end{Shaded}

Sample Run:

\begin{verbatim}
Enter your age: 21
Enter your height in meters: 1.72
Enter your first initial: A
Enter your name: Alice

--- Profile ---
Name: Alice
Initial: A
Age: 21 years
Height: 1.72 m
\end{verbatim}

\subsubsection{Why It Matters}\label{why-it-matters-9}

\begin{itemize}
\tightlist
\item
  \texttt{printf} makes your program communicate results.
\item
  \texttt{scanf} makes your program interactive, letting users supply
  data.
\item
  Together, they turn static programs into useful tools.
\end{itemize}

Almost every C program you'll write uses these functions in some way.

\subsubsection{Exercises}\label{exercises-9}

\begin{enumerate}
\def\labelenumi{\arabic{enumi}.}
\item
  Write a program that asks the user for two integers and prints their
  sum.
\item
  Ask the user for their name and favorite number, then print:

\begin{verbatim}
Hello NAME, your favorite number is N.
\end{verbatim}
\item
  Modify the full example to include weight in kilograms and print the
  BMI (Body Mass Index = weight / (height - height)).
\end{enumerate}

\subsection{Problems}\label{problems}

\subsubsection{1. My First Profile}\label{my-first-profile}

Write a program that asks the user for their name, age, and favorite
character, then prints them in a short introduction. Example run:

\begin{verbatim}
Enter your name: Alice
Enter your age: 21
Enter your favorite character: Z

Hello Alice! You are 21 years old and your favorite character is Z.
\end{verbatim}

\subsubsection{2. Temperature Converter}\label{temperature-converter}

Ask the user for a temperature in Celsius and print it in Fahrenheit
using the formula:

\begin{verbatim}
F = C × 9/5 + 32
\end{verbatim}

\subsubsection{3. Rectangle Calculator}\label{rectangle-calculator}

Read two integers from the user: \texttt{length} and \texttt{width}.
Print both the area and the perimeter of the rectangle.

\subsubsection{4. Circle Calculator (with
const)}\label{circle-calculator-with-const}

Use a \texttt{const\ float\ pi\ =\ 3.14159f;} to calculate the area and
circumference of a circle given its radius. Input the radius from the
user.

\subsubsection{5. Minutes in a Week (with
constexpr)}\label{minutes-in-a-week-with-constexpr}

Define a \texttt{constexpr\ int} to represent the number of minutes in a
week. Print the result.

\subsubsection{6. Swap Two Numbers}\label{swap-two-numbers}

Ask the user for two integers and print them before and after swapping
their values using a temporary variable.

\subsubsection{7. Character Codes}\label{character-codes}

Ask the user to type a character. Print both the character and its ASCII
code (integer value). Example:

\begin{verbatim}
Enter a character: A
You entered: A
ASCII code: 65
\end{verbatim}

\subsubsection{8. String Greeting}\label{string-greeting}

Ask the user for their first name and print:

\begin{verbatim}
Hello, NAME! Nice to meet you.
\end{verbatim}

\subsubsection{9. Simple BMI Calculator}\label{simple-bmi-calculator}

Ask the user for height (meters) and weight (kilograms). Calculate and
print their Body Mass Index (BMI = weight / (height - height)) with 2
decimal places.

\subsubsection{10. Sum of Three Numbers}\label{sum-of-three-numbers}

Ask the user for three integers and print their sum and average.

\subsubsection{11. Days to Hours and
Minutes}\label{days-to-hours-and-minutes}

Ask the user for a number of days. Calculate and print how many hours
and minutes that equals.

\subsubsection{12. Initials Program}\label{initials-program}

Ask the user for their first, middle, and last initials (as
\texttt{char}). Print them together as one string:

\begin{verbatim}
Enter your initials: A B C
Your initials are: ABC
\end{verbatim}

\subsubsection{13. Circle Comparison}\label{circle-comparison}

Ask the user for two radii, \texttt{r1} and \texttt{r2}. Use constants
for π. Print which circle is larger by area.

\subsubsection{14. Student Pass/Fail}\label{student-passfail}

Ask the user for an integer grade (0--100). Print \texttt{Pass} if it is
50 or above, otherwise \texttt{Fail}.

\subsubsection{15. Profile with Multiple
Types}\label{profile-with-multiple-types}

Declare and initialize:

\begin{itemize}
\tightlist
\item
  An \texttt{int} for your current year (e.g.~2025)
\item
  A \texttt{float} for your height
\item
  A \texttt{char} for your grade
\item
  A \texttt{char{[}{]}} string for your name
\end{itemize}

Print them all in a formatted way, like:

\begin{verbatim}
Name: Alice
Year: 2025
Height: 1.72 m
Grade: A
\end{verbatim}

\subsubsection{16. Age in Seconds}\label{age-in-seconds}

Ask the user for their age in years. Approximate and print how many
seconds they have lived, assuming 365 days per year.

\subsubsection{17. Favorite Number Game}\label{favorite-number-game}

Ask the user for their favorite integer. Print the number, its square,
and its cube.

\subsubsection{18. Welcome Banner}\label{welcome-banner}

Ask the user for their name and print it inside a ``banner'' made of
\texttt{-} symbols:

\begin{verbatim}
Enter your name: Bob

- Bob  -
\end{verbatim}

\subsubsection{19. Type Limits Exploration
(Bonus)}\label{type-limits-exploration-bonus}

Use \texttt{\textless{}limits.h\textgreater{}} and
\texttt{\textless{}float.h\textgreater{}} to print the minimum and
maximum values of \texttt{int}, \texttt{unsigned\ int}, \texttt{float},
and \texttt{double}. (Not beginner-essential, but a good exploration.)

\subsubsection{20. Combine Everything}\label{combine-everything}

Write a program that:

\begin{enumerate}
\def\labelenumi{\arabic{enumi}.}
\tightlist
\item
  Uses \texttt{const} for π.
\item
  Reads an integer \texttt{age}, a float \texttt{height}, a char
  \texttt{initial}, and a string \texttt{name}.
\item
  Prints them all back with labels. This ``mini-profile'' program should
  combine all concepts from Chapter 2.
\end{enumerate}

\section{Chapter 3. Expressions and
Operators}\label{chapter-3.-expressions-and-operators}

\subsection{3.1 Arithmetic Operators}\label{arithmetic-operators}

Arithmetic operators let your program calculate with numbers. In C, the
basic operators are:

\begin{itemize}
\tightlist
\item
  \texttt{+} (addition)
\item
  \texttt{-} (subtraction / unary negation)
\item
  \texttt{-} (multiplication)
\item
  \texttt{/} (division)
\item
  \texttt{\%} (remainder, modulo - integers only)
\end{itemize}

You'll use them with both integers (\texttt{int}, \texttt{long}, \ldots)
and floating-point numbers (\texttt{float}, \texttt{double}).

\subsubsection{Quick Examples}\label{quick-examples}

\begin{Shaded}
\begin{Highlighting}[]
\DataTypeTok{int}\NormalTok{ a }\OperatorTok{=} \DecValTok{7} \OperatorTok{+} \DecValTok{3}\OperatorTok{;}       \CommentTok{// 10}
\DataTypeTok{int}\NormalTok{ b }\OperatorTok{=} \DecValTok{7} \OperatorTok{{-}} \DecValTok{3}\OperatorTok{;}       \CommentTok{// 4}
\DataTypeTok{int}\NormalTok{ c }\OperatorTok{=} \DecValTok{7} \OperatorTok{{-}} \DecValTok{3}\OperatorTok{;}       \CommentTok{// 21}
\DataTypeTok{int}\NormalTok{ d }\OperatorTok{=} \DecValTok{7} \OperatorTok{/} \DecValTok{3}\OperatorTok{;}       \CommentTok{// 2  (integer division truncates the fraction)}
\DataTypeTok{int}\NormalTok{ r }\OperatorTok{=} \DecValTok{7} \OperatorTok{\%} \DecValTok{3}\OperatorTok{;}       \CommentTok{// 1  (remainder)}

\DataTypeTok{float}\NormalTok{ x }\OperatorTok{=} \FloatTok{7.0}\BuiltInTok{f} \OperatorTok{/} \DecValTok{3}\OperatorTok{;}  \CommentTok{// 2.333333 (floating division)}
\DataTypeTok{double}\NormalTok{ y }\OperatorTok{=} \DecValTok{7} \OperatorTok{/} \FloatTok{3.0}\OperatorTok{;}  \CommentTok{// 2.333333 (promoted to double)}
\end{Highlighting}
\end{Shaded}

\subsubsection{Unary Minus}\label{unary-minus}

You can flip the sign of a number with unary \texttt{-}:

\begin{Shaded}
\begin{Highlighting}[]
\DataTypeTok{int}\NormalTok{ k }\OperatorTok{=} \DecValTok{5}\OperatorTok{;}
\DataTypeTok{int}\NormalTok{ neg }\OperatorTok{=} \OperatorTok{{-}}\NormalTok{k}\OperatorTok{;}   \CommentTok{// {-}5}
\end{Highlighting}
\end{Shaded}

\subsubsection{\texorpdfstring{Modulo
(\texttt{\%})}{Modulo (\%)}}\label{modulo}

The \texttt{\%} operator gives the remainder after integer division:

\begin{Shaded}
\begin{Highlighting}[]
\DataTypeTok{int}\NormalTok{ r1 }\OperatorTok{=} \DecValTok{11} \OperatorTok{\%} \DecValTok{4}\OperatorTok{;}   \CommentTok{// 3  (11 = 2{-}4 + 3)}
\DataTypeTok{int}\NormalTok{ r2 }\OperatorTok{=} \DecValTok{7} \OperatorTok{\%} \DecValTok{2}\OperatorTok{;}    \CommentTok{// 1}
\end{Highlighting}
\end{Shaded}

Modulo is useful for tasks like checking even/odd (\texttt{n\ \%\ 2}).

\subsubsection{A Full Example}\label{a-full-example-4}

This program shows all arithmetic operators with two integers and also
demonstrates floating division:

\begin{Shaded}
\begin{Highlighting}[]
\PreprocessorTok{\#include }\ImportTok{\textless{}stdio.h\textgreater{}}

\DataTypeTok{int}\NormalTok{ main}\OperatorTok{(}\DataTypeTok{void}\OperatorTok{)} \OperatorTok{\{}
    \DataTypeTok{int}\NormalTok{ a}\OperatorTok{,}\NormalTok{ b}\OperatorTok{;}
\NormalTok{    printf}\OperatorTok{(}\StringTok{"Enter two integers (a b): "}\OperatorTok{);}
\NormalTok{    scanf}\OperatorTok{(}\StringTok{"}\SpecialCharTok{\%d}\StringTok{ }\SpecialCharTok{\%d}\StringTok{"}\OperatorTok{,} \OperatorTok{\&}\NormalTok{a}\OperatorTok{,} \OperatorTok{\&}\NormalTok{b}\OperatorTok{);}

    \DataTypeTok{int}\NormalTok{ sum  }\OperatorTok{=}\NormalTok{ a }\OperatorTok{+}\NormalTok{ b}\OperatorTok{;}
    \DataTypeTok{int}\NormalTok{ diff }\OperatorTok{=}\NormalTok{ a }\OperatorTok{{-}}\NormalTok{ b}\OperatorTok{;}
    \DataTypeTok{int}\NormalTok{ prod }\OperatorTok{=}\NormalTok{ a }\OperatorTok{{-}}\NormalTok{ b}\OperatorTok{;}
    \DataTypeTok{int}\NormalTok{ quot }\OperatorTok{=}\NormalTok{ a }\OperatorTok{/}\NormalTok{ b}\OperatorTok{;}      \CommentTok{// integer division}
    \DataTypeTok{int}\NormalTok{ rem  }\OperatorTok{=}\NormalTok{ a }\OperatorTok{\%}\NormalTok{ b}\OperatorTok{;}      \CommentTok{// remainder}
    \DataTypeTok{double}\NormalTok{ fquot }\OperatorTok{=} \OperatorTok{(}\DataTypeTok{double}\OperatorTok{)}\NormalTok{a }\OperatorTok{/}\NormalTok{ b}\OperatorTok{;} \CommentTok{// floating division}

\NormalTok{    printf}\OperatorTok{(}\StringTok{"}\SpecialCharTok{\textbackslash{}n}\StringTok{{-}{-}{-} Results {-}{-}{-}}\SpecialCharTok{\textbackslash{}n}\StringTok{"}\OperatorTok{);}
\NormalTok{    printf}\OperatorTok{(}\StringTok{"}\SpecialCharTok{\%d}\StringTok{ + }\SpecialCharTok{\%d}\StringTok{ = }\SpecialCharTok{\%d\textbackslash{}n}\StringTok{"}\OperatorTok{,}\NormalTok{ a}\OperatorTok{,}\NormalTok{ b}\OperatorTok{,}\NormalTok{ sum}\OperatorTok{);}
\NormalTok{    printf}\OperatorTok{(}\StringTok{"}\SpecialCharTok{\%d}\StringTok{ {-} }\SpecialCharTok{\%d}\StringTok{ = }\SpecialCharTok{\%d\textbackslash{}n}\StringTok{"}\OperatorTok{,}\NormalTok{ a}\OperatorTok{,}\NormalTok{ b}\OperatorTok{,}\NormalTok{ diff}\OperatorTok{);}
\NormalTok{    printf}\OperatorTok{(}\StringTok{"}\SpecialCharTok{\%d}\StringTok{ {-} }\SpecialCharTok{\%d}\StringTok{ = }\SpecialCharTok{\%d\textbackslash{}n}\StringTok{"}\OperatorTok{,}\NormalTok{ a}\OperatorTok{,}\NormalTok{ b}\OperatorTok{,}\NormalTok{ prod}\OperatorTok{);}
\NormalTok{    printf}\OperatorTok{(}\StringTok{"}\SpecialCharTok{\%d}\StringTok{ / }\SpecialCharTok{\%d}\StringTok{ = }\SpecialCharTok{\%d}\StringTok{ (integer division)}\SpecialCharTok{\textbackslash{}n}\StringTok{"}\OperatorTok{,}\NormalTok{ a}\OperatorTok{,}\NormalTok{ b}\OperatorTok{,}\NormalTok{ quot}\OperatorTok{);}
\NormalTok{    printf}\OperatorTok{(}\StringTok{"}\SpecialCharTok{\%d}\StringTok{ }\SpecialCharTok{\%\%}\StringTok{ }\SpecialCharTok{\%d}\StringTok{ = }\SpecialCharTok{\%d}\StringTok{ (remainder)}\SpecialCharTok{\textbackslash{}n}\StringTok{"}\OperatorTok{,}\NormalTok{ a}\OperatorTok{,}\NormalTok{ b}\OperatorTok{,}\NormalTok{ rem}\OperatorTok{);}
\NormalTok{    printf}\OperatorTok{(}\StringTok{"}\SpecialCharTok{\%d}\StringTok{ / }\SpecialCharTok{\%d}\StringTok{ = }\SpecialCharTok{\%.6f}\StringTok{ (floating division)}\SpecialCharTok{\textbackslash{}n}\StringTok{"}\OperatorTok{,}\NormalTok{ a}\OperatorTok{,}\NormalTok{ b}\OperatorTok{,}\NormalTok{ fquot}\OperatorTok{);}
\NormalTok{    printf}\OperatorTok{(}\StringTok{"Unary minus of }\SpecialCharTok{\%d}\StringTok{ is }\SpecialCharTok{\%d\textbackslash{}n}\StringTok{"}\OperatorTok{,}\NormalTok{ a}\OperatorTok{,} \OperatorTok{*}\NormalTok{a}\OperatorTok{);}

    \ControlFlowTok{return} \DecValTok{0}\OperatorTok{;}
\OperatorTok{\}}
\end{Highlighting}
\end{Shaded}

Example run:

\begin{Shaded}
\begin{Highlighting}[]
\ExtensionTok{Enter}\NormalTok{ two integers }\ErrorTok{(}\ExtensionTok{a}\NormalTok{ b}\KeywordTok{)}\BuiltInTok{:}\NormalTok{ 11 4}

\ExtensionTok{{-}{-}{-}}\NormalTok{ Results }\AttributeTok{{-}{-}{-}}
\ExtensionTok{11}\NormalTok{ + 4 = 15}
\ExtensionTok{11} \AttributeTok{{-}}\NormalTok{ 4 = 7}
\ExtensionTok{11} \AttributeTok{{-}}\NormalTok{ 4 = 44}
\ExtensionTok{11}\NormalTok{ / 4 = 2 }\ErrorTok{(}\ExtensionTok{integer}\NormalTok{ division}\KeywordTok{)}
\ExtensionTok{11}\NormalTok{ \% 4 = 3 }\ErrorTok{(}\ExtensionTok{remainder}\KeywordTok{)}
\ExtensionTok{11}\NormalTok{ / 4 = 2.750000 }\ErrorTok{(}\ExtensionTok{floating}\NormalTok{ division}\KeywordTok{)}
\ExtensionTok{Unary}\NormalTok{ minus of 11 is }\AttributeTok{{-}11}
\end{Highlighting}
\end{Shaded}

\subsubsection{Why It Matters}\label{why-it-matters-10}

Arithmetic operators are the foundation of programming. They let you
move beyond printing fixed text and start writing programs that
calculate answers. Understanding the difference between integer and
floating division is one of the first ``aha!'' moments for every C
beginner.

\subsubsection{Exercises}\label{exercises-10}

\begin{enumerate}
\def\labelenumi{\arabic{enumi}.}
\tightlist
\item
  Write a program that reads two integers and prints their sum,
  difference, product, integer division, remainder, and floating
  division.
\item
  Read one integer and print its square and cube.
\item
  Read one integer and print whether it is even or odd using
  \texttt{\%}. (Just print the remainder, we'll learn \texttt{if}
  later.)
\item
  Read three integers and print their total and average (as floating).
\item
  Read two integers and print the result of applying unary minus to
  each.
\end{enumerate}

\subsection{3.2 Assignment and
Precedence}\label{assignment-and-precedence}

So far we've calculated values and stored them in variables like this:

\begin{Shaded}
\begin{Highlighting}[]
\DataTypeTok{int}\NormalTok{ sum }\OperatorTok{=}\NormalTok{ a }\OperatorTok{+}\NormalTok{ b}\OperatorTok{;}
\end{Highlighting}
\end{Shaded}

This uses the assignment operator \texttt{=}. It means: -take the value
on the right-hand side and store it into the variable on the left-hand
side.-

\subsubsection{Basic Assignment}\label{basic-assignment}

\begin{Shaded}
\begin{Highlighting}[]
\DataTypeTok{int}\NormalTok{ x}\OperatorTok{;}     \CommentTok{// declare a variable}
\NormalTok{x }\OperatorTok{=} \DecValTok{5}\OperatorTok{;}     \CommentTok{// assign the value 5 to x}
\end{Highlighting}
\end{Shaded}

After this, \texttt{x} holds the number \texttt{5}.

You can change it later:

\begin{Shaded}
\begin{Highlighting}[]
\NormalTok{x }\OperatorTok{=} \DecValTok{10}\OperatorTok{;}    \CommentTok{// now x holds 10}
\end{Highlighting}
\end{Shaded}

\subsubsection{Combined Assignments}\label{combined-assignments}

C has shorthand operators that combine calculation and assignment:

\begin{itemize}
\tightlist
\item
  \texttt{x\ +=\ y;} → same as \texttt{x\ =\ x\ +\ y;}
\item
  \texttt{x\ -=\ y;} → same as \texttt{x\ =\ x\ -\ y;}
\item
  \texttt{x\ -=\ y;} → same as \texttt{x\ =\ x\ -\ y;}
\item
  \texttt{x\ /=\ y;} → same as \texttt{x\ =\ x\ /\ y;}
\item
  \texttt{x\ \%=\ y;} → same as \texttt{x\ =\ x\ \%\ y;}
\end{itemize}

These make code shorter and often clearer.

\subsubsection{Operator Precedence}\label{operator-precedence}

When an expression has multiple operators, C needs to decide which to do
first. This is called precedence.

General order (from higher to lower, the ones we've seen so far):

\begin{enumerate}
\def\labelenumi{\arabic{enumi}.}
\tightlist
\item
  Parentheses \texttt{(\ )}
\item
  Unary minus \texttt{-x}
\item
  Multiplication, division, modulo \texttt{-\ /\ \%}
\item
  Addition, subtraction \texttt{+\ -}
\item
  Assignment \texttt{=}
\end{enumerate}

So:

\begin{Shaded}
\begin{Highlighting}[]
\DataTypeTok{int}\NormalTok{ r }\OperatorTok{=} \DecValTok{2} \OperatorTok{+} \DecValTok{3} \OperatorTok{{-}} \DecValTok{4}\OperatorTok{;}    \CommentTok{// 3{-}4 happens first → 2 + 12 = 14}
\DataTypeTok{int}\NormalTok{ s }\OperatorTok{=} \OperatorTok{(}\DecValTok{2} \OperatorTok{+} \DecValTok{3}\OperatorTok{)} \OperatorTok{{-}} \DecValTok{4}\OperatorTok{;}  \CommentTok{// parentheses force addition first → 5{-}4 = 20}
\end{Highlighting}
\end{Shaded}

Always use parentheses when in doubt. They make code easier to read and
prevent mistakes.

\subsubsection{A Full Example}\label{a-full-example-5}

This program shows assignment, combined assignments, and precedence:

\begin{Shaded}
\begin{Highlighting}[]
\PreprocessorTok{\#include }\ImportTok{\textless{}stdio.h\textgreater{}}

\DataTypeTok{int}\NormalTok{ main}\OperatorTok{(}\DataTypeTok{void}\OperatorTok{)} \OperatorTok{\{}
    \DataTypeTok{int}\NormalTok{ x }\OperatorTok{=} \DecValTok{10}\OperatorTok{;}
    \DataTypeTok{int}\NormalTok{ y }\OperatorTok{=} \DecValTok{3}\OperatorTok{;}

\NormalTok{    printf}\OperatorTok{(}\StringTok{"Initial: x = }\SpecialCharTok{\%d}\StringTok{, y = }\SpecialCharTok{\%d\textbackslash{}n}\StringTok{"}\OperatorTok{,}\NormalTok{ x}\OperatorTok{,}\NormalTok{ y}\OperatorTok{);}

    \CommentTok{// Basic assignment}
\NormalTok{    x }\OperatorTok{=}\NormalTok{ x }\OperatorTok{+}\NormalTok{ y}\OperatorTok{;}
\NormalTok{    printf}\OperatorTok{(}\StringTok{"x = x + y → }\SpecialCharTok{\%d\textbackslash{}n}\StringTok{"}\OperatorTok{,}\NormalTok{ x}\OperatorTok{);}

    \CommentTok{// Reset x}
\NormalTok{    x }\OperatorTok{=} \DecValTok{10}\OperatorTok{;}

    \CommentTok{// Combined assignment}
\NormalTok{    x }\OperatorTok{+=}\NormalTok{ y}\OperatorTok{;}
\NormalTok{    printf}\OperatorTok{(}\StringTok{"x += y → }\SpecialCharTok{\%d\textbackslash{}n}\StringTok{"}\OperatorTok{,}\NormalTok{ x}\OperatorTok{);}

\NormalTok{    x }\OperatorTok{{-}=}\NormalTok{ y}\OperatorTok{;}
\NormalTok{    printf}\OperatorTok{(}\StringTok{"x {-}= y → }\SpecialCharTok{\%d\textbackslash{}n}\StringTok{"}\OperatorTok{,}\NormalTok{ x}\OperatorTok{);}

\NormalTok{    x }\OperatorTok{{-}=}\NormalTok{ y}\OperatorTok{;}
\NormalTok{    printf}\OperatorTok{(}\StringTok{"x {-}= y → }\SpecialCharTok{\%d\textbackslash{}n}\StringTok{"}\OperatorTok{,}\NormalTok{ x}\OperatorTok{);}

\NormalTok{    x }\OperatorTok{/=}\NormalTok{ y}\OperatorTok{;}
\NormalTok{    printf}\OperatorTok{(}\StringTok{"x /= y → }\SpecialCharTok{\%d\textbackslash{}n}\StringTok{"}\OperatorTok{,}\NormalTok{ x}\OperatorTok{);}

\NormalTok{    x }\OperatorTok{=} \DecValTok{10}\OperatorTok{;}
\NormalTok{    x }\OperatorTok{\%=}\NormalTok{ y}\OperatorTok{;}
\NormalTok{    printf}\OperatorTok{(}\StringTok{"x }\SpecialCharTok{\%\%}\StringTok{= y → }\SpecialCharTok{\%d\textbackslash{}n}\StringTok{"}\OperatorTok{,}\NormalTok{ x}\OperatorTok{);}

    \CommentTok{// Precedence}
    \DataTypeTok{int}\NormalTok{ a }\OperatorTok{=} \DecValTok{2} \OperatorTok{+} \DecValTok{3} \OperatorTok{{-}} \DecValTok{4}\OperatorTok{;}
    \DataTypeTok{int}\NormalTok{ b }\OperatorTok{=} \OperatorTok{(}\DecValTok{2} \OperatorTok{+} \DecValTok{3}\OperatorTok{)} \OperatorTok{{-}} \DecValTok{4}\OperatorTok{;}
\NormalTok{    printf}\OperatorTok{(}\StringTok{"2 + 3 {-} 4 = }\SpecialCharTok{\%d\textbackslash{}n}\StringTok{"}\OperatorTok{,}\NormalTok{ a}\OperatorTok{);}
\NormalTok{    printf}\OperatorTok{(}\StringTok{"(2 + 3) {-} 4 = }\SpecialCharTok{\%d\textbackslash{}n}\StringTok{"}\OperatorTok{,}\NormalTok{ b}\OperatorTok{);}

    \ControlFlowTok{return} \DecValTok{0}\OperatorTok{;}
\OperatorTok{\}}
\end{Highlighting}
\end{Shaded}

Example run:

\begin{Shaded}
\begin{Highlighting}[]
\ExtensionTok{Initial:}\NormalTok{ x = 10, y = 3}
\ExtensionTok{x}\NormalTok{ = x + y → 13}
\ExtensionTok{x}\NormalTok{ += y → 13}
\ExtensionTok{x} \AttributeTok{{-}}\OperatorTok{=}\NormalTok{ y → 10}
\ExtensionTok{x} \AttributeTok{{-}}\OperatorTok{=}\NormalTok{ y → 30}
\ExtensionTok{x}\NormalTok{ /= y → 10}
\ExtensionTok{x}\NormalTok{ \%= y → 1}
\ExtensionTok{2}\NormalTok{ + 3 }\AttributeTok{{-}}\NormalTok{ 4 = 14}
\KeywordTok{(}\ExtensionTok{2}\NormalTok{ + 3}\KeywordTok{)} \ExtensionTok{{-}}\NormalTok{ 4 = 20}
\end{Highlighting}
\end{Shaded}

\subsubsection{Why It Matters}\label{why-it-matters-11}

\begin{itemize}
\tightlist
\item
  Assignment is how you store results into variables.
\item
  Combined assignments make updates simpler (important in loops later).
\item
  Precedence ensures the computer reads your math the same way you do -
  or lets you override with parentheses.
\end{itemize}

\subsubsection{Exercises}\label{exercises-11}

\begin{enumerate}
\def\labelenumi{\arabic{enumi}.}
\tightlist
\item
  Start with \texttt{int\ n\ =\ 10;}. Use \texttt{+=}, \texttt{-=},
  \texttt{-=}, \texttt{/=}, \texttt{\%=} with another integer and print
  the result after each step.
\item
  Calculate \texttt{5\ +\ 2\ -\ 3} and \texttt{(5\ +\ 2)\ -\ 3} and
  print both results.
\item
  Write a program that computes \texttt{10\ -\ 4\ +\ 2\ -\ 3}. Then add
  parentheses in different places and print the different results.
\item
  Declare \texttt{int\ a\ =\ 7,\ b\ =\ 2;}. Compute and print
  \texttt{(a\ +\ b)\ -\ (a\ -\ b)}.
\item
  Show that \texttt{x\ =\ x\ +\ 1;} and \texttt{x\ +=\ 1;} give the same
  result.
\end{enumerate}

\subsection{3.3 Relational and Logical
Operators}\label{relational-and-logical-operators}

Programs often need to compare values or combine conditions. C provides
operators for this. They don't give you numbers like \texttt{5} or
\texttt{7} - instead they produce results that are either:

\begin{itemize}
\tightlist
\item
  1 → means \emph{true}
\item
  0 → means \emph{false}
\end{itemize}

This is how C represents truth values internally.

\subsubsection{Relational Operators}\label{relational-operators}

These compare two values:

\begin{itemize}
\tightlist
\item
  \texttt{==} → equal
\item
  \texttt{!=} → not equal
\item
  \texttt{\textless{}} → less than
\item
  \texttt{\textgreater{}} → greater than
\item
  \texttt{\textless{}=} → less than or equal
\item
  \texttt{\textgreater{}=} → greater than or equal
\end{itemize}

Examples:

\begin{Shaded}
\begin{Highlighting}[]
\DataTypeTok{int}\NormalTok{ a }\OperatorTok{=} \DecValTok{5}\OperatorTok{,}\NormalTok{ b }\OperatorTok{=} \DecValTok{3}\OperatorTok{;}

\DataTypeTok{int}\NormalTok{ r1 }\OperatorTok{=} \OperatorTok{(}\NormalTok{a }\OperatorTok{==}\NormalTok{ b}\OperatorTok{);}  \CommentTok{// 0 (false)}
\DataTypeTok{int}\NormalTok{ r2 }\OperatorTok{=} \OperatorTok{(}\NormalTok{a }\OperatorTok{\textgreater{}}\NormalTok{ b}\OperatorTok{);}   \CommentTok{// 1 (true)}
\DataTypeTok{int}\NormalTok{ r3 }\OperatorTok{=} \OperatorTok{(}\NormalTok{a }\OperatorTok{\textless{}=} \DecValTok{5}\OperatorTok{);}  \CommentTok{// 1 (true)}
\end{Highlighting}
\end{Shaded}

\subsubsection{Logical Operators}\label{logical-operators}

These combine true/false values:

\begin{itemize}
\tightlist
\item
  \texttt{\&\&} → logical AND (true only if both are true)
\item
  \texttt{\textbar{}\textbar{}} → logical OR (true if at least one is
  true)
\item
  \texttt{!} → logical NOT (flips true/false)
\end{itemize}

Examples:

\begin{Shaded}
\begin{Highlighting}[]
\DataTypeTok{int}\NormalTok{ x }\OperatorTok{=} \DecValTok{1}\OperatorTok{,}\NormalTok{ y }\OperatorTok{=} \DecValTok{0}\OperatorTok{;}

\DataTypeTok{int}\NormalTok{ r1 }\OperatorTok{=} \OperatorTok{(}\NormalTok{x }\OperatorTok{\&\&}\NormalTok{ y}\OperatorTok{);} \CommentTok{// 0 (1 AND 0 is false)}
\DataTypeTok{int}\NormalTok{ r2 }\OperatorTok{=} \OperatorTok{(}\NormalTok{x }\OperatorTok{||}\NormalTok{ y}\OperatorTok{);} \CommentTok{// 1 (1 OR 0 is true)}
\DataTypeTok{int}\NormalTok{ r3 }\OperatorTok{=} \OperatorTok{(!}\NormalTok{x}\OperatorTok{);}     \CommentTok{// 0 (NOT 1 is false)}
\end{Highlighting}
\end{Shaded}

\subsubsection{A Full Example}\label{a-full-example-6}

Here's a program that shows relational and logical operators in action:

\begin{Shaded}
\begin{Highlighting}[]
\PreprocessorTok{\#include }\ImportTok{\textless{}stdio.h\textgreater{}}

\DataTypeTok{int}\NormalTok{ main}\OperatorTok{(}\DataTypeTok{void}\OperatorTok{)} \OperatorTok{\{}
    \DataTypeTok{int}\NormalTok{ a}\OperatorTok{,}\NormalTok{ b}\OperatorTok{;}
\NormalTok{    printf}\OperatorTok{(}\StringTok{"Enter two integers (a b): "}\OperatorTok{);}
\NormalTok{    scanf}\OperatorTok{(}\StringTok{"}\SpecialCharTok{\%d}\StringTok{ }\SpecialCharTok{\%d}\StringTok{"}\OperatorTok{,} \OperatorTok{\&}\NormalTok{a}\OperatorTok{,} \OperatorTok{\&}\NormalTok{b}\OperatorTok{);}

\NormalTok{    printf}\OperatorTok{(}\StringTok{"}\SpecialCharTok{\textbackslash{}n}\StringTok{{-}{-}{-} Relational {-}{-}{-}}\SpecialCharTok{\textbackslash{}n}\StringTok{"}\OperatorTok{);}
\NormalTok{    printf}\OperatorTok{(}\StringTok{"}\SpecialCharTok{\%d}\StringTok{ == }\SpecialCharTok{\%d}\StringTok{ → }\SpecialCharTok{\%d\textbackslash{}n}\StringTok{"}\OperatorTok{,}\NormalTok{ a}\OperatorTok{,}\NormalTok{ b}\OperatorTok{,}\NormalTok{ a }\OperatorTok{==}\NormalTok{ b}\OperatorTok{);}
\NormalTok{    printf}\OperatorTok{(}\StringTok{"}\SpecialCharTok{\%d}\StringTok{ != }\SpecialCharTok{\%d}\StringTok{ → }\SpecialCharTok{\%d\textbackslash{}n}\StringTok{"}\OperatorTok{,}\NormalTok{ a}\OperatorTok{,}\NormalTok{ b}\OperatorTok{,}\NormalTok{ a }\OperatorTok{!=}\NormalTok{ b}\OperatorTok{);}
\NormalTok{    printf}\OperatorTok{(}\StringTok{"}\SpecialCharTok{\%d}\StringTok{ \textless{}  }\SpecialCharTok{\%d}\StringTok{ → }\SpecialCharTok{\%d\textbackslash{}n}\StringTok{"}\OperatorTok{,}\NormalTok{ a}\OperatorTok{,}\NormalTok{ b}\OperatorTok{,}\NormalTok{ a }\OperatorTok{\textless{}}\NormalTok{ b}\OperatorTok{);}
\NormalTok{    printf}\OperatorTok{(}\StringTok{"}\SpecialCharTok{\%d}\StringTok{ \textgreater{}  }\SpecialCharTok{\%d}\StringTok{ → }\SpecialCharTok{\%d\textbackslash{}n}\StringTok{"}\OperatorTok{,}\NormalTok{ a}\OperatorTok{,}\NormalTok{ b}\OperatorTok{,}\NormalTok{ a }\OperatorTok{\textgreater{}}\NormalTok{ b}\OperatorTok{);}
\NormalTok{    printf}\OperatorTok{(}\StringTok{"}\SpecialCharTok{\%d}\StringTok{ \textless{}= }\SpecialCharTok{\%d}\StringTok{ → }\SpecialCharTok{\%d\textbackslash{}n}\StringTok{"}\OperatorTok{,}\NormalTok{ a}\OperatorTok{,}\NormalTok{ b}\OperatorTok{,}\NormalTok{ a }\OperatorTok{\textless{}=}\NormalTok{ b}\OperatorTok{);}
\NormalTok{    printf}\OperatorTok{(}\StringTok{"}\SpecialCharTok{\%d}\StringTok{ \textgreater{}= }\SpecialCharTok{\%d}\StringTok{ → }\SpecialCharTok{\%d\textbackslash{}n}\StringTok{"}\OperatorTok{,}\NormalTok{ a}\OperatorTok{,}\NormalTok{ b}\OperatorTok{,}\NormalTok{ a }\OperatorTok{\textgreater{}=}\NormalTok{ b}\OperatorTok{);}

\NormalTok{    printf}\OperatorTok{(}\StringTok{"}\SpecialCharTok{\textbackslash{}n}\StringTok{{-}{-}{-} Logical {-}{-}{-}}\SpecialCharTok{\textbackslash{}n}\StringTok{"}\OperatorTok{);}
    \DataTypeTok{int}\NormalTok{ x }\OperatorTok{=} \OperatorTok{(}\NormalTok{a }\OperatorTok{\textgreater{}} \DecValTok{0}\OperatorTok{);}  \CommentTok{// true if a is positive}
    \DataTypeTok{int}\NormalTok{ y }\OperatorTok{=} \OperatorTok{(}\NormalTok{b }\OperatorTok{\textgreater{}} \DecValTok{0}\OperatorTok{);}  \CommentTok{// true if b is positive}

\NormalTok{    printf}\OperatorTok{(}\StringTok{"(a \textgreater{} 0) → }\SpecialCharTok{\%d\textbackslash{}n}\StringTok{"}\OperatorTok{,}\NormalTok{ x}\OperatorTok{);}
\NormalTok{    printf}\OperatorTok{(}\StringTok{"(b \textgreater{} 0) → }\SpecialCharTok{\%d\textbackslash{}n}\StringTok{"}\OperatorTok{,}\NormalTok{ y}\OperatorTok{);}
\NormalTok{    printf}\OperatorTok{(}\StringTok{"(a \textgreater{} 0) \&\& (b \textgreater{} 0) → }\SpecialCharTok{\%d\textbackslash{}n}\StringTok{"}\OperatorTok{,}\NormalTok{ x }\OperatorTok{\&\&}\NormalTok{ y}\OperatorTok{);}
\NormalTok{    printf}\OperatorTok{(}\StringTok{"(a \textgreater{} 0) || (b \textgreater{} 0) → }\SpecialCharTok{\%d\textbackslash{}n}\StringTok{"}\OperatorTok{,}\NormalTok{ x }\OperatorTok{||}\NormalTok{ y}\OperatorTok{);}
\NormalTok{    printf}\OperatorTok{(}\StringTok{"!(a \textgreater{} 0) → }\SpecialCharTok{\%d\textbackslash{}n}\StringTok{"}\OperatorTok{,} \OperatorTok{!}\NormalTok{x}\OperatorTok{);}

    \ControlFlowTok{return} \DecValTok{0}\OperatorTok{;}
\OperatorTok{\}}
\end{Highlighting}
\end{Shaded}

Example run:

\begin{Shaded}
\begin{Highlighting}[]
\ExtensionTok{Enter}\NormalTok{ two integers }\ErrorTok{(}\ExtensionTok{a}\NormalTok{ b}\KeywordTok{)}\BuiltInTok{:}\NormalTok{ 5 3}

\ExtensionTok{{-}{-}{-}}\NormalTok{ Relational }\AttributeTok{{-}{-}{-}}
\ExtensionTok{5}\NormalTok{ == 3 → 0}
\ExtensionTok{5}\NormalTok{ != 3 → 1}
\ExtensionTok{5} \OperatorTok{\textless{}}\NormalTok{  3 → 0}
\ExtensionTok{5} \OperatorTok{\textgreater{}}\NormalTok{  3 → 1}
\ExtensionTok{5} \OperatorTok{\textless{}}\NormalTok{= 3 → 0}
\ExtensionTok{5} \OperatorTok{\textgreater{}}\NormalTok{= 3 → 1}

\ExtensionTok{{-}{-}{-}}\NormalTok{ Logical }\AttributeTok{{-}{-}{-}}
\KeywordTok{(}\ExtensionTok{a} \OperatorTok{\textgreater{}}\NormalTok{ 0}\KeywordTok{)} \ExtensionTok{→}\NormalTok{ 1}
\KeywordTok{(}\ExtensionTok{b} \OperatorTok{\textgreater{}}\NormalTok{ 0}\KeywordTok{)} \ExtensionTok{→}\NormalTok{ 1}
\KeywordTok{(}\ExtensionTok{a} \OperatorTok{\textgreater{}}\NormalTok{ 0}\KeywordTok{)} \KeywordTok{\&\&} \KeywordTok{(}\ExtensionTok{b} \OperatorTok{\textgreater{}}\NormalTok{ 0}\KeywordTok{)} \ExtensionTok{→}\NormalTok{ 1}
\KeywordTok{(}\ExtensionTok{a} \OperatorTok{\textgreater{}}\NormalTok{ 0}\KeywordTok{)} \KeywordTok{||} \KeywordTok{(}\ExtensionTok{b} \OperatorTok{\textgreater{}}\NormalTok{ 0}\KeywordTok{)} \ExtensionTok{→}\NormalTok{ 1}
\ExtensionTok{!}\ErrorTok{(}\ExtensionTok{a} \OperatorTok{\textgreater{}}\NormalTok{ 0}\KeywordTok{)} \ExtensionTok{→}\NormalTok{ 0}
\end{Highlighting}
\end{Shaded}

\subsubsection{Why It Matter}\label{why-it-matter}

Relational and logical operators are how C evaluates conditions. They
let you compare numbers and combine truth values, producing results as
\texttt{1} (true) or \texttt{0} (false). This gives you a way to test
equality, check ranges, and express logical ideas directly in your code.

Even in simple printouts, these operators show how the computer
understands relationships and logic in numeric form.

\subsubsection{Exercises}\label{exercises-12}

\begin{enumerate}
\def\labelenumi{\arabic{enumi}.}
\tightlist
\item
  Read two integers and print the results of all six relational
  operators.
\item
  Read one integer and print \texttt{n\ \%\ 2\ ==\ 0} to check if it's
  even (you'll see 1 or 0).
\item
  Read two integers \texttt{a} and \texttt{b}, and print
  \texttt{(a\ \textgreater{}\ 0)\ \&\&\ (b\ \textgreater{}\ 0)}. Try
  positive and negative inputs.
\item
  Print the result of \texttt{!(a\ ==\ b)} and compare it with
  \texttt{a\ !=\ b}.
\item
  Experiment with
  \texttt{(a\ \textgreater{}\ 5)\ \textbar{}\textbar{}\ (b\ \textless{}\ 10)}
  and record the outputs for different inputs.
\end{enumerate}

\subsection{3.4 Working with Characters}\label{working-with-characters}

In C, characters are stored in variables of type \texttt{char}. Even
though they look like letters, underneath they are just small integers
representing codes (usually ASCII). This means you can compare them, add
or subtract from them, and print both the character and its numeric
code.

\subsubsection{Declaring and Printing
Characters}\label{declaring-and-printing-characters}

A character literal is written in single quotes:

\begin{Shaded}
\begin{Highlighting}[]
\DataTypeTok{char}\NormalTok{ letter }\OperatorTok{=} \CharTok{\textquotesingle{}A\textquotesingle{}}\OperatorTok{;}
\NormalTok{printf}\OperatorTok{(}\StringTok{"Letter: }\SpecialCharTok{\%c\textbackslash{}n}\StringTok{"}\OperatorTok{,}\NormalTok{ letter}\OperatorTok{);}
\end{Highlighting}
\end{Shaded}

Here \texttt{\%c} prints the character itself. If you use \texttt{\%d},
you'll see its integer code:

\begin{Shaded}
\begin{Highlighting}[]
\NormalTok{printf}\OperatorTok{(}\StringTok{"Code of }\SpecialCharTok{\%c}\StringTok{ is }\SpecialCharTok{\%d\textbackslash{}n}\StringTok{"}\OperatorTok{,}\NormalTok{ letter}\OperatorTok{,}\NormalTok{ letter}\OperatorTok{);}
\end{Highlighting}
\end{Shaded}

Output:

\begin{Shaded}
\begin{Highlighting}[]
\ExtensionTok{Code}\NormalTok{ of A is 65}
\end{Highlighting}
\end{Shaded}

\subsubsection{Character Arithmetic}\label{character-arithmetic}

Because characters are stored as numbers, you can do math with them.

\begin{Shaded}
\begin{Highlighting}[]
\DataTypeTok{char}\NormalTok{ letter }\OperatorTok{=} \CharTok{\textquotesingle{}A\textquotesingle{}}\OperatorTok{;}
\DataTypeTok{char}\NormalTok{ next }\OperatorTok{=}\NormalTok{ letter }\OperatorTok{+} \DecValTok{1}\OperatorTok{;}   \CommentTok{// \textquotesingle{}B\textquotesingle{}}
\end{Highlighting}
\end{Shaded}

Similarly:

\begin{Shaded}
\begin{Highlighting}[]
\DataTypeTok{char}\NormalTok{ digit }\OperatorTok{=} \CharTok{\textquotesingle{}3\textquotesingle{}}\OperatorTok{;}
\DataTypeTok{char}\NormalTok{ nextdigit }\OperatorTok{=}\NormalTok{ digit }\OperatorTok{+} \DecValTok{1}\OperatorTok{;}  \CommentTok{// \textquotesingle{}4\textquotesingle{}}
\end{Highlighting}
\end{Shaded}

\subsubsection{Character Comparisons}\label{character-comparisons}

You can compare characters with relational operators:

\begin{Shaded}
\begin{Highlighting}[]
\DataTypeTok{char}\NormalTok{ c }\OperatorTok{=} \CharTok{\textquotesingle{}m\textquotesingle{}}\OperatorTok{;}

\DataTypeTok{int}\NormalTok{ is\_lower }\OperatorTok{=} \OperatorTok{(}\NormalTok{c }\OperatorTok{\textgreater{}=} \CharTok{\textquotesingle{}a\textquotesingle{}} \OperatorTok{\&\&}\NormalTok{ c }\OperatorTok{\textless{}=} \CharTok{\textquotesingle{}z\textquotesingle{}}\OperatorTok{);}  \CommentTok{// 1 if lowercase}
\DataTypeTok{int}\NormalTok{ is\_upper }\OperatorTok{=} \OperatorTok{(}\NormalTok{c }\OperatorTok{\textgreater{}=} \CharTok{\textquotesingle{}A\textquotesingle{}} \OperatorTok{\&\&}\NormalTok{ c }\OperatorTok{\textless{}=} \CharTok{\textquotesingle{}Z\textquotesingle{}}\OperatorTok{);}  \CommentTok{// 1 if uppercase}
\end{Highlighting}
\end{Shaded}

This works because ASCII letters are stored in order.

\subsubsection{A Full Example}\label{a-full-example-7}

This program reads one character and shows its properties:

\begin{Shaded}
\begin{Highlighting}[]
\PreprocessorTok{\#include }\ImportTok{\textless{}stdio.h\textgreater{}}

\DataTypeTok{int}\NormalTok{ main}\OperatorTok{(}\DataTypeTok{void}\OperatorTok{)} \OperatorTok{\{}
    \DataTypeTok{char}\NormalTok{ c}\OperatorTok{;}
\NormalTok{    printf}\OperatorTok{(}\StringTok{"Enter a character: "}\OperatorTok{);}
\NormalTok{    scanf}\OperatorTok{(}\StringTok{" }\SpecialCharTok{\%c}\StringTok{"}\OperatorTok{,} \OperatorTok{\&}\NormalTok{c}\OperatorTok{);}   \CommentTok{// space before \%c skips whitespace}

\NormalTok{    printf}\OperatorTok{(}\StringTok{"}\SpecialCharTok{\textbackslash{}n}\StringTok{{-}{-}{-} Character Info {-}{-}{-}}\SpecialCharTok{\textbackslash{}n}\StringTok{"}\OperatorTok{);}
\NormalTok{    printf}\OperatorTok{(}\StringTok{"Character: }\SpecialCharTok{\%c\textbackslash{}n}\StringTok{"}\OperatorTok{,}\NormalTok{ c}\OperatorTok{);}
\NormalTok{    printf}\OperatorTok{(}\StringTok{"ASCII code: }\SpecialCharTok{\%d\textbackslash{}n}\StringTok{"}\OperatorTok{,}\NormalTok{ c}\OperatorTok{);}

\NormalTok{    printf}\OperatorTok{(}\StringTok{"Next character: }\SpecialCharTok{\%c\textbackslash{}n}\StringTok{"}\OperatorTok{,}\NormalTok{ c }\OperatorTok{+} \DecValTok{1}\OperatorTok{);}
\NormalTok{    printf}\OperatorTok{(}\StringTok{"Previous character: }\SpecialCharTok{\%c\textbackslash{}n}\StringTok{"}\OperatorTok{,}\NormalTok{ c }\OperatorTok{{-}} \DecValTok{1}\OperatorTok{);}

\NormalTok{    printf}\OperatorTok{(}\StringTok{"Is uppercase? }\SpecialCharTok{\%d\textbackslash{}n}\StringTok{"}\OperatorTok{,} \OperatorTok{(}\NormalTok{c }\OperatorTok{\textgreater{}=} \CharTok{\textquotesingle{}A\textquotesingle{}} \OperatorTok{\&\&}\NormalTok{ c }\OperatorTok{\textless{}=} \CharTok{\textquotesingle{}Z\textquotesingle{}}\OperatorTok{));}
\NormalTok{    printf}\OperatorTok{(}\StringTok{"Is lowercase? }\SpecialCharTok{\%d\textbackslash{}n}\StringTok{"}\OperatorTok{,} \OperatorTok{(}\NormalTok{c }\OperatorTok{\textgreater{}=} \CharTok{\textquotesingle{}a\textquotesingle{}} \OperatorTok{\&\&}\NormalTok{ c }\OperatorTok{\textless{}=} \CharTok{\textquotesingle{}z\textquotesingle{}}\OperatorTok{));}
\NormalTok{    printf}\OperatorTok{(}\StringTok{"Is digit?     }\SpecialCharTok{\%d\textbackslash{}n}\StringTok{"}\OperatorTok{,} \OperatorTok{(}\NormalTok{c }\OperatorTok{\textgreater{}=} \CharTok{\textquotesingle{}0\textquotesingle{}} \OperatorTok{\&\&}\NormalTok{ c }\OperatorTok{\textless{}=} \CharTok{\textquotesingle{}9\textquotesingle{}}\OperatorTok{));}

    \ControlFlowTok{return} \DecValTok{0}\OperatorTok{;}
\OperatorTok{\}}
\end{Highlighting}
\end{Shaded}

Example run:

\begin{Shaded}
\begin{Highlighting}[]
\ExtensionTok{Enter}\NormalTok{ a character: m}

\ExtensionTok{{-}{-}{-}}\NormalTok{ Character Info }\AttributeTok{{-}{-}{-}}
\ExtensionTok{Character:}\NormalTok{ m}
\ExtensionTok{ASCII}\NormalTok{ code: 109}
\ExtensionTok{Next}\NormalTok{ character: n}
\ExtensionTok{Previous}\NormalTok{ character: l}
\ExtensionTok{Is}\NormalTok{ uppercase}\PreprocessorTok{?}\NormalTok{ 0}
\ExtensionTok{Is}\NormalTok{ lowercase}\PreprocessorTok{?}\NormalTok{ 1}
\ExtensionTok{Is}\NormalTok{ digit}\PreprocessorTok{?}\NormalTok{     0}
\end{Highlighting}
\end{Shaded}

\subsubsection{Why It Matters}\label{why-it-matters-12}

Characters are the building blocks of text. Knowing that they are really
numbers helps you:

\begin{itemize}
\tightlist
\item
  understand comparisons
  (\texttt{\textquotesingle{}a\textquotesingle{}\ \textless{}\ \textquotesingle{}z\textquotesingle{}})
\item
  move through letters and digits by simple arithmetic
\item
  start working with strings later, since strings are arrays of
  characters
\end{itemize}

\subsubsection{Exercises}\label{exercises-13}

\begin{enumerate}
\def\labelenumi{\arabic{enumi}.}
\tightlist
\item
  Read one character and print its ASCII code.
\item
  Read a digit character
  (e.g.~\texttt{\textquotesingle{}5\textquotesingle{}}) and print the
  next digit.
\item
  Read a letter and print both its lowercase and uppercase version by
  adding or subtracting 32
  (\texttt{\textquotesingle{}A\textquotesingle{}\ +\ 32\ =\ \textquotesingle{}a\textquotesingle{}}).
\item
  Read one character and print whether it is between
  \texttt{\textquotesingle{}a\textquotesingle{}} and
  \texttt{\textquotesingle{}z\textquotesingle{}}.
\item
  Read one character and print the three following characters in
  sequence.
\end{enumerate}

\subsection{3.5 A First Calculator
Program}\label{a-first-calculator-program}

So far, you've learned how to use arithmetic operators, assignment,
precedence, and comparisons. Now let's combine them into a mini
calculator program that reads two numbers from the user and prints the
results of all the basic operations.

\subsubsection{The Idea}\label{the-idea}

\begin{itemize}
\tightlist
\item
  Input: two integers from the user (\texttt{a} and \texttt{b})
\item
  Output: sum, difference, product, integer division, remainder, and
  floating-point division
\end{itemize}

This lets you practice:

\begin{itemize}
\tightlist
\item
  \texttt{+\ -\ -\ /\ \%}
\item
  assignment to store results
\item
  formatted output with \texttt{printf}
\end{itemize}

\subsubsection{A Full Example}\label{a-full-example-8}

\begin{Shaded}
\begin{Highlighting}[]
\PreprocessorTok{\#include }\ImportTok{\textless{}stdio.h\textgreater{}}

\DataTypeTok{int}\NormalTok{ main}\OperatorTok{(}\DataTypeTok{void}\OperatorTok{)} \OperatorTok{\{}
    \DataTypeTok{int}\NormalTok{ a}\OperatorTok{,}\NormalTok{ b}\OperatorTok{;}

\NormalTok{    printf}\OperatorTok{(}\StringTok{"Enter two integers (a b): "}\OperatorTok{);}
\NormalTok{    scanf}\OperatorTok{(}\StringTok{"}\SpecialCharTok{\%d}\StringTok{ }\SpecialCharTok{\%d}\StringTok{"}\OperatorTok{,} \OperatorTok{\&}\NormalTok{a}\OperatorTok{,} \OperatorTok{\&}\NormalTok{b}\OperatorTok{);}

    \DataTypeTok{int}\NormalTok{ sum }\OperatorTok{=}\NormalTok{ a }\OperatorTok{+}\NormalTok{ b}\OperatorTok{;}
    \DataTypeTok{int}\NormalTok{ diff }\OperatorTok{=}\NormalTok{ a }\OperatorTok{{-}}\NormalTok{ b}\OperatorTok{;}
    \DataTypeTok{int}\NormalTok{ prod }\OperatorTok{=}\NormalTok{ a }\OperatorTok{{-}}\NormalTok{ b}\OperatorTok{;}
    \DataTypeTok{int}\NormalTok{ quot }\OperatorTok{=}\NormalTok{ a }\OperatorTok{/}\NormalTok{ b}\OperatorTok{;}         \CommentTok{// integer division}
    \DataTypeTok{int}\NormalTok{ rem  }\OperatorTok{=}\NormalTok{ a }\OperatorTok{\%}\NormalTok{ b}\OperatorTok{;}         \CommentTok{// remainder}
    \DataTypeTok{double}\NormalTok{ fquot }\OperatorTok{=} \OperatorTok{(}\DataTypeTok{double}\OperatorTok{)}\NormalTok{a }\OperatorTok{/}\NormalTok{ b}\OperatorTok{;}  \CommentTok{// floating{-}point division}

\NormalTok{    printf}\OperatorTok{(}\StringTok{"}\SpecialCharTok{\textbackslash{}n}\StringTok{{-}{-}{-} Calculator {-}{-}{-}}\SpecialCharTok{\textbackslash{}n}\StringTok{"}\OperatorTok{);}
\NormalTok{    printf}\OperatorTok{(}\StringTok{"}\SpecialCharTok{\%d}\StringTok{ + }\SpecialCharTok{\%d}\StringTok{ = }\SpecialCharTok{\%d\textbackslash{}n}\StringTok{"}\OperatorTok{,}\NormalTok{ a}\OperatorTok{,}\NormalTok{ b}\OperatorTok{,}\NormalTok{ sum}\OperatorTok{);}
\NormalTok{    printf}\OperatorTok{(}\StringTok{"}\SpecialCharTok{\%d}\StringTok{ {-} }\SpecialCharTok{\%d}\StringTok{ = }\SpecialCharTok{\%d\textbackslash{}n}\StringTok{"}\OperatorTok{,}\NormalTok{ a}\OperatorTok{,}\NormalTok{ b}\OperatorTok{,}\NormalTok{ diff}\OperatorTok{);}
\NormalTok{    printf}\OperatorTok{(}\StringTok{"}\SpecialCharTok{\%d}\StringTok{ {-} }\SpecialCharTok{\%d}\StringTok{ = }\SpecialCharTok{\%d\textbackslash{}n}\StringTok{"}\OperatorTok{,}\NormalTok{ a}\OperatorTok{,}\NormalTok{ b}\OperatorTok{,}\NormalTok{ prod}\OperatorTok{);}
\NormalTok{    printf}\OperatorTok{(}\StringTok{"}\SpecialCharTok{\%d}\StringTok{ / }\SpecialCharTok{\%d}\StringTok{ = }\SpecialCharTok{\%d}\StringTok{ (integer division)}\SpecialCharTok{\textbackslash{}n}\StringTok{"}\OperatorTok{,}\NormalTok{ a}\OperatorTok{,}\NormalTok{ b}\OperatorTok{,}\NormalTok{ quot}\OperatorTok{);}
\NormalTok{    printf}\OperatorTok{(}\StringTok{"}\SpecialCharTok{\%d}\StringTok{ }\SpecialCharTok{\%\%}\StringTok{ }\SpecialCharTok{\%d}\StringTok{ = }\SpecialCharTok{\%d}\StringTok{ (remainder)}\SpecialCharTok{\textbackslash{}n}\StringTok{"}\OperatorTok{,}\NormalTok{ a}\OperatorTok{,}\NormalTok{ b}\OperatorTok{,}\NormalTok{ rem}\OperatorTok{);}
\NormalTok{    printf}\OperatorTok{(}\StringTok{"}\SpecialCharTok{\%d}\StringTok{ / }\SpecialCharTok{\%d}\StringTok{ = }\SpecialCharTok{\%.6f}\StringTok{ (floating division)}\SpecialCharTok{\textbackslash{}n}\StringTok{"}\OperatorTok{,}\NormalTok{ a}\OperatorTok{,}\NormalTok{ b}\OperatorTok{,}\NormalTok{ fquot}\OperatorTok{);}

    \ControlFlowTok{return} \DecValTok{0}\OperatorTok{;}
\OperatorTok{\}}
\end{Highlighting}
\end{Shaded}

Example run:

\begin{Shaded}
\begin{Highlighting}[]
\ExtensionTok{Enter}\NormalTok{ two integers }\ErrorTok{(}\ExtensionTok{a}\NormalTok{ b}\KeywordTok{)}\BuiltInTok{:}\NormalTok{ 11 4}

\ExtensionTok{{-}{-}{-}}\NormalTok{ Calculator }\AttributeTok{{-}{-}{-}}
\ExtensionTok{11}\NormalTok{ + 4 = 15}
\ExtensionTok{11} \AttributeTok{{-}}\NormalTok{ 4 = 7}
\ExtensionTok{11} \AttributeTok{{-}}\NormalTok{ 4 = 44}
\ExtensionTok{11}\NormalTok{ / 4 = 2 }\ErrorTok{(}\ExtensionTok{integer}\NormalTok{ division}\KeywordTok{)}
\ExtensionTok{11}\NormalTok{ \% 4 = 3 }\ErrorTok{(}\ExtensionTok{remainder}\KeywordTok{)}
\ExtensionTok{11}\NormalTok{ / 4 = 2.750000 }\ErrorTok{(}\ExtensionTok{floating}\NormalTok{ division}\KeywordTok{)}
\end{Highlighting}
\end{Shaded}

\subsubsection{Why It Matters}\label{why-it-matters-13}

This small calculator shows how different operators behave side by side.
You see the difference between integer and floating division, how
remainders work, and how assignment stores intermediate results. It also
demonstrates how to format outputs clearly.

\subsubsection{Exercises}\label{exercises-14}

\begin{enumerate}
\def\labelenumi{\arabic{enumi}.}
\tightlist
\item
  Modify the program to also calculate the square and cube of each
  input.
\item
  Change the program to read two floating-point numbers and show their
  sum, difference, product, and division.
\item
  Add output that shows \texttt{a\ +\ b\ -\ 2} and
  \texttt{(a\ +\ b)\ -\ 2} to illustrate operator precedence.
\item
  Extend the program to ask for three integers and print their total and
  average (as floating).
\item
  Write a version that only prints the floating-point division result,
  but with 2, 4, and 8 decimal places.
\end{enumerate}

\subsection{Problems}\label{problems-1}

\subsubsection{1. Arithmetic Basics}\label{arithmetic-basics}

Write a program that reads two integers and prints their sum,
difference, product, integer division, remainder, and floating-point
division.

\subsubsection{2. Square and Cube}\label{square-and-cube}

Read one integer and print its square and cube.

\subsubsection{3. Average of Two}\label{average-of-two}

Read two integers and print both their integer average and their
floating average.

\subsubsection{\texorpdfstring{4. Even or Odd (with
\texttt{\%})}{4. Even or Odd (with \%)}}\label{even-or-odd-with}

Read one integer and print the result of \texttt{n\ \%\ 2}. Use this to
check even/odd (0 means even, 1 means odd).

\subsubsection{5. Precedence Practice}\label{precedence-practice}

Compute and print the results of:

\begin{itemize}
\tightlist
\item
  \texttt{5\ +\ 2\ -\ 3}
\item
  \texttt{(5\ +\ 2)\ -\ 3}
\item
  \texttt{10\ -\ 4\ +\ 2\ -\ 3}
\item
  \texttt{(10\ -\ 4\ +\ 2)\ -\ 3}
\end{itemize}

\subsubsection{6. Combined Assignment}\label{combined-assignment}

Start with \texttt{int\ x\ =\ 10;}. Then use \texttt{+=}, \texttt{-=},
\texttt{-=}, \texttt{/=}, \texttt{\%=} with another integer and print
the result after each step.

\subsubsection{7. Equality Test}\label{equality-test}

Read two integers and print the result of \texttt{a\ ==\ b} and
\texttt{a\ !=\ b}.

\subsubsection{8. Greater Than Check}\label{greater-than-check}

Read two integers and print the result of \texttt{a\ \textgreater{}\ b}
and \texttt{a\ \textless{}\ b}.

\subsubsection{9. Range Test}\label{range-test}

Read one integer and print whether it is between 1 and 100 using
\texttt{(n\ \textgreater{}=\ 1\ \&\&\ n\ \textless{}=\ 100)}.

\subsubsection{10. Logical OR}\label{logical-or}

Read two integers and print
\texttt{(a\ \textgreater{}\ 10\ \textbar{}\textbar{}\ b\ \textgreater{}\ 10)}.

\subsubsection{11. Logical NOT}\label{logical-not}

Read one integer and print both \texttt{n\ \textgreater{}\ 0} and
\texttt{!(n\ \textgreater{}\ 0)}.

\subsubsection{12. Character Info}\label{character-info}

Read one character and print:

\begin{itemize}
\tightlist
\item
  the character
\item
  its ASCII code
\item
  the next character (\texttt{c+1})
\item
  the previous character (\texttt{c-1})
\end{itemize}

\subsubsection{13. Uppercase or Lowercase}\label{uppercase-or-lowercase}

Read one character and print whether it is uppercase
(\texttt{\textquotesingle{}A\textquotesingle{}..\textquotesingle{}Z\textquotesingle{}})
or lowercase
(\texttt{\textquotesingle{}a\textquotesingle{}..\textquotesingle{}z\textquotesingle{}})
using relational operators.

\subsubsection{14. Digit Check}\label{digit-check}

Read one character and print whether it is a digit
(\texttt{\textquotesingle{}0\textquotesingle{}..\textquotesingle{}9\textquotesingle{}}).

\subsubsection{15. Character Math}\label{character-math}

Read a letter and print the letter 3 positions ahead (e.g., input
\texttt{A} → output \texttt{D}).

\subsubsection{16. Expression Explorer}\label{expression-explorer}

Read three integers \texttt{a,\ b,\ c} and print the result of:

\begin{itemize}
\tightlist
\item
  \texttt{a\ +\ b\ -\ c}
\item
  \texttt{(a\ +\ b)\ -\ c}
\item
  \texttt{a\ -\ b\ +\ c}
\end{itemize}

\subsubsection{\texorpdfstring{17. Increment with
\texttt{+=}}{17. Increment with +=}}\label{increment-with}

Read one integer \texttt{n}. Print \texttt{n}, then \texttt{n\ +=\ 1},
then \texttt{n\ +=\ 5}.

\subsubsection{18. Logical Combination}\label{logical-combination}

Read two integers \texttt{a,\ b}. Print the results of:

\begin{itemize}
\tightlist
\item
  \texttt{(a\ \textgreater{}\ 0\ \&\&\ b\ \textgreater{}\ 0)}
\item
  \texttt{(a\ \textgreater{}\ 0\ \textbar{}\textbar{}\ b\ \textgreater{}\ 0)}
\item
  \texttt{!(a\ \textgreater{}\ 0\ \&\&\ b\ \textgreater{}\ 0)}
\end{itemize}

\subsubsection{19. Mini Character Table}\label{mini-character-table}

Print the characters \texttt{\textquotesingle{}A\textquotesingle{}}
through \texttt{\textquotesingle{}Z\textquotesingle{}} with their ASCII
codes using a loop-free approach (declare them directly and print with
\texttt{\%c} and \texttt{\%d}).

\subsubsection{20. Mini Calculator}\label{mini-calculator}

Write a program that reads two integers and prints their sum,
difference, product, integer division, remainder, and floating division
- formatted like a calculator.

\bookmarksetup{startatroot}

\chapter{Part II. Building blocks}\label{part-ii.-building-blocks}

\section{Chapter 4. Control Flow}\label{chapter-4.-control-flow}

\subsection{\texorpdfstring{4.1 \texttt{if} and
\texttt{else}}{4.1 if and else}}\label{if-and-else}

Until now, all our programs did the same thing every time, no matter the
input. But useful programs often need to make choices:

\begin{itemize}
\tightlist
\item
  If the user is old enough → print ``Welcome.''
\item
  If a number is negative → print ``Error.''
\item
  Otherwise → do something else.
\end{itemize}

In C, decisions are made with the \texttt{if} statement.

\subsubsection{\texorpdfstring{The \texttt{if}
Statement}{The if Statement}}\label{the-if-statement}

\begin{Shaded}
\begin{Highlighting}[]
\ControlFlowTok{if} \OperatorTok{(}\NormalTok{condition}\OperatorTok{)} \OperatorTok{\{}
    \CommentTok{// statements run only if condition is true}
\OperatorTok{\}}
\end{Highlighting}
\end{Shaded}

\begin{itemize}
\tightlist
\item
  \texttt{condition} is an expression that evaluates to true (1) or
  false (0).
\item
  If it's true, the block inside \texttt{\{\}} runs.
\item
  If it's false, the block is skipped.
\end{itemize}

Example:

\begin{Shaded}
\begin{Highlighting}[]
\DataTypeTok{int}\NormalTok{ age }\OperatorTok{=} \DecValTok{20}\OperatorTok{;}
\ControlFlowTok{if} \OperatorTok{(}\NormalTok{age }\OperatorTok{\textgreater{}=} \DecValTok{18}\OperatorTok{)} \OperatorTok{\{}
\NormalTok{    printf}\OperatorTok{(}\StringTok{"You are an adult.}\SpecialCharTok{\textbackslash{}n}\StringTok{"}\OperatorTok{);}
\OperatorTok{\}}
\end{Highlighting}
\end{Shaded}

\subsubsection{\texorpdfstring{Adding
\texttt{else}}{Adding else}}\label{adding-else}

You can provide an alternative with \texttt{else}:

\begin{Shaded}
\begin{Highlighting}[]
\ControlFlowTok{if} \OperatorTok{(}\NormalTok{age }\OperatorTok{\textgreater{}=} \DecValTok{18}\OperatorTok{)} \OperatorTok{\{}
\NormalTok{    printf}\OperatorTok{(}\StringTok{"You are an adult.}\SpecialCharTok{\textbackslash{}n}\StringTok{"}\OperatorTok{);}
\OperatorTok{\}} \ControlFlowTok{else} \OperatorTok{\{}
\NormalTok{    printf}\OperatorTok{(}\StringTok{"You are a minor.}\SpecialCharTok{\textbackslash{}n}\StringTok{"}\OperatorTok{);}
\OperatorTok{\}}
\end{Highlighting}
\end{Shaded}

\subsubsection{\texorpdfstring{\texttt{if} \ldots{} \texttt{else\ if}
\ldots{}
\texttt{else}}{if \ldots{} else if \ldots{} else}}\label{if-else-if-else}

For multiple choices:

\begin{Shaded}
\begin{Highlighting}[]
\ControlFlowTok{if} \OperatorTok{(}\NormalTok{score }\OperatorTok{\textgreater{}=} \DecValTok{90}\OperatorTok{)} \OperatorTok{\{}
\NormalTok{    printf}\OperatorTok{(}\StringTok{"Grade A}\SpecialCharTok{\textbackslash{}n}\StringTok{"}\OperatorTok{);}
\OperatorTok{\}} \ControlFlowTok{else} \ControlFlowTok{if} \OperatorTok{(}\NormalTok{score }\OperatorTok{\textgreater{}=} \DecValTok{75}\OperatorTok{)} \OperatorTok{\{}
\NormalTok{    printf}\OperatorTok{(}\StringTok{"Grade B}\SpecialCharTok{\textbackslash{}n}\StringTok{"}\OperatorTok{);}
\OperatorTok{\}} \ControlFlowTok{else} \ControlFlowTok{if} \OperatorTok{(}\NormalTok{score }\OperatorTok{\textgreater{}=} \DecValTok{50}\OperatorTok{)} \OperatorTok{\{}
\NormalTok{    printf}\OperatorTok{(}\StringTok{"Grade C}\SpecialCharTok{\textbackslash{}n}\StringTok{"}\OperatorTok{);}
\OperatorTok{\}} \ControlFlowTok{else} \OperatorTok{\{}
\NormalTok{    printf}\OperatorTok{(}\StringTok{"Fail}\SpecialCharTok{\textbackslash{}n}\StringTok{"}\OperatorTok{);}
\OperatorTok{\}}
\end{Highlighting}
\end{Shaded}

The computer checks conditions in order. The first one that's true is
executed, and the rest are skipped.

\subsubsection{A Full Example}\label{a-full-example-9}

This program reads an integer and classifies it:

\begin{Shaded}
\begin{Highlighting}[]
\PreprocessorTok{\#include }\ImportTok{\textless{}stdio.h\textgreater{}}

\DataTypeTok{int}\NormalTok{ main}\OperatorTok{(}\DataTypeTok{void}\OperatorTok{)} \OperatorTok{\{}
    \DataTypeTok{int}\NormalTok{ n}\OperatorTok{;}
\NormalTok{    printf}\OperatorTok{(}\StringTok{"Enter a number: "}\OperatorTok{);}
\NormalTok{    scanf}\OperatorTok{(}\StringTok{"}\SpecialCharTok{\%d}\StringTok{"}\OperatorTok{,} \OperatorTok{\&}\NormalTok{n}\OperatorTok{);}

    \ControlFlowTok{if} \OperatorTok{(}\NormalTok{n }\OperatorTok{\textgreater{}} \DecValTok{0}\OperatorTok{)} \OperatorTok{\{}
\NormalTok{        printf}\OperatorTok{(}\StringTok{"}\SpecialCharTok{\%d}\StringTok{ is positive.}\SpecialCharTok{\textbackslash{}n}\StringTok{"}\OperatorTok{,}\NormalTok{ n}\OperatorTok{);}
    \OperatorTok{\}} \ControlFlowTok{else} \ControlFlowTok{if} \OperatorTok{(}\NormalTok{n }\OperatorTok{\textless{}} \DecValTok{0}\OperatorTok{)} \OperatorTok{\{}
\NormalTok{        printf}\OperatorTok{(}\StringTok{"}\SpecialCharTok{\%d}\StringTok{ is negative.}\SpecialCharTok{\textbackslash{}n}\StringTok{"}\OperatorTok{,}\NormalTok{ n}\OperatorTok{);}
    \OperatorTok{\}} \ControlFlowTok{else} \OperatorTok{\{}
\NormalTok{        printf}\OperatorTok{(}\StringTok{"The number is zero.}\SpecialCharTok{\textbackslash{}n}\StringTok{"}\OperatorTok{);}
    \OperatorTok{\}}

    \ControlFlowTok{return} \DecValTok{0}\OperatorTok{;}
\OperatorTok{\}}
\end{Highlighting}
\end{Shaded}

Example run:

\begin{Shaded}
\begin{Highlighting}[]
\ExtensionTok{Enter}\NormalTok{ a number: }\AttributeTok{{-}7}
\ExtensionTok{{-}7}\NormalTok{ is negative.}
\end{Highlighting}
\end{Shaded}

\subsubsection{Why It Matters}\label{why-it-matters-14}

\texttt{if} and \texttt{else} give programs the power to branch - to
take different paths depending on conditions. This is the foundation for
decision-making, error checking, and interactive behavior in all
real-world programs.

\subsubsection{Exercises}\label{exercises-15}

\begin{enumerate}
\def\labelenumi{\arabic{enumi}.}
\item
  Write a program that reads an integer and prints whether it is even or
  odd.
\item
  Write a program that reads two integers and prints which one is larger
  (or if they are equal).
\item
  Write a program that reads an exam score (0--100) and prints a letter
  grade (\texttt{A}, \texttt{B}, \texttt{C}, or \texttt{F}).
\item
  Write a program that reads a temperature in Celsius and prints:

  \begin{itemize}
  \tightlist
  \item
    \texttt{"Cold"} if less than 10
  \item
    \texttt{"Warm"} if between 10 and 25
  \item
    \texttt{"Hot"} if above 25
  \end{itemize}
\item
  Write a program that reads a character and prints whether it is
  uppercase, lowercase, or not a letter.
\end{enumerate}

\subsection{\texorpdfstring{4.2 \texttt{switch}
Statements}{4.2 switch Statements}}\label{switch-statements}

Sometimes you need to compare the same variable against several constant
values. Writing many \texttt{if\ …\ else\ if\ …\ else} lines can become
messy. The \texttt{switch} statement gives you a clearer way.

\subsubsection{The Structure}\label{the-structure}

\begin{Shaded}
\begin{Highlighting}[]
\ControlFlowTok{switch} \OperatorTok{(}\NormalTok{expression}\OperatorTok{)} \OperatorTok{\{}
    \ControlFlowTok{case}\NormalTok{ value1}\OperatorTok{:}
        \CommentTok{// code if expression == value1}
        \ControlFlowTok{break}\OperatorTok{;}
    \ControlFlowTok{case}\NormalTok{ value2}\OperatorTok{:}
        \CommentTok{// code if expression == value2}
        \ControlFlowTok{break}\OperatorTok{;}
    \OperatorTok{...}
    \ControlFlowTok{default}\OperatorTok{:}
        \CommentTok{// code if no case matches}
\OperatorTok{\}}
\end{Highlighting}
\end{Shaded}

\begin{itemize}
\tightlist
\item
  The \texttt{expression} is usually an integer or character.
\item
  Each \texttt{case} is compared against the expression.
\item
  \texttt{break;} ends that case so execution does not ``fall through''
  to the next one.
\item
  \texttt{default} runs if none of the cases match (like an ``else'').
\end{itemize}

\subsubsection{Example: Day of the Week}\label{example-day-of-the-week}

\begin{Shaded}
\begin{Highlighting}[]
\PreprocessorTok{\#include }\ImportTok{\textless{}stdio.h\textgreater{}}

\DataTypeTok{int}\NormalTok{ main}\OperatorTok{(}\DataTypeTok{void}\OperatorTok{)} \OperatorTok{\{}
    \DataTypeTok{int}\NormalTok{ day}\OperatorTok{;}
\NormalTok{    printf}\OperatorTok{(}\StringTok{"Enter a number (1–7): "}\OperatorTok{);}
\NormalTok{    scanf}\OperatorTok{(}\StringTok{"}\SpecialCharTok{\%d}\StringTok{"}\OperatorTok{,} \OperatorTok{\&}\NormalTok{day}\OperatorTok{);}

    \ControlFlowTok{switch} \OperatorTok{(}\NormalTok{day}\OperatorTok{)} \OperatorTok{\{}
        \ControlFlowTok{case} \DecValTok{1}\OperatorTok{:}\NormalTok{ printf}\OperatorTok{(}\StringTok{"Monday}\SpecialCharTok{\textbackslash{}n}\StringTok{"}\OperatorTok{);} \ControlFlowTok{break}\OperatorTok{;}
        \ControlFlowTok{case} \DecValTok{2}\OperatorTok{:}\NormalTok{ printf}\OperatorTok{(}\StringTok{"Tuesday}\SpecialCharTok{\textbackslash{}n}\StringTok{"}\OperatorTok{);} \ControlFlowTok{break}\OperatorTok{;}
        \ControlFlowTok{case} \DecValTok{3}\OperatorTok{:}\NormalTok{ printf}\OperatorTok{(}\StringTok{"Wednesday}\SpecialCharTok{\textbackslash{}n}\StringTok{"}\OperatorTok{);} \ControlFlowTok{break}\OperatorTok{;}
        \ControlFlowTok{case} \DecValTok{4}\OperatorTok{:}\NormalTok{ printf}\OperatorTok{(}\StringTok{"Thursday}\SpecialCharTok{\textbackslash{}n}\StringTok{"}\OperatorTok{);} \ControlFlowTok{break}\OperatorTok{;}
        \ControlFlowTok{case} \DecValTok{5}\OperatorTok{:}\NormalTok{ printf}\OperatorTok{(}\StringTok{"Friday}\SpecialCharTok{\textbackslash{}n}\StringTok{"}\OperatorTok{);} \ControlFlowTok{break}\OperatorTok{;}
        \ControlFlowTok{case} \DecValTok{6}\OperatorTok{:}\NormalTok{ printf}\OperatorTok{(}\StringTok{"Saturday}\SpecialCharTok{\textbackslash{}n}\StringTok{"}\OperatorTok{);} \ControlFlowTok{break}\OperatorTok{;}
        \ControlFlowTok{case} \DecValTok{7}\OperatorTok{:}\NormalTok{ printf}\OperatorTok{(}\StringTok{"Sunday}\SpecialCharTok{\textbackslash{}n}\StringTok{"}\OperatorTok{);} \ControlFlowTok{break}\OperatorTok{;}
        \ControlFlowTok{default}\OperatorTok{:}\NormalTok{ printf}\OperatorTok{(}\StringTok{"Invalid day number.}\SpecialCharTok{\textbackslash{}n}\StringTok{"}\OperatorTok{);}
    \OperatorTok{\}}

    \ControlFlowTok{return} \DecValTok{0}\OperatorTok{;}
\OperatorTok{\}}
\end{Highlighting}
\end{Shaded}

Example run:

\begin{Shaded}
\begin{Highlighting}[]
\ExtensionTok{Enter}\NormalTok{ a number }\ErrorTok{(}\ExtensionTok{1–7}\KeywordTok{)}\BuiltInTok{:}\NormalTok{ 3}
\ExtensionTok{Wednesday}
\end{Highlighting}
\end{Shaded}

\subsubsection{Example: Character Menu}\label{example-character-menu}

Because characters are really small integers, you can use them in
\texttt{switch}:

\begin{Shaded}
\begin{Highlighting}[]
\PreprocessorTok{\#include }\ImportTok{\textless{}stdio.h\textgreater{}}

\DataTypeTok{int}\NormalTok{ main}\OperatorTok{(}\DataTypeTok{void}\OperatorTok{)} \OperatorTok{\{}
    \DataTypeTok{char}\NormalTok{ choice}\OperatorTok{;}
\NormalTok{    printf}\OperatorTok{(}\StringTok{"Choose (a/b/c): "}\OperatorTok{);}
\NormalTok{    scanf}\OperatorTok{(}\StringTok{" }\SpecialCharTok{\%c}\StringTok{"}\OperatorTok{,} \OperatorTok{\&}\NormalTok{choice}\OperatorTok{);}

    \ControlFlowTok{switch} \OperatorTok{(}\NormalTok{choice}\OperatorTok{)} \OperatorTok{\{}
        \ControlFlowTok{case} \CharTok{\textquotesingle{}a\textquotesingle{}}\OperatorTok{:}\NormalTok{ printf}\OperatorTok{(}\StringTok{"You chose A.}\SpecialCharTok{\textbackslash{}n}\StringTok{"}\OperatorTok{);} \ControlFlowTok{break}\OperatorTok{;}
        \ControlFlowTok{case} \CharTok{\textquotesingle{}b\textquotesingle{}}\OperatorTok{:}\NormalTok{ printf}\OperatorTok{(}\StringTok{"You chose B.}\SpecialCharTok{\textbackslash{}n}\StringTok{"}\OperatorTok{);} \ControlFlowTok{break}\OperatorTok{;}
        \ControlFlowTok{case} \CharTok{\textquotesingle{}c\textquotesingle{}}\OperatorTok{:}\NormalTok{ printf}\OperatorTok{(}\StringTok{"You chose C.}\SpecialCharTok{\textbackslash{}n}\StringTok{"}\OperatorTok{);} \ControlFlowTok{break}\OperatorTok{;}
        \ControlFlowTok{default}\OperatorTok{:}\NormalTok{  printf}\OperatorTok{(}\StringTok{"Unknown choice.}\SpecialCharTok{\textbackslash{}n}\StringTok{"}\OperatorTok{);}
    \OperatorTok{\}}

    \ControlFlowTok{return} \DecValTok{0}\OperatorTok{;}
\OperatorTok{\}}
\end{Highlighting}
\end{Shaded}

\subsubsection{Why It Matters}\label{why-it-matters-15}

\begin{itemize}
\tightlist
\item
  \texttt{switch} is easier to read when one variable has many fixed
  options.
\item
  It avoids repetitive
  \texttt{if\ (x\ ==\ 1)\ …\ else\ if\ (x\ ==\ 2)\ …}.
\item
  Useful for menus, command selection, and state handling in larger
  programs.
\end{itemize}

\subsubsection{Exercises}\label{exercises-16}

\begin{enumerate}
\def\labelenumi{\arabic{enumi}.}
\tightlist
\item
  Write a program that reads a digit (0--9) and prints its English word
  (\texttt{"zero"}, \texttt{"one"}, \ldots).
\item
  Write a program that reads a grade character
  (\texttt{\textquotesingle{}A\textquotesingle{}},
  \texttt{\textquotesingle{}B\textquotesingle{}},
  \texttt{\textquotesingle{}C\textquotesingle{}},
  \texttt{\textquotesingle{}D\textquotesingle{}},
  \texttt{\textquotesingle{}F\textquotesingle{}}) and prints a message
  (e.g., \texttt{"Excellent"}, \texttt{"Good"}, \texttt{"Pass"},
  \texttt{"Fail"}).
\item
  Extend the day-of-week program to also print \texttt{"Weekend"} if the
  day is 6 or 7.
\item
  Write a simple calculator that reads an operator character
  (\texttt{+}, \texttt{-}, \texttt{-}, \texttt{/}) and two integers,
  then prints the result. Use \texttt{switch} to handle the operator.
\item
  Write a program that reads a character and prints whether it is a
  vowel (\texttt{a,\ e,\ i,\ o,\ u}) or consonant.
\end{enumerate}

\subsection{\texorpdfstring{4.3 \texttt{while}
Loops}{4.3 while Loops}}\label{while-loops}

Sometimes you want to do something over and over while a condition is
true. For example:

\begin{itemize}
\tightlist
\item
  Keep asking for input until the user types \texttt{0}.
\item
  Count from 1 to 10.
\item
  Process characters in a string one by one.
\end{itemize}

This is what the \texttt{while} loop is for.

\subsubsection{The Structure}\label{the-structure-1}

\begin{Shaded}
\begin{Highlighting}[]
\ControlFlowTok{while} \OperatorTok{(}\NormalTok{condition}\OperatorTok{)} \OperatorTok{\{}
    \CommentTok{// statements run repeatedly}
\OperatorTok{\}}
\end{Highlighting}
\end{Shaded}

\begin{itemize}
\tightlist
\item
  The \texttt{condition} is checked before each iteration.
\item
  If it's true, the loop body runs.
\item
  If it's false, the loop stops.
\end{itemize}

\subsubsection{Example: Counting to 5}\label{example-counting-to-5}

\begin{Shaded}
\begin{Highlighting}[]
\PreprocessorTok{\#include }\ImportTok{\textless{}stdio.h\textgreater{}}

\DataTypeTok{int}\NormalTok{ main}\OperatorTok{(}\DataTypeTok{void}\OperatorTok{)} \OperatorTok{\{}
    \DataTypeTok{int}\NormalTok{ i }\OperatorTok{=} \DecValTok{1}\OperatorTok{;}
    \ControlFlowTok{while} \OperatorTok{(}\NormalTok{i }\OperatorTok{\textless{}=} \DecValTok{5}\OperatorTok{)} \OperatorTok{\{}
\NormalTok{        printf}\OperatorTok{(}\StringTok{"}\SpecialCharTok{\%d\textbackslash{}n}\StringTok{"}\OperatorTok{,}\NormalTok{ i}\OperatorTok{);}
\NormalTok{        i }\OperatorTok{=}\NormalTok{ i }\OperatorTok{+} \DecValTok{1}\OperatorTok{;}  \CommentTok{// update}
    \OperatorTok{\}}
    \ControlFlowTok{return} \DecValTok{0}\OperatorTok{;}
\OperatorTok{\}}
\end{Highlighting}
\end{Shaded}

Output:

\begin{Shaded}
\begin{Highlighting}[]
\ExtensionTok{1}
\ExtensionTok{2}
\ExtensionTok{3}
\ExtensionTok{4}
\ExtensionTok{5}
\end{Highlighting}
\end{Shaded}

\subsubsection{Example: Summing Numbers}\label{example-summing-numbers}

\begin{Shaded}
\begin{Highlighting}[]
\PreprocessorTok{\#include }\ImportTok{\textless{}stdio.h\textgreater{}}

\DataTypeTok{int}\NormalTok{ main}\OperatorTok{(}\DataTypeTok{void}\OperatorTok{)} \OperatorTok{\{}
    \DataTypeTok{int}\NormalTok{ n}\OperatorTok{,}\NormalTok{ sum }\OperatorTok{=} \DecValTok{0}\OperatorTok{;}
\NormalTok{    printf}\OperatorTok{(}\StringTok{"Enter positive numbers (0 to stop):}\SpecialCharTok{\textbackslash{}n}\StringTok{"}\OperatorTok{);}
\NormalTok{    scanf}\OperatorTok{(}\StringTok{"}\SpecialCharTok{\%d}\StringTok{"}\OperatorTok{,} \OperatorTok{\&}\NormalTok{n}\OperatorTok{);}

    \ControlFlowTok{while} \OperatorTok{(}\NormalTok{n }\OperatorTok{!=} \DecValTok{0}\OperatorTok{)} \OperatorTok{\{}
\NormalTok{        sum }\OperatorTok{+=}\NormalTok{ n}\OperatorTok{;}
\NormalTok{        scanf}\OperatorTok{(}\StringTok{"}\SpecialCharTok{\%d}\StringTok{"}\OperatorTok{,} \OperatorTok{\&}\NormalTok{n}\OperatorTok{);}
    \OperatorTok{\}}

\NormalTok{    printf}\OperatorTok{(}\StringTok{"Sum = }\SpecialCharTok{\%d\textbackslash{}n}\StringTok{"}\OperatorTok{,}\NormalTok{ sum}\OperatorTok{);}
    \ControlFlowTok{return} \DecValTok{0}\OperatorTok{;}
\OperatorTok{\}}
\end{Highlighting}
\end{Shaded}

Run:

\begin{Shaded}
\begin{Highlighting}[]
\ExtensionTok{Enter}\NormalTok{ positive numbers }\ErrorTok{(}\ExtensionTok{0}\NormalTok{ to stop}\KeywordTok{)}\BuiltInTok{:}
\ExtensionTok{3}
\ExtensionTok{5}
\ExtensionTok{7}
\ExtensionTok{0}
\ExtensionTok{Sum}\NormalTok{ = 15}
\end{Highlighting}
\end{Shaded}

\subsubsection{Infinite Loops}\label{infinite-loops}

If the condition never becomes false, the loop runs forever. Example:

\begin{Shaded}
\begin{Highlighting}[]
\ControlFlowTok{while} \OperatorTok{(}\DecValTok{1}\OperatorTok{)} \OperatorTok{\{}
\NormalTok{    printf}\OperatorTok{(}\StringTok{"Looping forever!}\SpecialCharTok{\textbackslash{}n}\StringTok{"}\OperatorTok{);}
\OperatorTok{\}}
\end{Highlighting}
\end{Shaded}

You should almost always include an update step (like
\texttt{i\ =\ i\ +\ 1;}) so the loop eventually stops.

\subsubsection{Why It Matters}\label{why-it-matters-16}

The \texttt{while} loop introduces repetition - programs can keep
working until a condition changes. This is essential for tasks like
input validation, iterative calculations, and data processing.

\subsubsection{Exercises}\label{exercises-17}

\begin{enumerate}
\def\labelenumi{\arabic{enumi}.}
\tightlist
\item
  Write a program that prints the numbers from 1 to 10 using a
  \texttt{while} loop.
\item
  Write a program that reads integers until the user enters \texttt{0},
  then prints their total.
\item
  Write a program that prints the first 10 even numbers.
\item
  Write a program that asks for a password (string) and keeps asking
  until the correct one is entered.
\item
  Write a program that reads a number \texttt{n} and uses a
  \texttt{while} loop to print its multiplication table (from
  \texttt{n\ -\ 1} to \texttt{n\ -\ 10}).
\end{enumerate}

\subsection{\texorpdfstring{4.4 \texttt{for}
Loops}{4.4 for Loops}}\label{for-loops}

The \texttt{while} loop is flexible, but sometimes you want to count
through a range in a very compact form. For this, C provides the
\texttt{for} loop.

\subsubsection{The Structure}\label{the-structure-2}

\begin{Shaded}
\begin{Highlighting}[]
\ControlFlowTok{for} \OperatorTok{(}\NormalTok{initialization}\OperatorTok{;}\NormalTok{ condition}\OperatorTok{;}\NormalTok{ update}\OperatorTok{)} \OperatorTok{\{}
    \CommentTok{// loop body}
\OperatorTok{\}}
\end{Highlighting}
\end{Shaded}

\begin{itemize}
\tightlist
\item
  initialization → set the starting value
\item
  condition → loop continues while true
\item
  update → run after each iteration (e.g., increment)
\end{itemize}

It's just a compact way of writing the pattern:

\begin{Shaded}
\begin{Highlighting}[]
\NormalTok{initialization}\OperatorTok{;}
\ControlFlowTok{while} \OperatorTok{(}\NormalTok{condition}\OperatorTok{)} \OperatorTok{\{}
    \CommentTok{// body}
\NormalTok{    update}\OperatorTok{;}
\OperatorTok{\}}
\end{Highlighting}
\end{Shaded}

\subsubsection{Example: Counting to 5}\label{example-counting-to-5-1}

\begin{Shaded}
\begin{Highlighting}[]
\PreprocessorTok{\#include }\ImportTok{\textless{}stdio.h\textgreater{}}

\DataTypeTok{int}\NormalTok{ main}\OperatorTok{(}\DataTypeTok{void}\OperatorTok{)} \OperatorTok{\{}
    \ControlFlowTok{for} \OperatorTok{(}\DataTypeTok{int}\NormalTok{ i }\OperatorTok{=} \DecValTok{1}\OperatorTok{;}\NormalTok{ i }\OperatorTok{\textless{}=} \DecValTok{5}\OperatorTok{;}\NormalTok{ i}\OperatorTok{++)} \OperatorTok{\{}
\NormalTok{        printf}\OperatorTok{(}\StringTok{"}\SpecialCharTok{\%d\textbackslash{}n}\StringTok{"}\OperatorTok{,}\NormalTok{ i}\OperatorTok{);}
    \OperatorTok{\}}
    \ControlFlowTok{return} \DecValTok{0}\OperatorTok{;}
\OperatorTok{\}}
\end{Highlighting}
\end{Shaded}

Output:

\begin{verbatim}
1
2
3
4
5
\end{verbatim}

\subsubsection{Example: Sum of 1 to N}\label{example-sum-of-1-to-n}

\begin{Shaded}
\begin{Highlighting}[]
\PreprocessorTok{\#include }\ImportTok{\textless{}stdio.h\textgreater{}}

\DataTypeTok{int}\NormalTok{ main}\OperatorTok{(}\DataTypeTok{void}\OperatorTok{)} \OperatorTok{\{}
    \DataTypeTok{int}\NormalTok{ n}\OperatorTok{,}\NormalTok{ sum }\OperatorTok{=} \DecValTok{0}\OperatorTok{;}
\NormalTok{    printf}\OperatorTok{(}\StringTok{"Enter n: "}\OperatorTok{);}
\NormalTok{    scanf}\OperatorTok{(}\StringTok{"}\SpecialCharTok{\%d}\StringTok{"}\OperatorTok{,} \OperatorTok{\&}\NormalTok{n}\OperatorTok{);}

    \ControlFlowTok{for} \OperatorTok{(}\DataTypeTok{int}\NormalTok{ i }\OperatorTok{=} \DecValTok{1}\OperatorTok{;}\NormalTok{ i }\OperatorTok{\textless{}=}\NormalTok{ n}\OperatorTok{;}\NormalTok{ i}\OperatorTok{++)} \OperatorTok{\{}
\NormalTok{        sum }\OperatorTok{+=}\NormalTok{ i}\OperatorTok{;}
    \OperatorTok{\}}

\NormalTok{    printf}\OperatorTok{(}\StringTok{"Sum = }\SpecialCharTok{\%d\textbackslash{}n}\StringTok{"}\OperatorTok{,}\NormalTok{ sum}\OperatorTok{);}
    \ControlFlowTok{return} \DecValTok{0}\OperatorTok{;}
\OperatorTok{\}}
\end{Highlighting}
\end{Shaded}

Run:

\begin{Shaded}
\begin{Highlighting}[]
\ExtensionTok{Enter}\NormalTok{ n: 5}
\ExtensionTok{Sum}\NormalTok{ = 15}
\end{Highlighting}
\end{Shaded}

\subsubsection{Example: Multiplication
Table}\label{example-multiplication-table}

\begin{Shaded}
\begin{Highlighting}[]
\PreprocessorTok{\#include }\ImportTok{\textless{}stdio.h\textgreater{}}

\DataTypeTok{int}\NormalTok{ main}\OperatorTok{(}\DataTypeTok{void}\OperatorTok{)} \OperatorTok{\{}
    \DataTypeTok{int}\NormalTok{ n}\OperatorTok{;}
\NormalTok{    printf}\OperatorTok{(}\StringTok{"Enter a number: "}\OperatorTok{);}
\NormalTok{    scanf}\OperatorTok{(}\StringTok{"}\SpecialCharTok{\%d}\StringTok{"}\OperatorTok{,} \OperatorTok{\&}\NormalTok{n}\OperatorTok{);}

    \ControlFlowTok{for} \OperatorTok{(}\DataTypeTok{int}\NormalTok{ i }\OperatorTok{=} \DecValTok{1}\OperatorTok{;}\NormalTok{ i }\OperatorTok{\textless{}=} \DecValTok{10}\OperatorTok{;}\NormalTok{ i}\OperatorTok{++)} \OperatorTok{\{}
\NormalTok{        printf}\OperatorTok{(}\StringTok{"}\SpecialCharTok{\%d}\StringTok{ x }\SpecialCharTok{\%d}\StringTok{ = }\SpecialCharTok{\%d\textbackslash{}n}\StringTok{"}\OperatorTok{,}\NormalTok{ n}\OperatorTok{,}\NormalTok{ i}\OperatorTok{,}\NormalTok{ n }\OperatorTok{{-}}\NormalTok{ i}\OperatorTok{);}
    \OperatorTok{\}}
    \ControlFlowTok{return} \DecValTok{0}\OperatorTok{;}
\OperatorTok{\}}
\end{Highlighting}
\end{Shaded}

\subsubsection{Nested Loops}\label{nested-loops}

A loop inside another loop is called nested. Example: printing a
rectangle of \texttt{-} characters:

\begin{Shaded}
\begin{Highlighting}[]
\PreprocessorTok{\#include }\ImportTok{\textless{}stdio.h\textgreater{}}

\DataTypeTok{int}\NormalTok{ main}\OperatorTok{(}\DataTypeTok{void}\OperatorTok{)} \OperatorTok{\{}
    \ControlFlowTok{for} \OperatorTok{(}\DataTypeTok{int}\NormalTok{ row }\OperatorTok{=} \DecValTok{1}\OperatorTok{;}\NormalTok{ row }\OperatorTok{\textless{}=} \DecValTok{3}\OperatorTok{;}\NormalTok{ row}\OperatorTok{++)} \OperatorTok{\{}
        \ControlFlowTok{for} \OperatorTok{(}\DataTypeTok{int}\NormalTok{ col }\OperatorTok{=} \DecValTok{1}\OperatorTok{;}\NormalTok{ col }\OperatorTok{\textless{}=} \DecValTok{5}\OperatorTok{;}\NormalTok{ col}\OperatorTok{++)} \OperatorTok{\{}
\NormalTok{            printf}\OperatorTok{(}\StringTok{"{-}"}\OperatorTok{);}
        \OperatorTok{\}}
\NormalTok{        printf}\OperatorTok{(}\StringTok{"}\SpecialCharTok{\textbackslash{}n}\StringTok{"}\OperatorTok{);}
    \OperatorTok{\}}
    \ControlFlowTok{return} \DecValTok{0}\OperatorTok{;}
\OperatorTok{\}}
\end{Highlighting}
\end{Shaded}

Output:

\begin{verbatim}
-
-
-
\end{verbatim}

\subsubsection{Why It Matters}\label{why-it-matters-17}

\begin{itemize}
\tightlist
\item
  \texttt{for} loops are the standard tool for counting tasks.
\item
  They combine initialization, condition, and update neatly in one line.
\item
  Nested loops let you handle two dimensions (rows and columns).
\end{itemize}

\subsubsection{Exercises}\label{exercises-18}

\begin{enumerate}
\def\labelenumi{\arabic{enumi}.}
\tightlist
\item
  Write a program that prints the numbers from 1 to 20 using a
  \texttt{for} loop.
\item
  Write a program that prints all odd numbers from 1 to 19.
\item
  Write a program that computes the factorial of \texttt{n} (product of
  1 × 2 × \ldots{} × n) using a \texttt{for} loop.
\item
  Write a program that prints a multiplication table from 1 to 10 (all
  rows and columns).
\item
  Write a program that prints a right triangle of \texttt{-} with
  \texttt{n} rows, where \texttt{n} is read from input. Example for
  \texttt{n=4}:
\end{enumerate}

\subsection{4.5 Breaking and Continuing}\label{breaking-and-continuing}

Sometimes you don't want to finish a loop normally:

\begin{itemize}
\tightlist
\item
  You may want to stop early when a condition is met.
\item
  Or you may want to skip one iteration and continue with the next.
\end{itemize}

C gives two keywords for this: \texttt{break} and \texttt{continue}.

\subsubsection{\texorpdfstring{\texttt{break}}{break}}\label{break}

\texttt{break} immediately exits the nearest loop.

Example: stop when \texttt{i} reaches 5:

\begin{Shaded}
\begin{Highlighting}[]
\PreprocessorTok{\#include }\ImportTok{\textless{}stdio.h\textgreater{}}

\DataTypeTok{int}\NormalTok{ main}\OperatorTok{(}\DataTypeTok{void}\OperatorTok{)} \OperatorTok{\{}
    \ControlFlowTok{for} \OperatorTok{(}\DataTypeTok{int}\NormalTok{ i }\OperatorTok{=} \DecValTok{1}\OperatorTok{;}\NormalTok{ i }\OperatorTok{\textless{}=} \DecValTok{10}\OperatorTok{;}\NormalTok{ i}\OperatorTok{++)} \OperatorTok{\{}
        \ControlFlowTok{if} \OperatorTok{(}\NormalTok{i }\OperatorTok{==} \DecValTok{5}\OperatorTok{)} \OperatorTok{\{}
            \ControlFlowTok{break}\OperatorTok{;}
        \OperatorTok{\}}
\NormalTok{        printf}\OperatorTok{(}\StringTok{"}\SpecialCharTok{\%d\textbackslash{}n}\StringTok{"}\OperatorTok{,}\NormalTok{ i}\OperatorTok{);}
    \OperatorTok{\}}
    \ControlFlowTok{return} \DecValTok{0}\OperatorTok{;}
\OperatorTok{\}}
\end{Highlighting}
\end{Shaded}

Output:

\begin{Shaded}
\begin{Highlighting}[]
\ExtensionTok{1}
\ExtensionTok{2}
\ExtensionTok{3}
\ExtensionTok{4}
\end{Highlighting}
\end{Shaded}

The loop ends as soon as \texttt{i\ ==\ 5}.

\subsubsection{\texorpdfstring{\texttt{continue}}{continue}}\label{continue}

\texttt{continue} skips the rest of the current iteration and jumps to
the next one.

Example: skip even numbers:

\begin{Shaded}
\begin{Highlighting}[]
\PreprocessorTok{\#include }\ImportTok{\textless{}stdio.h\textgreater{}}

\DataTypeTok{int}\NormalTok{ main}\OperatorTok{(}\DataTypeTok{void}\OperatorTok{)} \OperatorTok{\{}
    \ControlFlowTok{for} \OperatorTok{(}\DataTypeTok{int}\NormalTok{ i }\OperatorTok{=} \DecValTok{1}\OperatorTok{;}\NormalTok{ i }\OperatorTok{\textless{}=} \DecValTok{10}\OperatorTok{;}\NormalTok{ i}\OperatorTok{++)} \OperatorTok{\{}
        \ControlFlowTok{if} \OperatorTok{(}\NormalTok{i }\OperatorTok{\%} \DecValTok{2} \OperatorTok{==} \DecValTok{0}\OperatorTok{)} \OperatorTok{\{}
            \ControlFlowTok{continue}\OperatorTok{;}  \CommentTok{// skip printing even numbers}
        \OperatorTok{\}}
\NormalTok{        printf}\OperatorTok{(}\StringTok{"}\SpecialCharTok{\%d\textbackslash{}n}\StringTok{"}\OperatorTok{,}\NormalTok{ i}\OperatorTok{);}
    \OperatorTok{\}}
    \ControlFlowTok{return} \DecValTok{0}\OperatorTok{;}
\OperatorTok{\}}
\end{Highlighting}
\end{Shaded}

Output:

\begin{verbatim}
1
3
5
7
9
\end{verbatim}

\subsubsection{\texorpdfstring{Combining with
\texttt{while}}{Combining with while}}\label{combining-with-while}

\texttt{break} and \texttt{continue} also work with \texttt{while}
loops.

Example: reading until user enters \texttt{0}:

\begin{Shaded}
\begin{Highlighting}[]
\PreprocessorTok{\#include }\ImportTok{\textless{}stdio.h\textgreater{}}

\DataTypeTok{int}\NormalTok{ main}\OperatorTok{(}\DataTypeTok{void}\OperatorTok{)} \OperatorTok{\{}
    \DataTypeTok{int}\NormalTok{ n}\OperatorTok{;}
    \ControlFlowTok{while} \OperatorTok{(}\DecValTok{1}\OperatorTok{)} \OperatorTok{\{}         \CommentTok{// infinite loop}
\NormalTok{        scanf}\OperatorTok{(}\StringTok{"}\SpecialCharTok{\%d}\StringTok{"}\OperatorTok{,} \OperatorTok{\&}\NormalTok{n}\OperatorTok{);}
        \ControlFlowTok{if} \OperatorTok{(}\NormalTok{n }\OperatorTok{==} \DecValTok{0}\OperatorTok{)} \OperatorTok{\{}
            \ControlFlowTok{break}\OperatorTok{;}      \CommentTok{// exit when n is 0}
        \OperatorTok{\}}
        \ControlFlowTok{if} \OperatorTok{(}\NormalTok{n }\OperatorTok{\textless{}} \DecValTok{0}\OperatorTok{)} \OperatorTok{\{}
            \ControlFlowTok{continue}\OperatorTok{;}   \CommentTok{// skip negatives}
        \OperatorTok{\}}
\NormalTok{        printf}\OperatorTok{(}\StringTok{"You entered: }\SpecialCharTok{\%d\textbackslash{}n}\StringTok{"}\OperatorTok{,}\NormalTok{ n}\OperatorTok{);}
    \OperatorTok{\}}
    \ControlFlowTok{return} \DecValTok{0}\OperatorTok{;}
\OperatorTok{\}}
\end{Highlighting}
\end{Shaded}

\subsubsection{Why It Matters}\label{why-it-matters-18}

\begin{itemize}
\tightlist
\item
  \texttt{break} lets you exit loops early, useful for searches or
  stopping when a goal is reached.
\item
  \texttt{continue} lets you skip specific cases without leaving the
  loop.
\item
  They give you finer control inside loops, making programs more
  efficient and easier to read.
\end{itemize}

\subsubsection{Exercises}\label{exercises-19}

\begin{enumerate}
\def\labelenumi{\arabic{enumi}.}
\tightlist
\item
  Write a program that prints numbers from 1 to 20 but stops at 13 using
  \texttt{break}.
\item
  Write a program that prints numbers from 1 to 20 but skips multiples
  of 3 using \texttt{continue}.
\item
  Write a program that reads integers until \texttt{0} is entered; skip
  negative numbers, and print only positives.
\item
  Write a program that searches for the first number divisible by 17
  between 1 and 100, then stops.
\item
  Write a program that prints all letters \texttt{A} to \texttt{Z} but
  skips vowels (\texttt{A,\ E,\ I,\ O,\ U}) using \texttt{continue}.
\end{enumerate}

\subsection{Problems}\label{problems-2}

\subsubsection{1. Positive, Negative, or
Zero}\label{positive-negative-or-zero}

Read an integer and print whether it is positive, negative, or zero.

\subsubsection{2. Maximum of Two}\label{maximum-of-two}

Read two integers and print the larger one (or print ``Equal'' if they
are the same).

\subsubsection{3. Grading System}\label{grading-system}

Read a score (0--100) and print the grade:

\begin{itemize}
\tightlist
\item
  \texttt{A} for 90--100
\item
  \texttt{B} for 75--89
\item
  \texttt{C} for 50--74
\item
  \texttt{F} for below 50
\end{itemize}

\subsubsection{4. Temperature Classifier}\label{temperature-classifier}

Read a Celsius temperature and print:

\begin{itemize}
\tightlist
\item
  ``Cold'' if \textless{} 10
\item
  ``Warm'' if 10--25
\item
  ``Hot'' if \textgreater{} 25
\end{itemize}

\subsubsection{5. Calculator with Switch}\label{calculator-with-switch}

Read two integers and an operator (\texttt{+}, \texttt{-}, \texttt{-},
\texttt{/}) and use a \texttt{switch} to perform the calculation.

\subsubsection{6. Digit to Word}\label{digit-to-word}

Read a single digit (0--9) and print its English word (``zero'',
``one'', \ldots). Use a \texttt{switch}.

\subsubsection{7. Vowel or Consonant}\label{vowel-or-consonant}

Read a character and print whether it is a vowel
(\texttt{a,\ e,\ i,\ o,\ u}) or consonant.

\subsubsection{8. Count from 1 to 10}\label{count-from-1-to-10}

Use a \texttt{while} loop to print the numbers 1 through 10.

\subsubsection{9. Sum Until Zero}\label{sum-until-zero}

Keep reading integers until the user enters \texttt{0}. Print their sum.

\subsubsection{10. Multiplication Table}\label{multiplication-table}

Read a number \texttt{n} and use a \texttt{for} loop to print its
multiplication table (from \texttt{n\ ×\ 1} to \texttt{n\ ×\ 10}).

\subsubsection{11. Factorial}\label{factorial}

Read a number \texttt{n} and compute its factorial using a \texttt{for}
loop.

\subsubsection{12. Print Odd Numbers}\label{print-odd-numbers}

Use a \texttt{for} loop with \texttt{continue} to print only the odd
numbers between 1 and 20.

\subsubsection{13. Stop at Thirteen}\label{stop-at-thirteen}

Use a \texttt{for} loop with \texttt{break} to print numbers from 1 to
20 but stop at 13.

\subsubsection{14. Skip Negatives}\label{skip-negatives}

Keep reading integers until \texttt{0} is entered. Skip negative numbers
with \texttt{continue} and print only positives.

\subsubsection{15. First Divisible by 17}\label{first-divisible-by-17}

Use a loop to find the first number between 1 and 100 that is divisible
by 17, then stop with \texttt{break}.

\subsubsection{16. Print a Right Triangle}\label{print-a-right-triangle}

Read \texttt{n} and use nested \texttt{for} loops to print a right
triangle of \texttt{-} with \texttt{n} rows. Example for \texttt{n=4}:

\subsubsection{17. Rectangle of Stars}\label{rectangle-of-stars}

Read \texttt{rows} and \texttt{cols} and print a rectangle of
\texttt{-}.

\subsubsection{18. Guessing Game}\label{guessing-game}

Pick a secret number (hardcode it, e.g., \texttt{42}). Use a
\texttt{while} loop to keep asking the user until they guess it
correctly. Print ``Too low'' or ``Too high'' for wrong guesses.

\subsubsection{19. Countdown}\label{countdown}

Read an integer \texttt{n} and use a \texttt{while} loop to count down
from \texttt{n} to 1, then print ``Blast off!''.

\subsubsection{20. Prime Check}\label{prime-check}

Read an integer \texttt{n} and check if it is prime by testing
divisibility in a loop. Print ``Prime'' or ``Not prime.''

\section{Chapter 5. Functions}\label{chapter-5.-functions}

\subsection{5.1 Why Functions Matter}\label{why-functions-matter}

As programs grow, putting everything into \texttt{main} becomes messy.
You end up with one giant block of code: hard to read, hard to change,
easy to break.

Functions are how we divide programs into small, clear pieces. Each
function has a name and does one job.

\subsubsection{Everyday Analogy}\label{everyday-analogy}

Think of a program like a kitchen. Instead of one person doing
everything, you split tasks:

\begin{itemize}
\tightlist
\item
  one function washes vegetables,
\item
  another cuts them,
\item
  another boils water,
\item
  another prepares sauce.
\end{itemize}

The recipe is easier to follow, and each part can be reused whenever
needed.

\subsubsection{What Functions Give You}\label{what-functions-give-you}

\begin{itemize}
\tightlist
\item
  Clarity → code is broken into named steps
\item
  Reuse → write once, use many times
\item
  Testing → check each part independently
\item
  Flexibility → update one function, all callers benefit
\end{itemize}

\subsubsection{A First Example}\label{a-first-example}

Here's a tiny function that squares an integer:

\begin{Shaded}
\begin{Highlighting}[]
\PreprocessorTok{\#include }\ImportTok{\textless{}stdio.h\textgreater{}}

\DataTypeTok{int}\NormalTok{ square}\OperatorTok{(}\DataTypeTok{int}\NormalTok{ n}\OperatorTok{)} \OperatorTok{\{}
    \ControlFlowTok{return}\NormalTok{ n }\OperatorTok{{-}}\NormalTok{ n}\OperatorTok{;}
\OperatorTok{\}}

\DataTypeTok{int}\NormalTok{ main}\OperatorTok{(}\DataTypeTok{void}\OperatorTok{)} \OperatorTok{\{}
    \DataTypeTok{int}\NormalTok{ x }\OperatorTok{=} \DecValTok{7}\OperatorTok{;}
\NormalTok{    printf}\OperatorTok{(}\StringTok{"square(}\SpecialCharTok{\%d}\StringTok{) = }\SpecialCharTok{\%d\textbackslash{}n}\StringTok{"}\OperatorTok{,}\NormalTok{ x}\OperatorTok{,}\NormalTok{ square}\OperatorTok{(}\NormalTok{x}\OperatorTok{));}
    \ControlFlowTok{return} \DecValTok{0}\OperatorTok{;}
\OperatorTok{\}}
\end{Highlighting}
\end{Shaded}

Notice how \texttt{square} gives a clear name to the operation. If you
see \texttt{square(7)}, you immediately know what it means.

\subsubsection{More Than One Function}\label{more-than-one-function}

You can define several functions in the same program. Each does one job,
and together they make the program easier to follow:

\begin{Shaded}
\begin{Highlighting}[]
\PreprocessorTok{\#include }\ImportTok{\textless{}stdio.h\textgreater{}}

\DataTypeTok{int}\NormalTok{ add}\OperatorTok{(}\DataTypeTok{int}\NormalTok{ a}\OperatorTok{,} \DataTypeTok{int}\NormalTok{ b}\OperatorTok{)} \OperatorTok{\{} \ControlFlowTok{return}\NormalTok{ a }\OperatorTok{+}\NormalTok{ b}\OperatorTok{;} \OperatorTok{\}}
\DataTypeTok{int}\NormalTok{ sub}\OperatorTok{(}\DataTypeTok{int}\NormalTok{ a}\OperatorTok{,} \DataTypeTok{int}\NormalTok{ b}\OperatorTok{)} \OperatorTok{\{} \ControlFlowTok{return}\NormalTok{ a }\OperatorTok{{-}}\NormalTok{ b}\OperatorTok{;} \OperatorTok{\}}
\DataTypeTok{void}\NormalTok{ print\_line}\OperatorTok{(}\DataTypeTok{void}\OperatorTok{)} \OperatorTok{\{}\NormalTok{ printf}\OperatorTok{(}\StringTok{"{-}{-}{-}{-}{-}{-}{-}{-}{-}{-}}\SpecialCharTok{\textbackslash{}n}\StringTok{"}\OperatorTok{);} \OperatorTok{\}}

\DataTypeTok{int}\NormalTok{ main}\OperatorTok{(}\DataTypeTok{void}\OperatorTok{)} \OperatorTok{\{}
    \DataTypeTok{int}\NormalTok{ a }\OperatorTok{=} \DecValTok{12}\OperatorTok{,}\NormalTok{ b }\OperatorTok{=} \DecValTok{5}\OperatorTok{;}
\NormalTok{    print\_line}\OperatorTok{();}
\NormalTok{    printf}\OperatorTok{(}\StringTok{"add = }\SpecialCharTok{\%d\textbackslash{}n}\StringTok{"}\OperatorTok{,}\NormalTok{ add}\OperatorTok{(}\NormalTok{a}\OperatorTok{,}\NormalTok{ b}\OperatorTok{));}
\NormalTok{    printf}\OperatorTok{(}\StringTok{"sub = }\SpecialCharTok{\%d\textbackslash{}n}\StringTok{"}\OperatorTok{,}\NormalTok{ sub}\OperatorTok{(}\NormalTok{a}\OperatorTok{,}\NormalTok{ b}\OperatorTok{));}
\NormalTok{    print\_line}\OperatorTok{();}
    \ControlFlowTok{return} \DecValTok{0}\OperatorTok{;}
\OperatorTok{\}}
\end{Highlighting}
\end{Shaded}

This reads like a story: draw a line, add numbers, subtract numbers,
draw a line.

\subsubsection{Why It Matters}\label{why-it-matters-19}

Functions are the building blocks of bigger programs. They let you:

\begin{itemize}
\tightlist
\item
  tell the computer -what to do- in small, named steps,
\item
  avoid repeating code,
\item
  and make your programs easier for humans to read and understand.
\end{itemize}

The rest of this chapter will show how to define, call, and organize
functions in C.

\subsubsection{Exercises}\label{exercises-20}

\begin{enumerate}
\def\labelenumi{\arabic{enumi}.}
\tightlist
\item
  Write a function \texttt{hello(void)} that prints ``Hello, world!''
  and call it from \texttt{main}.
\item
  Write \texttt{int\ double\_it(int\ n)} that returns twice the value of
  \texttt{n}.
\item
  Write two functions \texttt{line(void)} and \texttt{stars(void)} where
  \texttt{line} prints dashes and \texttt{stars} prints stars. Call them
  from \texttt{main} to decorate output.
\item
  Write a function \texttt{square(int\ n)} and use it to print the
  squares of numbers 1 through 5.
\item
  Write two functions \texttt{add(int\ a,int\ b)} and
  \texttt{mul(int\ a,int\ b)}. Call them with different inputs and print
  the results.
\end{enumerate}

\subsection{5.2 Defining and Calling
Functions}\label{defining-and-calling-functions}

Now that you know why functions matter, let's look at how to write and
use them in C.

\subsubsection{Function Definition}\label{function-definition}

A function has three parts:

\begin{enumerate}
\def\labelenumi{\arabic{enumi}.}
\tightlist
\item
  Return type - the kind of value it gives back (\texttt{int},
  \texttt{double}, \texttt{void}, \ldots)
\item
  Name - what you call it
\item
  Parameters - inputs inside parentheses
\end{enumerate}

\begin{Shaded}
\begin{Highlighting}[]
\NormalTok{return\_type name}\OperatorTok{(}\NormalTok{parameters}\OperatorTok{)} \OperatorTok{\{}
    \CommentTok{// body}
    \ControlFlowTok{return}\NormalTok{ value}\OperatorTok{;}   \CommentTok{// if not void}
\OperatorTok{\}}
\end{Highlighting}
\end{Shaded}

Example:

\begin{Shaded}
\begin{Highlighting}[]
\DataTypeTok{int}\NormalTok{ add}\OperatorTok{(}\DataTypeTok{int}\NormalTok{ a}\OperatorTok{,} \DataTypeTok{int}\NormalTok{ b}\OperatorTok{)} \OperatorTok{\{}
    \ControlFlowTok{return}\NormalTok{ a }\OperatorTok{+}\NormalTok{ b}\OperatorTok{;}
\OperatorTok{\}}
\end{Highlighting}
\end{Shaded}

\subsubsection{Calling a Function}\label{calling-a-function}

You call a function by writing its name followed by arguments:

\begin{Shaded}
\begin{Highlighting}[]
\DataTypeTok{int}\NormalTok{ result }\OperatorTok{=}\NormalTok{ add}\OperatorTok{(}\DecValTok{3}\OperatorTok{,} \DecValTok{4}\OperatorTok{);}
\NormalTok{printf}\OperatorTok{(}\StringTok{"}\SpecialCharTok{\%d\textbackslash{}n}\StringTok{"}\OperatorTok{,}\NormalTok{ result}\OperatorTok{);}
\end{Highlighting}
\end{Shaded}

The values \texttt{3} and \texttt{4} are arguments; inside the function
they are received as parameters (\texttt{a}, \texttt{b}).

\subsubsection{\texorpdfstring{Functions Defined Before
\texttt{main}}{Functions Defined Before main}}\label{functions-defined-before-main}

If a function is defined before \texttt{main}, the compiler already
knows it, so no extra declaration is needed:

\begin{Shaded}
\begin{Highlighting}[]
\PreprocessorTok{\#include }\ImportTok{\textless{}stdio.h\textgreater{}}

\DataTypeTok{int}\NormalTok{ square}\OperatorTok{(}\DataTypeTok{int}\NormalTok{ n}\OperatorTok{)} \OperatorTok{\{}
    \ControlFlowTok{return}\NormalTok{ n }\OperatorTok{{-}}\NormalTok{ n}\OperatorTok{;}
\OperatorTok{\}}

\DataTypeTok{int}\NormalTok{ main}\OperatorTok{(}\DataTypeTok{void}\OperatorTok{)} \OperatorTok{\{}
\NormalTok{    printf}\OperatorTok{(}\StringTok{"}\SpecialCharTok{\%d\textbackslash{}n}\StringTok{"}\OperatorTok{,}\NormalTok{ square}\OperatorTok{(}\DecValTok{7}\OperatorTok{));}
    \ControlFlowTok{return} \DecValTok{0}\OperatorTok{;}
\OperatorTok{\}}
\end{Highlighting}
\end{Shaded}

\subsubsection{\texorpdfstring{Prototypes and Defining After
\texttt{main}}{Prototypes and Defining After main}}\label{prototypes-and-defining-after-main}

If you prefer to put \texttt{main} first and functions later, you must
give the compiler a prototype before \texttt{main}.

A prototype tells the compiler the function's name, return type, and
parameter types.

\begin{Shaded}
\begin{Highlighting}[]
\PreprocessorTok{\#include }\ImportTok{\textless{}stdio.h\textgreater{}}

\DataTypeTok{int}\NormalTok{ square}\OperatorTok{(}\DataTypeTok{int}\NormalTok{ n}\OperatorTok{);}  \CommentTok{// prototype}

\DataTypeTok{int}\NormalTok{ main}\OperatorTok{(}\DataTypeTok{void}\OperatorTok{)} \OperatorTok{\{}
\NormalTok{    printf}\OperatorTok{(}\StringTok{"}\SpecialCharTok{\%d\textbackslash{}n}\StringTok{"}\OperatorTok{,}\NormalTok{ square}\OperatorTok{(}\DecValTok{7}\OperatorTok{));}
    \ControlFlowTok{return} \DecValTok{0}\OperatorTok{;}
\OperatorTok{\}}

\DataTypeTok{int}\NormalTok{ square}\OperatorTok{(}\DataTypeTok{int}\NormalTok{ n}\OperatorTok{)} \OperatorTok{\{}
    \ControlFlowTok{return}\NormalTok{ n }\OperatorTok{{-}}\NormalTok{ n}\OperatorTok{;}
\OperatorTok{\}}
\end{Highlighting}
\end{Shaded}

Without this, modern C (C99 and later) will not compile: every function
must be declared or defined before it is used.

\subsubsection{Matching Prototypes}\label{matching-prototypes}

The prototype must match the definition:

\begin{Shaded}
\begin{Highlighting}[]
\DataTypeTok{int}\NormalTok{ add}\OperatorTok{(}\DataTypeTok{int}\NormalTok{ a}\OperatorTok{,} \DataTypeTok{int}\NormalTok{ b}\OperatorTok{);}          \CommentTok{// OK}
\DataTypeTok{int}\NormalTok{ add}\OperatorTok{(}\DataTypeTok{int}\NormalTok{ a}\OperatorTok{,} \DataTypeTok{int}\NormalTok{ b}\OperatorTok{)} \OperatorTok{\{} \ControlFlowTok{return}\NormalTok{ a }\OperatorTok{+}\NormalTok{ b}\OperatorTok{;} \OperatorTok{\}}
\end{Highlighting}
\end{Shaded}

If the return type or parameters don't match, the compiler warns or
errors.

\subsubsection{Multiple Functions in One
Program}\label{multiple-functions-in-one-program}

You can organize many functions. Either define them all before
\texttt{main}, or put prototypes above \texttt{main} and definitions
after.

\begin{Shaded}
\begin{Highlighting}[]
\PreprocessorTok{\#include }\ImportTok{\textless{}stdio.h\textgreater{}}

\OperatorTok{/{-}}\NormalTok{ Prototypes }\OperatorTok{{-}/}
\DataTypeTok{int}\NormalTok{ add}\OperatorTok{(}\DataTypeTok{int}\NormalTok{ a}\OperatorTok{,} \DataTypeTok{int}\NormalTok{ b}\OperatorTok{);}
\DataTypeTok{int}\NormalTok{ sub}\OperatorTok{(}\DataTypeTok{int}\NormalTok{ a}\OperatorTok{,} \DataTypeTok{int}\NormalTok{ b}\OperatorTok{);}
\DataTypeTok{int}\NormalTok{ mul}\OperatorTok{(}\DataTypeTok{int}\NormalTok{ a}\OperatorTok{,} \DataTypeTok{int}\NormalTok{ b}\OperatorTok{);}

\DataTypeTok{int}\NormalTok{ main}\OperatorTok{(}\DataTypeTok{void}\OperatorTok{)} \OperatorTok{\{}
    \DataTypeTok{int}\NormalTok{ x }\OperatorTok{=} \DecValTok{10}\OperatorTok{,}\NormalTok{ y }\OperatorTok{=} \DecValTok{4}\OperatorTok{;}
\NormalTok{    printf}\OperatorTok{(}\StringTok{"add: }\SpecialCharTok{\%d\textbackslash{}n}\StringTok{"}\OperatorTok{,}\NormalTok{ add}\OperatorTok{(}\NormalTok{x}\OperatorTok{,}\NormalTok{ y}\OperatorTok{));}
\NormalTok{    printf}\OperatorTok{(}\StringTok{"sub: }\SpecialCharTok{\%d\textbackslash{}n}\StringTok{"}\OperatorTok{,}\NormalTok{ sub}\OperatorTok{(}\NormalTok{x}\OperatorTok{,}\NormalTok{ y}\OperatorTok{));}
\NormalTok{    printf}\OperatorTok{(}\StringTok{"mul: }\SpecialCharTok{\%d\textbackslash{}n}\StringTok{"}\OperatorTok{,}\NormalTok{ mul}\OperatorTok{(}\NormalTok{x}\OperatorTok{,}\NormalTok{ y}\OperatorTok{));}
    \ControlFlowTok{return} \DecValTok{0}\OperatorTok{;}
\OperatorTok{\}}

\OperatorTok{/{-}}\NormalTok{ Definitions }\OperatorTok{{-}/}
\DataTypeTok{int}\NormalTok{ add}\OperatorTok{(}\DataTypeTok{int}\NormalTok{ a}\OperatorTok{,} \DataTypeTok{int}\NormalTok{ b}\OperatorTok{)} \OperatorTok{\{} \ControlFlowTok{return}\NormalTok{ a }\OperatorTok{+}\NormalTok{ b}\OperatorTok{;} \OperatorTok{\}}
\DataTypeTok{int}\NormalTok{ sub}\OperatorTok{(}\DataTypeTok{int}\NormalTok{ a}\OperatorTok{,} \DataTypeTok{int}\NormalTok{ b}\OperatorTok{)} \OperatorTok{\{} \ControlFlowTok{return}\NormalTok{ a }\OperatorTok{{-}}\NormalTok{ b}\OperatorTok{;} \OperatorTok{\}}
\DataTypeTok{int}\NormalTok{ mul}\OperatorTok{(}\DataTypeTok{int}\NormalTok{ a}\OperatorTok{,} \DataTypeTok{int}\NormalTok{ b}\OperatorTok{)} \OperatorTok{\{} \ControlFlowTok{return}\NormalTok{ a }\OperatorTok{{-}}\NormalTok{ b}\OperatorTok{;} \OperatorTok{\}}
\end{Highlighting}
\end{Shaded}

\subsubsection{Why It Matters}\label{why-it-matters-20}

\begin{itemize}
\tightlist
\item
  The compiler must know a function before you call it.
\item
  You can achieve this by defining the function first, or by writing a
  prototype.
\item
  This rule becomes more important when we split code into multiple
  files - prototypes live in headers (\texttt{.h}), definitions in
  source files (\texttt{.c}).
\end{itemize}

\subsubsection{Exercises}\label{exercises-21}

\begin{enumerate}
\def\labelenumi{\arabic{enumi}.}
\tightlist
\item
  Write \texttt{int\ triple(int\ n)} that returns 3 × \texttt{n}. Define
  it before \texttt{main} and call it with different numbers.
\item
  Write \texttt{int\ max2(int\ a,\ int\ b)} but put its definition after
  \texttt{main}. Add a prototype before \texttt{main}.
\item
  Write \texttt{double\ area\_rectangle(double\ w,\ double\ h)} and
  \texttt{double\ perimeter\_rectangle(double\ w,\ double\ h)}. Call
  them from \texttt{main}.
\item
  Create a program with functions \texttt{add}, \texttt{sub},
  \texttt{mul}, and \texttt{div\_int} (assume divisor not zero). Place
  prototypes at the top, \texttt{main} next, definitions at the bottom.
\item
  Modify the previous program: move the prototypes into a separate file
  called \texttt{mathlib.h}, include it with
  \texttt{\#include\ "mathlib.h"}, and keep definitions in your
  \texttt{.c} file.
\end{enumerate}

\subsection{5.3 Arguments and Return
Values}\label{arguments-and-return-values}

A function is like a machine:

\begin{itemize}
\tightlist
\item
  You feed it inputs (arguments).
\item
  It processes them.
\item
  It may give you an output (return value).
\end{itemize}

\subsubsection{Parameters and Arguments}\label{parameters-and-arguments}

\begin{itemize}
\tightlist
\item
  Parameters are the variable names inside the function definition.
\item
  Arguments are the actual values you pass in when calling the function.
\end{itemize}

Example:

\begin{Shaded}
\begin{Highlighting}[]
\DataTypeTok{int}\NormalTok{ add}\OperatorTok{(}\DataTypeTok{int}\NormalTok{ a}\OperatorTok{,} \DataTypeTok{int}\NormalTok{ b}\OperatorTok{)} \OperatorTok{\{}   \CommentTok{// a, b are parameters}
    \ControlFlowTok{return}\NormalTok{ a }\OperatorTok{+}\NormalTok{ b}\OperatorTok{;}
\OperatorTok{\}}

\DataTypeTok{int}\NormalTok{ main}\OperatorTok{(}\DataTypeTok{void}\OperatorTok{)} \OperatorTok{\{}
    \DataTypeTok{int}\NormalTok{ x }\OperatorTok{=} \DecValTok{5}\OperatorTok{,}\NormalTok{ y }\OperatorTok{=} \DecValTok{7}\OperatorTok{;}
    \DataTypeTok{int}\NormalTok{ result }\OperatorTok{=}\NormalTok{ add}\OperatorTok{(}\NormalTok{x}\OperatorTok{,}\NormalTok{ y}\OperatorTok{);}  \CommentTok{// x, y are arguments}
\NormalTok{    printf}\OperatorTok{(}\StringTok{"}\SpecialCharTok{\%d\textbackslash{}n}\StringTok{"}\OperatorTok{,}\NormalTok{ result}\OperatorTok{);}
    \ControlFlowTok{return} \DecValTok{0}\OperatorTok{;}
\OperatorTok{\}}
\end{Highlighting}
\end{Shaded}

\subsubsection{Functions with No
Parameters}\label{functions-with-no-parameters}

If a function doesn't need input, you declare it with \texttt{void}:

\begin{Shaded}
\begin{Highlighting}[]
\DataTypeTok{void}\NormalTok{ greet}\OperatorTok{(}\DataTypeTok{void}\OperatorTok{)} \OperatorTok{\{}
\NormalTok{    printf}\OperatorTok{(}\StringTok{"Hello!}\SpecialCharTok{\textbackslash{}n}\StringTok{"}\OperatorTok{);}
\OperatorTok{\}}

\DataTypeTok{int}\NormalTok{ main}\OperatorTok{(}\DataTypeTok{void}\OperatorTok{)} \OperatorTok{\{}
\NormalTok{    greet}\OperatorTok{();}
    \ControlFlowTok{return} \DecValTok{0}\OperatorTok{;}
\OperatorTok{\}}
\end{Highlighting}
\end{Shaded}

\subsubsection{Functions with Multiple
Parameters}\label{functions-with-multiple-parameters}

You can pass as many inputs as you like, separated by commas:

\begin{Shaded}
\begin{Highlighting}[]
\DataTypeTok{int}\NormalTok{ max3}\OperatorTok{(}\DataTypeTok{int}\NormalTok{ a}\OperatorTok{,} \DataTypeTok{int}\NormalTok{ b}\OperatorTok{,} \DataTypeTok{int}\NormalTok{ c}\OperatorTok{)} \OperatorTok{\{}
    \DataTypeTok{int}\NormalTok{ m }\OperatorTok{=}\NormalTok{ a}\OperatorTok{;}
    \ControlFlowTok{if} \OperatorTok{(}\NormalTok{b }\OperatorTok{\textgreater{}}\NormalTok{ m}\OperatorTok{)}\NormalTok{ m }\OperatorTok{=}\NormalTok{ b}\OperatorTok{;}
    \ControlFlowTok{if} \OperatorTok{(}\NormalTok{c }\OperatorTok{\textgreater{}}\NormalTok{ m}\OperatorTok{)}\NormalTok{ m }\OperatorTok{=}\NormalTok{ c}\OperatorTok{;}
    \ControlFlowTok{return}\NormalTok{ m}\OperatorTok{;}
\OperatorTok{\}}
\end{Highlighting}
\end{Shaded}

\subsubsection{Return Values}\label{return-values}

The \texttt{return} statement sends a value back to the caller:

\begin{Shaded}
\begin{Highlighting}[]
\DataTypeTok{int}\NormalTok{ square}\OperatorTok{(}\DataTypeTok{int}\NormalTok{ n}\OperatorTok{)} \OperatorTok{\{}
    \ControlFlowTok{return}\NormalTok{ n }\OperatorTok{{-}}\NormalTok{ n}\OperatorTok{;}
\OperatorTok{\}}
\end{Highlighting}
\end{Shaded}

\begin{itemize}
\tightlist
\item
  The type of the return value must match the function's declared return
  type.
\item
  If a function is declared \texttt{void}, it should not return a value.
\end{itemize}

\subsubsection{Example: Average
Function}\label{example-average-function}

\begin{Shaded}
\begin{Highlighting}[]
\PreprocessorTok{\#include }\ImportTok{\textless{}stdio.h\textgreater{}}

\DataTypeTok{double}\NormalTok{ average3}\OperatorTok{(}\DataTypeTok{int}\NormalTok{ a}\OperatorTok{,} \DataTypeTok{int}\NormalTok{ b}\OperatorTok{,} \DataTypeTok{int}\NormalTok{ c}\OperatorTok{)} \OperatorTok{\{}
    \ControlFlowTok{return} \OperatorTok{(}\NormalTok{a }\OperatorTok{+}\NormalTok{ b }\OperatorTok{+}\NormalTok{ c}\OperatorTok{)} \OperatorTok{/} \FloatTok{3.0}\OperatorTok{;}
\OperatorTok{\}}

\DataTypeTok{int}\NormalTok{ main}\OperatorTok{(}\DataTypeTok{void}\OperatorTok{)} \OperatorTok{\{}
\NormalTok{    printf}\OperatorTok{(}\StringTok{"Average = }\SpecialCharTok{\%.2f\textbackslash{}n}\StringTok{"}\OperatorTok{,}\NormalTok{ average3}\OperatorTok{(}\DecValTok{4}\OperatorTok{,} \DecValTok{7}\OperatorTok{,} \DecValTok{10}\OperatorTok{));}
    \ControlFlowTok{return} \DecValTok{0}\OperatorTok{;}
\OperatorTok{\}}
\end{Highlighting}
\end{Shaded}

Output:

\begin{verbatim}
Average = 7.00
\end{verbatim}

\subsubsection{Ignoring Return Values}\label{ignoring-return-values}

You don't have to store the return value - you can call a function and
ignore it:

\begin{Shaded}
\begin{Highlighting}[]
\NormalTok{printf}\OperatorTok{(}\StringTok{"Result: }\SpecialCharTok{\%d\textbackslash{}n}\StringTok{"}\OperatorTok{,}\NormalTok{ square}\OperatorTok{(}\DecValTok{6}\OperatorTok{));}  \CommentTok{// no variable needed}
\end{Highlighting}
\end{Shaded}

\subsubsection{Why It Matters}\label{why-it-matters-21}

\begin{itemize}
\tightlist
\item
  Arguments let functions take in data.
\item
  Return values let functions produce results.
\item
  Together, they make functions reusable, flexible, and powerful - the
  building blocks of modular programs.
\end{itemize}

\subsubsection{Exercises}\label{exercises-22}

\begin{enumerate}
\def\labelenumi{\arabic{enumi}.}
\tightlist
\item
  Write \texttt{int\ is\_even(int\ n)} that returns \texttt{1} if
  \texttt{n} is even, \texttt{0} otherwise. Test it with numbers 1--10.
\item
  Write \texttt{double\ area\_circle(double\ r)} that returns π·r². Call
  it with several radii.
\item
  Write \texttt{int\ min2(int\ a,\ int\ b)} that returns the smaller of
  two integers.
\item
  Write \texttt{double\ convert\_c\_to\_f(double\ c)} that converts
  Celsius to Fahrenheit: \texttt{F\ =\ C\ -\ 9/5\ +\ 32}.
\item
  Write \texttt{void\ line(int\ n)} that prints \texttt{n} dashes in a
  row. Call it multiple times with different values of \texttt{n}.
\end{enumerate}

\subsection{5.4 Scope of Variables}\label{scope-of-variables}

When you write a program, you can declare variables in different places.
Where a variable is declared determines where it can be used - this is
called its scope.

\subsubsection{Local Variables}\label{local-variables}

A variable declared inside a function exists only in that function.

\begin{Shaded}
\begin{Highlighting}[]
\PreprocessorTok{\#include }\ImportTok{\textless{}stdio.h\textgreater{}}

\DataTypeTok{void}\NormalTok{ demo}\OperatorTok{(}\DataTypeTok{void}\OperatorTok{)} \OperatorTok{\{}
    \DataTypeTok{int}\NormalTok{ x }\OperatorTok{=} \DecValTok{10}\OperatorTok{;}   \CommentTok{// local to demo}
\NormalTok{    printf}\OperatorTok{(}\StringTok{"x in demo = }\SpecialCharTok{\%d\textbackslash{}n}\StringTok{"}\OperatorTok{,}\NormalTok{ x}\OperatorTok{);}
\OperatorTok{\}}

\DataTypeTok{int}\NormalTok{ main}\OperatorTok{(}\DataTypeTok{void}\OperatorTok{)} \OperatorTok{\{}
\NormalTok{    demo}\OperatorTok{();}
    \CommentTok{// printf("\%d", x);  // ❌ error: x not visible here}
    \ControlFlowTok{return} \DecValTok{0}\OperatorTok{;}
\OperatorTok{\}}
\end{Highlighting}
\end{Shaded}

Local variables are created when the function is called, and destroyed
when it ends.

\subsubsection{Function Parameters Are Local
Too}\label{function-parameters-are-local-too}

Parameters behave like local variables.

\begin{Shaded}
\begin{Highlighting}[]
\DataTypeTok{int}\NormalTok{ square}\OperatorTok{(}\DataTypeTok{int}\NormalTok{ n}\OperatorTok{)} \OperatorTok{\{}   \CommentTok{// n is local}
    \ControlFlowTok{return}\NormalTok{ n }\OperatorTok{{-}}\NormalTok{ n}\OperatorTok{;}
\OperatorTok{\}}
\end{Highlighting}
\end{Shaded}

Here \texttt{n} exists only while \texttt{square} runs.

\subsubsection{Global Variables}\label{global-variables}

A variable declared outside all functions is called a global variable.
It can be used by all functions in the file.

\begin{Shaded}
\begin{Highlighting}[]
\PreprocessorTok{\#include }\ImportTok{\textless{}stdio.h\textgreater{}}

\DataTypeTok{int}\NormalTok{ counter }\OperatorTok{=} \DecValTok{0}\OperatorTok{;}   \CommentTok{// global}

\DataTypeTok{void}\NormalTok{ increment}\OperatorTok{(}\DataTypeTok{void}\OperatorTok{)} \OperatorTok{\{}
\NormalTok{    counter}\OperatorTok{++;}
\OperatorTok{\}}

\DataTypeTok{int}\NormalTok{ main}\OperatorTok{(}\DataTypeTok{void}\OperatorTok{)} \OperatorTok{\{}
\NormalTok{    increment}\OperatorTok{();}
\NormalTok{    increment}\OperatorTok{();}
\NormalTok{    printf}\OperatorTok{(}\StringTok{"counter = }\SpecialCharTok{\%d\textbackslash{}n}\StringTok{"}\OperatorTok{,}\NormalTok{ counter}\OperatorTok{);}  \CommentTok{// prints 2}
    \ControlFlowTok{return} \DecValTok{0}\OperatorTok{;}
\OperatorTok{\}}
\end{Highlighting}
\end{Shaded}

Globals are created when the program starts and live until it ends.

\subsubsection{Shadowing}\label{shadowing}

A local variable can have the same name as a global. In that case, the
local one hides the global in its scope.

\begin{Shaded}
\begin{Highlighting}[]
\PreprocessorTok{\#include }\ImportTok{\textless{}stdio.h\textgreater{}}

\DataTypeTok{int}\NormalTok{ value }\OperatorTok{=} \DecValTok{100}\OperatorTok{;}   \CommentTok{// global}

\DataTypeTok{void}\NormalTok{ test}\OperatorTok{(}\DataTypeTok{void}\OperatorTok{)} \OperatorTok{\{}
    \DataTypeTok{int}\NormalTok{ value }\OperatorTok{=} \DecValTok{50}\OperatorTok{;}   \CommentTok{// shadows global}
\NormalTok{    printf}\OperatorTok{(}\StringTok{"local value = }\SpecialCharTok{\%d\textbackslash{}n}\StringTok{"}\OperatorTok{,}\NormalTok{ value}\OperatorTok{);}
\OperatorTok{\}}

\DataTypeTok{int}\NormalTok{ main}\OperatorTok{(}\DataTypeTok{void}\OperatorTok{)} \OperatorTok{\{}
\NormalTok{    test}\OperatorTok{();}
\NormalTok{    printf}\OperatorTok{(}\StringTok{"global value = }\SpecialCharTok{\%d\textbackslash{}n}\StringTok{"}\OperatorTok{,}\NormalTok{ value}\OperatorTok{);}
    \ControlFlowTok{return} \DecValTok{0}\OperatorTok{;}
\OperatorTok{\}}
\end{Highlighting}
\end{Shaded}

Output:

\begin{Shaded}
\begin{Highlighting}[]
\BuiltInTok{local} \VariableTok{value} \OperatorTok{=}\NormalTok{ 50}
\ExtensionTok{global}\NormalTok{ value = 100}
\end{Highlighting}
\end{Shaded}

\subsubsection{Block Scope}\label{block-scope}

Variables declared inside a block \texttt{\{\ ...\ \}} are visible only
inside that block.

\begin{Shaded}
\begin{Highlighting}[]
\PreprocessorTok{\#include }\ImportTok{\textless{}stdio.h\textgreater{}}

\DataTypeTok{int}\NormalTok{ main}\OperatorTok{(}\DataTypeTok{void}\OperatorTok{)} \OperatorTok{\{}
    \DataTypeTok{int}\NormalTok{ x }\OperatorTok{=} \DecValTok{10}\OperatorTok{;}
    \OperatorTok{\{}
        \DataTypeTok{int}\NormalTok{ y }\OperatorTok{=} \DecValTok{20}\OperatorTok{;}   \CommentTok{// visible only here}
\NormalTok{        printf}\OperatorTok{(}\StringTok{"}\SpecialCharTok{\%d}\StringTok{ }\SpecialCharTok{\%d\textbackslash{}n}\StringTok{"}\OperatorTok{,}\NormalTok{ x}\OperatorTok{,}\NormalTok{ y}\OperatorTok{);}
    \OperatorTok{\}}
    \CommentTok{// printf("\%d", y);  // ❌ error: y not visible here}
    \ControlFlowTok{return} \DecValTok{0}\OperatorTok{;}
\OperatorTok{\}}
\end{Highlighting}
\end{Shaded}

\subsubsection{Why It Matters}\label{why-it-matters-22}

\begin{itemize}
\tightlist
\item
  Local variables keep functions independent and safe.
\item
  Globals are shared, but overusing them makes code hard to manage.
\item
  Scope rules prevent name clashes and keep data where it belongs.
\item
  Understanding scope helps avoid bugs where variables ``mysteriously''
  change.
\end{itemize}

\subsubsection{Exercises}\label{exercises-23}

\begin{enumerate}
\def\labelenumi{\arabic{enumi}.}
\tightlist
\item
  Write a program with a global counter and a function \texttt{tick()}
  that increments it. Call \texttt{tick()} five times and print the
  result.
\item
  Write a program where a local variable shadows a global variable with
  the same name. Print both values.
\item
  Write a function \texttt{int\ cube(int\ n)} that uses only local
  variables. Show that the parameter is local.
\item
  Declare a variable inside a block \texttt{\{\ ...\ \}} in
  \texttt{main} and print it. Then try printing it outside the block to
  see the error.
\item
  Write a program with two functions, each with a local variable of the
  same name. Show that they do not interfere with each other.
\end{enumerate}

\subsection{5.5 Writing a Reusable Math
Library}\label{writing-a-reusable-math-library}

By now you know how to define functions, pass arguments, return values,
and handle scope. The next step is to organize functions into a library
that you can reuse across multiple programs.

\subsubsection{Splitting Code into
Files}\label{splitting-code-into-files}

A common C pattern is:

\begin{itemize}
\tightlist
\item
  Header file (\texttt{.h}) → contains function prototypes (the
  ``interface'')
\item
  Source file (\texttt{.c}) → contains function definitions (the
  ``implementation'')
\item
  Main program → uses the library by including the header
\end{itemize}

\subsubsection{\texorpdfstring{Example:
\texttt{mathlib.h}}{Example: mathlib.h}}\label{example-mathlib.h}

This file contains only prototypes:

\begin{Shaded}
\begin{Highlighting}[]
\PreprocessorTok{\#ifndef MATHLIB\_H}
\PreprocessorTok{\#define MATHLIB\_H}

\DataTypeTok{int}\NormalTok{ add}\OperatorTok{(}\DataTypeTok{int}\NormalTok{ a}\OperatorTok{,} \DataTypeTok{int}\NormalTok{ b}\OperatorTok{);}
\DataTypeTok{int}\NormalTok{ sub}\OperatorTok{(}\DataTypeTok{int}\NormalTok{ a}\OperatorTok{,} \DataTypeTok{int}\NormalTok{ b}\OperatorTok{);}
\DataTypeTok{int}\NormalTok{ mul}\OperatorTok{(}\DataTypeTok{int}\NormalTok{ a}\OperatorTok{,} \DataTypeTok{int}\NormalTok{ b}\OperatorTok{);}
\DataTypeTok{int}\NormalTok{ div\_int}\OperatorTok{(}\DataTypeTok{int}\NormalTok{ a}\OperatorTok{,} \DataTypeTok{int}\NormalTok{ b}\OperatorTok{);}
\DataTypeTok{int}\NormalTok{ square}\OperatorTok{(}\DataTypeTok{int}\NormalTok{ n}\OperatorTok{);}

\PreprocessorTok{\#endif}
\end{Highlighting}
\end{Shaded}

The \texttt{\#ifndef\ ...\ \#define\ ...\ \#endif} block is called an
-include guard-. It prevents multiple inclusion errors.

\subsubsection{\texorpdfstring{Example:
\texttt{mathlib.c}}{Example: mathlib.c}}\label{example-mathlib.c}

This file contains the function definitions:

\begin{Shaded}
\begin{Highlighting}[]
\PreprocessorTok{\#include }\ImportTok{"mathlib.h"}

\DataTypeTok{int}\NormalTok{ add}\OperatorTok{(}\DataTypeTok{int}\NormalTok{ a}\OperatorTok{,} \DataTypeTok{int}\NormalTok{ b}\OperatorTok{)} \OperatorTok{\{} \ControlFlowTok{return}\NormalTok{ a }\OperatorTok{+}\NormalTok{ b}\OperatorTok{;} \OperatorTok{\}}
\DataTypeTok{int}\NormalTok{ sub}\OperatorTok{(}\DataTypeTok{int}\NormalTok{ a}\OperatorTok{,} \DataTypeTok{int}\NormalTok{ b}\OperatorTok{)} \OperatorTok{\{} \ControlFlowTok{return}\NormalTok{ a }\OperatorTok{{-}}\NormalTok{ b}\OperatorTok{;} \OperatorTok{\}}
\DataTypeTok{int}\NormalTok{ mul}\OperatorTok{(}\DataTypeTok{int}\NormalTok{ a}\OperatorTok{,} \DataTypeTok{int}\NormalTok{ b}\OperatorTok{)} \OperatorTok{\{} \ControlFlowTok{return}\NormalTok{ a }\OperatorTok{{-}}\NormalTok{ b}\OperatorTok{;} \OperatorTok{\}}
\DataTypeTok{int}\NormalTok{ div\_int}\OperatorTok{(}\DataTypeTok{int}\NormalTok{ a}\OperatorTok{,} \DataTypeTok{int}\NormalTok{ b}\OperatorTok{)} \OperatorTok{\{} \ControlFlowTok{return}\NormalTok{ a }\OperatorTok{/}\NormalTok{ b}\OperatorTok{;} \OperatorTok{\}}   \CommentTok{// assumes b != 0}
\DataTypeTok{int}\NormalTok{ square}\OperatorTok{(}\DataTypeTok{int}\NormalTok{ n}\OperatorTok{)} \OperatorTok{\{} \ControlFlowTok{return}\NormalTok{ n }\OperatorTok{{-}}\NormalTok{ n}\OperatorTok{;} \OperatorTok{\}}
\end{Highlighting}
\end{Shaded}

\subsubsection{\texorpdfstring{Example:
\texttt{main.c}}{Example: main.c}}\label{example-main.c}

The main program includes the header and uses the library:

\begin{Shaded}
\begin{Highlighting}[]
\PreprocessorTok{\#include }\ImportTok{\textless{}stdio.h\textgreater{}}
\PreprocessorTok{\#include }\ImportTok{"mathlib.h"}

\DataTypeTok{int}\NormalTok{ main}\OperatorTok{(}\DataTypeTok{void}\OperatorTok{)} \OperatorTok{\{}
    \DataTypeTok{int}\NormalTok{ a }\OperatorTok{=} \DecValTok{12}\OperatorTok{,}\NormalTok{ b }\OperatorTok{=} \DecValTok{5}\OperatorTok{;}

\NormalTok{    printf}\OperatorTok{(}\StringTok{"a = }\SpecialCharTok{\%d}\StringTok{, b = }\SpecialCharTok{\%d\textbackslash{}n}\StringTok{"}\OperatorTok{,}\NormalTok{ a}\OperatorTok{,}\NormalTok{ b}\OperatorTok{);}
\NormalTok{    printf}\OperatorTok{(}\StringTok{"add = }\SpecialCharTok{\%d\textbackslash{}n}\StringTok{"}\OperatorTok{,}\NormalTok{ add}\OperatorTok{(}\NormalTok{a}\OperatorTok{,}\NormalTok{ b}\OperatorTok{));}
\NormalTok{    printf}\OperatorTok{(}\StringTok{"sub = }\SpecialCharTok{\%d\textbackslash{}n}\StringTok{"}\OperatorTok{,}\NormalTok{ sub}\OperatorTok{(}\NormalTok{a}\OperatorTok{,}\NormalTok{ b}\OperatorTok{));}
\NormalTok{    printf}\OperatorTok{(}\StringTok{"mul = }\SpecialCharTok{\%d\textbackslash{}n}\StringTok{"}\OperatorTok{,}\NormalTok{ mul}\OperatorTok{(}\NormalTok{a}\OperatorTok{,}\NormalTok{ b}\OperatorTok{));}
\NormalTok{    printf}\OperatorTok{(}\StringTok{"div = }\SpecialCharTok{\%d\textbackslash{}n}\StringTok{"}\OperatorTok{,}\NormalTok{ div\_int}\OperatorTok{(}\NormalTok{a}\OperatorTok{,}\NormalTok{ b}\OperatorTok{));}
\NormalTok{    printf}\OperatorTok{(}\StringTok{"square(a) = }\SpecialCharTok{\%d\textbackslash{}n}\StringTok{"}\OperatorTok{,}\NormalTok{ square}\OperatorTok{(}\NormalTok{a}\OperatorTok{));}

    \ControlFlowTok{return} \DecValTok{0}\OperatorTok{;}
\OperatorTok{\}}
\end{Highlighting}
\end{Shaded}

\subsubsection{Compiling Together}\label{compiling-together}

You compile both files and link them:

\begin{Shaded}
\begin{Highlighting}[]
\FunctionTok{gcc}\NormalTok{ main.c mathlib.c }\AttributeTok{{-}o}\NormalTok{ program}
\end{Highlighting}
\end{Shaded}

\subsubsection{Extending the Library}\label{extending-the-library}

You can add more functions later, just by:

\begin{enumerate}
\def\labelenumi{\arabic{enumi}.}
\tightlist
\item
  Adding the prototype to \texttt{mathlib.h}.
\item
  Adding the definition to \texttt{mathlib.c}.
\end{enumerate}

Any program that includes \texttt{mathlib.h} can then use the new
function.

\subsubsection{Why It Matters}\label{why-it-matters-23}

\begin{itemize}
\tightlist
\item
  Organizing functions into headers and source files is the first step
  toward modular programming.
\item
  It separates -what functions do- (interface) from -how they are
  implemented- (details).
\item
  This pattern scales from small projects to huge systems.
\end{itemize}

\subsubsection{Exercises}\label{exercises-24}

\begin{enumerate}
\def\labelenumi{\arabic{enumi}.}
\item
  Write a small library \texttt{shapes.h} and \texttt{shapes.c} with
  functions:

  \begin{itemize}
  \tightlist
  \item
    \texttt{area\_rectangle(w,h)}
  \item
    \texttt{perimeter\_rectangle(w,h)}
  \item
    \texttt{area\_circle(r)} (use \texttt{3.14159}) Use it in
    \texttt{main.c}.
  \end{itemize}
\item
  Add a function \texttt{int\ max2(int\ a,\ int\ b)} to
  \texttt{mathlib}. Update both the header and the source. Test it in
  \texttt{main.c}.
\item
  Add a function \texttt{int\ factorial(int\ n)} to \texttt{mathlib}.
  Demonstrate it in a program that prints factorials of numbers 1--10.
\item
  Move the \texttt{printf} line separator function
  \texttt{print\_line(void)} into its own library \texttt{util.h} /
  \texttt{util.c}. Include it in \texttt{main.c} alongside
  \texttt{mathlib.h}.
\item
  Combine \texttt{mathlib.c} and \texttt{util.c} into one program with
  \texttt{main.c}. Compile with:

\begin{Shaded}
\begin{Highlighting}[]
\FunctionTok{gcc}\NormalTok{ main.c mathlib.c util.c }\AttributeTok{{-}o}\NormalTok{ program}
\end{Highlighting}
\end{Shaded}
\end{enumerate}

\subsection{Problems}\label{problems-3}

\subsubsection{1. Hello Function}\label{hello-function}

Write a function \texttt{void\ hello(void)} that prints
\texttt{"Hello,\ world!"}. Call it from \texttt{main}.

\subsubsection{2. Double Function}\label{double-function}

Write \texttt{int\ double\_it(int\ n)} that returns twice the input.
Test it with several values.

\subsubsection{3. Maximum of Two}\label{maximum-of-two-1}

Write \texttt{int\ max2(int\ a,\ int\ b)} that returns the larger of two
integers. Demonstrate it in \texttt{main}.

\subsubsection{4. Minimum of Three}\label{minimum-of-three}

Write \texttt{int\ min3(int\ a,\ int\ b,\ int\ c)} that returns the
smallest of three integers.

\subsubsection{5. Even Check}\label{even-check}

Write \texttt{int\ is\_even(int\ n)} that returns \texttt{1} if the
number is even, \texttt{0} otherwise. Print results for numbers 1--10.

\subsubsection{6. Average of Three}\label{average-of-three}

Write \texttt{double\ average3(int\ a,\ int\ b,\ int\ c)} that returns
the average. Call it with several sets of numbers.

\subsubsection{7. Square Function}\label{square-function}

Write \texttt{int\ square(int\ n)} and use it to print squares of
numbers from 1 to 10 in a loop.

\subsubsection{8. Circle Area}\label{circle-area}

Write \texttt{double\ area\_circle(double\ r)} that returns π·r² (use
\texttt{3.14159}). Test it with different radii.

\subsubsection{9. Celsius to Fahrenheit}\label{celsius-to-fahrenheit}

Write \texttt{double\ c\_to\_f(double\ c)} that converts Celsius to
Fahrenheit. Demonstrate with 0, 25, 100.

\subsubsection{10. Print Line}\label{print-line}

Write \texttt{void\ line(int\ n)} that prints a row of \texttt{n}
dashes. Call it multiple times with different lengths.

\subsubsection{11. Scope Demonstration}\label{scope-demonstration}

Write a program with a global variable \texttt{counter} and a function
\texttt{tick()} that increments it. Call \texttt{tick()} five times from
\texttt{main} and print the result.

\subsubsection{12. Shadowing}\label{shadowing-1}

Write a program with a global variable \texttt{value\ =\ 100} and a
function that declares a local variable \texttt{value\ =\ 50}. Print
both the local and global values.

\subsubsection{13. Block Scope}\label{block-scope-1}

Write a program that declares a variable \texttt{y} inside a block
\texttt{\{\ ...\ \}}. Print it inside the block, then try printing it
outside (observe the compile error).

\subsubsection{14. Separate Functions}\label{separate-functions}

Write three functions: \texttt{add}, \texttt{sub}, \texttt{mul}. Place
their definitions before \texttt{main}. Call each with two integers.

\subsubsection{15. Prototypes and Later
Definitions}\label{prototypes-and-later-definitions}

Rewrite the previous program so that \texttt{main} comes first. Add
prototypes above \texttt{main} and put definitions after.

\subsubsection{16. Rectangle Functions}\label{rectangle-functions}

Write functions \texttt{double\ area\_rectangle(double\ w,\ double\ h)}
and \texttt{double\ perimeter\_rectangle(double\ w,\ double\ h)}. Test
them in \texttt{main}.

\subsubsection{17. Factorial Function}\label{factorial-function}

Write \texttt{int\ factorial(int\ n)} that returns \texttt{n!}. Use it
to print factorials of numbers 1--10.

\subsubsection{18. Function Reuse}\label{function-reuse}

Write \texttt{int\ sum\_range(int\ a,\ int\ b)} that returns the sum of
all integers between \texttt{a} and \texttt{b} inclusive. Use it in
\texttt{main} to compute the sum of 1--100.

\subsubsection{19. Tiny Math Library}\label{tiny-math-library}

Create two files:

\begin{itemize}
\tightlist
\item
  \texttt{mathlib.h} → prototypes for \texttt{add}, \texttt{sub},
  \texttt{mul}, \texttt{div\_int}, \texttt{square}
\item
  \texttt{mathlib.c} → definitions Write \texttt{main.c} that includes
  \texttt{mathlib.h} and uses the functions.
\end{itemize}

\subsubsection{20. Shapes Library}\label{shapes-library}

Create a header \texttt{shapes.h} and source \texttt{shapes.c} with:

\begin{itemize}
\tightlist
\item
  \texttt{area\_rectangle(w,h)}
\item
  \texttt{perimeter\_rectangle(w,h)}
\item
  \texttt{area\_circle(r)} Write \texttt{main.c} that uses these
  functions to print areas and perimeters for different shapes.
\end{itemize}

\section{Chapter 6. Arrays and
Strings}\label{chapter-6.-arrays-and-strings}

\subsection{6.1 Introduction to Arrays}\label{introduction-to-arrays}

So far, we've stored one value per variable: one \texttt{int}, one
\texttt{double}, one \texttt{char}. But many tasks need a collection of
values.

\begin{itemize}
\tightlist
\item
  A list of exam scores
\item
  The names of players on a team
\item
  The pixels in an image
\end{itemize}

In C, the simplest way to store such a collection is with an array.

\subsubsection{What Is an Array?}\label{what-is-an-array}

An array is a block of memory that holds multiple values of the same
type, arranged one after another.

\begin{itemize}
\tightlist
\item
  Each value is called an element.
\item
  You access elements using an index (position number).
\item
  Indexes start at 0 in C.
\end{itemize}

\subsubsection{Declaring an Array}\label{declaring-an-array}

\begin{Shaded}
\begin{Highlighting}[]
\DataTypeTok{int}\NormalTok{ scores}\OperatorTok{[}\DecValTok{5}\OperatorTok{];}
\end{Highlighting}
\end{Shaded}

This creates space for 5 integers:

\begin{itemize}
\tightlist
\item
  \texttt{scores{[}0{]}}
\item
  \texttt{scores{[}1{]}}
\item
  \texttt{scores{[}2{]}}
\item
  \texttt{scores{[}3{]}}
\item
  \texttt{scores{[}4{]}}
\end{itemize}

\subsubsection{Initializing an Array}\label{initializing-an-array}

You can set initial values at once:

\begin{Shaded}
\begin{Highlighting}[]
\DataTypeTok{int}\NormalTok{ numbers}\OperatorTok{[}\DecValTok{4}\OperatorTok{]} \OperatorTok{=} \OperatorTok{\{}\DecValTok{10}\OperatorTok{,} \DecValTok{20}\OperatorTok{,} \DecValTok{30}\OperatorTok{,} \DecValTok{40}\OperatorTok{\};}
\end{Highlighting}
\end{Shaded}

Or partially:

\begin{Shaded}
\begin{Highlighting}[]
\DataTypeTok{int}\NormalTok{ numbers}\OperatorTok{[}\DecValTok{4}\OperatorTok{]} \OperatorTok{=} \OperatorTok{\{}\DecValTok{10}\OperatorTok{,} \DecValTok{20}\OperatorTok{\};}  \CommentTok{// others become 0}
\end{Highlighting}
\end{Shaded}

You can also let the compiler count:

\begin{Shaded}
\begin{Highlighting}[]
\DataTypeTok{int}\NormalTok{ primes}\OperatorTok{[]} \OperatorTok{=} \OperatorTok{\{}\DecValTok{2}\OperatorTok{,} \DecValTok{3}\OperatorTok{,} \DecValTok{5}\OperatorTok{,} \DecValTok{7}\OperatorTok{,} \DecValTok{11}\OperatorTok{\};}
\end{Highlighting}
\end{Shaded}

\subsubsection{Accessing Elements}\label{accessing-elements}

You use the index in square brackets:

\begin{Shaded}
\begin{Highlighting}[]
\PreprocessorTok{\#include }\ImportTok{\textless{}stdio.h\textgreater{}}

\DataTypeTok{int}\NormalTok{ main}\OperatorTok{(}\DataTypeTok{void}\OperatorTok{)} \OperatorTok{\{}
    \DataTypeTok{int}\NormalTok{ nums}\OperatorTok{[}\DecValTok{3}\OperatorTok{]} \OperatorTok{=} \OperatorTok{\{}\DecValTok{4}\OperatorTok{,} \DecValTok{7}\OperatorTok{,} \DecValTok{9}\OperatorTok{\};}
\NormalTok{    printf}\OperatorTok{(}\StringTok{"First = }\SpecialCharTok{\%d\textbackslash{}n}\StringTok{"}\OperatorTok{,}\NormalTok{ nums}\OperatorTok{[}\DecValTok{0}\OperatorTok{]);}
\NormalTok{    printf}\OperatorTok{(}\StringTok{"Second = }\SpecialCharTok{\%d\textbackslash{}n}\StringTok{"}\OperatorTok{,}\NormalTok{ nums}\OperatorTok{[}\DecValTok{1}\OperatorTok{]);}
\NormalTok{    printf}\OperatorTok{(}\StringTok{"Third = }\SpecialCharTok{\%d\textbackslash{}n}\StringTok{"}\OperatorTok{,}\NormalTok{ nums}\OperatorTok{[}\DecValTok{2}\OperatorTok{]);}
    \ControlFlowTok{return} \DecValTok{0}\OperatorTok{;}
\OperatorTok{\}}
\end{Highlighting}
\end{Shaded}

Output:

\begin{verbatim}
First = 4
Second = 7
Third = 9
\end{verbatim}

\subsubsection{Changing Elements}\label{changing-elements}

\begin{Shaded}
\begin{Highlighting}[]
\NormalTok{nums}\OperatorTok{[}\DecValTok{1}\OperatorTok{]} \OperatorTok{=} \DecValTok{42}\OperatorTok{;}  \CommentTok{// changes the second element}
\end{Highlighting}
\end{Shaded}

Now \texttt{nums} is \texttt{\{4,\ 42,\ 9\}}.

\subsubsection{A Full Example}\label{a-full-example-10}

\begin{Shaded}
\begin{Highlighting}[]
\PreprocessorTok{\#include }\ImportTok{\textless{}stdio.h\textgreater{}}

\DataTypeTok{int}\NormalTok{ main}\OperatorTok{(}\DataTypeTok{void}\OperatorTok{)} \OperatorTok{\{}
    \DataTypeTok{int}\NormalTok{ scores}\OperatorTok{[}\DecValTok{5}\OperatorTok{];}

    \CommentTok{// read 5 scores}
    \ControlFlowTok{for} \OperatorTok{(}\DataTypeTok{int}\NormalTok{ i }\OperatorTok{=} \DecValTok{0}\OperatorTok{;}\NormalTok{ i }\OperatorTok{\textless{}} \DecValTok{5}\OperatorTok{;}\NormalTok{ i}\OperatorTok{++)} \OperatorTok{\{}
\NormalTok{        printf}\OperatorTok{(}\StringTok{"Enter score }\SpecialCharTok{\%d}\StringTok{: "}\OperatorTok{,}\NormalTok{ i }\OperatorTok{+} \DecValTok{1}\OperatorTok{);}
\NormalTok{        scanf}\OperatorTok{(}\StringTok{"}\SpecialCharTok{\%d}\StringTok{"}\OperatorTok{,} \OperatorTok{\&}\NormalTok{scores}\OperatorTok{[}\NormalTok{i}\OperatorTok{]);}
    \OperatorTok{\}}

    \CommentTok{// print them back}
\NormalTok{    printf}\OperatorTok{(}\StringTok{"You entered:}\SpecialCharTok{\textbackslash{}n}\StringTok{"}\OperatorTok{);}
    \ControlFlowTok{for} \OperatorTok{(}\DataTypeTok{int}\NormalTok{ i }\OperatorTok{=} \DecValTok{0}\OperatorTok{;}\NormalTok{ i }\OperatorTok{\textless{}} \DecValTok{5}\OperatorTok{;}\NormalTok{ i}\OperatorTok{++)} \OperatorTok{\{}
\NormalTok{        printf}\OperatorTok{(}\StringTok{"}\SpecialCharTok{\%d}\StringTok{ "}\OperatorTok{,}\NormalTok{ scores}\OperatorTok{[}\NormalTok{i}\OperatorTok{]);}
    \OperatorTok{\}}
\NormalTok{    printf}\OperatorTok{(}\StringTok{"}\SpecialCharTok{\textbackslash{}n}\StringTok{"}\OperatorTok{);}

    \ControlFlowTok{return} \DecValTok{0}\OperatorTok{;}
\OperatorTok{\}}
\end{Highlighting}
\end{Shaded}

\subsubsection{Why It Matters}\label{why-it-matters-24}

Arrays let you store and process many values efficiently without
creating dozens of separate variables. They are the foundation for
working with strings, data tables, and more complex structures.

\subsubsection{Exercises}\label{exercises-25}

\begin{enumerate}
\def\labelenumi{\arabic{enumi}.}
\tightlist
\item
  Declare an array of 10 integers and set each element to its index (0,
  1, 2, \ldots, 9). Print them.
\item
  Read 5 integers into an array, then print their sum.
\item
  Read 5 integers and print the largest.
\item
  Initialize an array of the first 6 even numbers and print them.
\item
  Read 10 integers and print them in reverse order.
\end{enumerate}

\subsection{6.2 Iterating Over Arrays}\label{iterating-over-arrays}

An array is most powerful when combined with loops. Instead of writing
code for each element, you let a loop handle all of them.

\subsubsection{Accessing Elements with a
Loop}\label{accessing-elements-with-a-loop}

\begin{Shaded}
\begin{Highlighting}[]
\PreprocessorTok{\#include }\ImportTok{\textless{}stdio.h\textgreater{}}

\DataTypeTok{int}\NormalTok{ main}\OperatorTok{(}\DataTypeTok{void}\OperatorTok{)} \OperatorTok{\{}
    \DataTypeTok{int}\NormalTok{ nums}\OperatorTok{[}\DecValTok{5}\OperatorTok{]} \OperatorTok{=} \OperatorTok{\{}\DecValTok{10}\OperatorTok{,} \DecValTok{20}\OperatorTok{,} \DecValTok{30}\OperatorTok{,} \DecValTok{40}\OperatorTok{,} \DecValTok{50}\OperatorTok{\};}

    \ControlFlowTok{for} \OperatorTok{(}\DataTypeTok{int}\NormalTok{ i }\OperatorTok{=} \DecValTok{0}\OperatorTok{;}\NormalTok{ i }\OperatorTok{\textless{}} \DecValTok{5}\OperatorTok{;}\NormalTok{ i}\OperatorTok{++)} \OperatorTok{\{}
\NormalTok{        printf}\OperatorTok{(}\StringTok{"nums[}\SpecialCharTok{\%d}\StringTok{] = }\SpecialCharTok{\%d\textbackslash{}n}\StringTok{"}\OperatorTok{,}\NormalTok{ i}\OperatorTok{,}\NormalTok{ nums}\OperatorTok{[}\NormalTok{i}\OperatorTok{]);}
    \OperatorTok{\}}

    \ControlFlowTok{return} \DecValTok{0}\OperatorTok{;}
\OperatorTok{\}}
\end{Highlighting}
\end{Shaded}

Output:

\begin{verbatim}
nums[0] = 10
nums[1] = 20
nums[2] = 30
nums[3] = 40
nums[4] = 50
\end{verbatim}

\subsubsection{Reading Into an Array}\label{reading-into-an-array}

\begin{Shaded}
\begin{Highlighting}[]
\PreprocessorTok{\#include }\ImportTok{\textless{}stdio.h\textgreater{}}

\DataTypeTok{int}\NormalTok{ main}\OperatorTok{(}\DataTypeTok{void}\OperatorTok{)} \OperatorTok{\{}
    \DataTypeTok{int}\NormalTok{ arr}\OperatorTok{[}\DecValTok{5}\OperatorTok{];}

    \ControlFlowTok{for} \OperatorTok{(}\DataTypeTok{int}\NormalTok{ i }\OperatorTok{=} \DecValTok{0}\OperatorTok{;}\NormalTok{ i }\OperatorTok{\textless{}} \DecValTok{5}\OperatorTok{;}\NormalTok{ i}\OperatorTok{++)} \OperatorTok{\{}
\NormalTok{        scanf}\OperatorTok{(}\StringTok{"}\SpecialCharTok{\%d}\StringTok{"}\OperatorTok{,} \OperatorTok{\&}\NormalTok{arr}\OperatorTok{[}\NormalTok{i}\OperatorTok{]);}  \CommentTok{// note the \&}
    \OperatorTok{\}}

    \ControlFlowTok{for} \OperatorTok{(}\DataTypeTok{int}\NormalTok{ i }\OperatorTok{=} \DecValTok{0}\OperatorTok{;}\NormalTok{ i }\OperatorTok{\textless{}} \DecValTok{5}\OperatorTok{;}\NormalTok{ i}\OperatorTok{++)} \OperatorTok{\{}
\NormalTok{        printf}\OperatorTok{(}\StringTok{"}\SpecialCharTok{\%d}\StringTok{ "}\OperatorTok{,}\NormalTok{ arr}\OperatorTok{[}\NormalTok{i}\OperatorTok{]);}
    \OperatorTok{\}}
\NormalTok{    printf}\OperatorTok{(}\StringTok{"}\SpecialCharTok{\textbackslash{}n}\StringTok{"}\OperatorTok{);}

    \ControlFlowTok{return} \DecValTok{0}\OperatorTok{;}
\OperatorTok{\}}
\end{Highlighting}
\end{Shaded}

\subsubsection{Calculating a Sum}\label{calculating-a-sum}

\begin{Shaded}
\begin{Highlighting}[]
\PreprocessorTok{\#include }\ImportTok{\textless{}stdio.h\textgreater{}}

\DataTypeTok{int}\NormalTok{ main}\OperatorTok{(}\DataTypeTok{void}\OperatorTok{)} \OperatorTok{\{}
    \DataTypeTok{int}\NormalTok{ scores}\OperatorTok{[}\DecValTok{5}\OperatorTok{]} \OperatorTok{=} \OperatorTok{\{}\DecValTok{80}\OperatorTok{,} \DecValTok{90}\OperatorTok{,} \DecValTok{70}\OperatorTok{,} \DecValTok{60}\OperatorTok{,} \DecValTok{85}\OperatorTok{\};}
    \DataTypeTok{int}\NormalTok{ sum }\OperatorTok{=} \DecValTok{0}\OperatorTok{;}

    \ControlFlowTok{for} \OperatorTok{(}\DataTypeTok{int}\NormalTok{ i }\OperatorTok{=} \DecValTok{0}\OperatorTok{;}\NormalTok{ i }\OperatorTok{\textless{}} \DecValTok{5}\OperatorTok{;}\NormalTok{ i}\OperatorTok{++)} \OperatorTok{\{}
\NormalTok{        sum }\OperatorTok{+=}\NormalTok{ scores}\OperatorTok{[}\NormalTok{i}\OperatorTok{];}
    \OperatorTok{\}}

\NormalTok{    printf}\OperatorTok{(}\StringTok{"Total = }\SpecialCharTok{\%d\textbackslash{}n}\StringTok{"}\OperatorTok{,}\NormalTok{ sum}\OperatorTok{);}
    \ControlFlowTok{return} \DecValTok{0}\OperatorTok{;}
\OperatorTok{\}}
\end{Highlighting}
\end{Shaded}

\subsubsection{Finding a Maximum}\label{finding-a-maximum}

\begin{Shaded}
\begin{Highlighting}[]
\PreprocessorTok{\#include }\ImportTok{\textless{}stdio.h\textgreater{}}

\DataTypeTok{int}\NormalTok{ main}\OperatorTok{(}\DataTypeTok{void}\OperatorTok{)} \OperatorTok{\{}
    \DataTypeTok{int}\NormalTok{ scores}\OperatorTok{[}\DecValTok{5}\OperatorTok{]} \OperatorTok{=} \OperatorTok{\{}\DecValTok{12}\OperatorTok{,} \DecValTok{45}\OperatorTok{,} \DecValTok{67}\OperatorTok{,} \DecValTok{23}\OperatorTok{,} \DecValTok{89}\OperatorTok{\};}
    \DataTypeTok{int}\NormalTok{ max }\OperatorTok{=}\NormalTok{ scores}\OperatorTok{[}\DecValTok{0}\OperatorTok{];}

    \ControlFlowTok{for} \OperatorTok{(}\DataTypeTok{int}\NormalTok{ i }\OperatorTok{=} \DecValTok{1}\OperatorTok{;}\NormalTok{ i }\OperatorTok{\textless{}} \DecValTok{5}\OperatorTok{;}\NormalTok{ i}\OperatorTok{++)} \OperatorTok{\{}
        \ControlFlowTok{if} \OperatorTok{(}\NormalTok{scores}\OperatorTok{[}\NormalTok{i}\OperatorTok{]} \OperatorTok{\textgreater{}}\NormalTok{ max}\OperatorTok{)} \OperatorTok{\{}
\NormalTok{            max }\OperatorTok{=}\NormalTok{ scores}\OperatorTok{[}\NormalTok{i}\OperatorTok{];}
        \OperatorTok{\}}
    \OperatorTok{\}}

\NormalTok{    printf}\OperatorTok{(}\StringTok{"Max = }\SpecialCharTok{\%d\textbackslash{}n}\StringTok{"}\OperatorTok{,}\NormalTok{ max}\OperatorTok{);}
    \ControlFlowTok{return} \DecValTok{0}\OperatorTok{;}
\OperatorTok{\}}
\end{Highlighting}
\end{Shaded}

\subsubsection{\texorpdfstring{Using \texttt{sizeof} to Get Array
Length}{Using sizeof to Get Array Length}}\label{using-sizeof-to-get-array-length}

Hardcoding the array size (\texttt{5} above) works, but we can calculate
it:

\begin{Shaded}
\begin{Highlighting}[]
\DataTypeTok{int}\NormalTok{ arr}\OperatorTok{[]} \OperatorTok{=} \OperatorTok{\{}\DecValTok{2}\OperatorTok{,} \DecValTok{4}\OperatorTok{,} \DecValTok{6}\OperatorTok{,} \DecValTok{8}\OperatorTok{,} \DecValTok{10}\OperatorTok{\};}
\DataTypeTok{int}\NormalTok{ len }\OperatorTok{=} \KeywordTok{sizeof}\OperatorTok{(}\NormalTok{arr}\OperatorTok{)} \OperatorTok{/} \KeywordTok{sizeof}\OperatorTok{(}\NormalTok{arr}\OperatorTok{[}\DecValTok{0}\OperatorTok{]);}
\end{Highlighting}
\end{Shaded}

This way, if the array changes size, the loop still works correctly.

\subsubsection{Why It Matters}\label{why-it-matters-25}

\begin{itemize}
\tightlist
\item
  Loops and arrays go hand in hand.
\item
  You can process data of any length with the same logic.
\item
  This is the foundation for algorithms like searching, sorting, and
  aggregation.
\end{itemize}

\subsubsection{Exercises}\label{exercises-26}

\begin{enumerate}
\def\labelenumi{\arabic{enumi}.}
\tightlist
\item
  Read 10 integers into an array and print their average.
\item
  Find the minimum value in an array of 8 integers.
\item
  Count how many numbers in an array of 10 are even.
\item
  Read 5 integers and print them in reverse order using a loop.
\item
  Given \texttt{int\ arr{[}{]}\ =\ \{1,2,3,4,5\}}, write a program to
  compute the sum of squares of all elements.
\end{enumerate}

\subsection{6.3 Strings as Character
Arrays}\label{strings-as-character-arrays}

In C, a string is not a special type. It's just an array of characters
ending with a special character:

\begin{itemize}
\tightlist
\item
  \texttt{\textquotesingle{}\textbackslash{}0\textquotesingle{}} (null
  terminator)
\end{itemize}

This null character marks the end of the string, so functions know where
to stop.

\subsubsection{Declaring Strings}\label{declaring-strings}

\begin{Shaded}
\begin{Highlighting}[]
\DataTypeTok{char}\NormalTok{ word}\OperatorTok{[}\DecValTok{6}\OperatorTok{]} \OperatorTok{=} \OperatorTok{\{}\CharTok{\textquotesingle{}H\textquotesingle{}}\OperatorTok{,} \CharTok{\textquotesingle{}e\textquotesingle{}}\OperatorTok{,} \CharTok{\textquotesingle{}l\textquotesingle{}}\OperatorTok{,} \CharTok{\textquotesingle{}l\textquotesingle{}}\OperatorTok{,} \CharTok{\textquotesingle{}o\textquotesingle{}}\OperatorTok{,} \CharTok{\textquotesingle{}}\SpecialCharTok{\textbackslash{}0}\CharTok{\textquotesingle{}}\OperatorTok{\};}
\end{Highlighting}
\end{Shaded}

Easier with double quotes:

\begin{Shaded}
\begin{Highlighting}[]
\DataTypeTok{char}\NormalTok{ word}\OperatorTok{[]} \OperatorTok{=} \StringTok{"Hello"}\OperatorTok{;}
\end{Highlighting}
\end{Shaded}

The compiler adds the
\texttt{\textquotesingle{}\textbackslash{}0\textquotesingle{}}
automatically.

\subsubsection{Printing Strings}\label{printing-strings}

Use \texttt{\%s} in \texttt{printf}:

\begin{Shaded}
\begin{Highlighting}[]
\PreprocessorTok{\#include }\ImportTok{\textless{}stdio.h\textgreater{}}

\DataTypeTok{int}\NormalTok{ main}\OperatorTok{(}\DataTypeTok{void}\OperatorTok{)} \OperatorTok{\{}
    \DataTypeTok{char}\NormalTok{ msg}\OperatorTok{[]} \OperatorTok{=} \StringTok{"C is fun!"}\OperatorTok{;}
\NormalTok{    printf}\OperatorTok{(}\StringTok{"}\SpecialCharTok{\%s\textbackslash{}n}\StringTok{"}\OperatorTok{,}\NormalTok{ msg}\OperatorTok{);}
    \ControlFlowTok{return} \DecValTok{0}\OperatorTok{;}
\OperatorTok{\}}
\end{Highlighting}
\end{Shaded}

Output:

\begin{Shaded}
\begin{Highlighting}[]
\ExtensionTok{C}\NormalTok{ is fun!}
\end{Highlighting}
\end{Shaded}

\subsubsection{Reading Strings}\label{reading-strings}

\begin{Shaded}
\begin{Highlighting}[]
\PreprocessorTok{\#include }\ImportTok{\textless{}stdio.h\textgreater{}}

\DataTypeTok{int}\NormalTok{ main}\OperatorTok{(}\DataTypeTok{void}\OperatorTok{)} \OperatorTok{\{}
    \DataTypeTok{char}\NormalTok{ name}\OperatorTok{[}\DecValTok{20}\OperatorTok{];}
\NormalTok{    printf}\OperatorTok{(}\StringTok{"Enter your name: "}\OperatorTok{);}
\NormalTok{    scanf}\OperatorTok{(}\StringTok{"}\SpecialCharTok{\%19s}\StringTok{"}\OperatorTok{,}\NormalTok{ name}\OperatorTok{);}   \CommentTok{// limit to 19 chars + \textquotesingle{}\textbackslash{}0\textquotesingle{}}
\NormalTok{    printf}\OperatorTok{(}\StringTok{"Hello, }\SpecialCharTok{\%s}\StringTok{!}\SpecialCharTok{\textbackslash{}n}\StringTok{"}\OperatorTok{,}\NormalTok{ name}\OperatorTok{);}
    \ControlFlowTok{return} \DecValTok{0}\OperatorTok{;}
\OperatorTok{\}}
\end{Highlighting}
\end{Shaded}

⚠️ \texttt{scanf("\%s",\ ...)} stops at the first space. For reading
full lines, safer functions like \texttt{fgets} are better (later
section).

\subsubsection{Character by Character}\label{character-by-character}

Strings are just arrays:

\begin{Shaded}
\begin{Highlighting}[]
\PreprocessorTok{\#include }\ImportTok{\textless{}stdio.h\textgreater{}}

\DataTypeTok{int}\NormalTok{ main}\OperatorTok{(}\DataTypeTok{void}\OperatorTok{)} \OperatorTok{\{}
    \DataTypeTok{char}\NormalTok{ word}\OperatorTok{[]} \OperatorTok{=} \StringTok{"Hi"}\OperatorTok{;}
\NormalTok{    printf}\OperatorTok{(}\StringTok{"}\SpecialCharTok{\%c}\StringTok{ }\SpecialCharTok{\%c}\StringTok{ }\SpecialCharTok{\%c\textbackslash{}n}\StringTok{"}\OperatorTok{,}\NormalTok{ word}\OperatorTok{[}\DecValTok{0}\OperatorTok{],}\NormalTok{ word}\OperatorTok{[}\DecValTok{1}\OperatorTok{],}\NormalTok{ word}\OperatorTok{[}\DecValTok{2}\OperatorTok{]);}
    \ControlFlowTok{return} \DecValTok{0}\OperatorTok{;}
\OperatorTok{\}}
\end{Highlighting}
\end{Shaded}

Output:

\begin{Shaded}
\begin{Highlighting}[]
\ExtensionTok{H}\NormalTok{ i }\DataTypeTok{\textbackslash{}0}
\end{Highlighting}
\end{Shaded}

Notice the third element is
\texttt{\textquotesingle{}\textbackslash{}0\textquotesingle{}}.

\subsubsection{Changing Characters}\label{changing-characters}

You can change individual letters:

\begin{Shaded}
\begin{Highlighting}[]
\DataTypeTok{char}\NormalTok{ greet}\OperatorTok{[]} \OperatorTok{=} \StringTok{"Cat"}\OperatorTok{;}
\NormalTok{greet}\OperatorTok{[}\DecValTok{0}\OperatorTok{]} \OperatorTok{=} \CharTok{\textquotesingle{}H\textquotesingle{}}\OperatorTok{;}  \CommentTok{// now "Hat"}
\end{Highlighting}
\end{Shaded}

\subsubsection{A Full Example}\label{a-full-example-11}

\begin{Shaded}
\begin{Highlighting}[]
\PreprocessorTok{\#include }\ImportTok{\textless{}stdio.h\textgreater{}}

\DataTypeTok{int}\NormalTok{ main}\OperatorTok{(}\DataTypeTok{void}\OperatorTok{)} \OperatorTok{\{}
    \DataTypeTok{char}\NormalTok{ city}\OperatorTok{[}\DecValTok{20}\OperatorTok{];}
\NormalTok{    printf}\OperatorTok{(}\StringTok{"Enter a city: "}\OperatorTok{);}
\NormalTok{    scanf}\OperatorTok{(}\StringTok{"}\SpecialCharTok{\%19s}\StringTok{"}\OperatorTok{,}\NormalTok{ city}\OperatorTok{);}

\NormalTok{    printf}\OperatorTok{(}\StringTok{"First letter: }\SpecialCharTok{\%c\textbackslash{}n}\StringTok{"}\OperatorTok{,}\NormalTok{ city}\OperatorTok{[}\DecValTok{0}\OperatorTok{]);}

    \DataTypeTok{int}\NormalTok{ i }\OperatorTok{=} \DecValTok{0}\OperatorTok{;}
    \ControlFlowTok{while} \OperatorTok{(}\NormalTok{city}\OperatorTok{[}\NormalTok{i}\OperatorTok{]} \OperatorTok{!=} \CharTok{\textquotesingle{}}\SpecialCharTok{\textbackslash{}0}\CharTok{\textquotesingle{}}\OperatorTok{)} \OperatorTok{\{}
\NormalTok{        i}\OperatorTok{++;}
    \OperatorTok{\}}
\NormalTok{    printf}\OperatorTok{(}\StringTok{"Length = }\SpecialCharTok{\%d\textbackslash{}n}\StringTok{"}\OperatorTok{,}\NormalTok{ i}\OperatorTok{);}

    \ControlFlowTok{return} \DecValTok{0}\OperatorTok{;}
\OperatorTok{\}}
\end{Highlighting}
\end{Shaded}

\subsubsection{Why It Matters}\label{why-it-matters-26}

\begin{itemize}
\tightlist
\item
  Strings are essential for text processing.
\item
  Understanding that they are arrays helps explain how input, output,
  and libraries like \texttt{\textless{}string.h\textgreater{}} work.
\item
  Remember: always leave room for the null terminator!
\end{itemize}

\subsubsection{Exercises}\label{exercises-27}

\begin{enumerate}
\def\labelenumi{\arabic{enumi}.}
\tightlist
\item
  Declare a string \texttt{"Hello"} and print its characters one by one
  in a loop.
\item
  Read a name into a char array and print a greeting.
\item
  Write a program that counts the number of characters in a string
  (without using \texttt{strlen}).
\item
  Modify a string \texttt{"dog"} into \texttt{"fog"} by changing its
  first character.
\item
  Read two strings and print them in reverse order (first the second,
  then the first).
\end{enumerate}

\subsection{6.4 Standard String
Functions}\label{standard-string-functions}

C provides many useful functions for handling strings in the header
\texttt{\textless{}string.h\textgreater{}}. These functions work with
arrays of characters that end with
\texttt{\textquotesingle{}\textbackslash{}0\textquotesingle{}}.

\subsubsection{\texorpdfstring{\texttt{strlen} - String
Length}{strlen - String Length}}\label{strlen---string-length}

Counts the number of characters before the null terminator.

\begin{Shaded}
\begin{Highlighting}[]
\PreprocessorTok{\#include }\ImportTok{\textless{}stdio.h\textgreater{}}
\PreprocessorTok{\#include }\ImportTok{\textless{}string.h\textgreater{}}

\DataTypeTok{int}\NormalTok{ main}\OperatorTok{(}\DataTypeTok{void}\OperatorTok{)} \OperatorTok{\{}
    \DataTypeTok{char}\NormalTok{ word}\OperatorTok{[]} \OperatorTok{=} \StringTok{"Hello"}\OperatorTok{;}
\NormalTok{    printf}\OperatorTok{(}\StringTok{"Length = }\SpecialCharTok{\%zu\textbackslash{}n}\StringTok{"}\OperatorTok{,}\NormalTok{ strlen}\OperatorTok{(}\NormalTok{word}\OperatorTok{));}
    \ControlFlowTok{return} \DecValTok{0}\OperatorTok{;}
\OperatorTok{\}}
\end{Highlighting}
\end{Shaded}

Output:

\begin{verbatim}
Length = 5
\end{verbatim}

\subsubsection{\texorpdfstring{\texttt{strcpy} - Copy a
String}{strcpy - Copy a String}}\label{strcpy---copy-a-string}

Copies characters from one string into another.

\begin{Shaded}
\begin{Highlighting}[]
\PreprocessorTok{\#include }\ImportTok{\textless{}stdio.h\textgreater{}}
\PreprocessorTok{\#include }\ImportTok{\textless{}string.h\textgreater{}}

\DataTypeTok{int}\NormalTok{ main}\OperatorTok{(}\DataTypeTok{void}\OperatorTok{)} \OperatorTok{\{}
    \DataTypeTok{char}\NormalTok{ src}\OperatorTok{[]} \OperatorTok{=} \StringTok{"world"}\OperatorTok{;}
    \DataTypeTok{char}\NormalTok{ dst}\OperatorTok{[}\DecValTok{20}\OperatorTok{];}
\NormalTok{    strcpy}\OperatorTok{(}\NormalTok{dst}\OperatorTok{,}\NormalTok{ src}\OperatorTok{);}
\NormalTok{    printf}\OperatorTok{(}\StringTok{"}\SpecialCharTok{\%s\textbackslash{}n}\StringTok{"}\OperatorTok{,}\NormalTok{ dst}\OperatorTok{);}
    \ControlFlowTok{return} \DecValTok{0}\OperatorTok{;}
\OperatorTok{\}}
\end{Highlighting}
\end{Shaded}

⚠️ Make sure the destination is large enough.

\subsubsection{\texorpdfstring{\texttt{strcat} - Concatenate
Strings}{strcat - Concatenate Strings}}\label{strcat---concatenate-strings}

Appends one string to another.

\begin{Shaded}
\begin{Highlighting}[]
\PreprocessorTok{\#include }\ImportTok{\textless{}stdio.h\textgreater{}}
\PreprocessorTok{\#include }\ImportTok{\textless{}string.h\textgreater{}}

\DataTypeTok{int}\NormalTok{ main}\OperatorTok{(}\DataTypeTok{void}\OperatorTok{)} \OperatorTok{\{}
    \DataTypeTok{char}\NormalTok{ a}\OperatorTok{[}\DecValTok{20}\OperatorTok{]} \OperatorTok{=} \StringTok{"Good"}\OperatorTok{;}
    \DataTypeTok{char}\NormalTok{ b}\OperatorTok{[]} \OperatorTok{=} \StringTok{" morning"}\OperatorTok{;}
\NormalTok{    strcat}\OperatorTok{(}\NormalTok{a}\OperatorTok{,}\NormalTok{ b}\OperatorTok{);}
\NormalTok{    printf}\OperatorTok{(}\StringTok{"}\SpecialCharTok{\%s\textbackslash{}n}\StringTok{"}\OperatorTok{,}\NormalTok{ a}\OperatorTok{);}
    \ControlFlowTok{return} \DecValTok{0}\OperatorTok{;}
\OperatorTok{\}}
\end{Highlighting}
\end{Shaded}

Output:

\begin{verbatim}
Good morning
\end{verbatim}

\subsubsection{\texorpdfstring{\texttt{strcmp} - Compare
Strings}{strcmp - Compare Strings}}\label{strcmp---compare-strings}

Compares two strings lexicographically (like dictionary order).

\begin{itemize}
\tightlist
\item
  Returns \texttt{0} if equal
\item
  Negative if first \textless{} second
\item
  Positive if first \textgreater{} second
\end{itemize}

\begin{Shaded}
\begin{Highlighting}[]
\PreprocessorTok{\#include }\ImportTok{\textless{}stdio.h\textgreater{}}
\PreprocessorTok{\#include }\ImportTok{\textless{}string.h\textgreater{}}

\DataTypeTok{int}\NormalTok{ main}\OperatorTok{(}\DataTypeTok{void}\OperatorTok{)} \OperatorTok{\{}
    \DataTypeTok{char}\NormalTok{ s1}\OperatorTok{[]} \OperatorTok{=} \StringTok{"apple"}\OperatorTok{;}
    \DataTypeTok{char}\NormalTok{ s2}\OperatorTok{[]} \OperatorTok{=} \StringTok{"banana"}\OperatorTok{;}

    \ControlFlowTok{if} \OperatorTok{(}\NormalTok{strcmp}\OperatorTok{(}\NormalTok{s1}\OperatorTok{,}\NormalTok{ s2}\OperatorTok{)} \OperatorTok{\textless{}} \DecValTok{0}\OperatorTok{)} \OperatorTok{\{}
\NormalTok{        printf}\OperatorTok{(}\StringTok{"}\SpecialCharTok{\%s}\StringTok{ comes before }\SpecialCharTok{\%s\textbackslash{}n}\StringTok{"}\OperatorTok{,}\NormalTok{ s1}\OperatorTok{,}\NormalTok{ s2}\OperatorTok{);}
    \OperatorTok{\}}
    \ControlFlowTok{return} \DecValTok{0}\OperatorTok{;}
\OperatorTok{\}}
\end{Highlighting}
\end{Shaded}

\subsubsection{Safer Variants}\label{safer-variants}

Many libraries provide safer versions:

\begin{itemize}
\tightlist
\item
  \texttt{strncpy} (copy with size limit)
\item
  \texttt{strncat} (concatenate with size limit)
\item
  \texttt{strncmp} (compare up to n chars)
\end{itemize}

Example:

\begin{Shaded}
\begin{Highlighting}[]
\DataTypeTok{char}\NormalTok{ src}\OperatorTok{[]} \OperatorTok{=} \StringTok{"hello"}\OperatorTok{;}
\DataTypeTok{char}\NormalTok{ dst}\OperatorTok{[}\DecValTok{10}\OperatorTok{];}
\NormalTok{strncpy}\OperatorTok{(}\NormalTok{dst}\OperatorTok{,}\NormalTok{ src}\OperatorTok{,} \KeywordTok{sizeof}\OperatorTok{(}\NormalTok{dst}\OperatorTok{){-}}\DecValTok{1}\OperatorTok{);}
\NormalTok{dst}\OperatorTok{[}\KeywordTok{sizeof}\OperatorTok{(}\NormalTok{dst}\OperatorTok{){-}}\DecValTok{1}\OperatorTok{]} \OperatorTok{=} \CharTok{\textquotesingle{}}\SpecialCharTok{\textbackslash{}0}\CharTok{\textquotesingle{}}\OperatorTok{;}   \CommentTok{// ensure termination}
\end{Highlighting}
\end{Shaded}

\subsubsection{A Full Example}\label{a-full-example-12}

\begin{Shaded}
\begin{Highlighting}[]
\PreprocessorTok{\#include }\ImportTok{\textless{}stdio.h\textgreater{}}
\PreprocessorTok{\#include }\ImportTok{\textless{}string.h\textgreater{}}

\DataTypeTok{int}\NormalTok{ main}\OperatorTok{(}\DataTypeTok{void}\OperatorTok{)} \OperatorTok{\{}
    \DataTypeTok{char}\NormalTok{ a}\OperatorTok{[}\DecValTok{30}\OperatorTok{]} \OperatorTok{=} \StringTok{"Hello"}\OperatorTok{;}
    \DataTypeTok{char}\NormalTok{ b}\OperatorTok{[]} \OperatorTok{=} \StringTok{"World"}\OperatorTok{;}

\NormalTok{    printf}\OperatorTok{(}\StringTok{"a length = }\SpecialCharTok{\%zu\textbackslash{}n}\StringTok{"}\OperatorTok{,}\NormalTok{ strlen}\OperatorTok{(}\NormalTok{a}\OperatorTok{));}
\NormalTok{    strcpy}\OperatorTok{(}\NormalTok{a}\OperatorTok{,} \StringTok{"Hi"}\OperatorTok{);}
\NormalTok{    strcat}\OperatorTok{(}\NormalTok{a}\OperatorTok{,} \StringTok{" there"}\OperatorTok{);}
\NormalTok{    printf}\OperatorTok{(}\StringTok{"a now = }\SpecialCharTok{\%s\textbackslash{}n}\StringTok{"}\OperatorTok{,}\NormalTok{ a}\OperatorTok{);}

    \ControlFlowTok{if} \OperatorTok{(}\NormalTok{strcmp}\OperatorTok{(}\NormalTok{a}\OperatorTok{,}\NormalTok{ b}\OperatorTok{)} \OperatorTok{==} \DecValTok{0}\OperatorTok{)}
\NormalTok{        printf}\OperatorTok{(}\StringTok{"a equals b}\SpecialCharTok{\textbackslash{}n}\StringTok{"}\OperatorTok{);}
    \ControlFlowTok{else}
\NormalTok{        printf}\OperatorTok{(}\StringTok{"a != b}\SpecialCharTok{\textbackslash{}n}\StringTok{"}\OperatorTok{);}

    \ControlFlowTok{return} \DecValTok{0}\OperatorTok{;}
\OperatorTok{\}}
\end{Highlighting}
\end{Shaded}

\subsubsection{Why It Matters}\label{why-it-matters-27}

\begin{itemize}
\tightlist
\item
  Without these helpers, you'd write long loops to process strings.
\item
  \texttt{\textless{}string.h\textgreater{}} functions are efficient and
  standard across all C compilers.
\item
  Learning them prepares you for real-world text processing.
\end{itemize}

\subsubsection{Exercises}\label{exercises-28}

\begin{enumerate}
\def\labelenumi{\arabic{enumi}.}
\tightlist
\item
  Read a string and print its length using \texttt{strlen}.
\item
  Copy one string into another and print both.
\item
  Concatenate \texttt{"Hello"} and \texttt{"World"} into a buffer and
  print the result.
\item
  Read two strings and print which comes first in dictionary order.
\item
  Use \texttt{strncpy} to safely copy \texttt{"C\ programming"} into a
  buffer of size 8, and print the result.
\end{enumerate}

\subsection{6.5 Building a Word Counter}\label{building-a-word-counter}

Now that we know how to use arrays and string functions, let's build a
simple word counter. This program will:

\begin{enumerate}
\def\labelenumi{\arabic{enumi}.}
\tightlist
\item
  Read a line of text.
\item
  Split it into words.
\item
  Count how many words there are.
\end{enumerate}

\subsubsection{Step 1: Reading a Line}\label{step-1-reading-a-line}

Use \texttt{fgets} to safely read a line of input (it includes spaces).

\begin{Shaded}
\begin{Highlighting}[]
\PreprocessorTok{\#include }\ImportTok{\textless{}stdio.h\textgreater{}}

\DataTypeTok{int}\NormalTok{ main}\OperatorTok{(}\DataTypeTok{void}\OperatorTok{)} \OperatorTok{\{}
    \DataTypeTok{char}\NormalTok{ line}\OperatorTok{[}\DecValTok{200}\OperatorTok{];}
\NormalTok{    printf}\OperatorTok{(}\StringTok{"Enter a sentence:}\SpecialCharTok{\textbackslash{}n}\StringTok{"}\OperatorTok{);}
\NormalTok{    fgets}\OperatorTok{(}\NormalTok{line}\OperatorTok{,} \KeywordTok{sizeof}\OperatorTok{(}\NormalTok{line}\OperatorTok{),}\NormalTok{ stdin}\OperatorTok{);}
\NormalTok{    printf}\OperatorTok{(}\StringTok{"You entered: }\SpecialCharTok{\%s}\StringTok{"}\OperatorTok{,}\NormalTok{ line}\OperatorTok{);}
    \ControlFlowTok{return} \DecValTok{0}\OperatorTok{;}
\OperatorTok{\}}
\end{Highlighting}
\end{Shaded}

\subsubsection{Step 2: Splitting Into
Words}\label{step-2-splitting-into-words}

The \texttt{\textless{}string.h\textgreater{}} function \texttt{strtok}
splits a string into tokens (pieces) separated by delimiters.

\begin{Shaded}
\begin{Highlighting}[]
\PreprocessorTok{\#include }\ImportTok{\textless{}stdio.h\textgreater{}}
\PreprocessorTok{\#include }\ImportTok{\textless{}string.h\textgreater{}}

\DataTypeTok{int}\NormalTok{ main}\OperatorTok{(}\DataTypeTok{void}\OperatorTok{)} \OperatorTok{\{}
    \DataTypeTok{char}\NormalTok{ line}\OperatorTok{[}\DecValTok{200}\OperatorTok{];}
\NormalTok{    printf}\OperatorTok{(}\StringTok{"Enter a sentence:}\SpecialCharTok{\textbackslash{}n}\StringTok{"}\OperatorTok{);}
\NormalTok{    fgets}\OperatorTok{(}\NormalTok{line}\OperatorTok{,} \KeywordTok{sizeof}\OperatorTok{(}\NormalTok{line}\OperatorTok{),}\NormalTok{ stdin}\OperatorTok{);}

    \DataTypeTok{int}\NormalTok{ count }\OperatorTok{=} \DecValTok{0}\OperatorTok{;}
    \DataTypeTok{char} \OperatorTok{{-}}\NormalTok{word }\OperatorTok{=}\NormalTok{ strtok}\OperatorTok{(}\NormalTok{line}\OperatorTok{,} \StringTok{" }\SpecialCharTok{\textbackslash{}t\textbackslash{}n}\StringTok{"}\OperatorTok{);}  \CommentTok{// split by space, tab, newline}
    \ControlFlowTok{while} \OperatorTok{(}\NormalTok{word }\OperatorTok{!=}\NormalTok{ NULL}\OperatorTok{)} \OperatorTok{\{}
\NormalTok{        printf}\OperatorTok{(}\StringTok{"Word: }\SpecialCharTok{\%s\textbackslash{}n}\StringTok{"}\OperatorTok{,}\NormalTok{ word}\OperatorTok{);}
\NormalTok{        count}\OperatorTok{++;}
\NormalTok{        word }\OperatorTok{=}\NormalTok{ strtok}\OperatorTok{(}\NormalTok{NULL}\OperatorTok{,} \StringTok{" }\SpecialCharTok{\textbackslash{}t\textbackslash{}n}\StringTok{"}\OperatorTok{);}
    \OperatorTok{\}}

\NormalTok{    printf}\OperatorTok{(}\StringTok{"Total words = }\SpecialCharTok{\%d\textbackslash{}n}\StringTok{"}\OperatorTok{,}\NormalTok{ count}\OperatorTok{);}
    \ControlFlowTok{return} \DecValTok{0}\OperatorTok{;}
\OperatorTok{\}}
\end{Highlighting}
\end{Shaded}

Example run:

\begin{verbatim}
Enter a sentence:
C makes low-level programming fun
Word: C
Word: makes
Word: low-level
Word: programming
Word: fun
Total words = 5
\end{verbatim}

\subsubsection{Step 3: Counting Word Frequencies
(Optional)}\label{step-3-counting-word-frequencies-optional}

We can go further: keep an array of words and their counts.

\begin{Shaded}
\begin{Highlighting}[]
\PreprocessorTok{\#include }\ImportTok{\textless{}stdio.h\textgreater{}}
\PreprocessorTok{\#include }\ImportTok{\textless{}string.h\textgreater{}}

\PreprocessorTok{\#define MAX\_WORDS }\DecValTok{50}
\PreprocessorTok{\#define MAX\_LEN   }\DecValTok{20}

\DataTypeTok{int}\NormalTok{ main}\OperatorTok{(}\DataTypeTok{void}\OperatorTok{)} \OperatorTok{\{}
    \DataTypeTok{char}\NormalTok{ line}\OperatorTok{[}\DecValTok{200}\OperatorTok{];}
    \DataTypeTok{char}\NormalTok{ words}\OperatorTok{[}\NormalTok{MAX\_WORDS}\OperatorTok{][}\NormalTok{MAX\_LEN}\OperatorTok{];}
    \DataTypeTok{int}\NormalTok{ counts}\OperatorTok{[}\NormalTok{MAX\_WORDS}\OperatorTok{]} \OperatorTok{=} \OperatorTok{\{}\DecValTok{0}\OperatorTok{\};}
    \DataTypeTok{int}\NormalTok{ total }\OperatorTok{=} \DecValTok{0}\OperatorTok{;}

\NormalTok{    printf}\OperatorTok{(}\StringTok{"Enter a sentence:}\SpecialCharTok{\textbackslash{}n}\StringTok{"}\OperatorTok{);}
\NormalTok{    fgets}\OperatorTok{(}\NormalTok{line}\OperatorTok{,} \KeywordTok{sizeof}\OperatorTok{(}\NormalTok{line}\OperatorTok{),}\NormalTok{ stdin}\OperatorTok{);}

    \DataTypeTok{char} \OperatorTok{{-}}\NormalTok{w }\OperatorTok{=}\NormalTok{ strtok}\OperatorTok{(}\NormalTok{line}\OperatorTok{,} \StringTok{" }\SpecialCharTok{\textbackslash{}t\textbackslash{}n}\StringTok{"}\OperatorTok{);}
    \ControlFlowTok{while} \OperatorTok{(}\NormalTok{w }\OperatorTok{!=}\NormalTok{ NULL }\OperatorTok{\&\&}\NormalTok{ total }\OperatorTok{\textless{}}\NormalTok{ MAX\_WORDS}\OperatorTok{)} \OperatorTok{\{}
        \DataTypeTok{int}\NormalTok{ found }\OperatorTok{=} \DecValTok{0}\OperatorTok{;}
        \ControlFlowTok{for} \OperatorTok{(}\DataTypeTok{int}\NormalTok{ i }\OperatorTok{=} \DecValTok{0}\OperatorTok{;}\NormalTok{ i }\OperatorTok{\textless{}}\NormalTok{ total}\OperatorTok{;}\NormalTok{ i}\OperatorTok{++)} \OperatorTok{\{}
            \ControlFlowTok{if} \OperatorTok{(}\NormalTok{strcmp}\OperatorTok{(}\NormalTok{words}\OperatorTok{[}\NormalTok{i}\OperatorTok{],}\NormalTok{ w}\OperatorTok{)} \OperatorTok{==} \DecValTok{0}\OperatorTok{)} \OperatorTok{\{}
\NormalTok{                counts}\OperatorTok{[}\NormalTok{i}\OperatorTok{]++;}
\NormalTok{                found }\OperatorTok{=} \DecValTok{1}\OperatorTok{;}
                \ControlFlowTok{break}\OperatorTok{;}
            \OperatorTok{\}}
        \OperatorTok{\}}
        \ControlFlowTok{if} \OperatorTok{(!}\NormalTok{found }\OperatorTok{\&\&}\NormalTok{ strlen}\OperatorTok{(}\NormalTok{w}\OperatorTok{)} \OperatorTok{\textless{}}\NormalTok{ MAX\_LEN}\OperatorTok{)} \OperatorTok{\{}
\NormalTok{            strcpy}\OperatorTok{(}\NormalTok{words}\OperatorTok{[}\NormalTok{total}\OperatorTok{],}\NormalTok{ w}\OperatorTok{);}
\NormalTok{            counts}\OperatorTok{[}\NormalTok{total}\OperatorTok{]} \OperatorTok{=} \DecValTok{1}\OperatorTok{;}
\NormalTok{            total}\OperatorTok{++;}
        \OperatorTok{\}}
\NormalTok{        w }\OperatorTok{=}\NormalTok{ strtok}\OperatorTok{(}\NormalTok{NULL}\OperatorTok{,} \StringTok{" }\SpecialCharTok{\textbackslash{}t\textbackslash{}n}\StringTok{"}\OperatorTok{);}
    \OperatorTok{\}}

\NormalTok{    printf}\OperatorTok{(}\StringTok{"}\SpecialCharTok{\textbackslash{}n}\StringTok{Word frequencies:}\SpecialCharTok{\textbackslash{}n}\StringTok{"}\OperatorTok{);}
    \ControlFlowTok{for} \OperatorTok{(}\DataTypeTok{int}\NormalTok{ i }\OperatorTok{=} \DecValTok{0}\OperatorTok{;}\NormalTok{ i }\OperatorTok{\textless{}}\NormalTok{ total}\OperatorTok{;}\NormalTok{ i}\OperatorTok{++)} \OperatorTok{\{}
\NormalTok{        printf}\OperatorTok{(}\StringTok{"}\SpecialCharTok{\%s}\StringTok{ : }\SpecialCharTok{\%d\textbackslash{}n}\StringTok{"}\OperatorTok{,}\NormalTok{ words}\OperatorTok{[}\NormalTok{i}\OperatorTok{],}\NormalTok{ counts}\OperatorTok{[}\NormalTok{i}\OperatorTok{]);}
    \OperatorTok{\}}

    \ControlFlowTok{return} \DecValTok{0}\OperatorTok{;}
\OperatorTok{\}}
\end{Highlighting}
\end{Shaded}

\subsubsection{Why It Matters}\label{why-it-matters-28}

\begin{itemize}
\tightlist
\item
  Combines arrays, strings, and library functions into one real project.
\item
  Shows how to process input text, split it, and analyze it.
\item
  This is a small taste of how text editors, search engines, and
  compilers start their work.
\end{itemize}

\subsubsection{Exercises}\label{exercises-29}

\begin{enumerate}
\def\labelenumi{\arabic{enumi}.}
\tightlist
\item
  Modify the word counter so it ignores case (treat \texttt{"C"} and
  \texttt{"c"} as the same).
\item
  Extend the program to print the longest word in the input.
\item
  Extend the program to print the average word length.
\item
  Modify it so it counts only unique words, and prints the total number
  of distinct words.
\item
  Write a version that reads from a file instead of user input, and
  counts words in the whole file.
\end{enumerate}

\subsection{Problems}\label{problems-4}

\subsubsection{1. Index Fill}\label{index-fill}

Declare an array of 10 integers. Use a loop to fill it so that
\texttt{arr{[}i{]}\ =\ i}. Print all elements.

\subsubsection{2. Array Sum}\label{array-sum}

Read 5 integers into an array and print their sum.

\subsubsection{3. Array Maximum}\label{array-maximum}

Read 8 integers into an array and print the largest.

\subsubsection{4. Reverse Print}\label{reverse-print}

Read 10 integers and print them in reverse order using a loop.

\subsubsection{5. Even Counter}\label{even-counter}

Read 10 integers into an array and count how many are even.

\subsubsection{6. Average of Numbers}\label{average-of-numbers}

Read 10 integers into an array and print their average as a
\texttt{double}.

\subsubsection{7. Sum of Squares}\label{sum-of-squares}

Given \texttt{int\ arr{[}{]}\ =\ \{1,2,3,4,5\}}, compute and print the
sum of their squares.

\subsubsection{8. Print Characters of a
String}\label{print-characters-of-a-string}

Declare a string \texttt{"Hello"} and print each character on a separate
line.

\subsubsection{9. Greeting with String}\label{greeting-with-string}

Read a string (name) into a char array and print
\texttt{"Hello,\ \textless{}name\textgreater{}!"}.

\subsubsection{10. Count Characters
(Manual)}\label{count-characters-manual}

Read a string and count its length manually (without \texttt{strlen}).

\subsubsection{11. Modify a String}\label{modify-a-string}

Declare \texttt{char\ word{[}{]}\ =\ "dog"} and change it to
\texttt{"fog"} by modifying one character. Print the result.

\subsubsection{12. Two Strings Reverse
Order}\label{two-strings-reverse-order}

Read two strings and print them in reverse order (second, then first).

\subsubsection{\texorpdfstring{13. Length with
\texttt{strlen}}{13. Length with strlen}}\label{length-with-strlen}

Read a string and print its length using \texttt{strlen}.

\subsubsection{14. String Copy}\label{string-copy}

Read a string into one buffer, copy it into another with
\texttt{strcpy}, and print both.

\subsubsection{15. String Concatenation}\label{string-concatenation}

Concatenate \texttt{"Hello"} and \texttt{"World"} into
\texttt{"HelloWorld"} using \texttt{strcat}. Print the result.

\subsubsection{16. String Comparison}\label{string-comparison}

Read two strings and print which one comes first in dictionary order
(use \texttt{strcmp}).

\subsubsection{\texorpdfstring{17. Safe Copy with
\texttt{strncpy}}{17. Safe Copy with strncpy}}\label{safe-copy-with-strncpy}

Copy \texttt{"C\ programming"} into a buffer of size 8 using
\texttt{strncpy}. Ensure the result is null-terminated, then print it.

\subsubsection{18. Word Counter (Basic)}\label{word-counter-basic}

Read a sentence with \texttt{fgets} and count how many words it has
(split by spaces). Print the count.

\subsubsection{19. Longest Word}\label{longest-word}

Extend the word counter: print the longest word in the sentence.

\subsubsection{20. Word Frequency}\label{word-frequency}

Read a line of text and count the frequency of each unique word. Print
the results.

\bookmarksetup{startatroot}

\chapter{Part III. Deeper into C}\label{part-iii.-deeper-into-c}

\section{Chapter 7. Pointers}\label{chapter-7.-pointers}

\subsection{7.1 What is a Pointer?}\label{what-is-a-pointer}

In C, a pointer is a variable that stores the *address- of another
variable.

Think of it like:

\begin{itemize}
\tightlist
\item
  A normal variable stores a value.
\item
  A pointer stores where that value lives in memory.
\end{itemize}

\subsubsection{Addresses in Memory}\label{addresses-in-memory}

Every variable in a program is stored in memory at some location. You
can get that location using the address-of operator \texttt{\&}.

\begin{Shaded}
\begin{Highlighting}[]
\PreprocessorTok{\#include }\ImportTok{\textless{}stdio.h\textgreater{}}

\DataTypeTok{int}\NormalTok{ main}\OperatorTok{(}\DataTypeTok{void}\OperatorTok{)} \OperatorTok{\{}
    \DataTypeTok{int}\NormalTok{ x }\OperatorTok{=} \DecValTok{42}\OperatorTok{;}
\NormalTok{    printf}\OperatorTok{(}\StringTok{"x = }\SpecialCharTok{\%d\textbackslash{}n}\StringTok{"}\OperatorTok{,}\NormalTok{ x}\OperatorTok{);}
\NormalTok{    printf}\OperatorTok{(}\StringTok{"address of x = }\SpecialCharTok{\%p\textbackslash{}n}\StringTok{"}\OperatorTok{,} \OperatorTok{(}\DataTypeTok{void}\OperatorTok{{-})\&}\NormalTok{x}\OperatorTok{);}
    \ControlFlowTok{return} \DecValTok{0}\OperatorTok{;}
\OperatorTok{\}}
\end{Highlighting}
\end{Shaded}

Example output:

\begin{Shaded}
\begin{Highlighting}[]
\ExtensionTok{x}\NormalTok{ = 42}
\ExtensionTok{address}\NormalTok{ of x = 0x7ffee8c48a7c}
\end{Highlighting}
\end{Shaded}

(The exact address will differ.)

\subsubsection{Declaring a Pointer}\label{declaring-a-pointer}

A pointer variable is declared with \texttt{-}:

\begin{Shaded}
\begin{Highlighting}[]
\DataTypeTok{int} \OperatorTok{{-}}\NormalTok{p}\OperatorTok{;}   \CommentTok{// p can hold the address of an int}
\end{Highlighting}
\end{Shaded}

To assign it:

\begin{Shaded}
\begin{Highlighting}[]
\DataTypeTok{int}\NormalTok{ x }\OperatorTok{=} \DecValTok{42}\OperatorTok{;}
\DataTypeTok{int} \OperatorTok{{-}}\NormalTok{p }\OperatorTok{=} \OperatorTok{\&}\NormalTok{x}\OperatorTok{;}   \CommentTok{// p points to x}
\end{Highlighting}
\end{Shaded}

\subsubsection{Dereferencing a Pointer}\label{dereferencing-a-pointer}

To get the value stored at the address, use \texttt{-} again
(dereference):

\begin{Shaded}
\begin{Highlighting}[]
\PreprocessorTok{\#include }\ImportTok{\textless{}stdio.h\textgreater{}}

\DataTypeTok{int}\NormalTok{ main}\OperatorTok{(}\DataTypeTok{void}\OperatorTok{)} \OperatorTok{\{}
    \DataTypeTok{int}\NormalTok{ x }\OperatorTok{=} \DecValTok{42}\OperatorTok{;}
    \DataTypeTok{int} \OperatorTok{{-}}\NormalTok{p }\OperatorTok{=} \OperatorTok{\&}\NormalTok{x}\OperatorTok{;}

\NormalTok{    printf}\OperatorTok{(}\StringTok{"x = }\SpecialCharTok{\%d\textbackslash{}n}\StringTok{"}\OperatorTok{,}\NormalTok{ x}\OperatorTok{);}
\NormalTok{    printf}\OperatorTok{(}\StringTok{"p points to }\SpecialCharTok{\%p\textbackslash{}n}\StringTok{"}\OperatorTok{,} \OperatorTok{(}\DataTypeTok{void}\OperatorTok{{-})}\NormalTok{p}\OperatorTok{);}
\NormalTok{    printf}\OperatorTok{(}\StringTok{"{-}p = }\SpecialCharTok{\%d\textbackslash{}n}\StringTok{"}\OperatorTok{,} \OperatorTok{{-}}\NormalTok{p}\OperatorTok{);}   \CommentTok{// value at that address}
    \ControlFlowTok{return} \DecValTok{0}\OperatorTok{;}
\OperatorTok{\}}
\end{Highlighting}
\end{Shaded}

Output:

\begin{verbatim}
x = 42
p points to 0x7ffee8c48a7c
-p = 42
\end{verbatim}

\subsubsection{Changing Through a
Pointer}\label{changing-through-a-pointer}

If you change \texttt{-p}, it changes the original variable:

\begin{Shaded}
\begin{Highlighting}[]
\DataTypeTok{int}\NormalTok{ x }\OperatorTok{=} \DecValTok{42}\OperatorTok{;}
\DataTypeTok{int} \OperatorTok{{-}}\NormalTok{p }\OperatorTok{=} \OperatorTok{\&}\NormalTok{x}\OperatorTok{;}
\OperatorTok{{-}}\NormalTok{p }\OperatorTok{=} \DecValTok{99}\OperatorTok{;}        \CommentTok{// modifies x}
\NormalTok{printf}\OperatorTok{(}\StringTok{"}\SpecialCharTok{\%d\textbackslash{}n}\StringTok{"}\OperatorTok{,}\NormalTok{ x}\OperatorTok{);}   \CommentTok{// prints 99}
\end{Highlighting}
\end{Shaded}

\subsubsection{Pointers and Types}\label{pointers-and-types}

The type of pointer must match the type it points to:

\begin{itemize}
\tightlist
\item
  \texttt{int\ -} → points to \texttt{int}
\item
  \texttt{double\ -} → points to \texttt{double}
\item
  \texttt{char\ -} → points to \texttt{char}
\end{itemize}

This tells the compiler how to interpret the memory at that address.

\subsubsection{Why It Matters}\label{why-it-matters-29}

\begin{itemize}
\tightlist
\item
  Pointers let you work directly with memory.
\item
  They are essential for arrays, strings, dynamic memory, and data
  structures.
\item
  Understanding pointers is the key to mastering C.
\end{itemize}

\subsubsection{Exercises}\label{exercises-30}

\begin{enumerate}
\def\labelenumi{\arabic{enumi}.}
\tightlist
\item
  Declare an \texttt{int\ x\ =\ 5} and a pointer \texttt{p} that points
  to it. Print both \texttt{x} and \texttt{-p}.
\item
  Change \texttt{x} by assigning to \texttt{-p} instead of \texttt{x}.
  Print the result.
\item
  Declare two integers \texttt{a=10}, \texttt{b=20} and pointers
  \texttt{pa}, \texttt{pb}. Print their addresses.
\item
  Write a program that reads an integer into \texttt{x} and prints its
  address.
\item
  Experiment: declare \texttt{double\ y\ =\ 3.14} and
  \texttt{double\ -py\ =\ \&y}. Print \texttt{y}, its address, and
  \texttt{-py}.
\end{enumerate}

\subsection{7.2 Pointers and Addresses}\label{pointers-and-addresses}

In the last section, we saw that a pointer stores the address of another
variable. Now let's explore that connection more carefully.

\subsubsection{Variables and Their
Addresses}\label{variables-and-their-addresses}

Every variable in C has:

\begin{enumerate}
\def\labelenumi{\arabic{enumi}.}
\tightlist
\item
  A value - what you store in it.
\item
  An address - where it lives in memory.
\end{enumerate}

\begin{Shaded}
\begin{Highlighting}[]
\PreprocessorTok{\#include }\ImportTok{\textless{}stdio.h\textgreater{}}

\DataTypeTok{int}\NormalTok{ main}\OperatorTok{(}\DataTypeTok{void}\OperatorTok{)} \OperatorTok{\{}
    \DataTypeTok{int}\NormalTok{ a }\OperatorTok{=} \DecValTok{123}\OperatorTok{;}
\NormalTok{    printf}\OperatorTok{(}\StringTok{"value = }\SpecialCharTok{\%d\textbackslash{}n}\StringTok{"}\OperatorTok{,}\NormalTok{ a}\OperatorTok{);}
\NormalTok{    printf}\OperatorTok{(}\StringTok{"address = }\SpecialCharTok{\%p\textbackslash{}n}\StringTok{"}\OperatorTok{,} \OperatorTok{(}\DataTypeTok{void}\OperatorTok{{-})\&}\NormalTok{a}\OperatorTok{);}
    \ControlFlowTok{return} \DecValTok{0}\OperatorTok{;}
\OperatorTok{\}}
\end{Highlighting}
\end{Shaded}

\subsubsection{Storing the Address in a
Pointer}\label{storing-the-address-in-a-pointer}

You can save that address inside a pointer:

\begin{Shaded}
\begin{Highlighting}[]
\DataTypeTok{int}\NormalTok{ a }\OperatorTok{=} \DecValTok{123}\OperatorTok{;}
\DataTypeTok{int} \OperatorTok{*}\NormalTok{pa }\OperatorTok{=} \OperatorTok{\&}\NormalTok{a}\OperatorTok{;}   \CommentTok{// pa points to a}
\end{Highlighting}
\end{Shaded}

Now:

\begin{itemize}
\tightlist
\item
  \texttt{pa} holds the address of \texttt{a}.
\item
  \texttt{*pa} is another way to refer to the value of \texttt{a}.
\end{itemize}

\subsubsection{Visualizing It}\label{visualizing-it}

Memory (simplified):

\begin{Shaded}
\begin{Highlighting}[]
 \ExtensionTok{a:}\NormalTok{  123}
\ExtensionTok{pa:}  \KeywordTok{\&}\ExtensionTok{a}
\end{Highlighting}
\end{Shaded}

When you write \texttt{*pa}, you are saying ``follow pa to where it
points'' → \texttt{123}.

\subsubsection{Example: Changing Through a
Pointer}\label{example-changing-through-a-pointer}

\begin{Shaded}
\begin{Highlighting}[]
\PreprocessorTok{\#include }\ImportTok{\textless{}stdio.h\textgreater{}}

\DataTypeTok{int}\NormalTok{ main}\OperatorTok{(}\DataTypeTok{void}\OperatorTok{)} \OperatorTok{\{}
    \DataTypeTok{int}\NormalTok{ x }\OperatorTok{=} \DecValTok{5}\OperatorTok{;}
    \DataTypeTok{int} \OperatorTok{*}\NormalTok{p }\OperatorTok{=} \OperatorTok{\&}\NormalTok{x}\OperatorTok{;}

\NormalTok{    printf}\OperatorTok{(}\StringTok{"x = }\SpecialCharTok{\%d\textbackslash{}n}\StringTok{"}\OperatorTok{,}\NormalTok{ x}\OperatorTok{);}
    \OperatorTok{*}\NormalTok{p }\OperatorTok{=} \DecValTok{42}\OperatorTok{;}        \CommentTok{// modifies x}
\NormalTok{    printf}\OperatorTok{(}\StringTok{"x = }\SpecialCharTok{\%d\textbackslash{}n}\StringTok{"}\OperatorTok{,}\NormalTok{ x}\OperatorTok{);}
    \ControlFlowTok{return} \DecValTok{0}\OperatorTok{;}
\OperatorTok{\}}
\end{Highlighting}
\end{Shaded}

Output:

\begin{Shaded}
\begin{Highlighting}[]
\ExtensionTok{x}\NormalTok{ = 5}
\ExtensionTok{x}\NormalTok{ = 42}
\end{Highlighting}
\end{Shaded}

\subsubsection{Multiple Pointers to the Same
Variable}\label{multiple-pointers-to-the-same-variable}

More than one pointer can point to the same place:

\begin{Shaded}
\begin{Highlighting}[]
\DataTypeTok{int}\NormalTok{ n }\OperatorTok{=} \DecValTok{7}\OperatorTok{;}
\DataTypeTok{int} \OperatorTok{*}\NormalTok{p1 }\OperatorTok{=} \OperatorTok{\&}\NormalTok{n}\OperatorTok{;}
\DataTypeTok{int} \OperatorTok{*}\NormalTok{p2 }\OperatorTok{=} \OperatorTok{\&}\NormalTok{n}\OperatorTok{;}

\OperatorTok{*}\NormalTok{p1 }\OperatorTok{=} \DecValTok{99}\OperatorTok{;}   \CommentTok{// changes n}
\NormalTok{printf}\OperatorTok{(}\StringTok{"}\SpecialCharTok{\%d\textbackslash{}n}\StringTok{"}\OperatorTok{,} \OperatorTok{*}\NormalTok{p2}\OperatorTok{);}  \CommentTok{// prints 99}
\end{Highlighting}
\end{Shaded}

\subsubsection{Pointer Assignment}\label{pointer-assignment}

Pointers can be reassigned to point to different variables:

\begin{Shaded}
\begin{Highlighting}[]
\DataTypeTok{int}\NormalTok{ a }\OperatorTok{=} \DecValTok{10}\OperatorTok{,}\NormalTok{ b }\OperatorTok{=} \DecValTok{20}\OperatorTok{;}
\DataTypeTok{int} \OperatorTok{*}\NormalTok{p }\OperatorTok{=} \OperatorTok{\&}\NormalTok{a}\OperatorTok{;}
\NormalTok{printf}\OperatorTok{(}\StringTok{"*p = }\SpecialCharTok{\%d\textbackslash{}n}\StringTok{"}\OperatorTok{,} \OperatorTok{*}\NormalTok{p}\OperatorTok{);}  \CommentTok{// 10}

\NormalTok{p }\OperatorTok{=} \OperatorTok{\&}\NormalTok{b}\OperatorTok{;}
\NormalTok{printf}\OperatorTok{(}\StringTok{"*p = }\SpecialCharTok{\%d\textbackslash{}n}\StringTok{"}\OperatorTok{,} \OperatorTok{*}\NormalTok{p}\OperatorTok{);}  \CommentTok{// 20}
\end{Highlighting}
\end{Shaded}

\subsubsection{Why It Matters}\label{why-it-matters-30}

\begin{itemize}
\tightlist
\item
  Pointers are *names for addresses-.
\item
  They let functions and data structures share and modify the same
  memory.
\item
  Understanding addresses is essential for arrays, strings, and dynamic
  memory.
\end{itemize}

\subsubsection{Exercises}\label{exercises-31}

\begin{enumerate}
\def\labelenumi{\arabic{enumi}.}
\tightlist
\item
  Declare an integer \texttt{n\ =\ 100}, and a pointer \texttt{pn}.
  Print \texttt{n}, \texttt{\&n}, \texttt{pn}, and \texttt{-pn}.
\item
  Write a program where two pointers point to the same integer. Modify
  the value through one pointer and print it through the other.
\item
  Declare two integers \texttt{a} and \texttt{b}. Make one pointer point
  first to \texttt{a}, then to \texttt{b}, printing values each time.
\item
  Write a program with three integers and an array of pointers
  (\texttt{int\ -ptrs{[}3{]}}). Make each pointer point to one integer
  and print their values.
\item
  Experiment: print the size of an \texttt{int\ -}, \texttt{double\ -},
  and \texttt{char\ -} using \texttt{sizeof}. Compare the results.
\end{enumerate}

\subsection{7.3 Arrays and Pointers}\label{arrays-and-pointers}

In C, an array name often behaves like a pointer. Understanding this
connection is key to working with strings, loops, and dynamic memory.

\subsubsection{Array Name as an Address}\label{array-name-as-an-address}

\begin{Shaded}
\begin{Highlighting}[]
\PreprocessorTok{\#include }\ImportTok{\textless{}stdio.h\textgreater{}}

\DataTypeTok{int}\NormalTok{ main}\OperatorTok{(}\DataTypeTok{void}\OperatorTok{)} \OperatorTok{\{}
    \DataTypeTok{int}\NormalTok{ arr}\OperatorTok{[}\DecValTok{3}\OperatorTok{]} \OperatorTok{=} \OperatorTok{\{}\DecValTok{10}\OperatorTok{,} \DecValTok{20}\OperatorTok{,} \DecValTok{30}\OperatorTok{\};}
\NormalTok{    printf}\OperatorTok{(}\StringTok{"arr = }\SpecialCharTok{\%p\textbackslash{}n}\StringTok{"}\OperatorTok{,} \OperatorTok{(}\DataTypeTok{void}\OperatorTok{*)}\NormalTok{arr}\OperatorTok{);}
\NormalTok{    printf}\OperatorTok{(}\StringTok{"\&arr[0] = }\SpecialCharTok{\%p\textbackslash{}n}\StringTok{"}\OperatorTok{,} \OperatorTok{(}\DataTypeTok{void}\OperatorTok{*)\&}\NormalTok{arr}\OperatorTok{[}\DecValTok{0}\OperatorTok{]);}
    \ControlFlowTok{return} \DecValTok{0}\OperatorTok{;}
\OperatorTok{\}}
\end{Highlighting}
\end{Shaded}

Output (addresses match):

\begin{Shaded}
\begin{Highlighting}[]
\ExtensionTok{arr}\NormalTok{ = 0x7ffee2c38930}
\KeywordTok{\&}\VariableTok{arr}\OperatorTok{[}\DecValTok{0}\OperatorTok{]} \ExtensionTok{=}\NormalTok{ 0x7ffee2c38930}
\end{Highlighting}
\end{Shaded}

The name \texttt{arr} is treated as the address of its first element.

\subsubsection{Accessing with Pointers}\label{accessing-with-pointers}

You can access array elements with either array indexing
(\texttt{arr{[}i{]}}) or pointer arithmetic (\texttt{-(arr\ +\ i)}).

\begin{Shaded}
\begin{Highlighting}[]
\DataTypeTok{int}\NormalTok{ arr}\OperatorTok{[}\DecValTok{3}\OperatorTok{]} \OperatorTok{=} \OperatorTok{\{}\DecValTok{10}\OperatorTok{,} \DecValTok{20}\OperatorTok{,} \DecValTok{30}\OperatorTok{\};}
\NormalTok{printf}\OperatorTok{(}\StringTok{"}\SpecialCharTok{\%d\textbackslash{}n}\StringTok{"}\OperatorTok{,}\NormalTok{ arr}\OperatorTok{[}\DecValTok{1}\OperatorTok{]);}       \CommentTok{// array indexing}
\NormalTok{printf}\OperatorTok{(}\StringTok{"}\SpecialCharTok{\%d\textbackslash{}n}\StringTok{"}\OperatorTok{,} \OperatorTok{{-}(}\NormalTok{arr }\OperatorTok{+} \DecValTok{1}\OperatorTok{));}   \CommentTok{// pointer arithmetic}
\end{Highlighting}
\end{Shaded}

Both print \texttt{20}.

\subsubsection{Using a Pointer Variable}\label{using-a-pointer-variable}

\begin{Shaded}
\begin{Highlighting}[]
\DataTypeTok{int}\NormalTok{ arr}\OperatorTok{[}\DecValTok{3}\OperatorTok{]} \OperatorTok{=} \OperatorTok{\{}\DecValTok{10}\OperatorTok{,} \DecValTok{20}\OperatorTok{,} \DecValTok{30}\OperatorTok{\};}
\DataTypeTok{int} \OperatorTok{{-}}\NormalTok{p }\OperatorTok{=}\NormalTok{ arr}\OperatorTok{;}     \CommentTok{// same as \&arr[0]}

\NormalTok{printf}\OperatorTok{(}\StringTok{"}\SpecialCharTok{\%d\textbackslash{}n}\StringTok{"}\OperatorTok{,} \OperatorTok{{-}}\NormalTok{p}\OperatorTok{);}     \CommentTok{// 10}
\NormalTok{printf}\OperatorTok{(}\StringTok{"}\SpecialCharTok{\%d\textbackslash{}n}\StringTok{"}\OperatorTok{,} \OperatorTok{{-}(}\NormalTok{p}\OperatorTok{+}\DecValTok{1}\OperatorTok{));} \CommentTok{// 20}
\NormalTok{printf}\OperatorTok{(}\StringTok{"}\SpecialCharTok{\%d\textbackslash{}n}\StringTok{"}\OperatorTok{,} \OperatorTok{{-}(}\NormalTok{p}\OperatorTok{+}\DecValTok{2}\OperatorTok{));} \CommentTok{// 30}
\end{Highlighting}
\end{Shaded}

\subsubsection{Iterating with a Pointer}\label{iterating-with-a-pointer}

\begin{Shaded}
\begin{Highlighting}[]
\PreprocessorTok{\#include }\ImportTok{\textless{}stdio.h\textgreater{}}

\DataTypeTok{int}\NormalTok{ main}\OperatorTok{(}\DataTypeTok{void}\OperatorTok{)} \OperatorTok{\{}
    \DataTypeTok{int}\NormalTok{ arr}\OperatorTok{[}\DecValTok{5}\OperatorTok{]} \OperatorTok{=} \OperatorTok{\{}\DecValTok{2}\OperatorTok{,} \DecValTok{4}\OperatorTok{,} \DecValTok{6}\OperatorTok{,} \DecValTok{8}\OperatorTok{,} \DecValTok{10}\OperatorTok{\};}
    \DataTypeTok{int} \OperatorTok{{-}}\NormalTok{p }\OperatorTok{=}\NormalTok{ arr}\OperatorTok{;}

    \ControlFlowTok{for} \OperatorTok{(}\DataTypeTok{int}\NormalTok{ i }\OperatorTok{=} \DecValTok{0}\OperatorTok{;}\NormalTok{ i }\OperatorTok{\textless{}} \DecValTok{5}\OperatorTok{;}\NormalTok{ i}\OperatorTok{++)} \OperatorTok{\{}
\NormalTok{        printf}\OperatorTok{(}\StringTok{"}\SpecialCharTok{\%d}\StringTok{ "}\OperatorTok{,} \OperatorTok{{-}(}\NormalTok{p}\OperatorTok{+}\NormalTok{i}\OperatorTok{));}
    \OperatorTok{\}}
\NormalTok{    printf}\OperatorTok{(}\StringTok{"}\SpecialCharTok{\textbackslash{}n}\StringTok{"}\OperatorTok{);}
    \ControlFlowTok{return} \DecValTok{0}\OperatorTok{;}
\OperatorTok{\}}
\end{Highlighting}
\end{Shaded}

\subsubsection{Arrays Are Not Pointers}\label{arrays-are-not-pointers}

Although array names -decay- to pointers in most expressions, they are
not the same thing:

\begin{itemize}
\tightlist
\item
  You cannot reassign an array name.
\item
  \texttt{sizeof(arr)} gives the full array size, while
  \texttt{sizeof(p)} (a pointer) gives only the size of the pointer
  type.
\end{itemize}

\begin{Shaded}
\begin{Highlighting}[]
\DataTypeTok{int}\NormalTok{ arr}\OperatorTok{[}\DecValTok{10}\OperatorTok{];}
\DataTypeTok{int} \OperatorTok{{-}}\NormalTok{p }\OperatorTok{=}\NormalTok{ arr}\OperatorTok{;}

\NormalTok{printf}\OperatorTok{(}\StringTok{"}\SpecialCharTok{\%zu\textbackslash{}n}\StringTok{"}\OperatorTok{,} \KeywordTok{sizeof}\OperatorTok{(}\NormalTok{arr}\OperatorTok{));} \CommentTok{// e.g., 40 (10 ints on 64{-}bit system)}
\NormalTok{printf}\OperatorTok{(}\StringTok{"}\SpecialCharTok{\%zu\textbackslash{}n}\StringTok{"}\OperatorTok{,} \KeywordTok{sizeof}\OperatorTok{(}\NormalTok{p}\OperatorTok{));}   \CommentTok{// e.g., 8 (pointer size)}
\end{Highlighting}
\end{Shaded}

\subsubsection{Strings and Pointers}\label{strings-and-pointers}

Strings are arrays of \texttt{char}. You can use pointer arithmetic to
walk through characters until the null terminator:

\begin{Shaded}
\begin{Highlighting}[]
\PreprocessorTok{\#include }\ImportTok{\textless{}stdio.h\textgreater{}}

\DataTypeTok{int}\NormalTok{ main}\OperatorTok{(}\DataTypeTok{void}\OperatorTok{)} \OperatorTok{\{}
    \DataTypeTok{char}\NormalTok{ word}\OperatorTok{[]} \OperatorTok{=} \StringTok{"Hello"}\OperatorTok{;}
    \DataTypeTok{char} \OperatorTok{{-}}\NormalTok{p }\OperatorTok{=}\NormalTok{ word}\OperatorTok{;}

    \ControlFlowTok{while} \OperatorTok{({-}}\NormalTok{p }\OperatorTok{!=} \CharTok{\textquotesingle{}}\SpecialCharTok{\textbackslash{}0}\CharTok{\textquotesingle{}}\OperatorTok{)} \OperatorTok{\{}
\NormalTok{        printf}\OperatorTok{(}\StringTok{"}\SpecialCharTok{\%c}\StringTok{ "}\OperatorTok{,} \OperatorTok{{-}}\NormalTok{p}\OperatorTok{);}
\NormalTok{        p}\OperatorTok{++;}
    \OperatorTok{\}}
\NormalTok{    printf}\OperatorTok{(}\StringTok{"}\SpecialCharTok{\textbackslash{}n}\StringTok{"}\OperatorTok{);}
    \ControlFlowTok{return} \DecValTok{0}\OperatorTok{;}
\OperatorTok{\}}
\end{Highlighting}
\end{Shaded}

Output:

\begin{Shaded}
\begin{Highlighting}[]
\ExtensionTok{H}\NormalTok{ e l l o}
\end{Highlighting}
\end{Shaded}

\subsubsection{Why It Matters}\label{why-it-matters-31}

\begin{itemize}
\tightlist
\item
  Arrays and pointers are two sides of the same coin in C.
\item
  Pointers give you flexibility in traversing and manipulating arrays.
\item
  This connection is the basis for string handling, dynamic memory, and
  data structures.
\end{itemize}

\subsubsection{Exercises}\label{exercises-32}

\begin{enumerate}
\def\labelenumi{\arabic{enumi}.}
\tightlist
\item
  Declare \texttt{int\ arr{[}5{]}\ =\ \{1,2,3,4,5\}}. Print all elements
  using pointer arithmetic (\texttt{-(arr+i)}).
\item
  Write a function \texttt{print\_array(int\ -p,\ int\ n)} that prints
  all elements of an integer array.
\item
  Read 5 integers into an array, then use a pointer to calculate their
  sum.
\item
  Create a string \texttt{"C\ language"} and use a pointer to print each
  character until
  \texttt{\textquotesingle{}\textbackslash{}0\textquotesingle{}}.
\item
  Compare \texttt{sizeof(arr)} and \texttt{sizeof(p)} where \texttt{p}
  is a pointer to the array. Print both results.
\end{enumerate}

\subsection{7.4 Pointers to Functions}\label{pointers-to-functions}

Just as you can have a pointer to a variable, you can also have a
pointer to a function. This lets you:

\begin{itemize}
\tightlist
\item
  Call functions dynamically,
\item
  Pass functions as arguments,
\item
  Build flexible libraries (like sort with a custom comparison).
\end{itemize}

\subsubsection{Function Names as
Addresses}\label{function-names-as-addresses}

The name of a function is its address in memory. So you can assign it to
a pointer:

\begin{Shaded}
\begin{Highlighting}[]
\PreprocessorTok{\#include }\ImportTok{\textless{}stdio.h\textgreater{}}

\DataTypeTok{int}\NormalTok{ add}\OperatorTok{(}\DataTypeTok{int}\NormalTok{ a}\OperatorTok{,} \DataTypeTok{int}\NormalTok{ b}\OperatorTok{)} \OperatorTok{\{} \ControlFlowTok{return}\NormalTok{ a }\OperatorTok{+}\NormalTok{ b}\OperatorTok{;} \OperatorTok{\}}

\DataTypeTok{int}\NormalTok{ main}\OperatorTok{(}\DataTypeTok{void}\OperatorTok{)} \OperatorTok{\{}
    \DataTypeTok{int} \OperatorTok{({-}}\NormalTok{fp}\OperatorTok{)(}\DataTypeTok{int}\OperatorTok{,} \DataTypeTok{int}\OperatorTok{)} \OperatorTok{=}\NormalTok{ add}\OperatorTok{;}   \CommentTok{// fp points to add}
\NormalTok{    printf}\OperatorTok{(}\StringTok{"}\SpecialCharTok{\%d\textbackslash{}n}\StringTok{"}\OperatorTok{,}\NormalTok{ fp}\OperatorTok{(}\DecValTok{3}\OperatorTok{,} \DecValTok{4}\OperatorTok{));}    \CommentTok{// call through pointer}
    \ControlFlowTok{return} \DecValTok{0}\OperatorTok{;}
\OperatorTok{\}}
\end{Highlighting}
\end{Shaded}

Output:

\begin{Shaded}
\begin{Highlighting}[]
\ExtensionTok{7}
\end{Highlighting}
\end{Shaded}

\subsubsection{Declaring a Function
Pointer}\label{declaring-a-function-pointer}

Syntax:

\begin{Shaded}
\begin{Highlighting}[]
\NormalTok{return\_type }\OperatorTok{({-}}\NormalTok{pointer\_name}\OperatorTok{)(}\NormalTok{parameter\_types}\OperatorTok{);}
\end{Highlighting}
\end{Shaded}

Example:

\begin{Shaded}
\begin{Highlighting}[]
\DataTypeTok{int} \OperatorTok{({-}}\NormalTok{f}\OperatorTok{)(}\DataTypeTok{int}\OperatorTok{,} \DataTypeTok{int}\OperatorTok{);}   \CommentTok{// f is a pointer to a function taking (int,int) and returning int}
\end{Highlighting}
\end{Shaded}

\subsubsection{Example: Multiple
Functions}\label{example-multiple-functions}

\begin{Shaded}
\begin{Highlighting}[]
\PreprocessorTok{\#include }\ImportTok{\textless{}stdio.h\textgreater{}}

\DataTypeTok{int}\NormalTok{ add}\OperatorTok{(}\DataTypeTok{int}\NormalTok{ a}\OperatorTok{,} \DataTypeTok{int}\NormalTok{ b}\OperatorTok{)} \OperatorTok{\{} \ControlFlowTok{return}\NormalTok{ a }\OperatorTok{+}\NormalTok{ b}\OperatorTok{;} \OperatorTok{\}}
\DataTypeTok{int}\NormalTok{ sub}\OperatorTok{(}\DataTypeTok{int}\NormalTok{ a}\OperatorTok{,} \DataTypeTok{int}\NormalTok{ b}\OperatorTok{)} \OperatorTok{\{} \ControlFlowTok{return}\NormalTok{ a }\OperatorTok{{-}}\NormalTok{ b}\OperatorTok{;} \OperatorTok{\}}

\DataTypeTok{int}\NormalTok{ main}\OperatorTok{(}\DataTypeTok{void}\OperatorTok{)} \OperatorTok{\{}
    \DataTypeTok{int} \OperatorTok{({-}}\NormalTok{op}\OperatorTok{)(}\DataTypeTok{int}\OperatorTok{,} \DataTypeTok{int}\OperatorTok{);}

\NormalTok{    op }\OperatorTok{=}\NormalTok{ add}\OperatorTok{;}
\NormalTok{    printf}\OperatorTok{(}\StringTok{"add: }\SpecialCharTok{\%d\textbackslash{}n}\StringTok{"}\OperatorTok{,}\NormalTok{ op}\OperatorTok{(}\DecValTok{10}\OperatorTok{,} \DecValTok{5}\OperatorTok{));}

\NormalTok{    op }\OperatorTok{=}\NormalTok{ sub}\OperatorTok{;}
\NormalTok{    printf}\OperatorTok{(}\StringTok{"sub: }\SpecialCharTok{\%d\textbackslash{}n}\StringTok{"}\OperatorTok{,}\NormalTok{ op}\OperatorTok{(}\DecValTok{10}\OperatorTok{,} \DecValTok{5}\OperatorTok{));}

    \ControlFlowTok{return} \DecValTok{0}\OperatorTok{;}
\OperatorTok{\}}
\end{Highlighting}
\end{Shaded}

\subsubsection{Passing Function Pointers to Other
Functions}\label{passing-function-pointers-to-other-functions}

\begin{Shaded}
\begin{Highlighting}[]
\PreprocessorTok{\#include }\ImportTok{\textless{}stdio.h\textgreater{}}

\DataTypeTok{int}\NormalTok{ add}\OperatorTok{(}\DataTypeTok{int}\NormalTok{ a}\OperatorTok{,} \DataTypeTok{int}\NormalTok{ b}\OperatorTok{)} \OperatorTok{\{} \ControlFlowTok{return}\NormalTok{ a }\OperatorTok{+}\NormalTok{ b}\OperatorTok{;} \OperatorTok{\}}
\DataTypeTok{int}\NormalTok{ mul}\OperatorTok{(}\DataTypeTok{int}\NormalTok{ a}\OperatorTok{,} \DataTypeTok{int}\NormalTok{ b}\OperatorTok{)} \OperatorTok{\{} \ControlFlowTok{return}\NormalTok{ a }\OperatorTok{{-}}\NormalTok{ b}\OperatorTok{;} \OperatorTok{\}}

\DataTypeTok{void}\NormalTok{ compute}\OperatorTok{(}\DataTypeTok{int} \OperatorTok{({-}}\NormalTok{f}\OperatorTok{)(}\DataTypeTok{int}\OperatorTok{,}\DataTypeTok{int}\OperatorTok{),} \DataTypeTok{int}\NormalTok{ x}\OperatorTok{,} \DataTypeTok{int}\NormalTok{ y}\OperatorTok{)} \OperatorTok{\{}
\NormalTok{    printf}\OperatorTok{(}\StringTok{"Result = }\SpecialCharTok{\%d\textbackslash{}n}\StringTok{"}\OperatorTok{,}\NormalTok{ f}\OperatorTok{(}\NormalTok{x}\OperatorTok{,}\NormalTok{y}\OperatorTok{));}
\OperatorTok{\}}

\DataTypeTok{int}\NormalTok{ main}\OperatorTok{(}\DataTypeTok{void}\OperatorTok{)} \OperatorTok{\{}
\NormalTok{    compute}\OperatorTok{(}\NormalTok{add}\OperatorTok{,} \DecValTok{3}\OperatorTok{,} \DecValTok{4}\OperatorTok{);}   \CommentTok{// pass add}
\NormalTok{    compute}\OperatorTok{(}\NormalTok{mul}\OperatorTok{,} \DecValTok{3}\OperatorTok{,} \DecValTok{4}\OperatorTok{);}   \CommentTok{// pass mul}
    \ControlFlowTok{return} \DecValTok{0}\OperatorTok{;}
\OperatorTok{\}}
\end{Highlighting}
\end{Shaded}

Output:

\begin{Shaded}
\begin{Highlighting}[]
\ExtensionTok{Result}\NormalTok{ = 7}
\ExtensionTok{Result}\NormalTok{ = 12}
\end{Highlighting}
\end{Shaded}

\subsubsection{\texorpdfstring{Real-World Example:
\texttt{qsort}}{Real-World Example: qsort}}\label{real-world-example-qsort}

The C standard library function \texttt{qsort} uses a function pointer
for custom comparison:

\begin{Shaded}
\begin{Highlighting}[]
\PreprocessorTok{\#include }\ImportTok{\textless{}stdio.h\textgreater{}}
\PreprocessorTok{\#include }\ImportTok{\textless{}stdlib.h\textgreater{}}

\DataTypeTok{int}\NormalTok{ cmp\_int}\OperatorTok{(}\DataTypeTok{const} \DataTypeTok{void} \OperatorTok{*}\NormalTok{a}\OperatorTok{,} \DataTypeTok{const} \DataTypeTok{void} \OperatorTok{*}\NormalTok{b}\OperatorTok{)} \OperatorTok{\{}
    \ControlFlowTok{return} \OperatorTok{(*(}\DataTypeTok{int}\OperatorTok{*)}\NormalTok{a }\OperatorTok{{-}} \OperatorTok{*(}\DataTypeTok{int}\OperatorTok{*)}\NormalTok{b}\OperatorTok{);}
\OperatorTok{\}}

\DataTypeTok{int}\NormalTok{ main}\OperatorTok{(}\DataTypeTok{void}\OperatorTok{)} \OperatorTok{\{}
    \DataTypeTok{int}\NormalTok{ arr}\OperatorTok{[}\DecValTok{5}\OperatorTok{]} \OperatorTok{=} \OperatorTok{\{}\DecValTok{4}\OperatorTok{,} \DecValTok{2}\OperatorTok{,} \DecValTok{5}\OperatorTok{,} \DecValTok{1}\OperatorTok{,} \DecValTok{3}\OperatorTok{\};}
\NormalTok{    qsort}\OperatorTok{(}\NormalTok{arr}\OperatorTok{,} \DecValTok{5}\OperatorTok{,} \KeywordTok{sizeof}\OperatorTok{(}\DataTypeTok{int}\OperatorTok{),}\NormalTok{ cmp\_int}\OperatorTok{);}

    \ControlFlowTok{for} \OperatorTok{(}\DataTypeTok{int}\NormalTok{ i }\OperatorTok{=} \DecValTok{0}\OperatorTok{;}\NormalTok{ i }\OperatorTok{\textless{}} \DecValTok{5}\OperatorTok{;}\NormalTok{ i}\OperatorTok{++)}\NormalTok{ printf}\OperatorTok{(}\StringTok{"}\SpecialCharTok{\%d}\StringTok{ "}\OperatorTok{,}\NormalTok{ arr}\OperatorTok{[}\NormalTok{i}\OperatorTok{]);}
\NormalTok{    printf}\OperatorTok{(}\StringTok{"}\SpecialCharTok{\textbackslash{}n}\StringTok{"}\OperatorTok{);}
    \ControlFlowTok{return} \DecValTok{0}\OperatorTok{;}
\OperatorTok{\}}
\end{Highlighting}
\end{Shaded}

Output:

\begin{Shaded}
\begin{Highlighting}[]
\ExtensionTok{1}\NormalTok{ 2 3 4 5}
\end{Highlighting}
\end{Shaded}

\subsubsection{Why It Matters}\label{why-it-matters-32}

\begin{itemize}
\tightlist
\item
  Function pointers enable flexibility - functions become data.
\item
  They are widely used in callbacks, event handlers, sorting, GUIs, and
  system programming.
\item
  Understanding them opens the door to advanced C patterns.
\end{itemize}

\subsubsection{Exercises}\label{exercises-33}

\begin{enumerate}
\def\labelenumi{\arabic{enumi}.}
\tightlist
\item
  Write two functions: \texttt{square(int)} and \texttt{cube(int)}. Use
  a function pointer to call each.
\item
  Write a function \texttt{apply(int\ (-f)(int),\ int\ x)} that applies
  \texttt{f} to \texttt{x} and prints the result.
\item
  Write an array of function pointers to basic operations (\texttt{add},
  \texttt{sub}, \texttt{mul}, \texttt{div\_int}) and call each in a
  loop.
\item
  Implement a program that takes two numbers and an operator
  (\texttt{+}, \texttt{-}, \texttt{-}, \texttt{/}), then uses function
  pointers to choose the correct operation.
\item
  Use \texttt{qsort} to sort an array of strings alphabetically. Write a
  comparison function for strings.
\end{enumerate}

\subsection{7.5 Safe Pointers in Modern
C}\label{safe-pointers-in-modern-c}

Pointers are powerful, but also dangerous. Misusing them can lead to
bugs, crashes, or security issues. Modern C (C11--C23) encourages safe
pointer practices to reduce risks.

\subsubsection{Null Pointers}\label{null-pointers}

A pointer that doesn't point anywhere should be set to a null pointer.

\begin{Shaded}
\begin{Highlighting}[]
\PreprocessorTok{\#include }\ImportTok{\textless{}stdio.h\textgreater{}}

\DataTypeTok{int}\NormalTok{ main}\OperatorTok{(}\DataTypeTok{void}\OperatorTok{)} \OperatorTok{\{}
    \DataTypeTok{int} \OperatorTok{{-}}\NormalTok{p }\OperatorTok{=}\NormalTok{ NULL}\OperatorTok{;}   \CommentTok{// points nowhere}
    \ControlFlowTok{if} \OperatorTok{(}\NormalTok{p }\OperatorTok{==}\NormalTok{ NULL}\OperatorTok{)} \OperatorTok{\{}
\NormalTok{        printf}\OperatorTok{(}\StringTok{"Pointer is null}\SpecialCharTok{\textbackslash{}n}\StringTok{"}\OperatorTok{);}
    \OperatorTok{\}}
    \ControlFlowTok{return} \DecValTok{0}\OperatorTok{;}
\OperatorTok{\}}
\end{Highlighting}
\end{Shaded}

\paragraph{\texorpdfstring{\texttt{NULL} vs \texttt{nullptr} in
C23}{NULL vs nullptr in C23}}\label{null-vs-nullptr-in-c23}

\begin{itemize}
\tightlist
\item
  In older C, \texttt{NULL} is used (defined in
  \texttt{\textless{}stddef.h\textgreater{}}).
\item
  C23 introduces \texttt{nullptr} (similar to C++), making null checks
  safer and clearer:
\end{itemize}

\begin{Shaded}
\begin{Highlighting}[]
\DataTypeTok{int} \OperatorTok{{-}}\NormalTok{p }\OperatorTok{=} \KeywordTok{nullptr}\OperatorTok{;}
\ControlFlowTok{if} \OperatorTok{(}\NormalTok{p }\OperatorTok{==} \KeywordTok{nullptr}\OperatorTok{)} \OperatorTok{\{} \OperatorTok{/{-}}\NormalTok{ safe }\OperatorTok{{-}/} \OperatorTok{\}}
\end{Highlighting}
\end{Shaded}

\subsubsection{Dangling Pointers}\label{dangling-pointers}

A pointer becomes dangling if the variable it points to goes out of
scope.

\begin{Shaded}
\begin{Highlighting}[]
\DataTypeTok{int} \OperatorTok{{-}}\NormalTok{bad\_pointer}\OperatorTok{(}\DataTypeTok{void}\OperatorTok{)} \OperatorTok{\{}
    \DataTypeTok{int}\NormalTok{ x }\OperatorTok{=} \DecValTok{10}\OperatorTok{;}
    \ControlFlowTok{return} \OperatorTok{\&}\NormalTok{x}\OperatorTok{;}   \CommentTok{// ❌ ERROR: x no longer exists after function ends}
\OperatorTok{\}}
\end{Highlighting}
\end{Shaded}

Rule: never return a pointer to a local variable.

\subsubsection{Wild Pointers}\label{wild-pointers}

A pointer that is uninitialized may point to random memory:

\begin{Shaded}
\begin{Highlighting}[]
\DataTypeTok{int} \OperatorTok{{-}}\NormalTok{p}\OperatorTok{;}  \CommentTok{// ❌ uninitialized}
\OperatorTok{{-}}\NormalTok{p }\OperatorTok{=} \DecValTok{42}\OperatorTok{;} \CommentTok{// undefined behavior}
\end{Highlighting}
\end{Shaded}

Always initialize pointers: either to a valid address or
\texttt{NULL}/\texttt{nullptr}.

\subsubsection{Double Free}\label{double-free}

If you \texttt{free} the same memory twice, the program may crash.

\begin{Shaded}
\begin{Highlighting}[]
\DataTypeTok{int} \OperatorTok{{-}}\NormalTok{p }\OperatorTok{=}\NormalTok{ malloc}\OperatorTok{(}\KeywordTok{sizeof}\OperatorTok{(}\DataTypeTok{int}\OperatorTok{));}
\NormalTok{free}\OperatorTok{(}\NormalTok{p}\OperatorTok{);}
\NormalTok{free}\OperatorTok{(}\NormalTok{p}\OperatorTok{);}   \CommentTok{// ❌ undefined behavior}
\end{Highlighting}
\end{Shaded}

Solution: after freeing, set \texttt{p\ =\ NULL} (or \texttt{nullptr}).

\subsubsection{Pointer Bounds}\label{pointer-bounds}

Pointers don't carry length information. Accessing outside an array is
undefined:

\begin{Shaded}
\begin{Highlighting}[]
\DataTypeTok{int}\NormalTok{ arr}\OperatorTok{[}\DecValTok{3}\OperatorTok{]} \OperatorTok{=} \OperatorTok{\{}\DecValTok{1}\OperatorTok{,}\DecValTok{2}\OperatorTok{,}\DecValTok{3}\OperatorTok{\};}
\DataTypeTok{int} \OperatorTok{{-}}\NormalTok{p }\OperatorTok{=}\NormalTok{ arr}\OperatorTok{;}
\NormalTok{printf}\OperatorTok{(}\StringTok{"}\SpecialCharTok{\%d\textbackslash{}n}\StringTok{"}\OperatorTok{,} \OperatorTok{{-}(}\NormalTok{p}\OperatorTok{+}\DecValTok{3}\OperatorTok{));} \CommentTok{// ❌ out of bounds}
\end{Highlighting}
\end{Shaded}

Always check bounds when iterating.

\subsubsection{Safer Practices}\label{safer-practices}

\begin{enumerate}
\def\labelenumi{\arabic{enumi}.}
\item
  Initialize pointers - set to \texttt{nullptr} or valid address.
\item
  Check before use - don't dereference a null pointer.
\item
  Don't return locals - never return a pointer to a local variable.
\item
  Set freed pointers to null - avoid double free.
\item
  Use \texttt{const} pointers if data should not change:

\begin{Shaded}
\begin{Highlighting}[]
\DataTypeTok{const} \DataTypeTok{char} \OperatorTok{*}\NormalTok{msg }\OperatorTok{=} \StringTok{"Hello"}\OperatorTok{;} \CommentTok{// prevents accidental modification}
\end{Highlighting}
\end{Shaded}
\end{enumerate}

\subsubsection{A Full Example}\label{a-full-example-13}

\begin{Shaded}
\begin{Highlighting}[]
\PreprocessorTok{\#include }\ImportTok{\textless{}stdio.h\textgreater{}}
\PreprocessorTok{\#include }\ImportTok{\textless{}stdlib.h\textgreater{}}

\DataTypeTok{int}\NormalTok{ main}\OperatorTok{(}\DataTypeTok{void}\OperatorTok{)} \OperatorTok{\{}
    \DataTypeTok{int} \OperatorTok{*}\NormalTok{p }\OperatorTok{=} \KeywordTok{nullptr}\OperatorTok{;}       \CommentTok{// safe initialization}
\NormalTok{    p }\OperatorTok{=}\NormalTok{ malloc}\OperatorTok{(}\KeywordTok{sizeof}\OperatorTok{(}\DataTypeTok{int}\OperatorTok{));}
    \ControlFlowTok{if} \OperatorTok{(}\NormalTok{p }\OperatorTok{==} \KeywordTok{nullptr}\OperatorTok{)} \OperatorTok{\{}     \CommentTok{// check allocation}
\NormalTok{        printf}\OperatorTok{(}\StringTok{"Memory allocation failed}\SpecialCharTok{\textbackslash{}n}\StringTok{"}\OperatorTok{);}
        \ControlFlowTok{return} \DecValTok{1}\OperatorTok{;}
    \OperatorTok{\}}

    \OperatorTok{*}\NormalTok{p }\OperatorTok{=} \DecValTok{42}\OperatorTok{;}
\NormalTok{    printf}\OperatorTok{(}\StringTok{"Value = }\SpecialCharTok{\%d\textbackslash{}n}\StringTok{"}\OperatorTok{,} \OperatorTok{*}\NormalTok{p}\OperatorTok{);}

\NormalTok{    free}\OperatorTok{(}\NormalTok{p}\OperatorTok{);}
\NormalTok{    p }\OperatorTok{=} \KeywordTok{nullptr}\OperatorTok{;}            \CommentTok{// avoid dangling pointer}

    \ControlFlowTok{return} \DecValTok{0}\OperatorTok{;}
\OperatorTok{\}}
\end{Highlighting}
\end{Shaded}

\subsubsection{Why It Matters}\label{why-it-matters-33}

\begin{itemize}
\tightlist
\item
  Pointers are the sharpest tool in C: essential but risky.
\item
  Safe usage prevents segmentation faults, memory leaks, and security
  vulnerabilities.
\item
  C23's \texttt{nullptr} makes code clearer and harder to misuse.
\end{itemize}

\subsubsection{Exercises}\label{exercises-34}

\begin{enumerate}
\def\labelenumi{\arabic{enumi}.}
\tightlist
\item
  Declare an uninitialized pointer, then fix it by setting it to
  \texttt{nullptr}. Print a check to confirm.
\item
  Write a function that safely allocates an integer, sets it to 10,
  prints it, and frees it.
\item
  Demonstrate a dangling pointer bug by returning the address of a local
  variable. Then fix it.
\item
  Allocate an array of 5 integers with \texttt{malloc}, set them to
  1--5, print them, then free.
\item
  Experiment: free a pointer twice without resetting it (observe
  crash/UB). Then fix it by setting to \texttt{nullptr}.
\end{enumerate}

\subsection{Problems}\label{problems-5}

\subsubsection{1. Basic Pointer Access}\label{basic-pointer-access}

Declare \texttt{int\ x\ =\ 10} and a pointer \texttt{p} to it. Print
\texttt{x}, the address of \texttt{x}, the value of \texttt{p}, and
\texttt{-p}.

\subsubsection{2. Modify Through a
Pointer}\label{modify-through-a-pointer}

Start with \texttt{int\ x\ =\ 5}. Use a pointer to change its value to
42. Print both \texttt{x} and \texttt{-p}.

\subsubsection{3. Two Pointers, One
Variable}\label{two-pointers-one-variable}

Make two pointers point to the same integer. Change the integer through
the first pointer, and print it through the second.

\subsubsection{4. Pointer Reassignment}\label{pointer-reassignment}

Declare two integers \texttt{a=10,\ b=20} and one pointer \texttt{p}.
First point \texttt{p} to \texttt{a}, then to \texttt{b}, printing
values each time.

\subsubsection{5. Array with Pointer
Arithmetic}\label{array-with-pointer-arithmetic}

Declare \texttt{int\ arr{[}5{]}\ =\ \{1,2,3,4,5\}}. Use pointer
arithmetic (\texttt{-(arr+i)}) to print all elements.

\subsubsection{6. Sum with Pointer
Traversal}\label{sum-with-pointer-traversal}

Read 5 integers into an array. Use a pointer to calculate their sum.

\subsubsection{7. Print String via
Pointer}\label{print-string-via-pointer}

Declare \texttt{char\ word{[}{]}\ =\ "Pointers"}. Use a pointer to walk
through the string character by character until
\texttt{\textquotesingle{}\textbackslash{}0\textquotesingle{}}.

\subsubsection{8. Compare Array vs Pointer
Sizes}\label{compare-array-vs-pointer-sizes}

Declare \texttt{int\ arr{[}10{]}} and \texttt{int\ -p\ =\ arr}. Print
\texttt{sizeof(arr)} and \texttt{sizeof(p)} to see the difference.

\subsubsection{9. Function Pointer
Basics}\label{function-pointer-basics}

Write two functions: \texttt{int\ square(int)} and
\texttt{int\ cube(int)}. Declare a function pointer and use it to call
each function.

\subsubsection{10. Function Pointer as
Argument}\label{function-pointer-as-argument}

Write a function \texttt{apply(int\ (-f)(int),\ int\ x)} that applies
\texttt{f} to \texttt{x} and prints the result. Test with both
\texttt{square} and \texttt{cube}.

\subsubsection{11. Array of Function
Pointers}\label{array-of-function-pointers}

Create an array of function pointers to four functions: \texttt{add},
\texttt{sub}, \texttt{mul}, \texttt{div\_int}. Call each in a loop with
two numbers.

\subsubsection{12. Operator with Function
Pointer}\label{operator-with-function-pointer}

Write a calculator program: read two integers and an operator
(\texttt{+\ -\ -\ /}), then use function pointers to call the correct
operation.

\subsubsection{13. Safe Null Pointer}\label{safe-null-pointer}

Declare an \texttt{int\ -p\ =\ nullptr} (or \texttt{NULL}). Print a
message if the pointer is null.

\subsubsection{14. Dangling Pointer Demo}\label{dangling-pointer-demo}

Write a function that returns the address of a local variable (dangling
pointer). Call it and print the result to observe the bug. Then fix it
by using \texttt{malloc}.

\subsubsection{15. Wild Pointer Example}\label{wild-pointer-example}

Declare an uninitialized pointer and try dereferencing it (expect
crash/UB). Then fix it by proper initialization.

\subsubsection{16. Double Free Experiment}\label{double-free-experiment}

Allocate an integer with \texttt{malloc}, free it twice (expect UB).
Then fix it by setting the pointer to \texttt{nullptr} after
\texttt{free}.

\subsubsection{17. Safe Dynamic Array}\label{safe-dynamic-array}

Allocate an array of 5 integers with \texttt{malloc}, fill with values
1--5, print them, then \texttt{free} safely.

\subsubsection{18. Struct with Pointer}\label{struct-with-pointer}

Define a \texttt{struct\ Point\ \{\ int\ x,y;\ \}}. Create one instance
and a pointer to it. Use the pointer with \texttt{-\textgreater{}} to
print and modify values.

\subsubsection{19. Passing Pointers to
Functions}\label{passing-pointers-to-functions}

Write a function \texttt{void\ swap(int\ *a,\ int\ -b)} that swaps two
integers using pointers. Demonstrate it in \texttt{main}.

\subsubsection{\texorpdfstring{20. Function Pointer with
\texttt{qsort}}{20. Function Pointer with qsort}}\label{function-pointer-with-qsort}

Use \texttt{qsort} to sort an array of integers. Write a comparison
function and pass it as a function pointer.

\section{Chapter 8. Structs and
Enums}\label{chapter-8.-structs-and-enums}

Here's Chapter 8.1: Grouping Data with \texttt{struct}, the first step
into C's way of building custom data types.

\subsection{\texorpdfstring{8.1 Grouping Data with
\texttt{struct}}{8.1 Grouping Data with struct}}\label{grouping-data-with-struct}

So far, we've worked with simple types: \texttt{int}, \texttt{double},
\texttt{char}, arrays, and pointers. But real-world programs often deal
with things that have multiple properties.

Example:

\begin{itemize}
\tightlist
\item
  A point has both an -x- and -y- coordinate.
\item
  A student has a \emph{name-, }age-, and *grade-.
\item
  A book has a -title-, *author-, and -year-.
\end{itemize}

In C, we can bundle related values into one structure using the keyword
\texttt{struct}.

\subsubsection{\texorpdfstring{Declaring a
\texttt{struct}}{Declaring a struct}}\label{declaring-a-struct}

\begin{Shaded}
\begin{Highlighting}[]
\KeywordTok{struct}\NormalTok{ Point }\OperatorTok{\{}
    \DataTypeTok{int}\NormalTok{ x}\OperatorTok{;}
    \DataTypeTok{int}\NormalTok{ y}\OperatorTok{;}
\OperatorTok{\};}
\end{Highlighting}
\end{Shaded}

This defines a new type \texttt{struct\ Point} with two fields:
\texttt{x} and \texttt{y}.

\subsubsection{Creating Variables}\label{creating-variables}

\begin{Shaded}
\begin{Highlighting}[]
\KeywordTok{struct}\NormalTok{ Point p1}\OperatorTok{;}
\NormalTok{p1}\OperatorTok{.}\NormalTok{x }\OperatorTok{=} \DecValTok{3}\OperatorTok{;}
\NormalTok{p1}\OperatorTok{.}\NormalTok{y }\OperatorTok{=} \DecValTok{4}\OperatorTok{;}
\end{Highlighting}
\end{Shaded}

You can also initialize at once:

\begin{Shaded}
\begin{Highlighting}[]
\KeywordTok{struct}\NormalTok{ Point p2 }\OperatorTok{=} \OperatorTok{\{}\DecValTok{10}\OperatorTok{,} \DecValTok{20}\OperatorTok{\};}
\end{Highlighting}
\end{Shaded}

\subsubsection{Accessing Fields}\label{accessing-fields}

Use the dot operator (\texttt{.}):

\begin{Shaded}
\begin{Highlighting}[]
\PreprocessorTok{\#include }\ImportTok{\textless{}stdio.h\textgreater{}}

\KeywordTok{struct}\NormalTok{ Point }\OperatorTok{\{}
    \DataTypeTok{int}\NormalTok{ x}\OperatorTok{;}
    \DataTypeTok{int}\NormalTok{ y}\OperatorTok{;}
\OperatorTok{\};}

\DataTypeTok{int}\NormalTok{ main}\OperatorTok{(}\DataTypeTok{void}\OperatorTok{)} \OperatorTok{\{}
    \KeywordTok{struct}\NormalTok{ Point p }\OperatorTok{=} \OperatorTok{\{}\DecValTok{5}\OperatorTok{,} \DecValTok{7}\OperatorTok{\};}
\NormalTok{    printf}\OperatorTok{(}\StringTok{"x=}\SpecialCharTok{\%d}\StringTok{ y=}\SpecialCharTok{\%d\textbackslash{}n}\StringTok{"}\OperatorTok{,}\NormalTok{ p}\OperatorTok{.}\NormalTok{x}\OperatorTok{,}\NormalTok{ p}\OperatorTok{.}\NormalTok{y}\OperatorTok{);}
    \ControlFlowTok{return} \DecValTok{0}\OperatorTok{;}
\OperatorTok{\}}
\end{Highlighting}
\end{Shaded}

Output:

\begin{Shaded}
\begin{Highlighting}[]
\VariableTok{x}\OperatorTok{=}\NormalTok{5 }\VariableTok{y}\OperatorTok{=}\NormalTok{7}
\end{Highlighting}
\end{Shaded}

\subsubsection{Example: Student Record}\label{example-student-record}

\begin{Shaded}
\begin{Highlighting}[]
\PreprocessorTok{\#include }\ImportTok{\textless{}stdio.h\textgreater{}}

\KeywordTok{struct}\NormalTok{ Student }\OperatorTok{\{}
    \DataTypeTok{char}\NormalTok{ name}\OperatorTok{[}\DecValTok{50}\OperatorTok{];}
    \DataTypeTok{int}\NormalTok{ age}\OperatorTok{;}
    \DataTypeTok{double}\NormalTok{ grade}\OperatorTok{;}
\OperatorTok{\};}

\DataTypeTok{int}\NormalTok{ main}\OperatorTok{(}\DataTypeTok{void}\OperatorTok{)} \OperatorTok{\{}
    \KeywordTok{struct}\NormalTok{ Student s }\OperatorTok{=} \OperatorTok{\{}\StringTok{"Alice"}\OperatorTok{,} \DecValTok{20}\OperatorTok{,} \FloatTok{3.8}\OperatorTok{\};}
\NormalTok{    printf}\OperatorTok{(}\StringTok{"Name: }\SpecialCharTok{\%s\textbackslash{}n}\StringTok{"}\OperatorTok{,}\NormalTok{ s}\OperatorTok{.}\NormalTok{name}\OperatorTok{);}
\NormalTok{    printf}\OperatorTok{(}\StringTok{"Age: }\SpecialCharTok{\%d\textbackslash{}n}\StringTok{"}\OperatorTok{,}\NormalTok{ s}\OperatorTok{.}\NormalTok{age}\OperatorTok{);}
\NormalTok{    printf}\OperatorTok{(}\StringTok{"Grade: }\SpecialCharTok{\%.2f\textbackslash{}n}\StringTok{"}\OperatorTok{,}\NormalTok{ s}\OperatorTok{.}\NormalTok{grade}\OperatorTok{);}
    \ControlFlowTok{return} \DecValTok{0}\OperatorTok{;}
\OperatorTok{\}}
\end{Highlighting}
\end{Shaded}

Output:

\begin{Shaded}
\begin{Highlighting}[]
\ExtensionTok{Name:}\NormalTok{ Alice}
\ExtensionTok{Age:}\NormalTok{ 20}
\ExtensionTok{Grade:}\NormalTok{ 3.80}
\end{Highlighting}
\end{Shaded}

\subsubsection{Arrays of Structs}\label{arrays-of-structs}

You can create an array of structs, just like an array of ints:

\begin{Shaded}
\begin{Highlighting}[]
\KeywordTok{struct}\NormalTok{ Point points}\OperatorTok{[}\DecValTok{3}\OperatorTok{]} \OperatorTok{=} \OperatorTok{\{}
    \OperatorTok{\{}\DecValTok{1}\OperatorTok{,} \DecValTok{2}\OperatorTok{\},}
    \OperatorTok{\{}\DecValTok{3}\OperatorTok{,} \DecValTok{4}\OperatorTok{\},}
    \OperatorTok{\{}\DecValTok{5}\OperatorTok{,} \DecValTok{6}\OperatorTok{\}}
\OperatorTok{\};}
\end{Highlighting}
\end{Shaded}

Loop through them:

\begin{Shaded}
\begin{Highlighting}[]
\ControlFlowTok{for} \OperatorTok{(}\DataTypeTok{int}\NormalTok{ i }\OperatorTok{=} \DecValTok{0}\OperatorTok{;}\NormalTok{ i }\OperatorTok{\textless{}} \DecValTok{3}\OperatorTok{;}\NormalTok{ i}\OperatorTok{++)} \OperatorTok{\{}
\NormalTok{    printf}\OperatorTok{(}\StringTok{"(}\SpecialCharTok{\%d}\StringTok{,}\SpecialCharTok{\%d}\StringTok{)}\SpecialCharTok{\textbackslash{}n}\StringTok{"}\OperatorTok{,}\NormalTok{ points}\OperatorTok{[}\NormalTok{i}\OperatorTok{].}\NormalTok{x}\OperatorTok{,}\NormalTok{ points}\OperatorTok{[}\NormalTok{i}\OperatorTok{].}\NormalTok{y}\OperatorTok{);}
\OperatorTok{\}}
\end{Highlighting}
\end{Shaded}

\subsubsection{Why It Matters}\label{why-it-matters-34}

\begin{itemize}
\tightlist
\item
  Structures let you represent real-world entities in code.
\item
  They keep related values together, making programs more organized and
  readable.
\item
  Almost every serious C program uses \texttt{struct} for data modeling.
\end{itemize}

\subsubsection{Exercises}\label{exercises-35}

\begin{enumerate}
\def\labelenumi{\arabic{enumi}.}
\tightlist
\item
  Define a \texttt{struct\ Point} with fields \texttt{x} and \texttt{y}.
  Create one, set values, and print them.
\item
  Define a \texttt{struct\ Student} with \texttt{name}, \texttt{age},
  and \texttt{gpa}. Initialize and print it.
\item
  Create an array of 3 \texttt{struct\ Point}s and print their
  coordinates.
\item
  Write a program that reads a student's name, age, and grade into a
  \texttt{struct} and prints them back.
\item
  Define a \texttt{struct\ Rectangle} with \texttt{width} and
  \texttt{height}. Write a function \texttt{area} that takes a rectangle
  and returns its area.
\end{enumerate}

\subsection{\texorpdfstring{8.2 Using
\texttt{typedef}}{8.2 Using typedef}}\label{using-typedef}

When you declare a \texttt{struct}, you normally have to write the
keyword \texttt{struct} every time:

\begin{Shaded}
\begin{Highlighting}[]
\KeywordTok{struct}\NormalTok{ Point }\OperatorTok{\{}
    \DataTypeTok{int}\NormalTok{ x}\OperatorTok{;}
    \DataTypeTok{int}\NormalTok{ y}\OperatorTok{;}
\OperatorTok{\};}

\KeywordTok{struct}\NormalTok{ Point p1 }\OperatorTok{=} \OperatorTok{\{}\DecValTok{3}\OperatorTok{,} \DecValTok{4}\OperatorTok{\};}
\end{Highlighting}
\end{Shaded}

This can get repetitive, especially in large programs. C provides
\texttt{typedef} to create a shorter alias for a type.

\subsubsection{\texorpdfstring{Basic
\texttt{typedef}}{Basic typedef}}\label{basic-typedef}

\begin{Shaded}
\begin{Highlighting}[]
\KeywordTok{typedef} \KeywordTok{struct}\NormalTok{ Point }\OperatorTok{\{}
    \DataTypeTok{int}\NormalTok{ x}\OperatorTok{;}
    \DataTypeTok{int}\NormalTok{ y}\OperatorTok{;}
\OperatorTok{\}}\NormalTok{ Point}\OperatorTok{;}
\end{Highlighting}
\end{Shaded}

Now you can write:

\begin{Shaded}
\begin{Highlighting}[]
\NormalTok{Point p1 }\OperatorTok{=} \OperatorTok{\{}\DecValTok{3}\OperatorTok{,} \DecValTok{4}\OperatorTok{\};}
\end{Highlighting}
\end{Shaded}

instead of \texttt{struct\ Point\ p1}.

\subsubsection{Example: Cleaner Code}\label{example-cleaner-code}

\begin{Shaded}
\begin{Highlighting}[]
\PreprocessorTok{\#include }\ImportTok{\textless{}stdio.h\textgreater{}}

\KeywordTok{typedef} \KeywordTok{struct} \OperatorTok{\{}
    \DataTypeTok{char}\NormalTok{ name}\OperatorTok{[}\DecValTok{50}\OperatorTok{];}
    \DataTypeTok{int}\NormalTok{ age}\OperatorTok{;}
    \DataTypeTok{double}\NormalTok{ gpa}\OperatorTok{;}
\OperatorTok{\}}\NormalTok{ Student}\OperatorTok{;}

\DataTypeTok{int}\NormalTok{ main}\OperatorTok{(}\DataTypeTok{void}\OperatorTok{)} \OperatorTok{\{}
\NormalTok{    Student s }\OperatorTok{=} \OperatorTok{\{}\StringTok{"Alice"}\OperatorTok{,} \DecValTok{20}\OperatorTok{,} \FloatTok{3.8}\OperatorTok{\};}
\NormalTok{    printf}\OperatorTok{(}\StringTok{"}\SpecialCharTok{\%s}\StringTok{, }\SpecialCharTok{\%d}\StringTok{ years, GPA=}\SpecialCharTok{\%.2f\textbackslash{}n}\StringTok{"}\OperatorTok{,}\NormalTok{ s}\OperatorTok{.}\NormalTok{name}\OperatorTok{,}\NormalTok{ s}\OperatorTok{.}\NormalTok{age}\OperatorTok{,}\NormalTok{ s}\OperatorTok{.}\NormalTok{gpa}\OperatorTok{);}
    \ControlFlowTok{return} \DecValTok{0}\OperatorTok{;}
\OperatorTok{\}}
\end{Highlighting}
\end{Shaded}

Notice: no need to say \texttt{struct\ Student}, just \texttt{Student}.

\subsubsection{Typedef with Pointers}\label{typedef-with-pointers}

\texttt{typedef} is also useful for pointer types:

\begin{Shaded}
\begin{Highlighting}[]
\KeywordTok{typedef} \KeywordTok{struct}\NormalTok{ Point }\OperatorTok{\{}
    \DataTypeTok{int}\NormalTok{ x}\OperatorTok{,}\NormalTok{ y}\OperatorTok{;}
\OperatorTok{\}}\NormalTok{ Point}\OperatorTok{;}

\KeywordTok{typedef}\NormalTok{ Point}\OperatorTok{{-}}\NormalTok{ PointPtr}\OperatorTok{;}

\DataTypeTok{int}\NormalTok{ main}\OperatorTok{(}\DataTypeTok{void}\OperatorTok{)} \OperatorTok{\{}
\NormalTok{    Point a }\OperatorTok{=} \OperatorTok{\{}\DecValTok{1}\OperatorTok{,} \DecValTok{2}\OperatorTok{\};}
\NormalTok{    PointPtr p }\OperatorTok{=} \OperatorTok{\&}\NormalTok{a}\OperatorTok{;}
\NormalTok{    printf}\OperatorTok{(}\StringTok{"(}\SpecialCharTok{\%d}\StringTok{,}\SpecialCharTok{\%d}\StringTok{)}\SpecialCharTok{\textbackslash{}n}\StringTok{"}\OperatorTok{,}\NormalTok{ p}\OperatorTok{{-}\textgreater{}}\NormalTok{x}\OperatorTok{,}\NormalTok{ p}\OperatorTok{{-}\textgreater{}}\NormalTok{y}\OperatorTok{);}
    \ControlFlowTok{return} \DecValTok{0}\OperatorTok{;}
\OperatorTok{\}}
\end{Highlighting}
\end{Shaded}

\subsubsection{\texorpdfstring{Typedef
vs.~\texttt{struct}}{Typedef vs.~struct}}\label{typedef-vs.-struct}

\begin{itemize}
\item
  Without typedef:

\begin{Shaded}
\begin{Highlighting}[]
\KeywordTok{struct}\NormalTok{ Point a}\OperatorTok{;}
\end{Highlighting}
\end{Shaded}
\item
  With typedef:

\begin{Shaded}
\begin{Highlighting}[]
\NormalTok{Point a}\OperatorTok{;}
\end{Highlighting}
\end{Shaded}
\end{itemize}

Both create the same kind of variable. Typedef just saves typing.

\subsubsection{Why It Matters}\label{why-it-matters-35}

\begin{itemize}
\tightlist
\item
  Makes code shorter and cleaner.
\item
  Common in APIs and libraries to simplify complex type names.
\item
  Improves readability by giving meaningful names (e.g., \texttt{Point},
  \texttt{Student}, \texttt{Matrix}).
\end{itemize}

\subsubsection{Exercises}\label{exercises-36}

\begin{enumerate}
\def\labelenumi{\arabic{enumi}.}
\tightlist
\item
  Rewrite the \texttt{struct\ Point} example from 8.1 using
  \texttt{typedef}.
\item
  Define a \texttt{typedef} for
  \texttt{struct\ Rectangle\ \{\ int\ w,h;\ \}} as \texttt{Rectangle}.
  Write a function that takes a \texttt{Rectangle} and returns its area.
\item
  Define a \texttt{typedef} for a pointer to \texttt{Student}. Create a
  \texttt{Student} and print values using the pointer.
\item
  Create a \texttt{typedef} for \texttt{unsigned\ long\ long} called
  \texttt{BigInt}. Use it to compute factorial of 10.
\item
  Experiment: write a \texttt{typedef} for a function pointer
  \texttt{int\ (-Op)(int,int)} and use it for add/subtract functions.
\end{enumerate}

\subsection{8.3 Enumerations}\label{enumerations}

Sometimes you need a variable that can take one of a small set of
related values:

\begin{itemize}
\tightlist
\item
  Days of the week
\item
  Traffic light colors
\item
  Error codes
\end{itemize}

In C, you can define these with enumerations, using the keyword
\texttt{enum}.

\subsubsection{\texorpdfstring{Declaring an
\texttt{enum}}{Declaring an enum}}\label{declaring-an-enum}

\begin{Shaded}
\begin{Highlighting}[]
\KeywordTok{enum}\NormalTok{ Color }\OperatorTok{\{}
\NormalTok{    RED}\OperatorTok{,}
\NormalTok{    GREEN}\OperatorTok{,}
\NormalTok{    BLUE}
\OperatorTok{\};}
\end{Highlighting}
\end{Shaded}

This defines \texttt{RED\ =\ 0}, \texttt{GREEN\ =\ 1},
\texttt{BLUE\ =\ 2}.

\subsubsection{\texorpdfstring{Using an
\texttt{enum}}{Using an enum}}\label{using-an-enum}

\begin{Shaded}
\begin{Highlighting}[]
\PreprocessorTok{\#include }\ImportTok{\textless{}stdio.h\textgreater{}}

\KeywordTok{enum}\NormalTok{ Color }\OperatorTok{\{}\NormalTok{ RED}\OperatorTok{,}\NormalTok{ GREEN}\OperatorTok{,}\NormalTok{ BLUE }\OperatorTok{\};}

\DataTypeTok{int}\NormalTok{ main}\OperatorTok{(}\DataTypeTok{void}\OperatorTok{)} \OperatorTok{\{}
    \KeywordTok{enum}\NormalTok{ Color c }\OperatorTok{=}\NormalTok{ GREEN}\OperatorTok{;}
    \ControlFlowTok{if} \OperatorTok{(}\NormalTok{c }\OperatorTok{==}\NormalTok{ GREEN}\OperatorTok{)} \OperatorTok{\{}
\NormalTok{        printf}\OperatorTok{(}\StringTok{"Go!}\SpecialCharTok{\textbackslash{}n}\StringTok{"}\OperatorTok{);}
    \OperatorTok{\}}
    \ControlFlowTok{return} \DecValTok{0}\OperatorTok{;}
\OperatorTok{\}}
\end{Highlighting}
\end{Shaded}

Output:

\begin{verbatim}
Go!
\end{verbatim}

\subsubsection{Custom Values}\label{custom-values}

You can assign specific integer values:

\begin{Shaded}
\begin{Highlighting}[]
\KeywordTok{enum}\NormalTok{ Status }\OperatorTok{\{}
\NormalTok{    OK }\OperatorTok{=} \DecValTok{200}\OperatorTok{,}
\NormalTok{    NOT\_FOUND }\OperatorTok{=} \DecValTok{404}\OperatorTok{,}
\NormalTok{    SERVER\_ERROR }\OperatorTok{=} \DecValTok{500}
\OperatorTok{\};}
\end{Highlighting}
\end{Shaded}

Now \texttt{OK} = 200, \texttt{NOT\_FOUND} = 404, etc.

\subsubsection{Sequential Values}\label{sequential-values}

Unspecified values continue from the last one:

\begin{Shaded}
\begin{Highlighting}[]
\KeywordTok{enum}\NormalTok{ Weekday }\OperatorTok{\{}
\NormalTok{    MON }\OperatorTok{=} \DecValTok{1}\OperatorTok{,}  \CommentTok{// start from 1}
\NormalTok{    TUE}\OperatorTok{,}      \CommentTok{// 2}
\NormalTok{    WED}\OperatorTok{,}      \CommentTok{// 3}
\NormalTok{    THU}\OperatorTok{,}\NormalTok{ FRI}\OperatorTok{,}\NormalTok{ SAT}\OperatorTok{,}\NormalTok{ SUN}
\OperatorTok{\};}
\end{Highlighting}
\end{Shaded}

\subsubsection{\texorpdfstring{\texttt{typedef} with
\texttt{enum}}{typedef with enum}}\label{typedef-with-enum}

You can combine \texttt{typedef} with \texttt{enum} to simplify usage:

\begin{Shaded}
\begin{Highlighting}[]
\KeywordTok{typedef} \KeywordTok{enum} \OperatorTok{\{}
\NormalTok{    RED}\OperatorTok{,}\NormalTok{ GREEN}\OperatorTok{,}\NormalTok{ BLUE}
\OperatorTok{\}}\NormalTok{ Color}\OperatorTok{;}

\DataTypeTok{int}\NormalTok{ main}\OperatorTok{(}\DataTypeTok{void}\OperatorTok{)} \OperatorTok{\{}
\NormalTok{    Color c }\OperatorTok{=}\NormalTok{ BLUE}\OperatorTok{;}
\NormalTok{    printf}\OperatorTok{(}\StringTok{"Color = }\SpecialCharTok{\%d\textbackslash{}n}\StringTok{"}\OperatorTok{,}\NormalTok{ c}\OperatorTok{);}
    \ControlFlowTok{return} \DecValTok{0}\OperatorTok{;}
\OperatorTok{\}}
\end{Highlighting}
\end{Shaded}

\subsubsection{Example: Traffic Light}\label{example-traffic-light}

\begin{Shaded}
\begin{Highlighting}[]
\PreprocessorTok{\#include }\ImportTok{\textless{}stdio.h\textgreater{}}

\KeywordTok{typedef} \KeywordTok{enum} \OperatorTok{\{}
\NormalTok{    RED}\OperatorTok{,}\NormalTok{ YELLOW}\OperatorTok{,}\NormalTok{ GREEN}
\OperatorTok{\}}\NormalTok{ TrafficLight}\OperatorTok{;}

\DataTypeTok{int}\NormalTok{ main}\OperatorTok{(}\DataTypeTok{void}\OperatorTok{)} \OperatorTok{\{}
\NormalTok{    TrafficLight light }\OperatorTok{=}\NormalTok{ RED}\OperatorTok{;}

    \ControlFlowTok{if} \OperatorTok{(}\NormalTok{light }\OperatorTok{==}\NormalTok{ RED}\OperatorTok{)} \OperatorTok{\{}
\NormalTok{        printf}\OperatorTok{(}\StringTok{"Stop}\SpecialCharTok{\textbackslash{}n}\StringTok{"}\OperatorTok{);}
    \OperatorTok{\}} \ControlFlowTok{else} \ControlFlowTok{if} \OperatorTok{(}\NormalTok{light }\OperatorTok{==}\NormalTok{ GREEN}\OperatorTok{)} \OperatorTok{\{}
\NormalTok{        printf}\OperatorTok{(}\StringTok{"Go}\SpecialCharTok{\textbackslash{}n}\StringTok{"}\OperatorTok{);}
    \OperatorTok{\}} \ControlFlowTok{else} \OperatorTok{\{}
\NormalTok{        printf}\OperatorTok{(}\StringTok{"Wait}\SpecialCharTok{\textbackslash{}n}\StringTok{"}\OperatorTok{);}
    \OperatorTok{\}}
    \ControlFlowTok{return} \DecValTok{0}\OperatorTok{;}
\OperatorTok{\}}
\end{Highlighting}
\end{Shaded}

\subsubsection{Why It Matters}\label{why-it-matters-36}

\begin{itemize}
\tightlist
\item
  Enums make code clearer and safer than using raw integers.
\item
  Useful for fixed sets of options.
\item
  Often combined with \texttt{switch} statements for clean logic.
\end{itemize}

\subsubsection{Exercises}\label{exercises-37}

\begin{enumerate}
\def\labelenumi{\arabic{enumi}.}
\tightlist
\item
  Define an \texttt{enum\ Weekday} with values for Monday--Sunday. Write
  a program that prints the number for Wednesday.
\item
  Define an
  \texttt{enum\ TrafficLight\ \{\ RED=1,\ YELLOW=2,\ GREEN=3\ \}}. Write
  a program that prints \texttt{"Stop"} for RED, \texttt{"Wait"} for
  YELLOW, \texttt{"Go"} for GREEN.
\item
  Use \texttt{typedef\ enum} to define \texttt{Color} with values \{RED,
  GREEN, BLUE\}. Declare a variable and print its integer value.
\item
  Create an \texttt{enum\ ErrorCode\ \{\ OK=0,\ FAIL=1,\ TIMEOUT=2\ \}}
  and use a \texttt{switch} to print an error message for each code.
\item
  Extend the \texttt{Weekday} enum so that MON=1, and write a loop to
  print all days with their numbers.
\end{enumerate}

\subsection{\texorpdfstring{8.4 \texttt{union} (and when to use
it)}{8.4 union (and when to use it)}}\label{union-and-when-to-use-it}

A \texttt{union} is like a \texttt{struct}, but instead of giving each
member its own storage, all members share the same memory location.

\begin{itemize}
\tightlist
\item
  At any moment, only one member holds a valid value.
\item
  The size of a union is the size of its largest member.
\end{itemize}

\subsubsection{Declaring a Union}\label{declaring-a-union}

\begin{Shaded}
\begin{Highlighting}[]
\KeywordTok{union}\NormalTok{ Number }\OperatorTok{\{}
    \DataTypeTok{int}\NormalTok{ i}\OperatorTok{;}
    \DataTypeTok{float}\NormalTok{ f}\OperatorTok{;}
\OperatorTok{\};}
\end{Highlighting}
\end{Shaded}

This defines a type \texttt{union\ Number} with two possible views of
the same memory.

\subsubsection{Example: One at a Time}\label{example-one-at-a-time}

\begin{Shaded}
\begin{Highlighting}[]
\PreprocessorTok{\#include }\ImportTok{\textless{}stdio.h\textgreater{}}

\KeywordTok{union}\NormalTok{ Number }\OperatorTok{\{}
    \DataTypeTok{int}\NormalTok{ i}\OperatorTok{;}
    \DataTypeTok{float}\NormalTok{ f}\OperatorTok{;}
\OperatorTok{\};}

\DataTypeTok{int}\NormalTok{ main}\OperatorTok{(}\DataTypeTok{void}\OperatorTok{)} \OperatorTok{\{}
    \KeywordTok{union}\NormalTok{ Number n}\OperatorTok{;}
\NormalTok{    n}\OperatorTok{.}\NormalTok{i }\OperatorTok{=} \DecValTok{42}\OperatorTok{;}
\NormalTok{    printf}\OperatorTok{(}\StringTok{"i = }\SpecialCharTok{\%d\textbackslash{}n}\StringTok{"}\OperatorTok{,}\NormalTok{ n}\OperatorTok{.}\NormalTok{i}\OperatorTok{);}

\NormalTok{    n}\OperatorTok{.}\NormalTok{f }\OperatorTok{=} \FloatTok{3.14}\BuiltInTok{f}\OperatorTok{;}          \CommentTok{// overwrites same memory}
\NormalTok{    printf}\OperatorTok{(}\StringTok{"f = }\SpecialCharTok{\%.2f\textbackslash{}n}\StringTok{"}\OperatorTok{,}\NormalTok{ n}\OperatorTok{.}\NormalTok{f}\OperatorTok{);}
\NormalTok{    printf}\OperatorTok{(}\StringTok{"i (garbled) = }\SpecialCharTok{\%d\textbackslash{}n}\StringTok{"}\OperatorTok{,}\NormalTok{ n}\OperatorTok{.}\NormalTok{i}\OperatorTok{);} \CommentTok{// old value no longer valid}
    \ControlFlowTok{return} \DecValTok{0}\OperatorTok{;}
\OperatorTok{\}}
\end{Highlighting}
\end{Shaded}

Output:

\begin{verbatim}
i = 42
f = 3.14
i (garbled) = some unpredictable value
\end{verbatim}

\subsubsection{Memory Layout}\label{memory-layout}

\begin{verbatim}
union Number:
+----------------+
| same memory    |  <-- interpreted as int or float
+----------------+
\end{verbatim}

\subsubsection{Practical Use: Variant
Data}\label{practical-use-variant-data}

Unions are useful when a value could be one of several types.

Example: storing a number that might be \texttt{int}, \texttt{float}, or
\texttt{double}.

\begin{Shaded}
\begin{Highlighting}[]
\PreprocessorTok{\#include }\ImportTok{\textless{}stdio.h\textgreater{}}

\KeywordTok{typedef} \KeywordTok{union} \OperatorTok{\{}
    \DataTypeTok{int}\NormalTok{ i}\OperatorTok{;}
    \DataTypeTok{float}\NormalTok{ f}\OperatorTok{;}
    \DataTypeTok{double}\NormalTok{ d}\OperatorTok{;}
\OperatorTok{\}}\NormalTok{ Value}\OperatorTok{;}

\DataTypeTok{int}\NormalTok{ main}\OperatorTok{(}\DataTypeTok{void}\OperatorTok{)} \OperatorTok{\{}
\NormalTok{    Value v}\OperatorTok{;}
\NormalTok{    v}\OperatorTok{.}\NormalTok{d }\OperatorTok{=} \FloatTok{12.34}\OperatorTok{;}
\NormalTok{    printf}\OperatorTok{(}\StringTok{"double = }\SpecialCharTok{\%.2f\textbackslash{}n}\StringTok{"}\OperatorTok{,}\NormalTok{ v}\OperatorTok{.}\NormalTok{d}\OperatorTok{);}
    \ControlFlowTok{return} \DecValTok{0}\OperatorTok{;}
\OperatorTok{\}}
\end{Highlighting}
\end{Shaded}

\subsubsection{\texorpdfstring{With \texttt{enum} +
\texttt{struct}}{With enum + struct}}\label{with-enum-struct}

To track which type is active, combine \texttt{union} with an
\texttt{enum}:

\begin{Shaded}
\begin{Highlighting}[]
\PreprocessorTok{\#include }\ImportTok{\textless{}stdio.h\textgreater{}}

\KeywordTok{typedef} \KeywordTok{enum} \OperatorTok{\{}\NormalTok{ INT}\OperatorTok{,}\NormalTok{ FLOAT }\OperatorTok{\}}\NormalTok{ Tag}\OperatorTok{;}

\KeywordTok{typedef} \KeywordTok{struct} \OperatorTok{\{}
\NormalTok{    Tag type}\OperatorTok{;}
    \KeywordTok{union} \OperatorTok{\{}
        \DataTypeTok{int}\NormalTok{ i}\OperatorTok{;}
        \DataTypeTok{float}\NormalTok{ f}\OperatorTok{;}
    \OperatorTok{\}}\NormalTok{ data}\OperatorTok{;}
\OperatorTok{\}}\NormalTok{ Variant}\OperatorTok{;}

\DataTypeTok{int}\NormalTok{ main}\OperatorTok{(}\DataTypeTok{void}\OperatorTok{)} \OperatorTok{\{}
\NormalTok{    Variant v}\OperatorTok{;}
\NormalTok{    v}\OperatorTok{.}\NormalTok{type }\OperatorTok{=}\NormalTok{ INT}\OperatorTok{;}
\NormalTok{    v}\OperatorTok{.}\NormalTok{data}\OperatorTok{.}\NormalTok{i }\OperatorTok{=} \DecValTok{42}\OperatorTok{;}

    \ControlFlowTok{if} \OperatorTok{(}\NormalTok{v}\OperatorTok{.}\NormalTok{type }\OperatorTok{==}\NormalTok{ INT}\OperatorTok{)} \OperatorTok{\{}
\NormalTok{        printf}\OperatorTok{(}\StringTok{"int=}\SpecialCharTok{\%d\textbackslash{}n}\StringTok{"}\OperatorTok{,}\NormalTok{ v}\OperatorTok{.}\NormalTok{data}\OperatorTok{.}\NormalTok{i}\OperatorTok{);}
    \OperatorTok{\}} \ControlFlowTok{else} \OperatorTok{\{}
\NormalTok{        printf}\OperatorTok{(}\StringTok{"float=}\SpecialCharTok{\%.2f\textbackslash{}n}\StringTok{"}\OperatorTok{,}\NormalTok{ v}\OperatorTok{.}\NormalTok{data}\OperatorTok{.}\NormalTok{f}\OperatorTok{);}
    \OperatorTok{\}}
    \ControlFlowTok{return} \DecValTok{0}\OperatorTok{;}
\OperatorTok{\}}
\end{Highlighting}
\end{Shaded}

\subsubsection{Why It Matters}\label{why-it-matters-37}

\begin{itemize}
\tightlist
\item
  \texttt{union} saves memory by reusing space.
\item
  It's the foundation of variant types and low-level data handling.
\item
  Useful when interfacing with hardware, binary files, or protocols
  where the same data may mean different things.
\end{itemize}

\subsubsection{Exercises}\label{exercises-38}

\begin{enumerate}
\def\labelenumi{\arabic{enumi}.}
\tightlist
\item
  Define a \texttt{union\ Data} with \texttt{int\ i} and
  \texttt{float\ f}. Assign and print each in turn.
\item
  Write a program that stores a value in a union as \texttt{int}, then
  as \texttt{float}, and prints both. Observe how the second overwrites
  the first.
\item
  Combine \texttt{enum} and \texttt{union} to represent a number that
  may be either \texttt{int} or \texttt{float}. Print it safely
  depending on the tag.
\item
  Create a union with \texttt{char\ c{[}4{]}} and \texttt{int\ n}.
  Assign to \texttt{n} and then print each \texttt{c{[}i{]}} (observe
  byte-level representation).
\item
  Define a \texttt{union\ Value\ \{\ int\ i;\ double\ d;\ \}}. Write a
  program that sets the double value and prints the size of the union
  using \texttt{sizeof}.
\end{enumerate}

\subsection{8.5 A Simple Contact Book}\label{a-simple-contact-book}

We've learned how to group data with \texttt{struct}, simplify with
\texttt{typedef}, and handle arrays. Now let's put it all together in a
small project: a contact book program.

\subsubsection{Defining the Contact
Structure}\label{defining-the-contact-structure}

\begin{Shaded}
\begin{Highlighting}[]
\PreprocessorTok{\#include }\ImportTok{\textless{}stdio.h\textgreater{}}

\KeywordTok{typedef} \KeywordTok{struct} \OperatorTok{\{}
    \DataTypeTok{char}\NormalTok{ name}\OperatorTok{[}\DecValTok{50}\OperatorTok{];}
    \DataTypeTok{char}\NormalTok{ phone}\OperatorTok{[}\DecValTok{20}\OperatorTok{];}
    \DataTypeTok{int}\NormalTok{ age}\OperatorTok{;}
\OperatorTok{\}}\NormalTok{ Contact}\OperatorTok{;}
\end{Highlighting}
\end{Shaded}

Each \texttt{Contact} stores a name, phone number, and age.

\subsubsection{Creating an Array of
Contacts}\label{creating-an-array-of-contacts}

We'll store multiple contacts in an array:

\begin{Shaded}
\begin{Highlighting}[]
\PreprocessorTok{\#define MAX\_CONTACTS }\DecValTok{5}
\NormalTok{Contact contacts}\OperatorTok{[}\NormalTok{MAX\_CONTACTS}\OperatorTok{];}
\DataTypeTok{int}\NormalTok{ count }\OperatorTok{=} \DecValTok{0}\OperatorTok{;}  \CommentTok{// how many we’ve added}
\end{Highlighting}
\end{Shaded}

\subsubsection{Adding a Contact}\label{adding-a-contact}

\begin{Shaded}
\begin{Highlighting}[]
\DataTypeTok{void}\NormalTok{ add\_contact}\OperatorTok{(}\NormalTok{Contact list}\OperatorTok{[],} \DataTypeTok{int} \OperatorTok{*}\NormalTok{count}\OperatorTok{,} \DataTypeTok{const} \DataTypeTok{char} \OperatorTok{*}\NormalTok{name}\OperatorTok{,} \DataTypeTok{const} \DataTypeTok{char} \OperatorTok{{-}}\NormalTok{phone}\OperatorTok{,} \DataTypeTok{int}\NormalTok{ age}\OperatorTok{)} \OperatorTok{\{}
    \ControlFlowTok{if} \OperatorTok{(*}\NormalTok{count }\OperatorTok{\textless{}}\NormalTok{ MAX\_CONTACTS}\OperatorTok{)} \OperatorTok{\{}
\NormalTok{        snprintf}\OperatorTok{(}\NormalTok{list}\OperatorTok{[*}\NormalTok{count}\OperatorTok{].}\NormalTok{name}\OperatorTok{,} \KeywordTok{sizeof}\OperatorTok{(}\NormalTok{list}\OperatorTok{[*}\NormalTok{count}\OperatorTok{].}\NormalTok{name}\OperatorTok{),} \StringTok{"}\SpecialCharTok{\%s}\StringTok{"}\OperatorTok{,}\NormalTok{ name}\OperatorTok{);}
\NormalTok{        snprintf}\OperatorTok{(}\NormalTok{list}\OperatorTok{[*}\NormalTok{count}\OperatorTok{].}\NormalTok{phone}\OperatorTok{,} \KeywordTok{sizeof}\OperatorTok{(}\NormalTok{list}\OperatorTok{[*}\NormalTok{count}\OperatorTok{].}\NormalTok{phone}\OperatorTok{),} \StringTok{"}\SpecialCharTok{\%s}\StringTok{"}\OperatorTok{,}\NormalTok{ phone}\OperatorTok{);}
\NormalTok{        list}\OperatorTok{[*}\NormalTok{count}\OperatorTok{].}\NormalTok{age }\OperatorTok{=}\NormalTok{ age}\OperatorTok{;}
        \OperatorTok{(*}\NormalTok{count}\OperatorTok{)++;}
    \OperatorTok{\}}
\OperatorTok{\}}
\end{Highlighting}
\end{Shaded}

\subsubsection{Printing Contacts}\label{printing-contacts}

\begin{Shaded}
\begin{Highlighting}[]
\DataTypeTok{void}\NormalTok{ print\_contacts}\OperatorTok{(}\DataTypeTok{const}\NormalTok{ Contact list}\OperatorTok{[],} \DataTypeTok{int}\NormalTok{ count}\OperatorTok{)} \OperatorTok{\{}
    \ControlFlowTok{for} \OperatorTok{(}\DataTypeTok{int}\NormalTok{ i }\OperatorTok{=} \DecValTok{0}\OperatorTok{;}\NormalTok{ i }\OperatorTok{\textless{}}\NormalTok{ count}\OperatorTok{;}\NormalTok{ i}\OperatorTok{++)} \OperatorTok{\{}
\NormalTok{        printf}\OperatorTok{(}\StringTok{"}\SpecialCharTok{\%s}\StringTok{, }\SpecialCharTok{\%s}\StringTok{, }\SpecialCharTok{\%d}\StringTok{ years}\SpecialCharTok{\textbackslash{}n}\StringTok{"}\OperatorTok{,}\NormalTok{ list}\OperatorTok{[}\NormalTok{i}\OperatorTok{].}\NormalTok{name}\OperatorTok{,}\NormalTok{ list}\OperatorTok{[}\NormalTok{i}\OperatorTok{].}\NormalTok{phone}\OperatorTok{,}\NormalTok{ list}\OperatorTok{[}\NormalTok{i}\OperatorTok{].}\NormalTok{age}\OperatorTok{);}
    \OperatorTok{\}}
\OperatorTok{\}}
\end{Highlighting}
\end{Shaded}

\subsubsection{Full Example}\label{full-example}

\begin{Shaded}
\begin{Highlighting}[]
\PreprocessorTok{\#include }\ImportTok{\textless{}stdio.h\textgreater{}}
\PreprocessorTok{\#include }\ImportTok{\textless{}string.h\textgreater{}}

\KeywordTok{typedef} \KeywordTok{struct} \OperatorTok{\{}
    \DataTypeTok{char}\NormalTok{ name}\OperatorTok{[}\DecValTok{50}\OperatorTok{];}
    \DataTypeTok{char}\NormalTok{ phone}\OperatorTok{[}\DecValTok{20}\OperatorTok{];}
    \DataTypeTok{int}\NormalTok{ age}\OperatorTok{;}
\OperatorTok{\}}\NormalTok{ Contact}\OperatorTok{;}

\PreprocessorTok{\#define MAX\_CONTACTS }\DecValTok{5}

\DataTypeTok{void}\NormalTok{ add\_contact}\OperatorTok{(}\NormalTok{Contact list}\OperatorTok{[],} \DataTypeTok{int} \OperatorTok{*}\NormalTok{count}\OperatorTok{,} \DataTypeTok{const} \DataTypeTok{char} \OperatorTok{*}\NormalTok{name}\OperatorTok{,} \DataTypeTok{const} \DataTypeTok{char} \OperatorTok{{-}}\NormalTok{phone}\OperatorTok{,} \DataTypeTok{int}\NormalTok{ age}\OperatorTok{)} \OperatorTok{\{}
    \ControlFlowTok{if} \OperatorTok{(*}\NormalTok{count }\OperatorTok{\textless{}}\NormalTok{ MAX\_CONTACTS}\OperatorTok{)} \OperatorTok{\{}
\NormalTok{        snprintf}\OperatorTok{(}\NormalTok{list}\OperatorTok{[*}\NormalTok{count}\OperatorTok{].}\NormalTok{name}\OperatorTok{,} \KeywordTok{sizeof}\OperatorTok{(}\NormalTok{list}\OperatorTok{[*}\NormalTok{count}\OperatorTok{].}\NormalTok{name}\OperatorTok{),} \StringTok{"}\SpecialCharTok{\%s}\StringTok{"}\OperatorTok{,}\NormalTok{ name}\OperatorTok{);}
\NormalTok{        snprintf}\OperatorTok{(}\NormalTok{list}\OperatorTok{[*}\NormalTok{count}\OperatorTok{].}\NormalTok{phone}\OperatorTok{,} \KeywordTok{sizeof}\OperatorTok{(}\NormalTok{list}\OperatorTok{[*}\NormalTok{count}\OperatorTok{].}\NormalTok{phone}\OperatorTok{),} \StringTok{"}\SpecialCharTok{\%s}\StringTok{"}\OperatorTok{,}\NormalTok{ phone}\OperatorTok{);}
\NormalTok{        list}\OperatorTok{[*}\NormalTok{count}\OperatorTok{].}\NormalTok{age }\OperatorTok{=}\NormalTok{ age}\OperatorTok{;}
        \OperatorTok{(*}\NormalTok{count}\OperatorTok{)++;}
    \OperatorTok{\}}
\OperatorTok{\}}

\DataTypeTok{void}\NormalTok{ print\_contacts}\OperatorTok{(}\DataTypeTok{const}\NormalTok{ Contact list}\OperatorTok{[],} \DataTypeTok{int}\NormalTok{ count}\OperatorTok{)} \OperatorTok{\{}
    \ControlFlowTok{for} \OperatorTok{(}\DataTypeTok{int}\NormalTok{ i }\OperatorTok{=} \DecValTok{0}\OperatorTok{;}\NormalTok{ i }\OperatorTok{\textless{}}\NormalTok{ count}\OperatorTok{;}\NormalTok{ i}\OperatorTok{++)} \OperatorTok{\{}
\NormalTok{        printf}\OperatorTok{(}\StringTok{"}\SpecialCharTok{\%s}\StringTok{, }\SpecialCharTok{\%s}\StringTok{, }\SpecialCharTok{\%d}\StringTok{ years}\SpecialCharTok{\textbackslash{}n}\StringTok{"}\OperatorTok{,}\NormalTok{ list}\OperatorTok{[}\NormalTok{i}\OperatorTok{].}\NormalTok{name}\OperatorTok{,}\NormalTok{ list}\OperatorTok{[}\NormalTok{i}\OperatorTok{].}\NormalTok{phone}\OperatorTok{,}\NormalTok{ list}\OperatorTok{[}\NormalTok{i}\OperatorTok{].}\NormalTok{age}\OperatorTok{);}
    \OperatorTok{\}}
\OperatorTok{\}}

\DataTypeTok{int}\NormalTok{ main}\OperatorTok{(}\DataTypeTok{void}\OperatorTok{)} \OperatorTok{\{}
\NormalTok{    Contact contacts}\OperatorTok{[}\NormalTok{MAX\_CONTACTS}\OperatorTok{];}
    \DataTypeTok{int}\NormalTok{ count }\OperatorTok{=} \DecValTok{0}\OperatorTok{;}

\NormalTok{    add\_contact}\OperatorTok{(}\NormalTok{contacts}\OperatorTok{,} \OperatorTok{\&}\NormalTok{count}\OperatorTok{,} \StringTok{"Alice"}\OperatorTok{,} \StringTok{"123{-}4567"}\OperatorTok{,} \DecValTok{20}\OperatorTok{);}
\NormalTok{    add\_contact}\OperatorTok{(}\NormalTok{contacts}\OperatorTok{,} \OperatorTok{\&}\NormalTok{count}\OperatorTok{,} \StringTok{"Bob"}\OperatorTok{,} \StringTok{"555{-}9876"}\OperatorTok{,} \DecValTok{25}\OperatorTok{);}

\NormalTok{    print\_contacts}\OperatorTok{(}\NormalTok{contacts}\OperatorTok{,}\NormalTok{ count}\OperatorTok{);}

    \ControlFlowTok{return} \DecValTok{0}\OperatorTok{;}
\OperatorTok{\}}
\end{Highlighting}
\end{Shaded}

Output:

\begin{Shaded}
\begin{Highlighting}[]
\ExtensionTok{Alice,}\NormalTok{ 123{-}4567, 20 years}
\ExtensionTok{Bob,}\NormalTok{ 555{-}9876, 25 years}
\end{Highlighting}
\end{Shaded}

\subsubsection{Why It Matters}\label{why-it-matters-38}

\begin{itemize}
\tightlist
\item
  Shows how structs can model real-world objects.
\item
  Demonstrates \texttt{typedef}, arrays, and function reuse.
\item
  This pattern (define a type → store in array → process with functions)
  is the basis of larger applications like databases and address books.
\end{itemize}

\subsubsection{Exercises}\label{exercises-39}

\begin{enumerate}
\def\labelenumi{\arabic{enumi}.}
\tightlist
\item
  Add a function
  \texttt{find\_contact(Contact\ list{[}{]},\ int\ count,\ const\ char\ *name)}
  that searches for a contact by name.
\item
  Add a function
  \texttt{delete\_contact(Contact\ list{[}{]},\ int\ *count,\ const\ char\ *name)}
  that removes a contact.
\item
  Extend \texttt{Contact} with an \texttt{email} field. Update
  \texttt{add\_contact} and \texttt{print\_contacts}.
\item
  Increase \texttt{MAX\_CONTACTS} to 100 and let the program read
  contacts from user input instead of hardcoding.
\item
  Save the contacts to a text file using \texttt{fprintf} and reload
  them with \texttt{fscanf}.
\end{enumerate}

\subsection{Problems}\label{problems-6}

\subsubsection{1. Basic Point Struct}\label{basic-point-struct}

Define a \texttt{struct\ Point} with \texttt{x} and \texttt{y} fields.
Create one, set values, and print them.

\subsubsection{2. Student Record}\label{student-record}

Define a \texttt{struct\ Student} with \texttt{name}, \texttt{age}, and
\texttt{gpa}. Initialize one student and print their details.

\subsubsection{3. Array of Structs}\label{array-of-structs}

Create an array of 3 \texttt{struct\ Point}s and print their
coordinates.

\subsubsection{4. Input into Struct}\label{input-into-struct}

Write a program that reads a student's \texttt{name}, \texttt{age}, and
\texttt{gpa} from the user into a struct and prints them back.

\subsubsection{5. Rectangle Area}\label{rectangle-area}

Define a \texttt{struct\ Rectangle} with \texttt{width} and
\texttt{height}. Write a function that takes a \texttt{Rectangle} and
returns its area.

\subsubsection{\texorpdfstring{6. Using \texttt{typedef} for
Point}{6. Using typedef for Point}}\label{using-typedef-for-point}

Rewrite the \texttt{struct\ Point} example using \texttt{typedef} so you
can declare variables as \texttt{Point\ p} instead of
\texttt{struct\ Point\ p}.

\subsubsection{7. Typedef with Pointers}\label{typedef-with-pointers-1}

Define a \texttt{typedef} for \texttt{struct\ Student} as
\texttt{Student}, and another typedef \texttt{StudentPtr} for a pointer
to \texttt{Student}. Create a student and print values through the
pointer.

\subsubsection{8. Typedef Alias for
Primitive}\label{typedef-alias-for-primitive}

Create a \texttt{typedef} called \texttt{BigInt} for
\texttt{unsigned\ long\ long}. Use it to compute factorial of 10.

\subsubsection{9. Enum for Weekdays}\label{enum-for-weekdays}

Define an \texttt{enum\ Weekday} with values for Monday--Sunday
(starting at 1). Print the integer value for Wednesday.

\subsubsection{10. Enum for Traffic Light}\label{enum-for-traffic-light}

Define an
\texttt{enum\ TrafficLight\ \{\ RED=1,\ YELLOW=2,\ GREEN=3\ \}}. Write a
program that prints \texttt{"Stop"}, \texttt{"Wait"}, or \texttt{"Go"}
depending on the enum value.

\subsubsection{11. Enum with Switch}\label{enum-with-switch}

Define an \texttt{enum\ ErrorCode\ \{\ OK=0,\ FAIL=1,\ TIMEOUT=2\ \}}
and use a \texttt{switch} to print a message for each code.

\subsubsection{12. Enum Loop}\label{enum-loop}

Use the \texttt{Weekday} enum from problem 9. Write a loop to print all
days with their numeric values.

\subsubsection{13. Basic Union}\label{basic-union}

Define a \texttt{union\ Data} with \texttt{int\ i} and
\texttt{float\ f}. Assign and print each in turn. Observe how one
overwrites the other.

\subsubsection{14. Union with Char Array}\label{union-with-char-array}

Create a union with \texttt{char\ c{[}4{]}} and \texttt{int\ n}. Assign
to \texttt{n} and then print each \texttt{c{[}i{]}} (observe byte-level
representation).

\subsubsection{15. Tagged Union}\label{tagged-union}

Combine \texttt{enum} and \texttt{union} into a \texttt{Variant} type
that may hold either an \texttt{int} or a \texttt{float}. Write a
program that sets and prints both kinds safely depending on the tag.

\subsubsection{16. Contact Struct Array}\label{contact-struct-array}

Define a
\texttt{struct\ Contact\ \{\ char\ name{[}50{]};\ char\ phone{[}20{]};\ int\ age;\ \}}.
Create an array of 3 contacts, initialize them, and print all.

\subsubsection{17. Contact Functions}\label{contact-functions}

Write a function \texttt{add\_contact} that inserts a new contact into
an array, and \texttt{print\_contacts} that prints them. Test with 2--3
contacts.

\subsubsection{18. Find Contact by Name}\label{find-contact-by-name}

Extend problem 17 with a function \texttt{find\_contact} that searches
for a contact by name and prints their details.

\subsubsection{19. Delete Contact}\label{delete-contact}

Extend problem 18 with a function \texttt{delete\_contact} that removes
a contact by shifting later ones down. Print the updated list.

\subsubsection{20. Save and Load Contacts}\label{save-and-load-contacts}

Write a program that saves contacts from an array to a text file with
\texttt{fprintf} and reloads them with \texttt{fscanf}.

\section{Chapter 9. Memory
Management}\label{chapter-9.-memory-management}

\subsection{9.1 Automatic vs.~Dynamic
Memory}\label{automatic-vs.-dynamic-memory}

Every program needs memory to store variables, arrays, and data. In C,
there are two main categories of memory you'll work with:

\begin{enumerate}
\def\labelenumi{\arabic{enumi}.}
\tightlist
\item
  Automatic memory (stack)
\item
  Dynamic memory (heap)
\end{enumerate}

Understanding the difference is essential for writing reliable and
efficient C programs.

\subsubsection{Automatic Memory (Stack)}\label{automatic-memory-stack}

\begin{itemize}
\tightlist
\item
  Variables declared inside a function (without \texttt{malloc}) are
  automatic.
\item
  Their lifetime is tied to the function call.
\item
  They are created when the function begins, and destroyed when it ends.
\end{itemize}

Example:

\begin{Shaded}
\begin{Highlighting}[]
\PreprocessorTok{\#include }\ImportTok{\textless{}stdio.h\textgreater{}}

\DataTypeTok{void}\NormalTok{ hello}\OperatorTok{(}\DataTypeTok{void}\OperatorTok{)} \OperatorTok{\{}
    \DataTypeTok{int}\NormalTok{ x }\OperatorTok{=} \DecValTok{42}\OperatorTok{;}  \CommentTok{// automatic variable}
\NormalTok{    printf}\OperatorTok{(}\StringTok{"x = }\SpecialCharTok{\%d\textbackslash{}n}\StringTok{"}\OperatorTok{,}\NormalTok{ x}\OperatorTok{);}
\OperatorTok{\}} \CommentTok{// x no longer exists here}

\DataTypeTok{int}\NormalTok{ main}\OperatorTok{(}\DataTypeTok{void}\OperatorTok{)} \OperatorTok{\{}
\NormalTok{    hello}\OperatorTok{();}
    \CommentTok{// printf("\%d", x);  // ❌ error: x is out of scope}
    \ControlFlowTok{return} \DecValTok{0}\OperatorTok{;}
\OperatorTok{\}}
\end{Highlighting}
\end{Shaded}

Automatic variables are usually stored on the stack, a region of memory
managed by the compiler.

\subsubsection{Dynamic Memory (Heap)}\label{dynamic-memory-heap}

\begin{itemize}
\tightlist
\item
  Allocated at runtime with functions like \texttt{malloc} and
  \texttt{free}.
\item
  Lifetime is controlled by the programmer (until explicitly freed).
\item
  Useful when you don't know the needed size at compile time.
\end{itemize}

Example:

\begin{Shaded}
\begin{Highlighting}[]
\PreprocessorTok{\#include }\ImportTok{\textless{}stdio.h\textgreater{}}
\PreprocessorTok{\#include }\ImportTok{\textless{}stdlib.h\textgreater{}}

\DataTypeTok{int}\NormalTok{ main}\OperatorTok{(}\DataTypeTok{void}\OperatorTok{)} \OperatorTok{\{}
    \DataTypeTok{int} \OperatorTok{{-}}\NormalTok{p }\OperatorTok{=}\NormalTok{ malloc}\OperatorTok{(}\KeywordTok{sizeof}\OperatorTok{(}\DataTypeTok{int}\OperatorTok{));}  \CommentTok{// dynamic allocation}
    \ControlFlowTok{if} \OperatorTok{(}\NormalTok{p }\OperatorTok{==}\NormalTok{ NULL}\OperatorTok{)} \ControlFlowTok{return} \DecValTok{1}\OperatorTok{;}       \CommentTok{// always check}
    \OperatorTok{{-}}\NormalTok{p }\OperatorTok{=} \DecValTok{42}\OperatorTok{;}
\NormalTok{    printf}\OperatorTok{(}\StringTok{"{-}p = }\SpecialCharTok{\%d\textbackslash{}n}\StringTok{"}\OperatorTok{,} \OperatorTok{{-}}\NormalTok{p}\OperatorTok{);}
\NormalTok{    free}\OperatorTok{(}\NormalTok{p}\OperatorTok{);}                       \CommentTok{// must free after use}
    \ControlFlowTok{return} \DecValTok{0}\OperatorTok{;}
\OperatorTok{\}}
\end{Highlighting}
\end{Shaded}

\subsubsection{Comparing the Two}\label{comparing-the-two}

\begin{longtable}[]{@{}lll@{}}
\toprule\noalign{}
Aspect & Automatic (stack) & Dynamic (heap) \\
\midrule\noalign{}
\endhead
\bottomrule\noalign{}
\endlastfoot
Lifetime & Ends when function ends & Until \texttt{free} is called \\
Allocation & Compiler-managed & Programmer-managed (\texttt{malloc}) \\
Speed & Very fast & Slower \\
Size & Usually small, fixed & Potentially large, flexible \\
\end{longtable}

\subsubsection{Example: Fixed vs.~Flexible
Arrays}\label{example-fixed-vs.-flexible-arrays}

Automatic array:

\begin{Shaded}
\begin{Highlighting}[]
\DataTypeTok{int}\NormalTok{ arr}\OperatorTok{[}\DecValTok{100}\OperatorTok{];}  \CommentTok{// size fixed at compile time}
\end{Highlighting}
\end{Shaded}

Dynamic array:

\begin{Shaded}
\begin{Highlighting}[]
\DataTypeTok{int}\NormalTok{ n}\OperatorTok{;}
\NormalTok{scanf}\OperatorTok{(}\StringTok{"}\SpecialCharTok{\%d}\StringTok{"}\OperatorTok{,} \OperatorTok{\&}\NormalTok{n}\OperatorTok{);}
\DataTypeTok{int} \OperatorTok{*}\NormalTok{arr }\OperatorTok{=}\NormalTok{ malloc}\OperatorTok{(}\NormalTok{n }\OperatorTok{{-}} \KeywordTok{sizeof}\OperatorTok{(}\DataTypeTok{int}\OperatorTok{));}  \CommentTok{// size chosen at runtime}
\end{Highlighting}
\end{Shaded}

\subsubsection{Why It Matters}\label{why-it-matters-39}

\begin{itemize}
\tightlist
\item
  Automatic memory is simple, but limited to function lifetimes and
  fixed sizes.
\item
  Dynamic memory gives flexibility, but requires discipline: every
  \texttt{malloc} should have a matching \texttt{free}.
\item
  Choosing the right kind of memory is a fundamental design decision in
  C programs.
\end{itemize}

\subsubsection{Exercises}\label{exercises-40}

\begin{enumerate}
\def\labelenumi{\arabic{enumi}.}
\tightlist
\item
  Write a function that declares an automatic variable, assigns it a
  value, and prints it. Call the function twice - what do you observe?
\item
  Allocate an \texttt{int} dynamically, assign it \texttt{99}, print it,
  then free it.
\item
  Write a program that asks the user for \texttt{n}, allocates an array
  of \texttt{n} integers dynamically, fills it with 1..n, and prints
  them.
\item
  Experiment: try to use a pointer to an automatic variable after the
  function returns. Why is this unsafe?
\item
  Compare the sizes of a stack array (\texttt{int\ a{[}1000{]};}) and a
  dynamically allocated array (\texttt{malloc(1000-sizeof(int))}). Use
  \texttt{sizeof} and print results.
\end{enumerate}

\subsection{\texorpdfstring{9.2 \texttt{malloc}, \texttt{calloc},
\texttt{free}}{9.2 malloc, calloc, free}}\label{malloc-calloc-free}

C gives you explicit control over memory allocation. The three most
important functions are:

\begin{itemize}
\tightlist
\item
  \texttt{malloc} - allocate memory
\item
  \texttt{calloc} - allocate and clear memory
\item
  \texttt{free} - release memory
\end{itemize}

All are declared in \texttt{\textless{}stdlib.h\textgreater{}}.

\subsubsection{\texorpdfstring{\texttt{malloc} - Allocate
Memory}{malloc - Allocate Memory}}\label{malloc---allocate-memory}

\begin{Shaded}
\begin{Highlighting}[]
\PreprocessorTok{\#include }\ImportTok{\textless{}stdlib.h\textgreater{}}
\DataTypeTok{void} \OperatorTok{{-}}\NormalTok{malloc}\OperatorTok{(}\DataTypeTok{size\_t}\NormalTok{ size}\OperatorTok{);}
\end{Highlighting}
\end{Shaded}

\begin{itemize}
\tightlist
\item
  Allocates a block of memory of given size (in bytes).
\item
  Returns a pointer to the beginning of the block, or \texttt{NULL} if
  it fails.
\item
  Contents of the memory are uninitialized (garbage).
\end{itemize}

Example:

\begin{Shaded}
\begin{Highlighting}[]
\PreprocessorTok{\#include }\ImportTok{\textless{}stdio.h\textgreater{}}
\PreprocessorTok{\#include }\ImportTok{\textless{}stdlib.h\textgreater{}}

\DataTypeTok{int}\NormalTok{ main}\OperatorTok{(}\DataTypeTok{void}\OperatorTok{)} \OperatorTok{\{}
    \DataTypeTok{int} \OperatorTok{{-}}\NormalTok{p }\OperatorTok{=}\NormalTok{ malloc}\OperatorTok{(}\DecValTok{5} \OperatorTok{{-}} \KeywordTok{sizeof}\OperatorTok{(}\DataTypeTok{int}\OperatorTok{));}
    \ControlFlowTok{if} \OperatorTok{(!}\NormalTok{p}\OperatorTok{)} \ControlFlowTok{return} \DecValTok{1}\OperatorTok{;}  \CommentTok{// check allocation}
    \ControlFlowTok{for} \OperatorTok{(}\DataTypeTok{int}\NormalTok{ i }\OperatorTok{=} \DecValTok{0}\OperatorTok{;}\NormalTok{ i }\OperatorTok{\textless{}} \DecValTok{5}\OperatorTok{;}\NormalTok{ i}\OperatorTok{++)}\NormalTok{ p}\OperatorTok{[}\NormalTok{i}\OperatorTok{]} \OperatorTok{=}\NormalTok{ i}\OperatorTok{+}\DecValTok{1}\OperatorTok{;}
    \ControlFlowTok{for} \OperatorTok{(}\DataTypeTok{int}\NormalTok{ i }\OperatorTok{=} \DecValTok{0}\OperatorTok{;}\NormalTok{ i }\OperatorTok{\textless{}} \DecValTok{5}\OperatorTok{;}\NormalTok{ i}\OperatorTok{++)}\NormalTok{ printf}\OperatorTok{(}\StringTok{"}\SpecialCharTok{\%d}\StringTok{ "}\OperatorTok{,}\NormalTok{ p}\OperatorTok{[}\NormalTok{i}\OperatorTok{]);}
\NormalTok{    printf}\OperatorTok{(}\StringTok{"}\SpecialCharTok{\textbackslash{}n}\StringTok{"}\OperatorTok{);}
\NormalTok{    free}\OperatorTok{(}\NormalTok{p}\OperatorTok{);}
    \ControlFlowTok{return} \DecValTok{0}\OperatorTok{;}
\OperatorTok{\}}
\end{Highlighting}
\end{Shaded}

\subsubsection{\texorpdfstring{\texttt{calloc} - Allocate and
Clear}{calloc - Allocate and Clear}}\label{calloc---allocate-and-clear}

\begin{Shaded}
\begin{Highlighting}[]
\PreprocessorTok{\#include }\ImportTok{\textless{}stdlib.h\textgreater{}}
\DataTypeTok{void} \OperatorTok{{-}}\NormalTok{calloc}\OperatorTok{(}\DataTypeTok{size\_t}\NormalTok{ n}\OperatorTok{,} \DataTypeTok{size\_t}\NormalTok{ size}\OperatorTok{);}
\end{Highlighting}
\end{Shaded}

\begin{itemize}
\tightlist
\item
  Allocates memory for an array of \texttt{n} elements, each of
  \texttt{size} bytes.
\item
  Initializes all bits to zero.
\end{itemize}

Example:

\begin{Shaded}
\begin{Highlighting}[]
\PreprocessorTok{\#include }\ImportTok{\textless{}stdio.h\textgreater{}}
\PreprocessorTok{\#include }\ImportTok{\textless{}stdlib.h\textgreater{}}

\DataTypeTok{int}\NormalTok{ main}\OperatorTok{(}\DataTypeTok{void}\OperatorTok{)} \OperatorTok{\{}
    \DataTypeTok{int} \OperatorTok{{-}}\NormalTok{p }\OperatorTok{=}\NormalTok{ calloc}\OperatorTok{(}\DecValTok{5}\OperatorTok{,} \KeywordTok{sizeof}\OperatorTok{(}\DataTypeTok{int}\OperatorTok{));}
    \ControlFlowTok{if} \OperatorTok{(!}\NormalTok{p}\OperatorTok{)} \ControlFlowTok{return} \DecValTok{1}\OperatorTok{;}
    \ControlFlowTok{for} \OperatorTok{(}\DataTypeTok{int}\NormalTok{ i }\OperatorTok{=} \DecValTok{0}\OperatorTok{;}\NormalTok{ i }\OperatorTok{\textless{}} \DecValTok{5}\OperatorTok{;}\NormalTok{ i}\OperatorTok{++)}\NormalTok{ printf}\OperatorTok{(}\StringTok{"}\SpecialCharTok{\%d}\StringTok{ "}\OperatorTok{,}\NormalTok{ p}\OperatorTok{[}\NormalTok{i}\OperatorTok{]);} \CommentTok{// all zero}
\NormalTok{    printf}\OperatorTok{(}\StringTok{"}\SpecialCharTok{\textbackslash{}n}\StringTok{"}\OperatorTok{);}
\NormalTok{    free}\OperatorTok{(}\NormalTok{p}\OperatorTok{);}
    \ControlFlowTok{return} \DecValTok{0}\OperatorTok{;}
\OperatorTok{\}}
\end{Highlighting}
\end{Shaded}

Output:

\begin{Shaded}
\begin{Highlighting}[]
\ExtensionTok{0}\NormalTok{ 0 0 0 0}
\end{Highlighting}
\end{Shaded}

\subsubsection{\texorpdfstring{\texttt{free} - Release
Memory}{free - Release Memory}}\label{free---release-memory}

\begin{Shaded}
\begin{Highlighting}[]
\PreprocessorTok{\#include }\ImportTok{\textless{}stdlib.h\textgreater{}}
\DataTypeTok{void}\NormalTok{ free}\OperatorTok{(}\DataTypeTok{void} \OperatorTok{{-}}\NormalTok{ptr}\OperatorTok{);}
\end{Highlighting}
\end{Shaded}

\begin{itemize}
\tightlist
\item
  Releases a block of memory previously allocated by \texttt{malloc} or
  \texttt{calloc}.
\item
  Does nothing if \texttt{ptr\ ==\ NULL}.
\item
  Accessing memory after \texttt{free} is undefined behavior.
\end{itemize}

Example:

\begin{Shaded}
\begin{Highlighting}[]
\PreprocessorTok{\#include }\ImportTok{\textless{}stdio.h\textgreater{}}
\PreprocessorTok{\#include }\ImportTok{\textless{}stdlib.h\textgreater{}}

\DataTypeTok{int}\NormalTok{ main}\OperatorTok{(}\DataTypeTok{void}\OperatorTok{)} \OperatorTok{\{}
    \DataTypeTok{int} \OperatorTok{{-}}\NormalTok{p }\OperatorTok{=}\NormalTok{ malloc}\OperatorTok{(}\DecValTok{3} \OperatorTok{{-}} \KeywordTok{sizeof}\OperatorTok{(}\DataTypeTok{int}\OperatorTok{));}
    \ControlFlowTok{if} \OperatorTok{(!}\NormalTok{p}\OperatorTok{)} \ControlFlowTok{return} \DecValTok{1}\OperatorTok{;}
\NormalTok{    p}\OperatorTok{[}\DecValTok{0}\OperatorTok{]=}\DecValTok{1}\OperatorTok{;}\NormalTok{ p}\OperatorTok{[}\DecValTok{1}\OperatorTok{]=}\DecValTok{2}\OperatorTok{;}\NormalTok{ p}\OperatorTok{[}\DecValTok{2}\OperatorTok{]=}\DecValTok{3}\OperatorTok{;}
\NormalTok{    free}\OperatorTok{(}\NormalTok{p}\OperatorTok{);}
    \CommentTok{// p is now dangling {-} should set to NULL}
\NormalTok{    p }\OperatorTok{=}\NormalTok{ NULL}\OperatorTok{;}
    \ControlFlowTok{return} \DecValTok{0}\OperatorTok{;}
\OperatorTok{\}}
\end{Highlighting}
\end{Shaded}

\subsubsection{Common Mistakes}\label{common-mistakes-1}

❌ Forgetting to call \texttt{free} → memory leak ❌ Using memory after
\texttt{free} → dangling pointer ❌ Freeing twice → double free error

Always follow the pattern:

\begin{Shaded}
\begin{Highlighting}[]
\NormalTok{p }\OperatorTok{=}\NormalTok{ malloc}\OperatorTok{(...);}
\ControlFlowTok{if} \OperatorTok{(!}\NormalTok{p}\OperatorTok{)} \OperatorTok{\{} \OperatorTok{/{-}}\NormalTok{ handle error }\OperatorTok{{-}/} \OperatorTok{\}}
\OperatorTok{...}
\NormalTok{free}\OperatorTok{(}\NormalTok{p}\OperatorTok{);}
\NormalTok{p }\OperatorTok{=}\NormalTok{ NULL}\OperatorTok{;}
\end{Highlighting}
\end{Shaded}

\subsubsection{Why It Matters}\label{why-it-matters-40}

\begin{itemize}
\tightlist
\item
  \texttt{malloc} and \texttt{calloc} let programs create data
  structures that grow and shrink at runtime.
\item
  \texttt{free} ensures memory is returned to the system.
\item
  Correct memory management is crucial in C - there is no garbage
  collector.
\end{itemize}

\subsubsection{Exercises}\label{exercises-41}

\begin{enumerate}
\def\labelenumi{\arabic{enumi}.}
\tightlist
\item
  Use \texttt{malloc} to allocate space for an array of 10 integers,
  fill with squares of 1..10, and print them.
\item
  Repeat using \texttt{calloc} and observe the difference in
  initialization.
\item
  Allocate an array of \texttt{double} of size entered by the user, set
  all to \texttt{3.14}, and print.
\item
  Write a program that allocates memory, frees it, then tries to use it
  again (observe error). Fix it by setting pointer to \texttt{NULL}.
\item
  Create a dynamic 2D array (array of \texttt{int-}) with 3 rows and 4
  columns using \texttt{malloc}. Fill it with numbers and print as a
  matrix.
\end{enumerate}

\subsection{9.3 Pointer Pitfalls (and how to avoid
them)}\label{pointer-pitfalls-and-how-to-avoid-them}

Pointers give you power over memory - but with great power comes great
responsibility. Misusing pointers often leads to undefined behavior,
crashes, or security bugs. Here are the most common pitfalls and how to
avoid them.

\subsubsection{1. Dangling Pointers}\label{dangling-pointers-1}

A dangling pointer points to memory that is no longer valid.

Example: returning a pointer to a local variable:

\begin{Shaded}
\begin{Highlighting}[]
\DataTypeTok{int}\OperatorTok{{-}}\NormalTok{ bad}\OperatorTok{(}\DataTypeTok{void}\OperatorTok{)} \OperatorTok{\{}
    \DataTypeTok{int}\NormalTok{ x }\OperatorTok{=} \DecValTok{42}\OperatorTok{;}
    \ControlFlowTok{return} \OperatorTok{\&}\NormalTok{x}\OperatorTok{;}   \CommentTok{// ❌ invalid after function ends}
\OperatorTok{\}}
\end{Highlighting}
\end{Shaded}

Fix: Allocate dynamically or pass results back by value.

\begin{Shaded}
\begin{Highlighting}[]
\DataTypeTok{int}\OperatorTok{{-}}\NormalTok{ good}\OperatorTok{(}\DataTypeTok{void}\OperatorTok{)} \OperatorTok{\{}
    \DataTypeTok{int} \OperatorTok{{-}}\NormalTok{p }\OperatorTok{=}\NormalTok{ malloc}\OperatorTok{(}\KeywordTok{sizeof}\OperatorTok{(}\DataTypeTok{int}\OperatorTok{));}
    \ControlFlowTok{if} \OperatorTok{(}\NormalTok{p}\OperatorTok{)} \OperatorTok{{-}}\NormalTok{p }\OperatorTok{=} \DecValTok{42}\OperatorTok{;}
    \ControlFlowTok{return}\NormalTok{ p}\OperatorTok{;}   \CommentTok{// caller must free}
\OperatorTok{\}}
\end{Highlighting}
\end{Shaded}

\subsubsection{2. Memory Leaks}\label{memory-leaks}

A memory leak happens when you lose all references to allocated memory
without freeing it.

\begin{Shaded}
\begin{Highlighting}[]
\DataTypeTok{int} \OperatorTok{{-}}\NormalTok{p }\OperatorTok{=}\NormalTok{ malloc}\OperatorTok{(}\DecValTok{100} \OperatorTok{{-}} \KeywordTok{sizeof}\OperatorTok{(}\DataTypeTok{int}\OperatorTok{));}
\NormalTok{p }\OperatorTok{=}\NormalTok{ NULL}\OperatorTok{;}  \CommentTok{// ❌ leak: cannot free the memory anymore}
\end{Highlighting}
\end{Shaded}

Fix: Always \texttt{free} before overwriting pointers.

\begin{Shaded}
\begin{Highlighting}[]
\NormalTok{free}\OperatorTok{(}\NormalTok{p}\OperatorTok{);}
\NormalTok{p }\OperatorTok{=}\NormalTok{ NULL}\OperatorTok{;}
\end{Highlighting}
\end{Shaded}

\subsubsection{3. Double Free}\label{double-free-1}

Freeing the same memory twice can corrupt the heap.

\begin{Shaded}
\begin{Highlighting}[]
\DataTypeTok{int} \OperatorTok{{-}}\NormalTok{p }\OperatorTok{=}\NormalTok{ malloc}\OperatorTok{(}\KeywordTok{sizeof}\OperatorTok{(}\DataTypeTok{int}\OperatorTok{));}
\NormalTok{free}\OperatorTok{(}\NormalTok{p}\OperatorTok{);}
\NormalTok{free}\OperatorTok{(}\NormalTok{p}\OperatorTok{);}   \CommentTok{// ❌ undefined behavior}
\end{Highlighting}
\end{Shaded}

Fix: After \texttt{free}, set pointer to \texttt{NULL}. Freeing a null
pointer is safe.

\subsubsection{4. Using After Free}\label{using-after-free}

Accessing memory after it's been freed:

\begin{Shaded}
\begin{Highlighting}[]
\DataTypeTok{int} \OperatorTok{{-}}\NormalTok{p }\OperatorTok{=}\NormalTok{ malloc}\OperatorTok{(}\KeywordTok{sizeof}\OperatorTok{(}\DataTypeTok{int}\OperatorTok{));}
\OperatorTok{{-}}\NormalTok{p }\OperatorTok{=} \DecValTok{7}\OperatorTok{;}
\NormalTok{free}\OperatorTok{(}\NormalTok{p}\OperatorTok{);}
\NormalTok{printf}\OperatorTok{(}\StringTok{"}\SpecialCharTok{\%d\textbackslash{}n}\StringTok{"}\OperatorTok{,} \OperatorTok{{-}}\NormalTok{p}\OperatorTok{);} \CommentTok{// ❌ use*after{-}free}
\end{Highlighting}
\end{Shaded}

Fix: Never dereference freed pointers. Set them to \texttt{NULL} to
prevent accidental use.

\subsubsection{5. Out-of-Bounds Access}\label{out-of-bounds-access}

Arrays don't track their size. Accessing outside is undefined.

\begin{Shaded}
\begin{Highlighting}[]
\DataTypeTok{int}\NormalTok{ a}\OperatorTok{[}\DecValTok{5}\OperatorTok{]} \OperatorTok{=} \OperatorTok{\{}\DecValTok{1}\OperatorTok{,}\DecValTok{2}\OperatorTok{,}\DecValTok{3}\OperatorTok{,}\DecValTok{4}\OperatorTok{,}\DecValTok{5}\OperatorTok{\};}
\NormalTok{printf}\OperatorTok{(}\StringTok{"}\SpecialCharTok{\%d\textbackslash{}n}\StringTok{"}\OperatorTok{,}\NormalTok{ a}\OperatorTok{[}\DecValTok{5}\OperatorTok{]);}  \CommentTok{// ❌ out of bounds}
\end{Highlighting}
\end{Shaded}

Fix: Always check indices against array bounds.

\subsubsection{6. Uninitialized Pointers}\label{uninitialized-pointers}

A pointer not assigned an address may point anywhere.

\begin{Shaded}
\begin{Highlighting}[]
\DataTypeTok{int} \OperatorTok{{-}}\NormalTok{p}\OperatorTok{;}   \CommentTok{// uninitialized}
\OperatorTok{{-}}\NormalTok{p }\OperatorTok{=} \DecValTok{5}\OperatorTok{;}   \CommentTok{// ❌ random crash}
\end{Highlighting}
\end{Shaded}

Fix: Initialize pointers to \texttt{NULL} or valid memory.

\subsubsection{Checklist for Safe
Pointers}\label{checklist-for-safe-pointers}

\begin{itemize}
\tightlist
\item[$\boxtimes$]
  Initialize all pointers.
\item[$\boxtimes$]
  Check return value of \texttt{malloc}/\texttt{calloc}.
\item[$\boxtimes$]
  \texttt{free} every allocation, exactly once.
\item[$\boxtimes$]
  Set pointers to \texttt{NULL} after \texttt{free}.
\item[$\boxtimes$]
  Always check bounds when indexing arrays.
\end{itemize}

\subsubsection{Example: Safe Allocation
Pattern}\label{example-safe-allocation-pattern}

\begin{Shaded}
\begin{Highlighting}[]
\PreprocessorTok{\#include }\ImportTok{\textless{}stdio.h\textgreater{}}
\PreprocessorTok{\#include }\ImportTok{\textless{}stdlib.h\textgreater{}}

\DataTypeTok{int}\NormalTok{ main}\OperatorTok{(}\DataTypeTok{void}\OperatorTok{)} \OperatorTok{\{}
    \DataTypeTok{int}\NormalTok{ n }\OperatorTok{=} \DecValTok{5}\OperatorTok{;}
    \DataTypeTok{int} \OperatorTok{*}\NormalTok{a }\OperatorTok{=}\NormalTok{ malloc}\OperatorTok{(}\NormalTok{n }\OperatorTok{{-}} \KeywordTok{sizeof}\OperatorTok{(}\DataTypeTok{int}\OperatorTok{));}
    \ControlFlowTok{if} \OperatorTok{(!}\NormalTok{a}\OperatorTok{)} \ControlFlowTok{return} \DecValTok{1}\OperatorTok{;}   \CommentTok{// check for failure}

    \ControlFlowTok{for} \OperatorTok{(}\DataTypeTok{int}\NormalTok{ i }\OperatorTok{=} \DecValTok{0}\OperatorTok{;}\NormalTok{ i }\OperatorTok{\textless{}}\NormalTok{ n}\OperatorTok{;}\NormalTok{ i}\OperatorTok{++)}\NormalTok{ a}\OperatorTok{[}\NormalTok{i}\OperatorTok{]} \OperatorTok{=}\NormalTok{ i }\OperatorTok{{-}}\NormalTok{ i}\OperatorTok{;}
    \ControlFlowTok{for} \OperatorTok{(}\DataTypeTok{int}\NormalTok{ i }\OperatorTok{=} \DecValTok{0}\OperatorTok{;}\NormalTok{ i }\OperatorTok{\textless{}}\NormalTok{ n}\OperatorTok{;}\NormalTok{ i}\OperatorTok{++)}\NormalTok{ printf}\OperatorTok{(}\StringTok{"}\SpecialCharTok{\%d}\StringTok{ "}\OperatorTok{,}\NormalTok{ a}\OperatorTok{[}\NormalTok{i}\OperatorTok{]);}
\NormalTok{    printf}\OperatorTok{(}\StringTok{"}\SpecialCharTok{\textbackslash{}n}\StringTok{"}\OperatorTok{);}

\NormalTok{    free}\OperatorTok{(}\NormalTok{a}\OperatorTok{);}
\NormalTok{    a }\OperatorTok{=}\NormalTok{ NULL}\OperatorTok{;}           \CommentTok{// prevent dangling}
    \ControlFlowTok{return} \DecValTok{0}\OperatorTok{;}
\OperatorTok{\}}
\end{Highlighting}
\end{Shaded}

\subsubsection{Why It Matters}\label{why-it-matters-41}

\begin{itemize}
\tightlist
\item
  Pointer mistakes are a leading cause of bugs in C programs.
\item
  Undefined behavior may look harmless in small tests, but break
  programs later.
\item
  Safe habits (null checks, freeing, bounds checking) make C code much
  more robust.
\end{itemize}

\subsubsection{Exercises}\label{exercises-42}

\begin{enumerate}
\def\labelenumi{\arabic{enumi}.}
\tightlist
\item
  Write a function that returns a pointer to a local variable. Run it
  and observe. Then fix it with \texttt{malloc}.
\item
  Create a program that allocates memory but forgets to \texttt{free}.
  Use a loop to make the leak visible. Then fix it.
\item
  Demonstrate a double free and then fix it with \texttt{p\ =\ NULL}.
\item
  Write a program that reads \texttt{n}, allocates an array of size
  \texttt{n}, and prints it. Add a check for out-of-bounds access.
\item
  Experiment: declare an uninitialized pointer and dereference it. Then
  fix by initializing with \texttt{NULL} and checking before use.
\end{enumerate}

\subsection{\texorpdfstring{9.4 Safer Allocations in C23
(\texttt{aligned\_alloc},
\texttt{nullptr})}{9.4 Safer Allocations in C23 (aligned\_alloc, nullptr)}}\label{safer-allocations-in-c23-aligned_alloc-nullptr}

C23 introduced improvements that make memory management less
error-prone. Two key features are:

\begin{enumerate}
\def\labelenumi{\arabic{enumi}.}
\tightlist
\item
  \texttt{aligned\_alloc} - request memory with a specific alignment
\item
  \texttt{nullptr} - a new null pointer constant
\end{enumerate}

\subsubsection{\texorpdfstring{\texttt{aligned\_alloc}}{aligned\_alloc}}\label{aligned_alloc}

Normal \texttt{malloc} returns memory suitable for any type, but
sometimes you need aligned memory (e.g., for SIMD instructions, hardware
buffers).

Prototype (in \texttt{\textless{}stdlib.h\textgreater{}}):

\begin{Shaded}
\begin{Highlighting}[]
\DataTypeTok{void} \OperatorTok{*}\NormalTok{aligned\_alloc}\OperatorTok{(}\DataTypeTok{size\_t}\NormalTok{ alignment}\OperatorTok{,} \DataTypeTok{size\_t}\NormalTok{ size}\OperatorTok{);}
\end{Highlighting}
\end{Shaded}

\begin{itemize}
\tightlist
\item
  \texttt{alignment} must be a power of two (e.g., 16, 32).
\item
  \texttt{size} must be a multiple of \texttt{alignment}.
\item
  Returns \texttt{NULL} on failure.
\end{itemize}

\subsubsection{Example: Allocate 32-byte aligned
array}\label{example-allocate-32-byte-aligned-array}

\begin{Shaded}
\begin{Highlighting}[]
\PreprocessorTok{\#include }\ImportTok{\textless{}stdio.h\textgreater{}}
\PreprocessorTok{\#include }\ImportTok{\textless{}stdlib.h\textgreater{}}

\DataTypeTok{int}\NormalTok{ main}\OperatorTok{(}\DataTypeTok{void}\OperatorTok{)} \OperatorTok{\{}
    \DataTypeTok{size\_t}\NormalTok{ n }\OperatorTok{=} \DecValTok{8}\OperatorTok{;}
    \DataTypeTok{int} \OperatorTok{*}\NormalTok{a }\OperatorTok{=}\NormalTok{ aligned\_alloc}\OperatorTok{(}\DecValTok{32}\OperatorTok{,}\NormalTok{ n }\OperatorTok{{-}} \KeywordTok{sizeof}\OperatorTok{(}\DataTypeTok{int}\OperatorTok{));}
    \ControlFlowTok{if} \OperatorTok{(!}\NormalTok{a}\OperatorTok{)} \ControlFlowTok{return} \DecValTok{1}\OperatorTok{;}
    \ControlFlowTok{for} \OperatorTok{(}\DataTypeTok{size\_t}\NormalTok{ i }\OperatorTok{=} \DecValTok{0}\OperatorTok{;}\NormalTok{ i }\OperatorTok{\textless{}}\NormalTok{ n}\OperatorTok{;}\NormalTok{ i}\OperatorTok{++)}\NormalTok{ a}\OperatorTok{[}\NormalTok{i}\OperatorTok{]} \OperatorTok{=}\NormalTok{ i}\OperatorTok{;}
    \ControlFlowTok{for} \OperatorTok{(}\DataTypeTok{size\_t}\NormalTok{ i }\OperatorTok{=} \DecValTok{0}\OperatorTok{;}\NormalTok{ i }\OperatorTok{\textless{}}\NormalTok{ n}\OperatorTok{;}\NormalTok{ i}\OperatorTok{++)}\NormalTok{ printf}\OperatorTok{(}\StringTok{"}\SpecialCharTok{\%d}\StringTok{ "}\OperatorTok{,}\NormalTok{ a}\OperatorTok{[}\NormalTok{i}\OperatorTok{]);}
\NormalTok{    printf}\OperatorTok{(}\StringTok{"}\SpecialCharTok{\textbackslash{}n}\StringTok{"}\OperatorTok{);}
\NormalTok{    free}\OperatorTok{(}\NormalTok{a}\OperatorTok{);}
    \ControlFlowTok{return} \DecValTok{0}\OperatorTok{;}
\OperatorTok{\}}
\end{Highlighting}
\end{Shaded}

\subsubsection{\texorpdfstring{\texttt{nullptr}}{nullptr}}\label{nullptr}

Traditionally, C used \texttt{NULL} (defined as \texttt{((void-)0)} or
\texttt{0}) to represent a null pointer. This could be confusing because
\texttt{0} is also an integer.

C23 introduces \texttt{nullptr} as a dedicated null pointer constant.

\begin{Shaded}
\begin{Highlighting}[]
\PreprocessorTok{\#include }\ImportTok{\textless{}stdio.h\textgreater{}}

\DataTypeTok{int}\NormalTok{ main}\OperatorTok{(}\DataTypeTok{void}\OperatorTok{)} \OperatorTok{\{}
    \DataTypeTok{int} \OperatorTok{{-}}\NormalTok{p }\OperatorTok{=} \KeywordTok{nullptr}\OperatorTok{;}  \CommentTok{// safe null initialization}
    \ControlFlowTok{if} \OperatorTok{(}\NormalTok{p }\OperatorTok{==} \KeywordTok{nullptr}\OperatorTok{)} \OperatorTok{\{}
\NormalTok{        printf}\OperatorTok{(}\StringTok{"Pointer is null}\SpecialCharTok{\textbackslash{}n}\StringTok{"}\OperatorTok{);}
    \OperatorTok{\}}
    \ControlFlowTok{return} \DecValTok{0}\OperatorTok{;}
\OperatorTok{\}}
\end{Highlighting}
\end{Shaded}

\subsubsection{\texorpdfstring{Why \texttt{nullptr} is
safer:}{Why nullptr is safer:}}\label{why-nullptr-is-safer}

\begin{itemize}
\tightlist
\item
  No confusion with integers.
\item
  Makes code more readable.
\item
  Aligns C with modern C++ style.
\end{itemize}

\subsubsection{Safer Practices in C23}\label{safer-practices-in-c23}

\begin{itemize}
\item
  Prefer \texttt{nullptr} instead of \texttt{NULL}.
\item
  Use \texttt{aligned\_alloc} when alignment matters (e.g., vectorized
  math, GPU buffers).
\item
  Always check for allocation failure:

\begin{Shaded}
\begin{Highlighting}[]
\ControlFlowTok{if} \OperatorTok{(}\NormalTok{p }\OperatorTok{==} \KeywordTok{nullptr}\OperatorTok{)} \OperatorTok{\{} \OperatorTok{/{-}}\NormalTok{ handle error }\OperatorTok{{-}/} \OperatorTok{\}}
\end{Highlighting}
\end{Shaded}
\end{itemize}

\subsubsection{Why It Matters}\label{why-it-matters-42}

\begin{itemize}
\tightlist
\item
  \texttt{aligned\_alloc} enables performance optimizations with aligned
  memory.
\item
  \texttt{nullptr} avoids common bugs with \texttt{NULL} being treated
  as integer \texttt{0}.
\item
  Together, they represent a step toward safer, clearer C code.
\end{itemize}

\subsubsection{Exercises}\label{exercises-43}

\begin{enumerate}
\def\labelenumi{\arabic{enumi}.}
\tightlist
\item
  Allocate a dynamic array of 16 \texttt{int}s using
  \texttt{aligned\_alloc(16,\ …)}. Fill and print it.
\item
  Write a program that initializes a pointer with \texttt{nullptr} and
  checks it before use.
\item
  Compare \texttt{p\ =\ NULL;} and \texttt{p\ =\ nullptr;} in a simple
  program. Print the result of \texttt{(p\ ==\ 0)}.
\item
  Allocate a block of 64 bytes with alignment 32. Print its address to
  confirm divisibility by 32.
\item
  Write a program that combines both: initialize a pointer with
  \texttt{nullptr}, then allocate aligned memory and use it.
\end{enumerate}

\subsection{9.5 A Mini Dynamic Array
Implementation}\label{a-mini-dynamic-array-implementation}

Dynamic arrays are one of the most common data structures. In C, there
is no built-in ``vector'' type like in C++ - but we can build one using
pointers, \texttt{malloc}, \texttt{realloc}, and \texttt{free}.

\subsubsection{Step 1: Define a Struct for the
Array}\label{step-1-define-a-struct-for-the-array}

We need to track:

\begin{itemize}
\tightlist
\item
  A pointer to the data
\item
  The number of elements (\texttt{size})
\item
  The capacity (allocated space)
\end{itemize}

\begin{Shaded}
\begin{Highlighting}[]
\PreprocessorTok{\#include }\ImportTok{\textless{}stddef.h\textgreater{}}\PreprocessorTok{  }\CommentTok{// for size\_t}

\KeywordTok{typedef} \KeywordTok{struct} \OperatorTok{\{}
    \DataTypeTok{int} \OperatorTok{{-}}\NormalTok{data}\OperatorTok{;}
    \DataTypeTok{size\_t}\NormalTok{ size}\OperatorTok{;}
    \DataTypeTok{size\_t}\NormalTok{ capacity}\OperatorTok{;}
\OperatorTok{\}}\NormalTok{ DynArray}\OperatorTok{;}
\end{Highlighting}
\end{Shaded}

\subsubsection{Step 2: Initialize}\label{step-2-initialize}

\begin{Shaded}
\begin{Highlighting}[]
\PreprocessorTok{\#include }\ImportTok{\textless{}stdlib.h\textgreater{}}

\DataTypeTok{void}\NormalTok{ init\_array}\OperatorTok{(}\NormalTok{DynArray }\OperatorTok{*}\NormalTok{a}\OperatorTok{,} \DataTypeTok{size\_t}\NormalTok{ initial\_capacity}\OperatorTok{)} \OperatorTok{\{}
\NormalTok{    a}\OperatorTok{{-}\textgreater{}}\NormalTok{data }\OperatorTok{=}\NormalTok{ malloc}\OperatorTok{(}\NormalTok{initial\_capacity }\OperatorTok{{-}} \KeywordTok{sizeof}\OperatorTok{(}\DataTypeTok{int}\OperatorTok{));}
\NormalTok{    a}\OperatorTok{{-}\textgreater{}}\NormalTok{size }\OperatorTok{=} \DecValTok{0}\OperatorTok{;}
\NormalTok{    a}\OperatorTok{{-}\textgreater{}}\NormalTok{capacity }\OperatorTok{=} \OperatorTok{(}\NormalTok{a}\OperatorTok{{-}\textgreater{}}\NormalTok{data }\OperatorTok{?}\NormalTok{ initial\_capacity }\OperatorTok{:} \DecValTok{0}\OperatorTok{);}
\OperatorTok{\}}
\end{Highlighting}
\end{Shaded}

\subsubsection{Step 3: Add Elements (Resize if
Full)}\label{step-3-add-elements-resize-if-full}

\begin{Shaded}
\begin{Highlighting}[]
\PreprocessorTok{\#include }\ImportTok{\textless{}string.h\textgreater{}}

\DataTypeTok{int}\NormalTok{ push\_back}\OperatorTok{(}\NormalTok{DynArray }\OperatorTok{*}\NormalTok{a}\OperatorTok{,} \DataTypeTok{int}\NormalTok{ value}\OperatorTok{)} \OperatorTok{\{}
    \ControlFlowTok{if} \OperatorTok{(}\NormalTok{a}\OperatorTok{{-}\textgreater{}}\NormalTok{size }\OperatorTok{==}\NormalTok{ a}\OperatorTok{{-}\textgreater{}}\NormalTok{capacity}\OperatorTok{)} \OperatorTok{\{}
        \DataTypeTok{size\_t}\NormalTok{ new\_capacity }\OperatorTok{=} \OperatorTok{(}\NormalTok{a}\OperatorTok{{-}\textgreater{}}\NormalTok{capacity }\OperatorTok{==} \DecValTok{0}\OperatorTok{)} \OperatorTok{?} \DecValTok{1} \OperatorTok{:}\NormalTok{ a}\OperatorTok{{-}\textgreater{}}\NormalTok{capacity }\OperatorTok{{-}} \DecValTok{2}\OperatorTok{;}
        \DataTypeTok{int} \OperatorTok{{-}}\NormalTok{new\_data }\OperatorTok{=}\NormalTok{ realloc}\OperatorTok{(}\NormalTok{a}\OperatorTok{{-}\textgreater{}}\NormalTok{data}\OperatorTok{,}\NormalTok{ new\_capacity }\OperatorTok{{-}} \KeywordTok{sizeof}\OperatorTok{(}\DataTypeTok{int}\OperatorTok{));}
        \ControlFlowTok{if} \OperatorTok{(!}\NormalTok{new\_data}\OperatorTok{)} \ControlFlowTok{return} \DecValTok{0}\OperatorTok{;}  \CommentTok{// fail}
\NormalTok{        a}\OperatorTok{{-}\textgreater{}}\NormalTok{data }\OperatorTok{=}\NormalTok{ new\_data}\OperatorTok{;}
\NormalTok{        a}\OperatorTok{{-}\textgreater{}}\NormalTok{capacity }\OperatorTok{=}\NormalTok{ new\_capacity}\OperatorTok{;}
    \OperatorTok{\}}
\NormalTok{    a}\OperatorTok{{-}\textgreater{}}\NormalTok{data}\OperatorTok{[}\NormalTok{a}\OperatorTok{{-}\textgreater{}}\NormalTok{size}\OperatorTok{++]} \OperatorTok{=}\NormalTok{ value}\OperatorTok{;}
    \ControlFlowTok{return} \DecValTok{1}\OperatorTok{;} \CommentTok{// success}
\OperatorTok{\}}
\end{Highlighting}
\end{Shaded}

\subsubsection{Step 4: Free the Array}\label{step-4-free-the-array}

\begin{Shaded}
\begin{Highlighting}[]
\DataTypeTok{void}\NormalTok{ free\_array}\OperatorTok{(}\NormalTok{DynArray }\OperatorTok{*}\NormalTok{a}\OperatorTok{)} \OperatorTok{\{}
\NormalTok{    free}\OperatorTok{(}\NormalTok{a}\OperatorTok{{-}\textgreater{}}\NormalTok{data}\OperatorTok{);}
\NormalTok{    a}\OperatorTok{{-}\textgreater{}}\NormalTok{data }\OperatorTok{=}\NormalTok{ NULL}\OperatorTok{;}
\NormalTok{    a}\OperatorTok{{-}\textgreater{}}\NormalTok{size }\OperatorTok{=} \DecValTok{0}\OperatorTok{;}
\NormalTok{    a}\OperatorTok{{-}\textgreater{}}\NormalTok{capacity }\OperatorTok{=} \DecValTok{0}\OperatorTok{;}
\OperatorTok{\}}
\end{Highlighting}
\end{Shaded}

\subsubsection{Step 5: Example Program}\label{step-5-example-program}

\begin{Shaded}
\begin{Highlighting}[]
\PreprocessorTok{\#include }\ImportTok{\textless{}stdio.h\textgreater{}}
\PreprocessorTok{\#include }\ImportTok{\textless{}stdlib.h\textgreater{}}

\KeywordTok{typedef} \KeywordTok{struct} \OperatorTok{\{}
    \DataTypeTok{int} \OperatorTok{{-}}\NormalTok{data}\OperatorTok{;}
    \DataTypeTok{size\_t}\NormalTok{ size}\OperatorTok{;}
    \DataTypeTok{size\_t}\NormalTok{ capacity}\OperatorTok{;}
\OperatorTok{\}}\NormalTok{ DynArray}\OperatorTok{;}

\DataTypeTok{void}\NormalTok{ init\_array}\OperatorTok{(}\NormalTok{DynArray }\OperatorTok{*}\NormalTok{a}\OperatorTok{,} \DataTypeTok{size\_t}\NormalTok{ initial\_capacity}\OperatorTok{)} \OperatorTok{\{}
\NormalTok{    a}\OperatorTok{{-}\textgreater{}}\NormalTok{data }\OperatorTok{=}\NormalTok{ malloc}\OperatorTok{(}\NormalTok{initial\_capacity }\OperatorTok{{-}} \KeywordTok{sizeof}\OperatorTok{(}\DataTypeTok{int}\OperatorTok{));}
\NormalTok{    a}\OperatorTok{{-}\textgreater{}}\NormalTok{size }\OperatorTok{=} \DecValTok{0}\OperatorTok{;}
\NormalTok{    a}\OperatorTok{{-}\textgreater{}}\NormalTok{capacity }\OperatorTok{=} \OperatorTok{(}\NormalTok{a}\OperatorTok{{-}\textgreater{}}\NormalTok{data }\OperatorTok{?}\NormalTok{ initial\_capacity }\OperatorTok{:} \DecValTok{0}\OperatorTok{);}
\OperatorTok{\}}

\DataTypeTok{int}\NormalTok{ push\_back}\OperatorTok{(}\NormalTok{DynArray }\OperatorTok{*}\NormalTok{a}\OperatorTok{,} \DataTypeTok{int}\NormalTok{ value}\OperatorTok{)} \OperatorTok{\{}
    \ControlFlowTok{if} \OperatorTok{(}\NormalTok{a}\OperatorTok{{-}\textgreater{}}\NormalTok{size }\OperatorTok{==}\NormalTok{ a}\OperatorTok{{-}\textgreater{}}\NormalTok{capacity}\OperatorTok{)} \OperatorTok{\{}
        \DataTypeTok{size\_t}\NormalTok{ new\_capacity }\OperatorTok{=} \OperatorTok{(}\NormalTok{a}\OperatorTok{{-}\textgreater{}}\NormalTok{capacity }\OperatorTok{==} \DecValTok{0}\OperatorTok{)} \OperatorTok{?} \DecValTok{1} \OperatorTok{:}\NormalTok{ a}\OperatorTok{{-}\textgreater{}}\NormalTok{capacity }\OperatorTok{{-}} \DecValTok{2}\OperatorTok{;}
        \DataTypeTok{int} \OperatorTok{{-}}\NormalTok{new\_data }\OperatorTok{=}\NormalTok{ realloc}\OperatorTok{(}\NormalTok{a}\OperatorTok{{-}\textgreater{}}\NormalTok{data}\OperatorTok{,}\NormalTok{ new\_capacity }\OperatorTok{{-}} \KeywordTok{sizeof}\OperatorTok{(}\DataTypeTok{int}\OperatorTok{));}
        \ControlFlowTok{if} \OperatorTok{(!}\NormalTok{new\_data}\OperatorTok{)} \ControlFlowTok{return} \DecValTok{0}\OperatorTok{;}
\NormalTok{        a}\OperatorTok{{-}\textgreater{}}\NormalTok{data }\OperatorTok{=}\NormalTok{ new\_data}\OperatorTok{;}
\NormalTok{        a}\OperatorTok{{-}\textgreater{}}\NormalTok{capacity }\OperatorTok{=}\NormalTok{ new\_capacity}\OperatorTok{;}
    \OperatorTok{\}}
\NormalTok{    a}\OperatorTok{{-}\textgreater{}}\NormalTok{data}\OperatorTok{[}\NormalTok{a}\OperatorTok{{-}\textgreater{}}\NormalTok{size}\OperatorTok{++]} \OperatorTok{=}\NormalTok{ value}\OperatorTok{;}
    \ControlFlowTok{return} \DecValTok{1}\OperatorTok{;}
\OperatorTok{\}}

\DataTypeTok{void}\NormalTok{ free\_array}\OperatorTok{(}\NormalTok{DynArray }\OperatorTok{*}\NormalTok{a}\OperatorTok{)} \OperatorTok{\{}
\NormalTok{    free}\OperatorTok{(}\NormalTok{a}\OperatorTok{{-}\textgreater{}}\NormalTok{data}\OperatorTok{);}
\NormalTok{    a}\OperatorTok{{-}\textgreater{}}\NormalTok{data }\OperatorTok{=}\NormalTok{ NULL}\OperatorTok{;}
\NormalTok{    a}\OperatorTok{{-}\textgreater{}}\NormalTok{size }\OperatorTok{=} \DecValTok{0}\OperatorTok{;}
\NormalTok{    a}\OperatorTok{{-}\textgreater{}}\NormalTok{capacity }\OperatorTok{=} \DecValTok{0}\OperatorTok{;}
\OperatorTok{\}}

\DataTypeTok{int}\NormalTok{ main}\OperatorTok{(}\DataTypeTok{void}\OperatorTok{)} \OperatorTok{\{}
\NormalTok{    DynArray arr}\OperatorTok{;}
\NormalTok{    init\_array}\OperatorTok{(\&}\NormalTok{arr}\OperatorTok{,} \DecValTok{2}\OperatorTok{);}

    \ControlFlowTok{for} \OperatorTok{(}\DataTypeTok{int}\NormalTok{ i }\OperatorTok{=} \DecValTok{1}\OperatorTok{;}\NormalTok{ i }\OperatorTok{\textless{}=} \DecValTok{10}\OperatorTok{;}\NormalTok{ i}\OperatorTok{++)} \OperatorTok{\{}
\NormalTok{        push\_back}\OperatorTok{(\&}\NormalTok{arr}\OperatorTok{,}\NormalTok{ i }\OperatorTok{{-}}\NormalTok{ i}\OperatorTok{);}
    \OperatorTok{\}}

    \ControlFlowTok{for} \OperatorTok{(}\DataTypeTok{size\_t}\NormalTok{ i }\OperatorTok{=} \DecValTok{0}\OperatorTok{;}\NormalTok{ i }\OperatorTok{\textless{}}\NormalTok{ arr}\OperatorTok{.}\NormalTok{size}\OperatorTok{;}\NormalTok{ i}\OperatorTok{++)} \OperatorTok{\{}
\NormalTok{        printf}\OperatorTok{(}\StringTok{"}\SpecialCharTok{\%d}\StringTok{ "}\OperatorTok{,}\NormalTok{ arr}\OperatorTok{.}\NormalTok{data}\OperatorTok{[}\NormalTok{i}\OperatorTok{]);}
    \OperatorTok{\}}
\NormalTok{    printf}\OperatorTok{(}\StringTok{"}\SpecialCharTok{\textbackslash{}n}\StringTok{"}\OperatorTok{);}

\NormalTok{    free\_array}\OperatorTok{(\&}\NormalTok{arr}\OperatorTok{);}
    \ControlFlowTok{return} \DecValTok{0}\OperatorTok{;}
\OperatorTok{\}}
\end{Highlighting}
\end{Shaded}

Output:

\begin{verbatim}
1 4 9 16 25 36 49 64 81 100
\end{verbatim}

\subsubsection{Why It Matters}\label{why-it-matters-43}

\begin{itemize}
\tightlist
\item
  Shows how to build a resizable container in C.
\item
  Demonstrates safe use of \texttt{malloc}, \texttt{realloc}, and
  \texttt{free}.
\item
  This pattern underlies real libraries like glib dynamic arrays or C++
  vectors.
\end{itemize}

\subsubsection{Exercises}\label{exercises-44}

\begin{enumerate}
\def\labelenumi{\arabic{enumi}.}
\tightlist
\item
  Extend the \texttt{DynArray} with a function
  \texttt{get(DynArray\ *a,\ size\_t\ index)} that safely returns an
  element (or error).
\item
  Write a function \texttt{pop\_back} that removes the last element.
\item
  Modify \texttt{push\_back} so that it shrinks the array when too empty
  (optional).
\item
  Store \texttt{double} instead of \texttt{int}. What changes?
\item
  Write a small program that reads numbers from the user until EOF and
  stores them in a \texttt{DynArray}. Print them back.
\end{enumerate}

\subsection{Problems}\label{problems-7}

\subsubsection{1. Automatic Variable
Lifetime}\label{automatic-variable-lifetime}

Write a function that declares an automatic variable, assigns it a
value, and prints it. Call the function twice. What do you observe?

\subsubsection{2. Pointer to Automatic
Variable}\label{pointer-to-automatic-variable}

Write a function that returns a pointer to a local variable. Use it in
\texttt{main}. What happens? Fix it with \texttt{malloc}.

\subsubsection{\texorpdfstring{3. Simple \texttt{malloc}
Allocation}{3. Simple malloc Allocation}}\label{simple-malloc-allocation}

Use \texttt{malloc} to allocate space for an integer, assign it
\texttt{42}, print it, and then free the memory.

\subsubsection{\texorpdfstring{4. Array with
\texttt{malloc}}{4. Array with malloc}}\label{array-with-malloc}

Ask the user for \texttt{n}, allocate an array of \texttt{n} integers
with \texttt{malloc}, fill with 1..n, and print them. Free the memory
afterward.

\subsubsection{\texorpdfstring{5. Array with
\texttt{calloc}}{5. Array with calloc}}\label{array-with-calloc}

Repeat problem 4, but use \texttt{calloc}. Print the values right after
allocation to show they are initialized to zero.

\subsubsection{\texorpdfstring{6. Compare \texttt{malloc}
vs.~\texttt{calloc}}{6. Compare malloc vs.~calloc}}\label{compare-malloc-vs.-calloc}

Allocate the same array with \texttt{malloc} and \texttt{calloc}. Print
the contents before writing anything. What's the difference?

\subsubsection{7. Freeing and Reusing
Memory}\label{freeing-and-reusing-memory}

Allocate memory for an integer, free it, then try to use it again. What
happens? Fix it by setting the pointer to \texttt{NULL}.

\subsubsection{8. Double Free Error}\label{double-free-error}

Write a program that frees the same pointer twice. Observe the behavior
(may crash). Then fix it.

\subsubsection{9. Memory Leak Demo}\label{memory-leak-demo}

Write a loop that calls \texttt{malloc} repeatedly without calling
\texttt{free}. Watch memory usage grow. Then fix it by freeing properly.

\subsubsection{10. Out-of-Bounds Access}\label{out-of-bounds-access-1}

Allocate an array of size 5 with \texttt{malloc}. Try to access index 5
(6th element). Observe the result. Then fix with proper bounds checking.

\subsubsection{11. Safe Allocation
Pattern}\label{safe-allocation-pattern}

Write a function that allocates an array with \texttt{malloc}, checks
for \texttt{NULL}, fills it, and returns it to the caller. Free it in
\texttt{main}.

\subsubsection{12. Aligned Allocation}\label{aligned-allocation}

Use \texttt{aligned\_alloc} to allocate an array of 16 integers aligned
to 32 bytes. Print the address and confirm divisibility by 32.

\subsubsection{\texorpdfstring{13. \texttt{nullptr}
vs.~\texttt{NULL}}{13. nullptr vs.~NULL}}\label{nullptr-vs.-null}

Write a program that declares one pointer with \texttt{NULL} and another
with \texttt{nullptr}. Print both and compare.

\subsubsection{14. Safe Null Checks}\label{safe-null-checks}

Initialize a pointer with \texttt{nullptr}, check it before
dereferencing, then assign memory with \texttt{malloc} and use it
safely.

\subsubsection{\texorpdfstring{15. Shrinking with
\texttt{realloc}}{15. Shrinking with realloc}}\label{shrinking-with-realloc}

Allocate an array of 10 integers with \texttt{malloc}. Fill with 1..10.
Then shrink it to size 5 with \texttt{realloc}. Print results.

\subsubsection{\texorpdfstring{16. Growing with
\texttt{realloc}}{16. Growing with realloc}}\label{growing-with-realloc}

Allocate an array of 5 integers. Fill with 1..5. Then grow it to size 10
with \texttt{realloc}. Fill the new elements with squares of 6..10 and
print all.

\subsubsection{17. Memory Fragmentation}\label{memory-fragmentation}

Allocate and free blocks of different sizes in a loop. Observe if
allocations succeed. Discuss fragmentation risk.

\subsubsection{18. Mini Dynamic Array - Push
Back}\label{mini-dynamic-array---push-back}

Implement a dynamic array with a \texttt{push\_back} function that
doubles capacity when full. Test with 10 numbers.

\subsubsection{19. Mini Dynamic Array - Pop
Back}\label{mini-dynamic-array---pop-back}

Extend problem 18 with a \texttt{pop\_back} function that removes the
last element. Print results after each operation.

\subsubsection{20. Mini Dynamic Array -
Generalize}\label{mini-dynamic-array---generalize}

Modify the dynamic array to store \texttt{double} instead of
\texttt{int}. Test by pushing 10 floating-point numbers.

\bookmarksetup{startatroot}

\chapter{Part IV. Working with the Real
World}\label{part-iv.-working-with-the-real-world}

\section{Chapter 10. Files}\label{chapter-10.-files}

\subsection{10.1 Reading and Writing
Files}\label{reading-and-writing-files}

Programs often need to save data (to a file) and load it later. C does
this through the standard I/O library
\texttt{\textless{}stdio.h\textgreater{}} using a \texttt{FILE\ -}
handle.

\subsubsection{Opening a File}\label{opening-a-file}

Use \texttt{fopen(path,\ mode)} to get a \texttt{FILE-}. Check for
\texttt{NULL} (open failed).

Common text modes:

\begin{itemize}
\tightlist
\item
  \texttt{"r"} → read (file must exist)
\item
  \texttt{"w"} → write (creates/truncates)
\item
  \texttt{"a"} → append (creates if missing)
\item
  Add \texttt{+} for read/write (e.g., \texttt{"r+"}, \texttt{"w+"},
  \texttt{"a+"})
\end{itemize}

\begin{Shaded}
\begin{Highlighting}[]
\DataTypeTok{FILE} \OperatorTok{{-}}\NormalTok{f }\OperatorTok{=}\NormalTok{ fopen}\OperatorTok{(}\StringTok{"data.txt"}\OperatorTok{,} \StringTok{"r"}\OperatorTok{);}
\ControlFlowTok{if} \OperatorTok{(}\NormalTok{f }\OperatorTok{==}\NormalTok{ NULL}\OperatorTok{)} \OperatorTok{\{} \OperatorTok{/{-}}\NormalTok{ handle error }\OperatorTok{{-}/} \OperatorTok{\}}
\end{Highlighting}
\end{Shaded}

Close with:

\begin{Shaded}
\begin{Highlighting}[]
\NormalTok{fclose}\OperatorTok{(}\NormalTok{f}\OperatorTok{);}
\end{Highlighting}
\end{Shaded}

\subsubsection{Writing Text}\label{writing-text}

\texttt{fprintf} works like \texttt{printf}, but to a file.
\texttt{fputs} writes a string, \texttt{fputc} a character.

\begin{Shaded}
\begin{Highlighting}[]
\PreprocessorTok{\#include }\ImportTok{\textless{}stdio.h\textgreater{}}

\DataTypeTok{int}\NormalTok{ main}\OperatorTok{(}\DataTypeTok{void}\OperatorTok{)} \OperatorTok{\{}
    \DataTypeTok{FILE} \OperatorTok{{-}}\NormalTok{f }\OperatorTok{=}\NormalTok{ fopen}\OperatorTok{(}\StringTok{"scores.txt"}\OperatorTok{,} \StringTok{"w"}\OperatorTok{);}
    \ControlFlowTok{if} \OperatorTok{(!}\NormalTok{f}\OperatorTok{)} \OperatorTok{\{}\NormalTok{ perror}\OperatorTok{(}\StringTok{"open"}\OperatorTok{);} \ControlFlowTok{return} \DecValTok{1}\OperatorTok{;} \OperatorTok{\}}

\NormalTok{    fprintf}\OperatorTok{(}\NormalTok{f}\OperatorTok{,} \StringTok{"Alice }\SpecialCharTok{\%d\textbackslash{}n}\StringTok{"}\OperatorTok{,} \DecValTok{95}\OperatorTok{);}
\NormalTok{    fprintf}\OperatorTok{(}\NormalTok{f}\OperatorTok{,} \StringTok{"Bob }\SpecialCharTok{\%d\textbackslash{}n}\StringTok{"}\OperatorTok{,}   \DecValTok{82}\OperatorTok{);}
\NormalTok{    fputs}\OperatorTok{(}\StringTok{"Carol 88}\SpecialCharTok{\textbackslash{}n}\StringTok{"}\OperatorTok{,}\NormalTok{ f}\OperatorTok{);}

    \ControlFlowTok{if} \OperatorTok{(}\NormalTok{fclose}\OperatorTok{(}\NormalTok{f}\OperatorTok{)} \OperatorTok{==}\NormalTok{ EOF}\OperatorTok{)} \OperatorTok{\{}\NormalTok{ perror}\OperatorTok{(}\StringTok{"close"}\OperatorTok{);} \OperatorTok{\}}
    \ControlFlowTok{return} \DecValTok{0}\OperatorTok{;}
\OperatorTok{\}}
\end{Highlighting}
\end{Shaded}

\subsubsection{Reading Text}\label{reading-text}

\texttt{fscanf} parses formatted text. \texttt{fgets} reads a whole line
(including spaces) into a buffer.

\begin{Shaded}
\begin{Highlighting}[]
\PreprocessorTok{\#include }\ImportTok{\textless{}stdio.h\textgreater{}}

\DataTypeTok{int}\NormalTok{ main}\OperatorTok{(}\DataTypeTok{void}\OperatorTok{)} \OperatorTok{\{}
    \DataTypeTok{FILE} \OperatorTok{{-}}\NormalTok{f }\OperatorTok{=}\NormalTok{ fopen}\OperatorTok{(}\StringTok{"scores.txt"}\OperatorTok{,} \StringTok{"r"}\OperatorTok{);}
    \ControlFlowTok{if} \OperatorTok{(!}\NormalTok{f}\OperatorTok{)} \OperatorTok{\{}\NormalTok{ perror}\OperatorTok{(}\StringTok{"open"}\OperatorTok{);} \ControlFlowTok{return} \DecValTok{1}\OperatorTok{;} \OperatorTok{\}}

    \DataTypeTok{char}\NormalTok{ name}\OperatorTok{[}\DecValTok{64}\OperatorTok{];}
    \DataTypeTok{int}\NormalTok{ score}\OperatorTok{;}

    \ControlFlowTok{while} \OperatorTok{(}\NormalTok{fscanf}\OperatorTok{(}\NormalTok{f}\OperatorTok{,} \StringTok{"}\SpecialCharTok{\%63s}\StringTok{ }\SpecialCharTok{\%d}\StringTok{"}\OperatorTok{,}\NormalTok{ name}\OperatorTok{,} \OperatorTok{\&}\NormalTok{score}\OperatorTok{)} \OperatorTok{==} \DecValTok{2}\OperatorTok{)} \OperatorTok{\{}
\NormalTok{        printf}\OperatorTok{(}\StringTok{"}\SpecialCharTok{\%s}\StringTok{ {-}\textgreater{} }\SpecialCharTok{\%d\textbackslash{}n}\StringTok{"}\OperatorTok{,}\NormalTok{ name}\OperatorTok{,}\NormalTok{ score}\OperatorTok{);}
    \OperatorTok{\}}

\NormalTok{    fclose}\OperatorTok{(}\NormalTok{f}\OperatorTok{);}
    \ControlFlowTok{return} \DecValTok{0}\OperatorTok{;}
\OperatorTok{\}}
\end{Highlighting}
\end{Shaded}

Reading full lines with \texttt{fgets}:

\begin{Shaded}
\begin{Highlighting}[]
\DataTypeTok{char}\NormalTok{ line}\OperatorTok{[}\DecValTok{128}\OperatorTok{];}
\ControlFlowTok{while} \OperatorTok{(}\NormalTok{fgets}\OperatorTok{(}\NormalTok{line}\OperatorTok{,} \KeywordTok{sizeof}\NormalTok{ line}\OperatorTok{,}\NormalTok{ f}\OperatorTok{))} \OperatorTok{\{}
\NormalTok{    printf}\OperatorTok{(}\StringTok{"line: }\SpecialCharTok{\%s}\StringTok{"}\OperatorTok{,}\NormalTok{ line}\OperatorTok{);}  \CommentTok{// line already has \textquotesingle{}\textbackslash{}n\textquotesingle{} if present}
\OperatorTok{\}}
\end{Highlighting}
\end{Shaded}

\subsubsection{Checking Errors and EOF}\label{checking-errors-and-eof}

Most file functions signal problems:

\begin{itemize}
\tightlist
\item
  Return \texttt{NULL}, \texttt{EOF}, or a short count.
\item
  Use \texttt{feof(f)} (end-of-file) and \texttt{ferror(f)} (error
  happened).
\item
  \texttt{perror("msg")} prints a human-readable error based on
  \texttt{errno}.
\end{itemize}

\begin{Shaded}
\begin{Highlighting}[]
\ControlFlowTok{if} \OperatorTok{(}\NormalTok{ferror}\OperatorTok{(}\NormalTok{f}\OperatorTok{))} \OperatorTok{\{}\NormalTok{ perror}\OperatorTok{(}\StringTok{"read error"}\OperatorTok{);} \OperatorTok{\}}
\end{Highlighting}
\end{Shaded}

\subsubsection{A Full Example: Copy Lines With
Numbers}\label{a-full-example-copy-lines-with-numbers}

This program:

\begin{enumerate}
\def\labelenumi{\arabic{enumi}.}
\tightlist
\item
  Writes a few lines to a file.
\item
  Reopens it to read and prepend line numbers, writing to a second file.
\end{enumerate}

\begin{Shaded}
\begin{Highlighting}[]
\PreprocessorTok{\#include }\ImportTok{\textless{}stdio.h\textgreater{}}
\PreprocessorTok{\#include }\ImportTok{\textless{}stdlib.h\textgreater{}}

\DataTypeTok{int}\NormalTok{ main}\OperatorTok{(}\DataTypeTok{void}\OperatorTok{)} \OperatorTok{\{}
    \CommentTok{// 1) Write sample file}
    \OperatorTok{\{}
        \DataTypeTok{FILE} \OperatorTok{{-}}\NormalTok{out }\OperatorTok{=}\NormalTok{ fopen}\OperatorTok{(}\StringTok{"poem.txt"}\OperatorTok{,} \StringTok{"w"}\OperatorTok{);}
        \ControlFlowTok{if} \OperatorTok{(!}\NormalTok{out}\OperatorTok{)} \OperatorTok{\{}\NormalTok{ perror}\OperatorTok{(}\StringTok{"open poem.txt for write"}\OperatorTok{);} \ControlFlowTok{return} \DecValTok{1}\OperatorTok{;} \OperatorTok{\}}
\NormalTok{        fputs}\OperatorTok{(}\StringTok{"Roses are red}\SpecialCharTok{\textbackslash{}n}\StringTok{"}\OperatorTok{,}\NormalTok{ out}\OperatorTok{);}
\NormalTok{        fputs}\OperatorTok{(}\StringTok{"Violets are blue}\SpecialCharTok{\textbackslash{}n}\StringTok{"}\OperatorTok{,}\NormalTok{ out}\OperatorTok{);}
\NormalTok{        fputs}\OperatorTok{(}\StringTok{"C is powerful}\SpecialCharTok{\textbackslash{}n}\StringTok{"}\OperatorTok{,}\NormalTok{ out}\OperatorTok{);}
\NormalTok{        fputs}\OperatorTok{(}\StringTok{"And speedy too}\SpecialCharTok{\textbackslash{}n}\StringTok{"}\OperatorTok{,}\NormalTok{ out}\OperatorTok{);}
        \ControlFlowTok{if} \OperatorTok{(}\NormalTok{fclose}\OperatorTok{(}\NormalTok{out}\OperatorTok{)} \OperatorTok{==}\NormalTok{ EOF}\OperatorTok{)} \OperatorTok{\{}\NormalTok{ perror}\OperatorTok{(}\StringTok{"close poem.txt"}\OperatorTok{);} \ControlFlowTok{return} \DecValTok{1}\OperatorTok{;} \OperatorTok{\}}
    \OperatorTok{\}}

    \CommentTok{// 2) Read poem.txt, write numbered\_poem.txt}
    \DataTypeTok{FILE} \OperatorTok{{-}}\NormalTok{in  }\OperatorTok{=}\NormalTok{ fopen}\OperatorTok{(}\StringTok{"poem.txt"}\OperatorTok{,} \StringTok{"r"}\OperatorTok{);}
    \ControlFlowTok{if} \OperatorTok{(!}\NormalTok{in}\OperatorTok{)} \OperatorTok{\{}\NormalTok{ perror}\OperatorTok{(}\StringTok{"open poem.txt for read"}\OperatorTok{);} \ControlFlowTok{return} \DecValTok{1}\OperatorTok{;} \OperatorTok{\}}

    \DataTypeTok{FILE} \OperatorTok{{-}}\NormalTok{out }\OperatorTok{=}\NormalTok{ fopen}\OperatorTok{(}\StringTok{"numbered\_poem.txt"}\OperatorTok{,} \StringTok{"w"}\OperatorTok{);}
    \ControlFlowTok{if} \OperatorTok{(!}\NormalTok{out}\OperatorTok{)} \OperatorTok{\{}\NormalTok{ perror}\OperatorTok{(}\StringTok{"open numbered\_poem.txt"}\OperatorTok{);}\NormalTok{ fclose}\OperatorTok{(}\NormalTok{in}\OperatorTok{);} \ControlFlowTok{return} \DecValTok{1}\OperatorTok{;} \OperatorTok{\}}

    \DataTypeTok{char}\NormalTok{ line}\OperatorTok{[}\DecValTok{256}\OperatorTok{];}
    \DataTypeTok{int}\NormalTok{ n }\OperatorTok{=} \DecValTok{1}\OperatorTok{;}

    \ControlFlowTok{while} \OperatorTok{(}\NormalTok{fgets}\OperatorTok{(}\NormalTok{line}\OperatorTok{,} \KeywordTok{sizeof}\NormalTok{ line}\OperatorTok{,}\NormalTok{ in}\OperatorTok{))} \OperatorTok{\{}
\NormalTok{        fprintf}\OperatorTok{(}\NormalTok{out}\OperatorTok{,} \StringTok{"}\SpecialCharTok{\%02d}\StringTok{: }\SpecialCharTok{\%s}\StringTok{"}\OperatorTok{,}\NormalTok{ n}\OperatorTok{++,}\NormalTok{ line}\OperatorTok{);} \CommentTok{// line already ends with \textquotesingle{}\textbackslash{}n\textquotesingle{} (usually)}
    \OperatorTok{\}}

    \ControlFlowTok{if} \OperatorTok{(}\NormalTok{ferror}\OperatorTok{(}\NormalTok{in}\OperatorTok{))}  \OperatorTok{\{}\NormalTok{ perror}\OperatorTok{(}\StringTok{"read poem.txt"}\OperatorTok{);} \OperatorTok{\}}
    \ControlFlowTok{if} \OperatorTok{(}\NormalTok{fclose}\OperatorTok{(}\NormalTok{in}\OperatorTok{)} \OperatorTok{==}\NormalTok{ EOF}\OperatorTok{)}  \OperatorTok{\{}\NormalTok{ perror}\OperatorTok{(}\StringTok{"close poem.txt"}\OperatorTok{);} \OperatorTok{\}}
    \ControlFlowTok{if} \OperatorTok{(}\NormalTok{fclose}\OperatorTok{(}\NormalTok{out}\OperatorTok{)} \OperatorTok{==}\NormalTok{ EOF}\OperatorTok{)} \OperatorTok{\{}\NormalTok{ perror}\OperatorTok{(}\StringTok{"close numbered\_poem.txt"}\OperatorTok{);} \OperatorTok{\}}

\NormalTok{    puts}\OperatorTok{(}\StringTok{"Wrote numbered\_poem.txt"}\OperatorTok{);}
    \ControlFlowTok{return} \DecValTok{0}\OperatorTok{;}
\OperatorTok{\}}
\end{Highlighting}
\end{Shaded}

\subsubsection{Tips \& Gotchas}\label{tips-gotchas}

\begin{itemize}
\tightlist
\item
  Always check \texttt{fopen} for \texttt{NULL}.
\item
  Always close files with \texttt{fclose}.
\item
  When parsing with \texttt{fscanf}, check the return value (how many
  items read).
\item
  Prefer \texttt{fgets} + manual parsing if lines may contain spaces or
  complicated formats.
\item
  Text vs.~binary differences (line endings, encoding) are covered in
  10.2.
\end{itemize}

\subsubsection{Why It Matters}\label{why-it-matters-44}

File I/O lets your programs persist data, process logs, import/export
information, and communicate with other tools. It's a core skill in C.

\subsubsection{Exercises}\label{exercises-45}

\begin{enumerate}
\def\labelenumi{\arabic{enumi}.}
\tightlist
\item
  Create \texttt{numbers.txt} with the integers 1--10, each on its own
  line (use \texttt{fprintf}).
\item
  Read \texttt{numbers.txt}, compute the sum, and print it to stdout.
\item
  Read lines from \texttt{stdin} with \texttt{fgets} and write only
  lines longer than 10 characters to \texttt{long.txt}.
\item
  Make a program that copies a text file to another file, preserving
  content exactly (use \texttt{fgets}/\texttt{fputs}).
\item
  Read a file containing \texttt{name\ score} pairs and print the
  highest score and the name that achieved it.
\end{enumerate}

\subsection{10.2 Text vs.~Binary Files}\label{text-vs.-binary-files}

Files in C can be opened in text mode or binary mode. The difference is
how the data is interpreted and possibly transformed by the operating
system.

\subsubsection{Text Mode}\label{text-mode}

\begin{itemize}
\tightlist
\item
  Default mode if you use \texttt{"r"}, \texttt{"w"}, or \texttt{"a"}.
\item
  Data is stored as human-readable characters.
\item
  On Windows, \texttt{\textbackslash{}n} is converted to
  \texttt{\textbackslash{}r\textbackslash{}n} when writing, and reversed
  when reading.
\item
  Portable for structured text: logs, CSV, configuration files.
\end{itemize}

Example:

\begin{Shaded}
\begin{Highlighting}[]
\PreprocessorTok{\#include }\ImportTok{\textless{}stdio.h\textgreater{}}

\DataTypeTok{int}\NormalTok{ main}\OperatorTok{(}\DataTypeTok{void}\OperatorTok{)} \OperatorTok{\{}
    \DataTypeTok{FILE} \OperatorTok{{-}}\NormalTok{f }\OperatorTok{=}\NormalTok{ fopen}\OperatorTok{(}\StringTok{"hello.txt"}\OperatorTok{,} \StringTok{"w"}\OperatorTok{);}
    \ControlFlowTok{if} \OperatorTok{(!}\NormalTok{f}\OperatorTok{)} \ControlFlowTok{return} \DecValTok{1}\OperatorTok{;}
\NormalTok{    fprintf}\OperatorTok{(}\NormalTok{f}\OperatorTok{,} \StringTok{"Line1}\SpecialCharTok{\textbackslash{}n}\StringTok{Line2}\SpecialCharTok{\textbackslash{}n}\StringTok{"}\OperatorTok{);}
\NormalTok{    fclose}\OperatorTok{(}\NormalTok{f}\OperatorTok{);}
    \ControlFlowTok{return} \DecValTok{0}\OperatorTok{;}
\OperatorTok{\}}
\end{Highlighting}
\end{Shaded}

\subsubsection{Binary Mode}\label{binary-mode}

\begin{itemize}
\tightlist
\item
  Use \texttt{"rb"}, \texttt{"wb"}, \texttt{"ab"} (append \texttt{b} for
  binary).
\item
  Data is stored exactly as bytes in memory.
\item
  No translation of line endings.
\item
  Useful for images, audio, compiled data, or any non-text format.
\end{itemize}

Example:

\begin{Shaded}
\begin{Highlighting}[]
\PreprocessorTok{\#include }\ImportTok{\textless{}stdio.h\textgreater{}}

\DataTypeTok{int}\NormalTok{ main}\OperatorTok{(}\DataTypeTok{void}\OperatorTok{)} \OperatorTok{\{}
    \DataTypeTok{FILE} \OperatorTok{{-}}\NormalTok{f }\OperatorTok{=}\NormalTok{ fopen}\OperatorTok{(}\StringTok{"data.bin"}\OperatorTok{,} \StringTok{"wb"}\OperatorTok{);}
    \ControlFlowTok{if} \OperatorTok{(!}\NormalTok{f}\OperatorTok{)} \ControlFlowTok{return} \DecValTok{1}\OperatorTok{;}

    \DataTypeTok{int}\NormalTok{ nums}\OperatorTok{[}\DecValTok{3}\OperatorTok{]} \OperatorTok{=} \OperatorTok{\{}\DecValTok{10}\OperatorTok{,} \DecValTok{20}\OperatorTok{,} \DecValTok{30}\OperatorTok{\};}
\NormalTok{    fwrite}\OperatorTok{(}\NormalTok{nums}\OperatorTok{,} \KeywordTok{sizeof}\OperatorTok{(}\DataTypeTok{int}\OperatorTok{),} \DecValTok{3}\OperatorTok{,}\NormalTok{ f}\OperatorTok{);} \CommentTok{// write raw bytes}
\NormalTok{    fclose}\OperatorTok{(}\NormalTok{f}\OperatorTok{);}
    \ControlFlowTok{return} \DecValTok{0}\OperatorTok{;}
\OperatorTok{\}}
\end{Highlighting}
\end{Shaded}

Reading back:

\begin{Shaded}
\begin{Highlighting}[]
\PreprocessorTok{\#include }\ImportTok{\textless{}stdio.h\textgreater{}}

\DataTypeTok{int}\NormalTok{ main}\OperatorTok{(}\DataTypeTok{void}\OperatorTok{)} \OperatorTok{\{}
    \DataTypeTok{FILE} \OperatorTok{{-}}\NormalTok{f }\OperatorTok{=}\NormalTok{ fopen}\OperatorTok{(}\StringTok{"data.bin"}\OperatorTok{,} \StringTok{"rb"}\OperatorTok{);}
    \ControlFlowTok{if} \OperatorTok{(!}\NormalTok{f}\OperatorTok{)} \ControlFlowTok{return} \DecValTok{1}\OperatorTok{;}

    \DataTypeTok{int}\NormalTok{ nums}\OperatorTok{[}\DecValTok{3}\OperatorTok{];}
\NormalTok{    fread}\OperatorTok{(}\NormalTok{nums}\OperatorTok{,} \KeywordTok{sizeof}\OperatorTok{(}\DataTypeTok{int}\OperatorTok{),} \DecValTok{3}\OperatorTok{,}\NormalTok{ f}\OperatorTok{);}
\NormalTok{    fclose}\OperatorTok{(}\NormalTok{f}\OperatorTok{);}

    \ControlFlowTok{for} \OperatorTok{(}\DataTypeTok{int}\NormalTok{ i }\OperatorTok{=} \DecValTok{0}\OperatorTok{;}\NormalTok{ i }\OperatorTok{\textless{}} \DecValTok{3}\OperatorTok{;}\NormalTok{ i}\OperatorTok{++)}\NormalTok{ printf}\OperatorTok{(}\StringTok{"}\SpecialCharTok{\%d}\StringTok{ "}\OperatorTok{,}\NormalTok{ nums}\OperatorTok{[}\NormalTok{i}\OperatorTok{]);}
\NormalTok{    printf}\OperatorTok{(}\StringTok{"}\SpecialCharTok{\textbackslash{}n}\StringTok{"}\OperatorTok{);}
    \ControlFlowTok{return} \DecValTok{0}\OperatorTok{;}
\OperatorTok{\}}
\end{Highlighting}
\end{Shaded}

\subsubsection{When to Use Text
vs.~Binary}\label{when-to-use-text-vs.-binary}

\begin{itemize}
\tightlist
\item
  Text mode: when humans will read or edit the file.
\item
  Binary mode: when performance matters, or you need to store raw data
  compactly.
\item
  Same data may take more space in text mode (numbers as characters)
  than binary.
\end{itemize}

\subsubsection{Exercises}\label{exercises-46}

\begin{enumerate}
\def\labelenumi{\arabic{enumi}.}
\tightlist
\item
  Write numbers 1--10 to a file in text mode using \texttt{fprintf}.
  Open the file in a text editor and inspect.
\item
  Write the same numbers to a binary file using \texttt{fwrite}. Open
  the file in a text editor and compare.
\item
  Create a struct with two fields (\texttt{id}, \texttt{score}) and
  write an array of 3 structs to a binary file.
\item
  Read the binary file from exercise 3 back and print the values.
\item
  Measure the size of a text file storing 1000 integers and compare it
  to the binary version.
\end{enumerate}

\subsection{10.3 Error Handling in File
Operations}\label{error-handling-in-file-operations}

When working with files, many things can go wrong:

\begin{itemize}
\tightlist
\item
  The file might not exist.
\item
  The disk might be full.
\item
  A read or write may fail.
\end{itemize}

C provides mechanisms in \texttt{\textless{}stdio.h\textgreater{}} to
detect and handle these conditions.

\subsubsection{Return Values}\label{return-values-1}

Most file functions signal errors through return values:

\begin{itemize}
\tightlist
\item
  \texttt{fopen} → returns \texttt{NULL} if the file cannot be opened.
\item
  \texttt{fclose} → returns \texttt{EOF} on failure.
\item
  \texttt{fread} / \texttt{fwrite} → return the number of items actually
  read or written (less than requested on error or end-of-file).
\item
  \texttt{fprintf} / \texttt{fputs} / \texttt{fputc} → return a negative
  value on failure.
\end{itemize}

Always check these return values.

\subsubsection{Checking Error State}\label{checking-error-state}

Two functions let you query the file stream:

\begin{itemize}
\tightlist
\item
  \texttt{feof(FILE\ -f)} → nonzero if end-of-file was reached.
\item
  \texttt{ferror(FILE\ -f)} → nonzero if an error occurred.
\end{itemize}

Resetting errors:

\begin{Shaded}
\begin{Highlighting}[]
\NormalTok{clearerr}\OperatorTok{(}\NormalTok{f}\OperatorTok{);}
\end{Highlighting}
\end{Shaded}

\subsubsection{\texorpdfstring{Using \texttt{perror} and
\texttt{errno}}{Using perror and errno}}\label{using-perror-and-errno}

Many library calls set the global variable \texttt{errno} on error. Use
\texttt{perror("msg")} to print a descriptive error message.

Example:

\begin{Shaded}
\begin{Highlighting}[]
\PreprocessorTok{\#include }\ImportTok{\textless{}stdio.h\textgreater{}}

\DataTypeTok{int}\NormalTok{ main}\OperatorTok{(}\DataTypeTok{void}\OperatorTok{)} \OperatorTok{\{}
    \DataTypeTok{FILE} \OperatorTok{{-}}\NormalTok{f }\OperatorTok{=}\NormalTok{ fopen}\OperatorTok{(}\StringTok{"nofile.txt"}\OperatorTok{,} \StringTok{"r"}\OperatorTok{);}
    \ControlFlowTok{if} \OperatorTok{(!}\NormalTok{f}\OperatorTok{)} \OperatorTok{\{}
\NormalTok{        perror}\OperatorTok{(}\StringTok{"open failed"}\OperatorTok{);}
        \ControlFlowTok{return} \DecValTok{1}\OperatorTok{;}
    \OperatorTok{\}}
\NormalTok{    fclose}\OperatorTok{(}\NormalTok{f}\OperatorTok{);}
    \ControlFlowTok{return} \DecValTok{0}\OperatorTok{;}
\OperatorTok{\}}
\end{Highlighting}
\end{Shaded}

Output might look like:

\begin{verbatim}
open failed: No such file or directory
\end{verbatim}

\subsubsection{Example: Robust File
Copy}\label{example-robust-file-copy}

\begin{Shaded}
\begin{Highlighting}[]
\PreprocessorTok{\#include }\ImportTok{\textless{}stdio.h\textgreater{}}
\PreprocessorTok{\#include }\ImportTok{\textless{}stdlib.h\textgreater{}}

\DataTypeTok{int}\NormalTok{ main}\OperatorTok{(}\DataTypeTok{void}\OperatorTok{)} \OperatorTok{\{}
    \DataTypeTok{FILE} \OperatorTok{{-}}\NormalTok{in }\OperatorTok{=}\NormalTok{ fopen}\OperatorTok{(}\StringTok{"source.txt"}\OperatorTok{,} \StringTok{"r"}\OperatorTok{);}
    \ControlFlowTok{if} \OperatorTok{(!}\NormalTok{in}\OperatorTok{)} \OperatorTok{\{}\NormalTok{ perror}\OperatorTok{(}\StringTok{"open source.txt"}\OperatorTok{);} \ControlFlowTok{return} \DecValTok{1}\OperatorTok{;} \OperatorTok{\}}

    \DataTypeTok{FILE} \OperatorTok{{-}}\NormalTok{out }\OperatorTok{=}\NormalTok{ fopen}\OperatorTok{(}\StringTok{"dest.txt"}\OperatorTok{,} \StringTok{"w"}\OperatorTok{);}
    \ControlFlowTok{if} \OperatorTok{(!}\NormalTok{out}\OperatorTok{)} \OperatorTok{\{}\NormalTok{ perror}\OperatorTok{(}\StringTok{"open dest.txt"}\OperatorTok{);}\NormalTok{ fclose}\OperatorTok{(}\NormalTok{in}\OperatorTok{);} \ControlFlowTok{return} \DecValTok{1}\OperatorTok{;} \OperatorTok{\}}

    \DataTypeTok{char}\NormalTok{ buf}\OperatorTok{[}\DecValTok{256}\OperatorTok{];}
    \DataTypeTok{size\_t}\NormalTok{ n}\OperatorTok{;}
    \ControlFlowTok{while} \OperatorTok{((}\NormalTok{n }\OperatorTok{=}\NormalTok{ fread}\OperatorTok{(}\NormalTok{buf}\OperatorTok{,} \DecValTok{1}\OperatorTok{,} \KeywordTok{sizeof}\NormalTok{ buf}\OperatorTok{,}\NormalTok{ in}\OperatorTok{))} \OperatorTok{\textgreater{}} \DecValTok{0}\OperatorTok{)} \OperatorTok{\{}
        \ControlFlowTok{if} \OperatorTok{(}\NormalTok{fwrite}\OperatorTok{(}\NormalTok{buf}\OperatorTok{,} \DecValTok{1}\OperatorTok{,}\NormalTok{ n}\OperatorTok{,}\NormalTok{ out}\OperatorTok{)} \OperatorTok{!=}\NormalTok{ n}\OperatorTok{)} \OperatorTok{\{}
\NormalTok{            perror}\OperatorTok{(}\StringTok{"write error"}\OperatorTok{);}
\NormalTok{            fclose}\OperatorTok{(}\NormalTok{in}\OperatorTok{);}
\NormalTok{            fclose}\OperatorTok{(}\NormalTok{out}\OperatorTok{);}
            \ControlFlowTok{return} \DecValTok{1}\OperatorTok{;}
        \OperatorTok{\}}
    \OperatorTok{\}}

    \ControlFlowTok{if} \OperatorTok{(}\NormalTok{ferror}\OperatorTok{(}\NormalTok{in}\OperatorTok{))}\NormalTok{ perror}\OperatorTok{(}\StringTok{"read error"}\OperatorTok{);}

    \ControlFlowTok{if} \OperatorTok{(}\NormalTok{fclose}\OperatorTok{(}\NormalTok{in}\OperatorTok{)} \OperatorTok{==}\NormalTok{ EOF}\OperatorTok{)}\NormalTok{ perror}\OperatorTok{(}\StringTok{"close in"}\OperatorTok{);}
    \ControlFlowTok{if} \OperatorTok{(}\NormalTok{fclose}\OperatorTok{(}\NormalTok{out}\OperatorTok{)} \OperatorTok{==}\NormalTok{ EOF}\OperatorTok{)}\NormalTok{ perror}\OperatorTok{(}\StringTok{"close out"}\OperatorTok{);}

    \ControlFlowTok{return} \DecValTok{0}\OperatorTok{;}
\OperatorTok{\}}
\end{Highlighting}
\end{Shaded}

\subsubsection{Why It Matters}\label{why-it-matters-45}

Ignoring return values can lead to silent data loss. Robust programs
check errors and handle them gracefully, printing helpful messages for
debugging or recovery.

\subsubsection{Exercises}\label{exercises-47}

\begin{enumerate}
\def\labelenumi{\arabic{enumi}.}
\tightlist
\item
  Attempt to open a non-existent file with
  \texttt{fopen("missing.txt",\ "r")}. Detect the error with
  \texttt{perror}.
\item
  Write a program that reads integers from a file with \texttt{fscanf}.
  Stop correctly on end-of-file, not on error.
\item
  Modify the file copy program to detect and report both read and write
  errors.
\item
  Use \texttt{clearerr} to reset an error state on a file and continue
  reading.
\item
  Experiment: open a file in read-only mode and try to write to it.
  Capture the error with \texttt{ferror}.
\end{enumerate}

\subsection{10.4 Building a Simple Log
Writer}\label{building-a-simple-log-writer}

A common task in programs is writing logs: recording events, errors, or
data points to a file. This section shows how to build a minimal log
writer in C.

\subsubsection{Opening a Log File}\label{opening-a-log-file}

A log usually grows over time, so use append mode (\texttt{"a"}) to add
new entries without overwriting old ones.

\begin{Shaded}
\begin{Highlighting}[]
\PreprocessorTok{\#include }\ImportTok{\textless{}stdio.h\textgreater{}}

\DataTypeTok{int}\NormalTok{ main}\OperatorTok{(}\DataTypeTok{void}\OperatorTok{)} \OperatorTok{\{}
    \DataTypeTok{FILE} \OperatorTok{{-}}\NormalTok{log }\OperatorTok{=}\NormalTok{ fopen}\OperatorTok{(}\StringTok{"app.log"}\OperatorTok{,} \StringTok{"a"}\OperatorTok{);}
    \ControlFlowTok{if} \OperatorTok{(!}\NormalTok{log}\OperatorTok{)} \OperatorTok{\{}\NormalTok{ perror}\OperatorTok{(}\StringTok{"open log"}\OperatorTok{);} \ControlFlowTok{return} \DecValTok{1}\OperatorTok{;} \OperatorTok{\}}
\NormalTok{    fprintf}\OperatorTok{(}\NormalTok{log}\OperatorTok{,} \StringTok{"Program started}\SpecialCharTok{\textbackslash{}n}\StringTok{"}\OperatorTok{);}
\NormalTok{    fclose}\OperatorTok{(}\NormalTok{log}\OperatorTok{);}
    \ControlFlowTok{return} \DecValTok{0}\OperatorTok{;}
\OperatorTok{\}}
\end{Highlighting}
\end{Shaded}

\subsubsection{Adding Timestamps}\label{adding-timestamps}

Logs are more useful with timestamps. Use
\texttt{\textless{}time.h\textgreater{}} to get the current time.

\begin{Shaded}
\begin{Highlighting}[]
\PreprocessorTok{\#include }\ImportTok{\textless{}stdio.h\textgreater{}}
\PreprocessorTok{\#include }\ImportTok{\textless{}time.h\textgreater{}}

\DataTypeTok{void}\NormalTok{ write\_log}\OperatorTok{(}\DataTypeTok{const} \DataTypeTok{char} \OperatorTok{{-}}\NormalTok{msg}\OperatorTok{)} \OperatorTok{\{}
    \DataTypeTok{FILE} \OperatorTok{{-}}\NormalTok{log }\OperatorTok{=}\NormalTok{ fopen}\OperatorTok{(}\StringTok{"app.log"}\OperatorTok{,} \StringTok{"a"}\OperatorTok{);}
    \ControlFlowTok{if} \OperatorTok{(!}\NormalTok{log}\OperatorTok{)} \ControlFlowTok{return}\OperatorTok{;}

    \DataTypeTok{time\_t}\NormalTok{ now }\OperatorTok{=}\NormalTok{ time}\OperatorTok{(}\NormalTok{NULL}\OperatorTok{);}
    \DataTypeTok{char} \OperatorTok{{-}}\NormalTok{ts }\OperatorTok{=}\NormalTok{ ctime}\OperatorTok{(\&}\NormalTok{now}\OperatorTok{);}      \CommentTok{// returns string with \textquotesingle{}\textbackslash{}n\textquotesingle{}}
\NormalTok{    ts}\OperatorTok{[}\DecValTok{24}\OperatorTok{]} \OperatorTok{=} \CharTok{\textquotesingle{}}\SpecialCharTok{\textbackslash{}0}\CharTok{\textquotesingle{}}\OperatorTok{;}               \CommentTok{// remove newline}

\NormalTok{    fprintf}\OperatorTok{(}\NormalTok{log}\OperatorTok{,} \StringTok{"[}\SpecialCharTok{\%s}\StringTok{] }\SpecialCharTok{\%s\textbackslash{}n}\StringTok{"}\OperatorTok{,}\NormalTok{ ts}\OperatorTok{,}\NormalTok{ msg}\OperatorTok{);}
\NormalTok{    fclose}\OperatorTok{(}\NormalTok{log}\OperatorTok{);}
\OperatorTok{\}}
\end{Highlighting}
\end{Shaded}

Usage:

\begin{Shaded}
\begin{Highlighting}[]
\DataTypeTok{int}\NormalTok{ main}\OperatorTok{(}\DataTypeTok{void}\OperatorTok{)} \OperatorTok{\{}
\NormalTok{    write\_log}\OperatorTok{(}\StringTok{"Application started"}\OperatorTok{);}
\NormalTok{    write\_log}\OperatorTok{(}\StringTok{"An event occurred"}\OperatorTok{);}
    \ControlFlowTok{return} \DecValTok{0}\OperatorTok{;}
\OperatorTok{\}}
\end{Highlighting}
\end{Shaded}

\subsubsection{Flushing the Log}\label{flushing-the-log}

If the program might crash or be interrupted, flush data immediately
after writing.

\begin{Shaded}
\begin{Highlighting}[]
\NormalTok{fprintf}\OperatorTok{(}\NormalTok{log}\OperatorTok{,} \StringTok{"Critical error}\SpecialCharTok{\textbackslash{}n}\StringTok{"}\OperatorTok{);}
\NormalTok{fflush}\OperatorTok{(}\NormalTok{log}\OperatorTok{);}
\end{Highlighting}
\end{Shaded}

\texttt{fflush} forces buffered output to be written to the file.

\subsubsection{Log Levels}\label{log-levels}

Use levels (\texttt{INFO}, \texttt{WARN}, \texttt{ERROR}) for clarity.

\begin{Shaded}
\begin{Highlighting}[]
\DataTypeTok{void}\NormalTok{ log\_msg}\OperatorTok{(}\DataTypeTok{const} \DataTypeTok{char} \OperatorTok{{-}}\NormalTok{level}\OperatorTok{,} \DataTypeTok{const} \DataTypeTok{char} \OperatorTok{{-}}\NormalTok{msg}\OperatorTok{)} \OperatorTok{\{}
    \DataTypeTok{FILE} \OperatorTok{{-}}\NormalTok{log }\OperatorTok{=}\NormalTok{ fopen}\OperatorTok{(}\StringTok{"app.log"}\OperatorTok{,} \StringTok{"a"}\OperatorTok{);}
    \ControlFlowTok{if} \OperatorTok{(!}\NormalTok{log}\OperatorTok{)} \ControlFlowTok{return}\OperatorTok{;}

    \DataTypeTok{time\_t}\NormalTok{ now }\OperatorTok{=}\NormalTok{ time}\OperatorTok{(}\NormalTok{NULL}\OperatorTok{);}
    \DataTypeTok{char} \OperatorTok{{-}}\NormalTok{ts }\OperatorTok{=}\NormalTok{ ctime}\OperatorTok{(\&}\NormalTok{now}\OperatorTok{);}\NormalTok{ ts}\OperatorTok{[}\DecValTok{24}\OperatorTok{]} \OperatorTok{=} \CharTok{\textquotesingle{}}\SpecialCharTok{\textbackslash{}0}\CharTok{\textquotesingle{}}\OperatorTok{;}

\NormalTok{    fprintf}\OperatorTok{(}\NormalTok{log}\OperatorTok{,} \StringTok{"[}\SpecialCharTok{\%s}\StringTok{] }\SpecialCharTok{\%s}\StringTok{: }\SpecialCharTok{\%s\textbackslash{}n}\StringTok{"}\OperatorTok{,}\NormalTok{ ts}\OperatorTok{,}\NormalTok{ level}\OperatorTok{,}\NormalTok{ msg}\OperatorTok{);}
\NormalTok{    fclose}\OperatorTok{(}\NormalTok{log}\OperatorTok{);}
\OperatorTok{\}}
\end{Highlighting}
\end{Shaded}

Example:

\begin{Shaded}
\begin{Highlighting}[]
\NormalTok{log\_msg}\OperatorTok{(}\StringTok{"INFO"}\OperatorTok{,} \StringTok{"Application started"}\OperatorTok{);}
\NormalTok{log\_msg}\OperatorTok{(}\StringTok{"ERROR"}\OperatorTok{,} \StringTok{"Could not open file"}\OperatorTok{);}
\end{Highlighting}
\end{Shaded}

\subsubsection{Why It Matters}\label{why-it-matters-46}

\begin{itemize}
\tightlist
\item
  Logging helps diagnose problems after a program runs.
\item
  Appending ensures past events are preserved.
\item
  Timestamps and levels make logs readable and actionable.
\end{itemize}

\subsubsection{Exercises}\label{exercises-48}

\begin{enumerate}
\def\labelenumi{\arabic{enumi}.}
\tightlist
\item
  Write a \texttt{log\_info(const\ char\ -msg)} function that logs with
  \texttt{"INFO"}.
\item
  Extend with \texttt{log\_warn} and \texttt{log\_error}.
\item
  Write a loop that logs 5 numbered messages. Open the log afterward in
  a text editor to inspect.
\item
  Modify the logging function to also print to \texttt{stdout} while
  writing to the file.
\item
  Add a command-line argument: if \texttt{-\/-clear} is passed, start by
  truncating the log file (\texttt{"w"} mode), otherwise append.
\end{enumerate}

\subsubsection{10.5 File APIs in C23}\label{file-apis-in-c23}

C23 did not radically change file I/O, but it brought some cleanups and
clarifications that make file handling safer and clearer.

\subsubsection{\texorpdfstring{\texttt{nullptr} with File
Pointers}{nullptr with File Pointers}}\label{nullptr-with-file-pointers}

Traditionally, \texttt{fopen} returned \texttt{NULL} on failure. In C23,
you can also use the new null constant \texttt{nullptr} for comparisons,
which avoids confusion with integers.

\begin{Shaded}
\begin{Highlighting}[]
\PreprocessorTok{\#include }\ImportTok{\textless{}stdio.h\textgreater{}}

\DataTypeTok{int}\NormalTok{ main}\OperatorTok{(}\DataTypeTok{void}\OperatorTok{)} \OperatorTok{\{}
    \DataTypeTok{FILE} \OperatorTok{{-}}\NormalTok{f }\OperatorTok{=}\NormalTok{ fopen}\OperatorTok{(}\StringTok{"missing.txt"}\OperatorTok{,} \StringTok{"r"}\OperatorTok{);}
    \ControlFlowTok{if} \OperatorTok{(}\NormalTok{f }\OperatorTok{==} \KeywordTok{nullptr}\OperatorTok{)} \OperatorTok{\{}   \CommentTok{// C23 style}
\NormalTok{        perror}\OperatorTok{(}\StringTok{"open failed"}\OperatorTok{);}
        \ControlFlowTok{return} \DecValTok{1}\OperatorTok{;}
    \OperatorTok{\}}
\NormalTok{    fclose}\OperatorTok{(}\NormalTok{f}\OperatorTok{);}
    \ControlFlowTok{return} \DecValTok{0}\OperatorTok{;}
\OperatorTok{\}}
\end{Highlighting}
\end{Shaded}

\subsubsection{\texorpdfstring{\texttt{fopen\_s} (Optional, Annex
K)}{fopen\_s (Optional, Annex K)}}\label{fopen_s-optional-annex-k}

Some platforms provide safer alternatives like \texttt{fopen\_s}, which
require explicit buffer sizes or return codes. These are optional in the
C standard library. When available:

\begin{Shaded}
\begin{Highlighting}[]
\DataTypeTok{FILE} \OperatorTok{{-}}\NormalTok{f}\OperatorTok{;}
\ControlFlowTok{if} \OperatorTok{(}\NormalTok{fopen\_s}\OperatorTok{(\&}\NormalTok{f}\OperatorTok{,} \StringTok{"data.txt"}\OperatorTok{,} \StringTok{"r"}\OperatorTok{)} \OperatorTok{!=} \DecValTok{0}\OperatorTok{)} \OperatorTok{\{}
\NormalTok{    perror}\OperatorTok{(}\StringTok{"fopen\_s failed"}\OperatorTok{);}
\OperatorTok{\}}
\end{Highlighting}
\end{Shaded}

But note: Annex K functions are not universally supported and may not be
portable.

\subsubsection{\texorpdfstring{\texttt{tmpfile} and
\texttt{tmpnam}}{tmpfile and tmpnam}}\label{tmpfile-and-tmpnam}

\begin{itemize}
\tightlist
\item
  \texttt{tmpfile()} creates a temporary file that is automatically
  removed when closed.
\item
  Useful for scratch space where the file does not need to persist.
\end{itemize}

\begin{Shaded}
\begin{Highlighting}[]
\PreprocessorTok{\#include }\ImportTok{\textless{}stdio.h\textgreater{}}

\DataTypeTok{int}\NormalTok{ main}\OperatorTok{(}\DataTypeTok{void}\OperatorTok{)} \OperatorTok{\{}
    \DataTypeTok{FILE} \OperatorTok{{-}}\NormalTok{t }\OperatorTok{=}\NormalTok{ tmpfile}\OperatorTok{();}
    \ControlFlowTok{if} \OperatorTok{(!}\NormalTok{t}\OperatorTok{)} \OperatorTok{\{}\NormalTok{ perror}\OperatorTok{(}\StringTok{"tmpfile"}\OperatorTok{);} \ControlFlowTok{return} \DecValTok{1}\OperatorTok{;} \OperatorTok{\}}
\NormalTok{    fputs}\OperatorTok{(}\StringTok{"temporary content}\SpecialCharTok{\textbackslash{}n}\StringTok{"}\OperatorTok{,}\NormalTok{ t}\OperatorTok{);}
\NormalTok{    rewind}\OperatorTok{(}\NormalTok{t}\OperatorTok{);}  \CommentTok{// move back to start}
    \DataTypeTok{char}\NormalTok{ buf}\OperatorTok{[}\DecValTok{64}\OperatorTok{];}
\NormalTok{    fgets}\OperatorTok{(}\NormalTok{buf}\OperatorTok{,} \KeywordTok{sizeof}\NormalTok{ buf}\OperatorTok{,}\NormalTok{ t}\OperatorTok{);}
\NormalTok{    printf}\OperatorTok{(}\StringTok{"read: }\SpecialCharTok{\%s}\StringTok{"}\OperatorTok{,}\NormalTok{ buf}\OperatorTok{);}
\NormalTok{    fclose}\OperatorTok{(}\NormalTok{t}\OperatorTok{);}
    \ControlFlowTok{return} \DecValTok{0}\OperatorTok{;}
\OperatorTok{\}}
\end{Highlighting}
\end{Shaded}

\subsubsection{\texorpdfstring{\texttt{freopen} for
Redirection}{freopen for Redirection}}\label{freopen-for-redirection}

\texttt{freopen} can reassign \texttt{stdin}, \texttt{stdout}, or
\texttt{stderr} to a file.

\begin{Shaded}
\begin{Highlighting}[]
\PreprocessorTok{\#include }\ImportTok{\textless{}stdio.h\textgreater{}}

\DataTypeTok{int}\NormalTok{ main}\OperatorTok{(}\DataTypeTok{void}\OperatorTok{)} \OperatorTok{\{}
\NormalTok{    freopen}\OperatorTok{(}\StringTok{"out.txt"}\OperatorTok{,} \StringTok{"w"}\OperatorTok{,}\NormalTok{ stdout}\OperatorTok{);}
\NormalTok{    printf}\OperatorTok{(}\StringTok{"This goes into out.txt}\SpecialCharTok{\textbackslash{}n}\StringTok{"}\OperatorTok{);}
    \ControlFlowTok{return} \DecValTok{0}\OperatorTok{;}
\OperatorTok{\}}
\end{Highlighting}
\end{Shaded}

\subsubsection{Wide Character File I/O}\label{wide-character-file-io}

C23 clarifies wide character I/O functions (\texttt{fwprintf},
\texttt{fwscanf}, etc.) for Unicode text handling. These are useful for
programs that must support international text.

\subsubsection{Why It Matters}\label{why-it-matters-47}

\begin{itemize}
\tightlist
\item
  \texttt{nullptr} makes code cleaner and less error-prone.
\item
  \texttt{tmpfile} and \texttt{freopen} are powerful but often
  overlooked parts of \texttt{\textless{}stdio.h\textgreater{}}.
\item
  Awareness of Annex K APIs helps, but portability should guide choices.
\end{itemize}

\subsubsection{Exercises}\label{exercises-49}

\begin{enumerate}
\def\labelenumi{\arabic{enumi}.}
\tightlist
\item
  Write a program that opens a file using \texttt{fopen}, checks with
  \texttt{nullptr}, and prints \texttt{"ok"} if successful.
\item
  Use \texttt{tmpfile} to create a scratch file, write \texttt{"hello"},
  rewind, and read it back.
\item
  Redirect \texttt{stdout} to a file using \texttt{freopen} and confirm
  output goes there.
\item
  Try writing and reading wide characters with \texttt{fwprintf} and
  \texttt{fwscanf}.
\item
  Research whether your compiler supports \texttt{fopen\_s}. Try it if
  available.
\end{enumerate}

Here's a unified problem set for Chapter 10 (Files). These exercises
cover basic file I/O, text vs.~binary handling, error detection,
logging, and modern C23 file APIs.

\subsection{Problems}\label{problems-8}

\subsubsection{1. Hello File}\label{hello-file}

Write a program that creates \texttt{hello.txt} and writes
\texttt{"Hello,\ File!"} into it.

\subsubsection{2. Write Numbers}\label{write-numbers}

Open \texttt{numbers.txt} for writing. Write integers 1--10, one per
line.

\subsubsection{3. Read Numbers and Sum}\label{read-numbers-and-sum}

Read the numbers back from \texttt{numbers.txt}, compute their sum, and
print it.

\subsubsection{4. Copy Text File}\label{copy-text-file}

Write a program that copies the contents of \texttt{source.txt} into
\texttt{dest.txt} line by line using \texttt{fgets} and \texttt{fputs}.

\subsubsection{5. Long Line Filter}\label{long-line-filter}

Read lines from \texttt{input.txt} and write only those longer than 10
characters to \texttt{long.txt}.

\subsubsection{6. Text vs.~Binary (Write)}\label{text-vs.-binary-write}

Write integers 1--5 to \texttt{ints.txt} using \texttt{fprintf}. Then
write them to \texttt{ints.bin} using \texttt{fwrite}.

\subsubsection{7. Text vs.~Binary (Read)}\label{text-vs.-binary-read}

Read the numbers back from both \texttt{ints.txt} and \texttt{ints.bin},
print them, and confirm they match.

\subsubsection{8. Struct to Binary}\label{struct-to-binary}

Define a struct \texttt{\{\ int\ id;\ double\ score;\ \}}. Write an
array of 3 structs to \texttt{scores.bin} with \texttt{fwrite}.

\subsubsection{9. Read Structs Back}\label{read-structs-back}

Read \texttt{scores.bin} with \texttt{fread} and print the values.

\subsubsection{10. File Error Check}\label{file-error-check}

Attempt to open a missing file with \texttt{fopen("nofile.txt","r")}.
Detect the failure with \texttt{perror}.

\subsubsection{11. Robust Copy with Error
Handling}\label{robust-copy-with-error-handling}

Write a file copy program that reads from \texttt{source.txt} and writes
to \texttt{dest.txt}, checking for read/write errors and reporting them
with \texttt{perror}.

\subsubsection{12. Log Writer}\label{log-writer}

Write a \texttt{log\_msg(const\ char\ -level,\ const\ char\ -msg)}
function that appends entries to \texttt{app.log} with timestamps.

\subsubsection{13. Multiple Log Levels}\label{multiple-log-levels}

Add helpers \texttt{log\_info}, \texttt{log\_warn}, and
\texttt{log\_error} that call \texttt{log\_msg} with the appropriate
level.

\subsubsection{14. Logging Loop}\label{logging-loop}

Write a loop that logs 5 numbered \texttt{"INFO"} messages to
\texttt{app.log}.

\subsubsection{15. Log Flush}\label{log-flush}

Modify \texttt{log\_msg} to \texttt{fflush} after each write, ensuring
messages appear in the log immediately.

\subsubsection{16. Clear vs.~Append}\label{clear-vs.-append}

Modify the logging program so that if run with \texttt{-\/-clear}, it
truncates \texttt{app.log} (open in \texttt{"w"} mode). Otherwise, it
appends.

\subsubsection{\texorpdfstring{17. Use
\texttt{tmpfile}}{17. Use tmpfile}}\label{use-tmpfile}

Write a program that creates a temporary file with \texttt{tmpfile()},
writes \texttt{"scratch\ data"}, rewinds, reads, and prints it.

\subsubsection{\texorpdfstring{18. Use
\texttt{freopen}}{18. Use freopen}}\label{use-freopen}

Redirect \texttt{stdout} to \texttt{out.txt} with \texttt{freopen} and
write \texttt{"redirected\ output"}.

\subsubsection{19. Wide Character I/O}\label{wide-character-io}

Write wide characters to \texttt{wide.txt} using \texttt{fwprintf}, then
read them back with \texttt{fwscanf}.

\subsubsection{\texorpdfstring{20. \texttt{nullptr} in File Handling
(C23)}{20. nullptr in File Handling (C23)}}\label{nullptr-in-file-handling-c23}

Write a program that tries to open a non-existent file. Compare the file
pointer against \texttt{nullptr} instead of \texttt{NULL}, and report
the error with \texttt{perror}.

\section{Chapter 11. Modular
Programming}\label{chapter-11.-modular-programming}

\subsection{11.1 Splitting Code into Multiple
Files}\label{splitting-code-into-multiple-files}

Large programs are easier to manage when you separate interfaces (what
is provided) from implementations (how it works). In C, this usually
means:

\begin{itemize}
\tightlist
\item
  Header files (\texttt{.h}) with -declarations- (function prototypes,
  type definitions, constants).
\item
  Source files (\texttt{.c}) with -definitions- (function bodies,
  private helpers).
\item
  A main file that uses the interface.
\end{itemize}

\subsubsection{Translation Units and Declarations
vs.~Definitions}\label{translation-units-and-declarations-vs.-definitions}

\begin{itemize}
\item
  A translation unit is one \texttt{.c} file after preprocessing
  (\texttt{\#include} expanded).
\item
  Declaration tells the compiler a symbol exists (type, parameters).
  Example (prototype): \texttt{int\ add(int\ a,\ int\ b);}
\item
  Definition provides the actual storage or function body. Example
  (function):

\begin{Shaded}
\begin{Highlighting}[]
\DataTypeTok{int}\NormalTok{ add}\OperatorTok{(}\DataTypeTok{int}\NormalTok{ a}\OperatorTok{,} \DataTypeTok{int}\NormalTok{ b}\OperatorTok{)} \OperatorTok{\{} \ControlFlowTok{return}\NormalTok{ a }\OperatorTok{+}\NormalTok{ b}\OperatorTok{;} \OperatorTok{\}}
\end{Highlighting}
\end{Shaded}
\end{itemize}

You declare in headers so -multiple- \texttt{.c} files can see the
interface, and define in exactly one \texttt{.c} file.

\subsubsection{Include Guards}\label{include-guards}

Headers must prevent double inclusion:

\begin{Shaded}
\begin{Highlighting}[]
\PreprocessorTok{\#ifndef MATHX\_H}
\PreprocessorTok{\#define MATHX\_H}

\DataTypeTok{int}\NormalTok{ add}\OperatorTok{(}\DataTypeTok{int}\NormalTok{ a}\OperatorTok{,} \DataTypeTok{int}\NormalTok{ b}\OperatorTok{);}
\DataTypeTok{int}\NormalTok{ mul}\OperatorTok{(}\DataTypeTok{int}\NormalTok{ a}\OperatorTok{,} \DataTypeTok{int}\NormalTok{ b}\OperatorTok{);}

\PreprocessorTok{\#endif }\CommentTok{// MATHX\_H}
\end{Highlighting}
\end{Shaded}

(Alternatively: \texttt{\#pragma\ once} on compilers that support it.)

\subsubsection{A Minimal 3-File Example}\label{a-minimal-3-file-example}

Goal: split a tiny math library from \texttt{main.c}.

\paragraph{\texorpdfstring{\texttt{mathx.h} - the interface
(declarations)}{mathx.h - the interface (declarations)}}\label{mathx.h---the-interface-declarations}

\begin{Shaded}
\begin{Highlighting}[]
\PreprocessorTok{\#ifndef MATHX\_H}
\PreprocessorTok{\#define MATHX\_H}

\CommentTok{// Function prototypes (declarations)}
\DataTypeTok{int}\NormalTok{ add}\OperatorTok{(}\DataTypeTok{int}\NormalTok{ a}\OperatorTok{,} \DataTypeTok{int}\NormalTok{ b}\OperatorTok{);}
\DataTypeTok{int}\NormalTok{ mul}\OperatorTok{(}\DataTypeTok{int}\NormalTok{ a}\OperatorTok{,} \DataTypeTok{int}\NormalTok{ b}\OperatorTok{);}

\PreprocessorTok{\#endif}
\end{Highlighting}
\end{Shaded}

\paragraph{\texorpdfstring{\texttt{mathx.c} - the implementation
(definitions)}{mathx.c - the implementation (definitions)}}\label{mathx.c---the-implementation-definitions}

\begin{Shaded}
\begin{Highlighting}[]
\PreprocessorTok{\#include }\ImportTok{"mathx.h"}

\CommentTok{// Definitions (bodies)}
\DataTypeTok{int}\NormalTok{ add}\OperatorTok{(}\DataTypeTok{int}\NormalTok{ a}\OperatorTok{,} \DataTypeTok{int}\NormalTok{ b}\OperatorTok{)} \OperatorTok{\{} \ControlFlowTok{return}\NormalTok{ a }\OperatorTok{+}\NormalTok{ b}\OperatorTok{;} \OperatorTok{\}}
\DataTypeTok{int}\NormalTok{ mul}\OperatorTok{(}\DataTypeTok{int}\NormalTok{ a}\OperatorTok{,} \DataTypeTok{int}\NormalTok{ b}\OperatorTok{)} \OperatorTok{\{} \ControlFlowTok{return}\NormalTok{ a }\OperatorTok{{-}}\NormalTok{ b}\OperatorTok{;} \OperatorTok{\}}
\end{Highlighting}
\end{Shaded}

\paragraph{\texorpdfstring{\texttt{main.c} - user of the
library}{main.c - user of the library}}\label{main.c---user-of-the-library}

\begin{Shaded}
\begin{Highlighting}[]
\PreprocessorTok{\#include }\ImportTok{\textless{}stdio.h\textgreater{}}
\PreprocessorTok{\#include }\ImportTok{"mathx.h"}

\DataTypeTok{int}\NormalTok{ main}\OperatorTok{(}\DataTypeTok{void}\OperatorTok{)} \OperatorTok{\{}
\NormalTok{    printf}\OperatorTok{(}\StringTok{"add(3,4) = }\SpecialCharTok{\%d\textbackslash{}n}\StringTok{"}\OperatorTok{,}\NormalTok{ add}\OperatorTok{(}\DecValTok{3}\OperatorTok{,}\DecValTok{4}\OperatorTok{));}
\NormalTok{    printf}\OperatorTok{(}\StringTok{"mul(3,4) = }\SpecialCharTok{\%d\textbackslash{}n}\StringTok{"}\OperatorTok{,}\NormalTok{ mul}\OperatorTok{(}\DecValTok{3}\OperatorTok{,}\DecValTok{4}\OperatorTok{));}
    \ControlFlowTok{return} \DecValTok{0}\OperatorTok{;}
\OperatorTok{\}}
\end{Highlighting}
\end{Shaded}

Compile \& link:

\begin{Shaded}
\begin{Highlighting}[]
\FunctionTok{gcc} \AttributeTok{{-}std}\OperatorTok{=}\NormalTok{c23 }\AttributeTok{{-}Wall} \AttributeTok{{-}Wextra} \AttributeTok{{-}c}\NormalTok{ mathx.c    }\CommentTok{\# compile to mathx.o}
\FunctionTok{gcc} \AttributeTok{{-}std}\OperatorTok{=}\NormalTok{c23 }\AttributeTok{{-}Wall} \AttributeTok{{-}Wextra} \AttributeTok{{-}c}\NormalTok{ main.c     }\CommentTok{\# compile to main.o}
\FunctionTok{gcc}\NormalTok{ main.o mathx.o }\AttributeTok{{-}o}\NormalTok{ app                }\CommentTok{\# link to executable}
\CommentTok{\# or one{-}liner:}
\FunctionTok{gcc} \AttributeTok{{-}std}\OperatorTok{=}\NormalTok{c23 }\AttributeTok{{-}Wall} \AttributeTok{{-}Wextra}\NormalTok{ main.c mathx.c }\AttributeTok{{-}o}\NormalTok{ app}
\end{Highlighting}
\end{Shaded}

Run:

\begin{verbatim}
./app
add(3,4) = 7
mul(3,4) = 12
\end{verbatim}

\subsubsection{Visibility and ``Private''
Helpers}\label{visibility-and-private-helpers}

Functions only used inside one \texttt{.c} file can be marked
\texttt{static} to give them internal linkage (not visible to other
translation units):

\begin{Shaded}
\begin{Highlighting}[]
\CommentTok{// inside mathx.c}
\DataTypeTok{static} \DataTypeTok{int}\NormalTok{ clamp16}\OperatorTok{(}\DataTypeTok{int}\NormalTok{ x}\OperatorTok{)} \OperatorTok{\{}
    \ControlFlowTok{if} \OperatorTok{(}\NormalTok{x }\OperatorTok{\textless{}} \OperatorTok{{-}}\DecValTok{32768}\OperatorTok{)} \ControlFlowTok{return} \OperatorTok{{-}}\DecValTok{32768}\OperatorTok{;}
    \ControlFlowTok{if} \OperatorTok{(}\NormalTok{x }\OperatorTok{\textgreater{}}  \DecValTok{32767}\OperatorTok{)} \ControlFlowTok{return}  \DecValTok{32767}\OperatorTok{;}
    \ControlFlowTok{return}\NormalTok{ x}\OperatorTok{;}
\OperatorTok{\}}
\end{Highlighting}
\end{Shaded}

Do not place such helpers in the header unless they're intended for
public use.

\subsubsection{Common Pitfalls}\label{common-pitfalls}

\begin{itemize}
\tightlist
\item
  Multiple definitions: putting a function body in a header and
  including it in multiple \texttt{.c} files causes duplicate symbol
  errors. Keep -definitions- in exactly one \texttt{.c}.
\item
  Missing prototypes: calling a function without a visible prototype can
  cause wrong calling conventions or implicit declarations*always
  include the header.
\item
  Mismatched signatures: prototype and definition must match exactly
  (return type, parameters).
\end{itemize}

\subsubsection{Folder Layout and
Includes}\label{folder-layout-and-includes}

A tidy layout helps:

\begin{verbatim}
/project
  src/
    main.c
    mathx.c
  include/
    mathx.h
\end{verbatim}

Compile with an include path:

\begin{Shaded}
\begin{Highlighting}[]
\FunctionTok{gcc} \AttributeTok{{-}Iinclude} \AttributeTok{{-}std}\OperatorTok{=}\NormalTok{c23 }\AttributeTok{{-}Wall} \AttributeTok{{-}Wextra}\NormalTok{ src/main.c src/mathx.c }\AttributeTok{{-}o}\NormalTok{ app}
\end{Highlighting}
\end{Shaded}

\subsubsection{(Optional) Tiny Makefile}\label{optional-tiny-makefile}

\begin{Shaded}
\begin{Highlighting}[]
\NormalTok{CC=gcc}
\NormalTok{CFLAGS={-}std=c23 {-}Wall {-}Wextra {-}Iinclude}
\NormalTok{OBJ=build/main.o build/mathx.o}

\NormalTok{app: $(OBJ)}
\NormalTok{    $(CC) $(OBJ) {-}o app}

\NormalTok{build/main.o: src/main.c include/mathx.h}
\NormalTok{    mkdir {-}p build}
\NormalTok{    $(CC) $(CFLAGS) {-}c src/main.c {-}o $@}

\NormalTok{build/mathx.o: src/mathx.c include/mathx.h}
\NormalTok{    mkdir {-}p build}
\NormalTok{    $(CC) $(CFLAGS) {-}c src/mathx.c {-}o $@}

\NormalTok{clean:}
\NormalTok{    rm {-}rf build app}
\end{Highlighting}
\end{Shaded}

\subsubsection{Why This Matters}\label{why-this-matters}

\begin{itemize}
\tightlist
\item
  Encourages modularity and reuse.
\item
  Speeds up builds (only changed \texttt{.c} recompiles).
\item
  Makes interfaces clear, reduces coupling, and prevents
  multiple-definition errors.
\end{itemize}

\subsubsection{Exercises}\label{exercises-50}

\begin{enumerate}
\def\labelenumi{\arabic{enumi}.}
\tightlist
\item
  Split a small math library (\texttt{add}, \texttt{sub}, \texttt{mul},
  \texttt{div\_int}) into \texttt{calc.h}, \texttt{calc.c}, and
  \texttt{main.c}.
\item
  Add a \texttt{static} helper in \texttt{calc.c} that is not exposed
  publicly.
\item
  Introduce a signature mismatch intentionally, observe the compiler
  error, then fix it.
\item
  Move headers to \texttt{include/} and sources to \texttt{src/},
  compile with \texttt{-Iinclude}.
\item
  Convert your shell commands into a minimal Makefile with separate
  compile/link steps.
\end{enumerate}

\subsection{\texorpdfstring{11.2 Header Files (\texttt{.h}) and
Declarations}{11.2 Header Files (.h) and Declarations}}\label{header-files-.h-and-declarations}

In C, header files (\texttt{.h}) provide a way to share declarations
across multiple source files. They act as the ``contract'' between
modules: what functions, constants, and types are available.

\subsubsection{What Goes in a Header?}\label{what-goes-in-a-header}

A header typically contains:

\begin{itemize}
\tightlist
\item
  Function prototypes (declarations, not definitions).
\item
  Type definitions (\texttt{typedef}, \texttt{struct}, \texttt{enum}).
\item
  Constants and macros (\texttt{\#define} or \texttt{const}).
\item
  Extern variables (rare, used sparingly).
\end{itemize}

Example \texttt{mathx.h}:

\begin{Shaded}
\begin{Highlighting}[]
\PreprocessorTok{\#ifndef MATHX\_H}
\PreprocessorTok{\#define MATHX\_H}

\CommentTok{// Function declarations}
\DataTypeTok{int}\NormalTok{ add}\OperatorTok{(}\DataTypeTok{int}\NormalTok{ a}\OperatorTok{,} \DataTypeTok{int}\NormalTok{ b}\OperatorTok{);}
\DataTypeTok{int}\NormalTok{ mul}\OperatorTok{(}\DataTypeTok{int}\NormalTok{ a}\OperatorTok{,} \DataTypeTok{int}\NormalTok{ b}\OperatorTok{);}

\CommentTok{// Type definition}
\KeywordTok{typedef} \KeywordTok{struct} \OperatorTok{\{}
    \DataTypeTok{int}\NormalTok{ x}\OperatorTok{,}\NormalTok{ y}\OperatorTok{;}
\OperatorTok{\}}\NormalTok{ Point}\OperatorTok{;}

\CommentTok{// Constant}
\PreprocessorTok{\#define PI }\FloatTok{3.14159}

\PreprocessorTok{\#endif}
\end{Highlighting}
\end{Shaded}

\subsubsection{Why Use Headers?}\label{why-use-headers}

\begin{itemize}
\tightlist
\item
  Prevents copy-pasting declarations into multiple files.
\item
  Ensures consistency: prototype in \texttt{.h} must match definition in
  \texttt{.c}.
\item
  Allows separation of interface (header) from implementation
  (\texttt{.c}).
\item
  Makes large projects manageable by defining clear module boundaries.
\end{itemize}

\subsubsection{Include Guards}\label{include-guards-1}

Always protect headers against double inclusion:

\begin{Shaded}
\begin{Highlighting}[]
\PreprocessorTok{\#ifndef HEADER\_NAME\_H}
\PreprocessorTok{\#define HEADER\_NAME\_H}
\CommentTok{// ... contents ...}
\PreprocessorTok{\#endif}
\end{Highlighting}
\end{Shaded}

Or modern alternative (supported by most compilers):

\begin{Shaded}
\begin{Highlighting}[]
\PreprocessorTok{\#pragma once}
\end{Highlighting}
\end{Shaded}

\subsubsection{How to Use a Header}\label{how-to-use-a-header}

\begin{enumerate}
\def\labelenumi{\arabic{enumi}.}
\tightlist
\item
  Place declarations in \texttt{foo.h}.
\item
  Place definitions in \texttt{foo.c}.
\item
  In \texttt{main.c} (or another module), include the header:
\end{enumerate}

\begin{Shaded}
\begin{Highlighting}[]
\PreprocessorTok{\#include }\ImportTok{"foo.h"}
\end{Highlighting}
\end{Shaded}

This tells the compiler what functions/types exist, even though the code
lives in another file.

\subsubsection{Example: Library with
Header}\label{example-library-with-header}

\texttt{mathx.h}:

\begin{Shaded}
\begin{Highlighting}[]
\PreprocessorTok{\#ifndef MATHX\_H}
\PreprocessorTok{\#define MATHX\_H}

\DataTypeTok{int}\NormalTok{ add}\OperatorTok{(}\DataTypeTok{int}\NormalTok{ a}\OperatorTok{,} \DataTypeTok{int}\NormalTok{ b}\OperatorTok{);}
\DataTypeTok{int}\NormalTok{ mul}\OperatorTok{(}\DataTypeTok{int}\NormalTok{ a}\OperatorTok{,} \DataTypeTok{int}\NormalTok{ b}\OperatorTok{);}

\PreprocessorTok{\#endif}
\end{Highlighting}
\end{Shaded}

\texttt{mathx.c}:

\begin{Shaded}
\begin{Highlighting}[]
\PreprocessorTok{\#include }\ImportTok{"mathx.h"}

\DataTypeTok{int}\NormalTok{ add}\OperatorTok{(}\DataTypeTok{int}\NormalTok{ a}\OperatorTok{,} \DataTypeTok{int}\NormalTok{ b}\OperatorTok{)} \OperatorTok{\{} \ControlFlowTok{return}\NormalTok{ a }\OperatorTok{+}\NormalTok{ b}\OperatorTok{;} \OperatorTok{\}}
\DataTypeTok{int}\NormalTok{ mul}\OperatorTok{(}\DataTypeTok{int}\NormalTok{ a}\OperatorTok{,} \DataTypeTok{int}\NormalTok{ b}\OperatorTok{)} \OperatorTok{\{} \ControlFlowTok{return}\NormalTok{ a }\OperatorTok{{-}}\NormalTok{ b}\OperatorTok{;} \OperatorTok{\}}
\end{Highlighting}
\end{Shaded}

\texttt{main.c}:

\begin{Shaded}
\begin{Highlighting}[]
\PreprocessorTok{\#include }\ImportTok{\textless{}stdio.h\textgreater{}}
\PreprocessorTok{\#include }\ImportTok{"mathx.h"}

\DataTypeTok{int}\NormalTok{ main}\OperatorTok{(}\DataTypeTok{void}\OperatorTok{)} \OperatorTok{\{}
\NormalTok{    printf}\OperatorTok{(}\StringTok{"}\SpecialCharTok{\%d\textbackslash{}n}\StringTok{"}\OperatorTok{,}\NormalTok{ add}\OperatorTok{(}\DecValTok{2}\OperatorTok{,}\DecValTok{3}\OperatorTok{));}
\NormalTok{    printf}\OperatorTok{(}\StringTok{"}\SpecialCharTok{\%d\textbackslash{}n}\StringTok{"}\OperatorTok{,}\NormalTok{ mul}\OperatorTok{(}\DecValTok{2}\OperatorTok{,}\DecValTok{3}\OperatorTok{));}
    \ControlFlowTok{return} \DecValTok{0}\OperatorTok{;}
\OperatorTok{\}}
\end{Highlighting}
\end{Shaded}

Compile:

\begin{Shaded}
\begin{Highlighting}[]
\FunctionTok{gcc} \AttributeTok{{-}std}\OperatorTok{=}\NormalTok{c23 main.c mathx.c }\AttributeTok{{-}o}\NormalTok{ app}
\end{Highlighting}
\end{Shaded}

\subsubsection{Common Mistakes}\label{common-mistakes-2}

\begin{itemize}
\tightlist
\item
  Placing function definitions in headers: If the header is included in
  multiple \texttt{.c} files, this causes multiple-definition errors.
  Keep -definitions- in \texttt{.c}, -declarations- in \texttt{.h}.
\item
  Forgetting include guards: Leads to redefinition errors.
\item
  Using \texttt{\#include} on \texttt{.c} files: Bad practice - include
  headers, compile \texttt{.c} separately.
\end{itemize}

\subsubsection{Why It Matters}\label{why-it-matters-48}

Headers are the backbone of modular programming in C. They let teams
share interfaces without exposing implementation details, leading to
clearer, more maintainable code.

\subsubsection{Exercises}\label{exercises-51}

\begin{enumerate}
\def\labelenumi{\arabic{enumi}.}
\tightlist
\item
  Write \texttt{shapes.h} declaring functions
  \texttt{area\_rect(int\ w,\ int\ h)} and
  \texttt{area\_circle(int\ r)}.
\item
  Implement \texttt{shapes.c} with the definitions.
\item
  Write \texttt{main.c} that uses \texttt{shapes.h} and prints areas.
\item
  Add include guards to \texttt{shapes.h} and test by including it twice
  in \texttt{main.c}.
\item
  Create a macro \texttt{\#define\ SQUARE(x)\ ((x)-(x))} in
  \texttt{shapes.h} and use it in \texttt{main.c}.
\end{enumerate}

Do you want me to keep going with 11.3 The Preprocessor
(\texttt{\#include}, \texttt{\#define}) next?

\subsection{\texorpdfstring{11.3 The Preprocessor (\texttt{\#include},
\texttt{\#define})}{11.3 The Preprocessor (\#include, \#define)}}\label{the-preprocessor-include-define}

Before the compiler processes your C source code, the preprocessor runs.
It handles directives that start with \texttt{\#}, such as
\texttt{\#include} and \texttt{\#define}. Understanding the preprocessor
is essential for managing modular code and constants.

\subsubsection{\texorpdfstring{\texttt{\#include}}{\#include}}\label{include}

Tells the preprocessor to insert the contents of another file.

\begin{itemize}
\item
  System headers (use angle brackets):

\begin{Shaded}
\begin{Highlighting}[]
\PreprocessorTok{\#include }\ImportTok{\textless{}stdio.h\textgreater{}}
\end{Highlighting}
\end{Shaded}

  The compiler searches system include paths.
\item
  User headers (use quotes):

\begin{Shaded}
\begin{Highlighting}[]
\PreprocessorTok{\#include }\ImportTok{"mathx.h"}
\end{Highlighting}
\end{Shaded}

  The compiler searches the current directory first, then system paths.
\end{itemize}

This is how you bring in function prototypes, macros, and type
definitions from headers.

\subsubsection{\texorpdfstring{\texttt{\#define}}{\#define}}\label{define}

Defines symbolic constants or macros. The preprocessor replaces
occurrences of the symbol before compilation.

Constants:

\begin{Shaded}
\begin{Highlighting}[]
\PreprocessorTok{\#define PI }\FloatTok{3.14159}
\PreprocessorTok{\#define MAX\_SIZE }\DecValTok{100}
\end{Highlighting}
\end{Shaded}

Macros with parameters:

\begin{Shaded}
\begin{Highlighting}[]
\PreprocessorTok{\#define SQUARE}\OperatorTok{(}\PreprocessorTok{x}\OperatorTok{)}\PreprocessorTok{ }\OperatorTok{((}\PreprocessorTok{x}\OperatorTok{){-}(}\PreprocessorTok{x}\OperatorTok{))}
\end{Highlighting}
\end{Shaded}

Expands inline, no function call overhead. Be careful with parentheses
to avoid precedence bugs.

Example:

\begin{Shaded}
\begin{Highlighting}[]
\NormalTok{printf}\OperatorTok{(}\StringTok{"}\SpecialCharTok{\%d\textbackslash{}n}\StringTok{"}\OperatorTok{,}\NormalTok{ SQUARE}\OperatorTok{(}\DecValTok{3}\OperatorTok{+}\DecValTok{1}\OperatorTok{));}  \CommentTok{// expands to ((3+1){-}(3+1)) = 16}
\end{Highlighting}
\end{Shaded}

\subsubsection{Conditional Compilation}\label{conditional-compilation}

Directives can include or exclude code:

\begin{Shaded}
\begin{Highlighting}[]
\PreprocessorTok{\#ifdef DEBUG}
\NormalTok{    printf}\OperatorTok{(}\StringTok{"Debug info}\SpecialCharTok{\textbackslash{}n}\StringTok{"}\OperatorTok{);}
\PreprocessorTok{\#endif}
\end{Highlighting}
\end{Shaded}

\begin{Shaded}
\begin{Highlighting}[]
\PreprocessorTok{\#if defined(WIN32)}
\NormalTok{    printf}\OperatorTok{(}\StringTok{"Running on Windows}\SpecialCharTok{\textbackslash{}n}\StringTok{"}\OperatorTok{);}
\PreprocessorTok{\#elif defined(\_\_linux\_\_)}
\NormalTok{    printf}\OperatorTok{(}\StringTok{"Running on Linux}\SpecialCharTok{\textbackslash{}n}\StringTok{"}\OperatorTok{);}
\PreprocessorTok{\#else}
\NormalTok{    printf}\OperatorTok{(}\StringTok{"Unknown OS}\SpecialCharTok{\textbackslash{}n}\StringTok{"}\OperatorTok{);}
\PreprocessorTok{\#endif}
\end{Highlighting}
\end{Shaded}

\subsubsection{\texorpdfstring{\texttt{\#undef}}{\#undef}}\label{undef}

Removes a macro definition:

\begin{Shaded}
\begin{Highlighting}[]
\PreprocessorTok{\#undef PI}
\end{Highlighting}
\end{Shaded}

\subsubsection{Example: Using Preprocessor
Features}\label{example-using-preprocessor-features}

\texttt{config.h}:

\begin{Shaded}
\begin{Highlighting}[]
\PreprocessorTok{\#ifndef CONFIG\_H}
\PreprocessorTok{\#define CONFIG\_H}

\PreprocessorTok{\#define VERSION }\StringTok{"1.0"}
\PreprocessorTok{\#define DEBUG }\DecValTok{1}

\PreprocessorTok{\#endif}
\end{Highlighting}
\end{Shaded}

\texttt{main.c}:

\begin{Shaded}
\begin{Highlighting}[]
\PreprocessorTok{\#include }\ImportTok{\textless{}stdio.h\textgreater{}}
\PreprocessorTok{\#include }\ImportTok{"config.h"}

\PreprocessorTok{\#define SQUARE}\OperatorTok{(}\PreprocessorTok{x}\OperatorTok{)}\PreprocessorTok{ }\OperatorTok{((}\PreprocessorTok{x}\OperatorTok{){-}(}\PreprocessorTok{x}\OperatorTok{))}

\DataTypeTok{int}\NormalTok{ main}\OperatorTok{(}\DataTypeTok{void}\OperatorTok{)} \OperatorTok{\{}
\NormalTok{    printf}\OperatorTok{(}\StringTok{"Version: }\SpecialCharTok{\%s\textbackslash{}n}\StringTok{"}\OperatorTok{,}\NormalTok{ VERSION}\OperatorTok{);}

\PreprocessorTok{\#if DEBUG}
\NormalTok{    printf}\OperatorTok{(}\StringTok{"Debug mode on}\SpecialCharTok{\textbackslash{}n}\StringTok{"}\OperatorTok{);}
\PreprocessorTok{\#endif}

\NormalTok{    printf}\OperatorTok{(}\StringTok{"Square of 5: }\SpecialCharTok{\%d\textbackslash{}n}\StringTok{"}\OperatorTok{,}\NormalTok{ SQUARE}\OperatorTok{(}\DecValTok{5}\OperatorTok{));}
    \ControlFlowTok{return} \DecValTok{0}\OperatorTok{;}
\OperatorTok{\}}
\end{Highlighting}
\end{Shaded}

Compile and run:

\begin{verbatim}
Version: 1.0
Debug mode on
Square of 5: 25
\end{verbatim}

\subsubsection{Why It Matters}\label{why-it-matters-49}

\begin{itemize}
\tightlist
\item
  \texttt{\#include} keeps code modular by reusing declarations from
  headers.
\item
  \texttt{\#define} creates symbolic names and macros for readability
  and performance.
\item
  Conditional compilation allows platform-specific or debug code without
  duplicating files.
\end{itemize}

\subsubsection{Exercises}\label{exercises-52}

\begin{enumerate}
\def\labelenumi{\arabic{enumi}.}
\tightlist
\item
  Create a header \texttt{config.h} with
  \texttt{\#define\ APP\_NAME\ "MyApp"} and print it in \texttt{main.c}.
\item
  Define a macro \texttt{CUBE(x)} that computes the cube of \texttt{x}.
  Test with \texttt{CUBE(3+1)}.
\item
  Use \texttt{\#ifdef} to include extra debugging output only when
  \texttt{DEBUG} is defined.
\item
  Write a program that prints \texttt{"Windows"} if compiled on Windows
  (\texttt{\_WIN32} defined) and \texttt{"Linux"} if compiled on Linux
  (\texttt{\_\_linux\_\_} defined).
\item
  Experiment with \texttt{\#undef} by defining a constant, undefining
  it, and checking compilation behavior.
\end{enumerate}

\subsection{11.4 Macros vs.~Inline
Functions}\label{macros-vs.-inline-functions}

Both macros (\texttt{\#define}) and inline functions (introduced in C99,
refined in later standards) can be used to avoid repeated code and
eliminate function call overhead. But they differ in safety,
type-checking, and readability.

\subsubsection{Macros}\label{macros}

A macro is expanded by the preprocessor before compilation.

\begin{Shaded}
\begin{Highlighting}[]
\PreprocessorTok{\#define SQUARE}\OperatorTok{(}\PreprocessorTok{x}\OperatorTok{)}\PreprocessorTok{ }\OperatorTok{((}\PreprocessorTok{x}\OperatorTok{){-}(}\PreprocessorTok{x}\OperatorTok{))}
\end{Highlighting}
\end{Shaded}

Usage:

\begin{Shaded}
\begin{Highlighting}[]
\NormalTok{printf}\OperatorTok{(}\StringTok{"}\SpecialCharTok{\%d\textbackslash{}n}\StringTok{"}\OperatorTok{,}\NormalTok{ SQUARE}\OperatorTok{(}\DecValTok{3}\OperatorTok{));}    \CommentTok{// expands to ((3){-}(3)) = 9}
\NormalTok{printf}\OperatorTok{(}\StringTok{"}\SpecialCharTok{\%d\textbackslash{}n}\StringTok{"}\OperatorTok{,}\NormalTok{ SQUARE}\OperatorTok{(}\DecValTok{3}\OperatorTok{+}\DecValTok{1}\OperatorTok{));}  \CommentTok{// expands to ((3+1){-}(3+1)) = 16}
\end{Highlighting}
\end{Shaded}

Pros:

\begin{itemize}
\tightlist
\item
  Very fast, just text substitution.
\item
  Can be used for constants, inline code, conditional compilation.
\end{itemize}

Cons:

\begin{itemize}
\item
  No type checking.
\item
  Can cause subtle bugs if arguments have side effects:

\begin{Shaded}
\begin{Highlighting}[]
\DataTypeTok{int}\NormalTok{ i }\OperatorTok{=} \DecValTok{2}\OperatorTok{;}
\NormalTok{printf}\OperatorTok{(}\StringTok{"}\SpecialCharTok{\%d\textbackslash{}n}\StringTok{"}\OperatorTok{,}\NormalTok{ SQUARE}\OperatorTok{(}\NormalTok{i}\OperatorTok{++));}  \CommentTok{// expands to ((i++){-}(i++)), increments twice!}
\end{Highlighting}
\end{Shaded}
\end{itemize}

\subsubsection{Inline Functions}\label{inline-functions}

An \texttt{inline} function is a real function, but the compiler may
expand it directly into code at the call site.

\begin{Shaded}
\begin{Highlighting}[]
\KeywordTok{inline} \DataTypeTok{int}\NormalTok{ square}\OperatorTok{(}\DataTypeTok{int}\NormalTok{ x}\OperatorTok{)} \OperatorTok{\{}
    \ControlFlowTok{return}\NormalTok{ x }\OperatorTok{{-}}\NormalTok{ x}\OperatorTok{;}
\OperatorTok{\}}
\end{Highlighting}
\end{Shaded}

Usage:

\begin{Shaded}
\begin{Highlighting}[]
\NormalTok{printf}\OperatorTok{(}\StringTok{"}\SpecialCharTok{\%d\textbackslash{}n}\StringTok{"}\OperatorTok{,}\NormalTok{ square}\OperatorTok{(}\DecValTok{3}\OperatorTok{));}
\NormalTok{printf}\OperatorTok{(}\StringTok{"}\SpecialCharTok{\%d\textbackslash{}n}\StringTok{"}\OperatorTok{,}\NormalTok{ square}\OperatorTok{(}\DecValTok{3}\OperatorTok{+}\DecValTok{1}\OperatorTok{));}
\end{Highlighting}
\end{Shaded}

Pros:

\begin{itemize}
\tightlist
\item
  Type checking is enforced.
\item
  Safe from multiple evaluation bugs.
\item
  Debuggable, works like a normal function.
\end{itemize}

Cons:

\begin{itemize}
\tightlist
\item
  Slightly more verbose.
\item
  The compiler may decide not to inline it (but usually does for small
  functions).
\end{itemize}

\subsubsection{\texorpdfstring{Constants: \texttt{\#define} vs
\texttt{const}}{Constants: \#define vs const}}\label{constants-define-vs-const}

Old style:

\begin{Shaded}
\begin{Highlighting}[]
\PreprocessorTok{\#define PI }\FloatTok{3.14159}
\end{Highlighting}
\end{Shaded}

Modern style:

\begin{Shaded}
\begin{Highlighting}[]
\DataTypeTok{const} \DataTypeTok{double}\NormalTok{ PI }\OperatorTok{=} \FloatTok{3.14159}\OperatorTok{;}
\end{Highlighting}
\end{Shaded}

\begin{itemize}
\tightlist
\item
  \texttt{const} has a type, checked by the compiler.
\item
  \texttt{\#define} is just text substitution.
\end{itemize}

\subsubsection{Example: Macro vs Inline}\label{example-macro-vs-inline}

\begin{Shaded}
\begin{Highlighting}[]
\PreprocessorTok{\#include }\ImportTok{\textless{}stdio.h\textgreater{}}

\PreprocessorTok{\#define CUBE\_MACRO}\OperatorTok{(}\PreprocessorTok{x}\OperatorTok{)}\PreprocessorTok{ }\OperatorTok{((}\PreprocessorTok{x}\OperatorTok{){-}(}\PreprocessorTok{x}\OperatorTok{){-}(}\PreprocessorTok{x}\OperatorTok{))}

\KeywordTok{inline} \DataTypeTok{int}\NormalTok{ cube\_func}\OperatorTok{(}\DataTypeTok{int}\NormalTok{ x}\OperatorTok{)} \OperatorTok{\{}
    \ControlFlowTok{return}\NormalTok{ x }\OperatorTok{{-}}\NormalTok{ x }\OperatorTok{{-}}\NormalTok{ x}\OperatorTok{;}
\OperatorTok{\}}

\DataTypeTok{int}\NormalTok{ main}\OperatorTok{(}\DataTypeTok{void}\OperatorTok{)} \OperatorTok{\{}
    \DataTypeTok{int}\NormalTok{ a }\OperatorTok{=} \DecValTok{3}\OperatorTok{;}
\NormalTok{    printf}\OperatorTok{(}\StringTok{"macro: }\SpecialCharTok{\%d\textbackslash{}n}\StringTok{"}\OperatorTok{,}\NormalTok{ CUBE\_MACRO}\OperatorTok{(}\NormalTok{a}\OperatorTok{));}
\NormalTok{    printf}\OperatorTok{(}\StringTok{"func:  }\SpecialCharTok{\%d\textbackslash{}n}\StringTok{"}\OperatorTok{,}\NormalTok{ cube\_func}\OperatorTok{(}\NormalTok{a}\OperatorTok{));}

    \DataTypeTok{int}\NormalTok{ i }\OperatorTok{=} \DecValTok{2}\OperatorTok{;}
\NormalTok{    printf}\OperatorTok{(}\StringTok{"macro side effect: }\SpecialCharTok{\%d\textbackslash{}n}\StringTok{"}\OperatorTok{,}\NormalTok{ CUBE\_MACRO}\OperatorTok{(}\NormalTok{i}\OperatorTok{++));} \CommentTok{// unsafe}
    \CommentTok{// printf("func side effect: \%d\textbackslash{}n", cube\_func(i++)); // safe, only one increment}
    \ControlFlowTok{return} \DecValTok{0}\OperatorTok{;}
\OperatorTok{\}}
\end{Highlighting}
\end{Shaded}

Output:

\begin{verbatim}
macro: 27
func:  27
macro side effect: 60
\end{verbatim}

\subsubsection{Guidelines}\label{guidelines}

\begin{itemize}
\tightlist
\item
  Prefer \texttt{const} variables over \texttt{\#define} for constants.
\item
  Prefer \texttt{inline\ functions} over macros for computations.
\item
  Use macros only for conditional compilation or cases where inline
  functions are not possible (e.g., include guards).
\end{itemize}

\subsubsection{Exercises}\label{exercises-53}

\begin{enumerate}
\def\labelenumi{\arabic{enumi}.}
\tightlist
\item
  Write a macro \texttt{ABS(x)} that computes the absolute value. Test
  it with negative and positive numbers.
\item
  Write an inline function \texttt{abs\_inline(int\ x)} that does the
  same. Compare outputs.
\item
  Experiment with \texttt{ABS(i++)} vs \texttt{abs\_inline(i++)}. What
  difference do you see?
\item
  Replace \texttt{\#define\ PI\ 3.14159} with
  \texttt{const\ double\ PI\ =\ 3.14159;}. Try using it in
  \texttt{printf}.
\item
  Benchmark (simple loop) calling a macro cube and an inline cube 1
  million times - do you see performance differences? \#\#\# 11.5
  Building a Small Project
\end{enumerate}

By this point, you've learned how to split code into files, use headers,
and control compilation with the preprocessor. Let's put everything
together into a small modular project.

\subsubsection{Project: A Simple Calculator
Library}\label{project-a-simple-calculator-library}

We'll build a calculator with four operations (\texttt{add},
\texttt{sub}, \texttt{mul}, \texttt{div\_int}), place it into a library
(\texttt{calc.c} + \texttt{calc.h}), and use it from a \texttt{main.c}.

\paragraph{\texorpdfstring{\texttt{calc.h} - header
file}{calc.h - header file}}\label{calc.h---header-file}

\begin{Shaded}
\begin{Highlighting}[]
\PreprocessorTok{\#ifndef CALC\_H}
\PreprocessorTok{\#define CALC\_H}

\CommentTok{// Function prototypes}
\DataTypeTok{int}\NormalTok{ add}\OperatorTok{(}\DataTypeTok{int}\NormalTok{ a}\OperatorTok{,} \DataTypeTok{int}\NormalTok{ b}\OperatorTok{);}
\DataTypeTok{int}\NormalTok{ sub}\OperatorTok{(}\DataTypeTok{int}\NormalTok{ a}\OperatorTok{,} \DataTypeTok{int}\NormalTok{ b}\OperatorTok{);}
\DataTypeTok{int}\NormalTok{ mul}\OperatorTok{(}\DataTypeTok{int}\NormalTok{ a}\OperatorTok{,} \DataTypeTok{int}\NormalTok{ b}\OperatorTok{);}
\DataTypeTok{int}\NormalTok{ div\_int}\OperatorTok{(}\DataTypeTok{int}\NormalTok{ a}\OperatorTok{,} \DataTypeTok{int}\NormalTok{ b}\OperatorTok{);} \CommentTok{// integer division, b must not be 0}

\PreprocessorTok{\#endif}
\end{Highlighting}
\end{Shaded}

\paragraph{\texorpdfstring{\texttt{calc.c} -
implementation}{calc.c - implementation}}\label{calc.c---implementation}

\begin{Shaded}
\begin{Highlighting}[]
\PreprocessorTok{\#include }\ImportTok{"calc.h"}

\DataTypeTok{int}\NormalTok{ add}\OperatorTok{(}\DataTypeTok{int}\NormalTok{ a}\OperatorTok{,} \DataTypeTok{int}\NormalTok{ b}\OperatorTok{)} \OperatorTok{\{} \ControlFlowTok{return}\NormalTok{ a }\OperatorTok{+}\NormalTok{ b}\OperatorTok{;} \OperatorTok{\}}
\DataTypeTok{int}\NormalTok{ sub}\OperatorTok{(}\DataTypeTok{int}\NormalTok{ a}\OperatorTok{,} \DataTypeTok{int}\NormalTok{ b}\OperatorTok{)} \OperatorTok{\{} \ControlFlowTok{return}\NormalTok{ a }\OperatorTok{{-}}\NormalTok{ b}\OperatorTok{;} \OperatorTok{\}}
\DataTypeTok{int}\NormalTok{ mul}\OperatorTok{(}\DataTypeTok{int}\NormalTok{ a}\OperatorTok{,} \DataTypeTok{int}\NormalTok{ b}\OperatorTok{)} \OperatorTok{\{} \ControlFlowTok{return}\NormalTok{ a }\OperatorTok{{-}}\NormalTok{ b}\OperatorTok{;} \OperatorTok{\}}
\DataTypeTok{int}\NormalTok{ div\_int}\OperatorTok{(}\DataTypeTok{int}\NormalTok{ a}\OperatorTok{,} \DataTypeTok{int}\NormalTok{ b}\OperatorTok{)} \OperatorTok{\{} \ControlFlowTok{return} \OperatorTok{(}\NormalTok{b }\OperatorTok{!=} \DecValTok{0}\OperatorTok{)} \OperatorTok{?}\NormalTok{ a }\OperatorTok{/}\NormalTok{ b }\OperatorTok{:} \DecValTok{0}\OperatorTok{;} \OperatorTok{\}}
\end{Highlighting}
\end{Shaded}

\paragraph{\texorpdfstring{\texttt{main.c} - program entry
point}{main.c - program entry point}}\label{main.c---program-entry-point}

\begin{Shaded}
\begin{Highlighting}[]
\PreprocessorTok{\#include }\ImportTok{\textless{}stdio.h\textgreater{}}
\PreprocessorTok{\#include }\ImportTok{"calc.h"}

\DataTypeTok{int}\NormalTok{ main}\OperatorTok{(}\DataTypeTok{void}\OperatorTok{)} \OperatorTok{\{}
    \DataTypeTok{int}\NormalTok{ x }\OperatorTok{=} \DecValTok{12}\OperatorTok{,}\NormalTok{ y }\OperatorTok{=} \DecValTok{4}\OperatorTok{;}
\NormalTok{    printf}\OperatorTok{(}\StringTok{"}\SpecialCharTok{\%d}\StringTok{ + }\SpecialCharTok{\%d}\StringTok{ = }\SpecialCharTok{\%d\textbackslash{}n}\StringTok{"}\OperatorTok{,}\NormalTok{ x}\OperatorTok{,}\NormalTok{ y}\OperatorTok{,}\NormalTok{ add}\OperatorTok{(}\NormalTok{x}\OperatorTok{,}\NormalTok{ y}\OperatorTok{));}
\NormalTok{    printf}\OperatorTok{(}\StringTok{"}\SpecialCharTok{\%d}\StringTok{ {-} }\SpecialCharTok{\%d}\StringTok{ = }\SpecialCharTok{\%d\textbackslash{}n}\StringTok{"}\OperatorTok{,}\NormalTok{ x}\OperatorTok{,}\NormalTok{ y}\OperatorTok{,}\NormalTok{ sub}\OperatorTok{(}\NormalTok{x}\OperatorTok{,}\NormalTok{ y}\OperatorTok{));}
\NormalTok{    printf}\OperatorTok{(}\StringTok{"}\SpecialCharTok{\%d}\StringTok{ {-} }\SpecialCharTok{\%d}\StringTok{ = }\SpecialCharTok{\%d\textbackslash{}n}\StringTok{"}\OperatorTok{,}\NormalTok{ x}\OperatorTok{,}\NormalTok{ y}\OperatorTok{,}\NormalTok{ mul}\OperatorTok{(}\NormalTok{x}\OperatorTok{,}\NormalTok{ y}\OperatorTok{));}
\NormalTok{    printf}\OperatorTok{(}\StringTok{"}\SpecialCharTok{\%d}\StringTok{ / }\SpecialCharTok{\%d}\StringTok{ = }\SpecialCharTok{\%d\textbackslash{}n}\StringTok{"}\OperatorTok{,}\NormalTok{ x}\OperatorTok{,}\NormalTok{ y}\OperatorTok{,}\NormalTok{ div\_int}\OperatorTok{(}\NormalTok{x}\OperatorTok{,}\NormalTok{ y}\OperatorTok{));}
    \ControlFlowTok{return} \DecValTok{0}\OperatorTok{;}
\OperatorTok{\}}
\end{Highlighting}
\end{Shaded}

\subsubsection{Compiling and Linking}\label{compiling-and-linking}

Compile each \texttt{.c} separately and then link:

\begin{Shaded}
\begin{Highlighting}[]
\FunctionTok{gcc} \AttributeTok{{-}std}\OperatorTok{=}\NormalTok{c23 }\AttributeTok{{-}Wall} \AttributeTok{{-}Wextra} \AttributeTok{{-}c}\NormalTok{ calc.c   }\CommentTok{\# produce calc.o}
\FunctionTok{gcc} \AttributeTok{{-}std}\OperatorTok{=}\NormalTok{c23 }\AttributeTok{{-}Wall} \AttributeTok{{-}Wextra} \AttributeTok{{-}c}\NormalTok{ main.c   }\CommentTok{\# produce main.o}
\FunctionTok{gcc}\NormalTok{ main.o calc.o }\AttributeTok{{-}o}\NormalTok{ app               }\CommentTok{\# link to executable}
\end{Highlighting}
\end{Shaded}

Run:

\begin{verbatim}
12 + 4 = 16
12 - 4 = 8
12 - 4 = 48
12 / 4 = 3
\end{verbatim}

\subsubsection{Organizing the Project}\label{organizing-the-project}

A good folder structure:

\begin{verbatim}
/project
  include/
    calc.h
  src/
    calc.c
    main.c
\end{verbatim}

Compile with an include path:

\begin{Shaded}
\begin{Highlighting}[]
\FunctionTok{gcc} \AttributeTok{{-}Iinclude} \AttributeTok{{-}std}\OperatorTok{=}\NormalTok{c23 src/main.c src/calc.c }\AttributeTok{{-}o}\NormalTok{ app}
\end{Highlighting}
\end{Shaded}

\subsubsection{Example Makefile}\label{example-makefile}

\begin{Shaded}
\begin{Highlighting}[]
\NormalTok{CC=gcc}
\NormalTok{CFLAGS={-}std=c23 {-}Wall {-}Wextra {-}Iinclude}
\NormalTok{OBJ=build/main.o build/calc.o}

\NormalTok{app: $(OBJ)}
\NormalTok{    $(CC) $(OBJ) {-}o app}

\NormalTok{build/main.o: src/main.c include/calc.h}
\NormalTok{    mkdir {-}p build}
\NormalTok{    $(CC) $(CFLAGS) {-}c src/main.c {-}o $@}

\NormalTok{build/calc.o: src/calc.c include/calc.h}
\NormalTok{    mkdir {-}p build}
\NormalTok{    $(CC) $(CFLAGS) {-}c src/calc.c {-}o $@}

\NormalTok{clean:}
\NormalTok{    rm {-}rf build app}
\end{Highlighting}
\end{Shaded}

\subsubsection{Why It Matters}\label{why-it-matters-50}

\begin{itemize}
\tightlist
\item
  Shows how real C projects are organized.
\item
  Demonstrates the role of headers as contracts and source files as
  implementations.
\item
  Teaches separation of concerns: math code is reusable, \texttt{main}
  is just orchestration.
\end{itemize}

\subsubsection{Exercises}\label{exercises-54}

\begin{enumerate}
\def\labelenumi{\arabic{enumi}.}
\tightlist
\item
  Extend the calculator with a \texttt{mod(int\ a,\ int\ b)} function.
\item
  Add \texttt{pow\_int(int\ base,\ int\ exp)} that computes powers with
  a loop.
\item
  Update the Makefile so it automatically recompiles when
  \texttt{calc.h} changes.
\item
  Move calculator code into a static library (\texttt{libcalc.a}) and
  link it.
\item
  Try compiling the project with \texttt{clang} or MSVC - confirm
  portability. Here's a comprehensive problem set for Chapter 11
  (Modular Programming). These exercises cover splitting code into
  multiple files, using headers, preprocessor usage, macros vs inline,
  and building small projects.
\end{enumerate}

\subsection{Problems}\label{problems-9}

\subsubsection{1. Split a Math Library}\label{split-a-math-library}

Write a program that computes \texttt{add}, \texttt{sub}, \texttt{mul},
and \texttt{div\_int}. Place declarations in \texttt{mathlib.h},
definitions in \texttt{mathlib.c}, and usage in \texttt{main.c}.

\subsubsection{2. Include Guards}\label{include-guards-2}

Modify \texttt{mathlib.h} to use an include guard. Test by including
\texttt{\#include\ "mathlib.h"} twice in \texttt{main.c}. Confirm no
errors.

\subsubsection{3. Static Helper}\label{static-helper}

Add a \texttt{static} helper function in \texttt{mathlib.c} (e.g.,
\texttt{clamp16(int\ x)}) that is not visible outside the file. Use it
inside \texttt{div\_int} to prevent division overflow.

\subsubsection{4. Conditional Debug
Output}\label{conditional-debug-output}

Use \texttt{\#ifdef\ DEBUG} in \texttt{mathlib.c} so that when compiled
with \texttt{-DDEBUG}, the library prints
\texttt{"debug:\ add\ called"}. Otherwise, it stays silent.

\subsubsection{\texorpdfstring{5. Constant with
\texttt{\#define}}{5. Constant with \#define}}\label{constant-with-define}

In \texttt{mathlib.h}, define \texttt{\#define\ PI\ 3.14159} and use it
in \texttt{main.c} to compute the area of a circle. Then replace it with
\texttt{const\ double\ PI}.

\subsubsection{6. Macro vs Inline
Function}\label{macro-vs-inline-function}

Write a macro \texttt{SQUARE(x)} and an inline function
\texttt{square(int\ x)}. In \texttt{main.c}, test them with \texttt{3},
\texttt{3+1}, and \texttt{i++}. Compare results.

\subsubsection{7. Platform-Specific Code}\label{platform-specific-code}

Use \texttt{\#if\ defined(\_WIN32)} and
\texttt{\#elif\ defined(\_\_linux\_\_)} in \texttt{main.c} to print
\texttt{"Windows"} or \texttt{"Linux"} depending on platform.

\subsubsection{\texorpdfstring{8. Redefining with
\texttt{\#undef}}{8. Redefining with \#undef}}\label{redefining-with-undef}

Define a macro \texttt{LIMIT\ 100}, print it, then
\texttt{\#undef\ LIMIT} and try printing again. Observe compilation
behavior.

\subsubsection{9. Simple Header +
Implementation}\label{simple-header-implementation}

Create \texttt{shapes.h} with \texttt{area\_rect(int\ w,\ int\ h)} and
\texttt{shapes.c} with its definition. Use it in \texttt{main.c} to
print \texttt{area\_rect(3,4)}.

\subsubsection{10. Enum in Header}\label{enum-in-header}

Declare an \texttt{enum\ Color\ \{\ RED,\ GREEN,\ BLUE\ \};} in
\texttt{shapes.h}. In \texttt{main.c}, declare a variable
\texttt{Color\ c\ =\ RED;} and print its value.

\subsubsection{11. Struct in Header}\label{struct-in-header}

Declare a struct \texttt{Point\ \{\ int\ x;\ int\ y;\ \};} in
\texttt{shapes.h}. In \texttt{main.c}, create a point, assign values,
and print coordinates.

\subsubsection{12. Preprocessor Version
Macro}\label{preprocessor-version-macro}

In \texttt{config.h}, define \texttt{\#define\ VERSION\ "1.0"}. In
\texttt{main.c}, include \texttt{config.h} and print the version string.

\subsubsection{13. Multiple Translation
Units}\label{multiple-translation-units}

Write \texttt{file1.c} with \texttt{void\ f1(void)\ \{\ puts("f1");\ \}}
and \texttt{file2.c} with \texttt{void\ f2(void)\ \{\ puts("f2");\ \}}.
Declare them in \texttt{file.h}. Call both from \texttt{main.c}.

\subsubsection{14. Build with Separate
Compilation}\label{build-with-separate-compilation}

Compile problem 13 with three commands (\texttt{gcc\ -c\ file1.c},
\texttt{gcc\ -c\ file2.c}, \texttt{gcc\ -c\ main.c}) and link into
\texttt{prog}.

\subsubsection{15. Mini Project with
Makefile}\label{mini-project-with-makefile}

Write a \texttt{Makefile} that builds \texttt{prog} from
\texttt{main.c}, \texttt{file1.c}, \texttt{file2.c}, and
\texttt{file.h}. Include \texttt{clean} target.

\subsubsection{16. Inline vs Non-Inline
Performance}\label{inline-vs-non-inline-performance}

Write a loop that calls a \texttt{cube\_inline(int)} inline function 1
million times. Then try a \texttt{\#define\ CUBE(x)} macro. Time the
difference with \texttt{clock()}.

\subsubsection{\texorpdfstring{17. Header in \texttt{include/}
Directory}{17. Header in include/ Directory}}\label{header-in-include-directory}

Move headers to \texttt{include/} and sources to \texttt{src/}. Compile
with \texttt{gcc\ -Iinclude\ src/-.c\ -o\ app}.

\subsubsection{18. Nested Includes}\label{nested-includes}

Make \texttt{shapes.h} include \texttt{mathlib.h} and confirm that
\texttt{main.c} can use both libraries with just one include.

\subsubsection{19. Header Misuse Example}\label{header-misuse-example}

Put a function definition directly in a header file (\texttt{bad.h}) and
include it twice in different \texttt{.c} files. Observe the ``multiple
definition'' error. Fix it by moving the definition into \texttt{.c}.

\subsubsection{20. Expand Calculator
Project}\label{expand-calculator-project}

Expand the calculator project:

\begin{itemize}
\tightlist
\item
  Add \texttt{mod(int,int)} and \texttt{pow\_int(int,int)} in
  \texttt{calc.c}.
\item
  Declare them in \texttt{calc.h}.
\item
  Use them in \texttt{main.c} to print results.
\end{itemize}

\section{Chapter 12. Standard Library
Essentials}\label{chapter-12.-standard-library-essentials}

\subsection{\texorpdfstring{12.1 Input/Output
(\texttt{stdio.h})}{12.1 Input/Output (stdio.h)}}\label{inputoutput-stdio.h}

\texttt{\textless{}stdio.h\textgreater{}} is C's standard input/output
library. It provides:

\begin{itemize}
\tightlist
\item
  Streams: \texttt{stdin}, \texttt{stdout}, \texttt{stderr}
\item
  Printing: \texttt{printf}, \texttt{fprintf}, \texttt{puts},
  \texttt{putchar}
\item
  Reading: \texttt{scanf}, \texttt{fgets}, \texttt{getchar}
\item
  Formatted I/O with -conversion specifiers-
\end{itemize}

\subsubsection{Streams at a glance}\label{streams-at-a-glance}

\begin{itemize}
\tightlist
\item
  \texttt{stdin} → default input (keyboard or redirected file)
\item
  \texttt{stdout} → normal output (console or redirected file)
\item
  \texttt{stderr} → error output (usually unbuffered; keeps errors
  separate)
\end{itemize}

You can redirect on the command line:

\begin{verbatim}
./app < input.txt > output.txt 2> errors.txt
\end{verbatim}

\subsubsection{\texorpdfstring{Printing with \texttt{printf} /
\texttt{puts} /
\texttt{putchar}}{Printing with printf / puts / putchar}}\label{printing-with-printf-puts-putchar}

\begin{Shaded}
\begin{Highlighting}[]
\PreprocessorTok{\#include }\ImportTok{\textless{}stdio.h\textgreater{}}

\DataTypeTok{int}\NormalTok{ main}\OperatorTok{(}\DataTypeTok{void}\OperatorTok{)} \OperatorTok{\{}
    \DataTypeTok{int}\NormalTok{ n }\OperatorTok{=} \DecValTok{42}\OperatorTok{;}
    \DataTypeTok{double}\NormalTok{ x }\OperatorTok{=} \FloatTok{3.14159}\OperatorTok{;}

\NormalTok{    printf}\OperatorTok{(}\StringTok{"n=}\SpecialCharTok{\%d}\StringTok{ x=}\SpecialCharTok{\%.2f\textbackslash{}n}\StringTok{"}\OperatorTok{,}\NormalTok{ n}\OperatorTok{,}\NormalTok{ x}\OperatorTok{);}   \CommentTok{// formatted}
\NormalTok{    puts}\OperatorTok{(}\StringTok{"hello"}\OperatorTok{);}                   \CommentTok{// string + newline}
\NormalTok{    putchar}\OperatorTok{(}\CharTok{\textquotesingle{}A\textquotesingle{}}\OperatorTok{);}\NormalTok{ putchar}\OperatorTok{(}\CharTok{\textquotesingle{}}\SpecialCharTok{\textbackslash{}n}\CharTok{\textquotesingle{}}\OperatorTok{);}     \CommentTok{// single chars}
    \ControlFlowTok{return} \DecValTok{0}\OperatorTok{;}
\OperatorTok{\}}
\end{Highlighting}
\end{Shaded}

Common format specifiers:

\begin{itemize}
\tightlist
\item
  \texttt{\%d} (int), \texttt{\%ld} (long), \texttt{\%lld} (long long)
\item
  \texttt{\%u} (unsigned), \texttt{\%x} (hex), \texttt{\%o} (octal)
\item
  \texttt{\%f} / \texttt{\%e} / \texttt{\%g} (double)
\item
  \texttt{\%c} (char), \texttt{\%s} (string)
\item
  \texttt{\%p} (pointer), \texttt{\%zu} (size\_t)
\item
  Width/precision: \texttt{\%.2f}, \texttt{\%8d}, \texttt{\%-10s},
  \texttt{\%08x}
\end{itemize}

\subsubsection{\texorpdfstring{Reading with \texttt{scanf} (quick, but
picky)}{Reading with scanf (quick, but picky)}}\label{reading-with-scanf-quick-but-picky}

\begin{Shaded}
\begin{Highlighting}[]
\PreprocessorTok{\#include }\ImportTok{\textless{}stdio.h\textgreater{}}

\DataTypeTok{int}\NormalTok{ main}\OperatorTok{(}\DataTypeTok{void}\OperatorTok{)} \OperatorTok{\{}
    \DataTypeTok{int}\NormalTok{ a}\OperatorTok{;} \DataTypeTok{double}\NormalTok{ b}\OperatorTok{;}
    \ControlFlowTok{if} \OperatorTok{(}\NormalTok{scanf}\OperatorTok{(}\StringTok{"}\SpecialCharTok{\%d}\StringTok{ }\SpecialCharTok{\%lf}\StringTok{"}\OperatorTok{,} \OperatorTok{\&}\NormalTok{a}\OperatorTok{,} \OperatorTok{\&}\NormalTok{b}\OperatorTok{)} \OperatorTok{==} \DecValTok{2}\OperatorTok{)} \OperatorTok{\{}
\NormalTok{        printf}\OperatorTok{(}\StringTok{"a=}\SpecialCharTok{\%d}\StringTok{ b=}\SpecialCharTok{\%.3f\textbackslash{}n}\StringTok{"}\OperatorTok{,}\NormalTok{ a}\OperatorTok{,}\NormalTok{ b}\OperatorTok{);}
    \OperatorTok{\}} \ControlFlowTok{else} \OperatorTok{\{}
\NormalTok{        fprintf}\OperatorTok{(}\NormalTok{stderr}\OperatorTok{,} \StringTok{"input error}\SpecialCharTok{\textbackslash{}n}\StringTok{"}\OperatorTok{);}
    \OperatorTok{\}}
    \ControlFlowTok{return} \DecValTok{0}\OperatorTok{;}
\OperatorTok{\}}
\end{Highlighting}
\end{Shaded}

Tips:

\begin{itemize}
\tightlist
\item
  Always check the return value (number of items read).
\item
  Beware of leftover newlines when mixing \texttt{\%d}/\texttt{\%lf}
  with \texttt{\%c}/\texttt{\%s}.
\end{itemize}

\subsubsection{\texorpdfstring{Safer line-based input with
\texttt{fgets}}{Safer line-based input with fgets}}\label{safer-line-based-input-with-fgets}

Use \texttt{fgets} to read a whole line, then parse it.

\begin{Shaded}
\begin{Highlighting}[]
\PreprocessorTok{\#include }\ImportTok{\textless{}stdio.h\textgreater{}}

\DataTypeTok{int}\NormalTok{ main}\OperatorTok{(}\DataTypeTok{void}\OperatorTok{)} \OperatorTok{\{}
    \DataTypeTok{char}\NormalTok{ buf}\OperatorTok{[}\DecValTok{128}\OperatorTok{];}
    \ControlFlowTok{if} \OperatorTok{(}\NormalTok{fgets}\OperatorTok{(}\NormalTok{buf}\OperatorTok{,} \KeywordTok{sizeof}\NormalTok{ buf}\OperatorTok{,}\NormalTok{ stdin}\OperatorTok{))} \OperatorTok{\{}
\NormalTok{        printf}\OperatorTok{(}\StringTok{"line: }\SpecialCharTok{\%s}\StringTok{"}\OperatorTok{,}\NormalTok{ buf}\OperatorTok{);}  \CommentTok{// includes \textquotesingle{}\textbackslash{}n\textquotesingle{} if it fit}
    \OperatorTok{\}}
    \ControlFlowTok{return} \DecValTok{0}\OperatorTok{;}
\OperatorTok{\}}
\end{Highlighting}
\end{Shaded}

Parsing a line (e.g., with \texttt{sscanf}):

\begin{Shaded}
\begin{Highlighting}[]
\PreprocessorTok{\#include }\ImportTok{\textless{}stdio.h\textgreater{}}

\DataTypeTok{int}\NormalTok{ main}\OperatorTok{(}\DataTypeTok{void}\OperatorTok{)} \OperatorTok{\{}
    \DataTypeTok{char}\NormalTok{ buf}\OperatorTok{[}\DecValTok{128}\OperatorTok{];}
    \ControlFlowTok{if} \OperatorTok{(!}\NormalTok{fgets}\OperatorTok{(}\NormalTok{buf}\OperatorTok{,} \KeywordTok{sizeof}\NormalTok{ buf}\OperatorTok{,}\NormalTok{ stdin}\OperatorTok{))} \ControlFlowTok{return} \DecValTok{0}\OperatorTok{;}

    \DataTypeTok{int}\NormalTok{ a}\OperatorTok{;} \DataTypeTok{double}\NormalTok{ b}\OperatorTok{;}
    \ControlFlowTok{if} \OperatorTok{(}\NormalTok{sscanf}\OperatorTok{(}\NormalTok{buf}\OperatorTok{,} \StringTok{"}\SpecialCharTok{\%d}\StringTok{ }\SpecialCharTok{\%lf}\StringTok{"}\OperatorTok{,} \OperatorTok{\&}\NormalTok{a}\OperatorTok{,} \OperatorTok{\&}\NormalTok{b}\OperatorTok{)} \OperatorTok{==} \DecValTok{2}\OperatorTok{)} \OperatorTok{\{}
\NormalTok{        printf}\OperatorTok{(}\StringTok{"ok: a=}\SpecialCharTok{\%d}\StringTok{ b=}\SpecialCharTok{\%.2f\textbackslash{}n}\StringTok{"}\OperatorTok{,}\NormalTok{ a}\OperatorTok{,}\NormalTok{ b}\OperatorTok{);}
    \OperatorTok{\}} \ControlFlowTok{else} \OperatorTok{\{}
\NormalTok{        puts}\OperatorTok{(}\StringTok{"parse error"}\OperatorTok{);}
    \OperatorTok{\}}
    \ControlFlowTok{return} \DecValTok{0}\OperatorTok{;}
\OperatorTok{\}}
\end{Highlighting}
\end{Shaded}

\subsubsection{\texorpdfstring{Buffered I/O and
\texttt{fflush}}{Buffered I/O and fflush}}\label{buffered-io-and-fflush}

\begin{itemize}
\tightlist
\item
  \texttt{stdout} is usually line-buffered on terminals; fully buffered
  when redirected.
\item
  Force output now: \texttt{fflush(stdout);}
\item
  \texttt{stderr} is typically unbuffered (appears immediately).
\end{itemize}

\begin{Shaded}
\begin{Highlighting}[]
\NormalTok{printf}\OperatorTok{(}\StringTok{"Working..."}\OperatorTok{);}\NormalTok{ fflush}\OperatorTok{(}\NormalTok{stdout}\OperatorTok{);} \CommentTok{// ensure user sees this now}
\end{Highlighting}
\end{Shaded}

\subsubsection{Example: Echo with numbering (mixed
I/O)}\label{example-echo-with-numbering-mixed-io}

\begin{Shaded}
\begin{Highlighting}[]
\PreprocessorTok{\#include }\ImportTok{\textless{}stdio.h\textgreater{}}
\PreprocessorTok{\#include }\ImportTok{\textless{}string.h\textgreater{}}

\DataTypeTok{int}\NormalTok{ main}\OperatorTok{(}\DataTypeTok{void}\OperatorTok{)} \OperatorTok{\{}
    \DataTypeTok{char}\NormalTok{ line}\OperatorTok{[}\DecValTok{256}\OperatorTok{];}
    \DataTypeTok{int}\NormalTok{ n }\OperatorTok{=} \DecValTok{1}\OperatorTok{;}
    \ControlFlowTok{while} \OperatorTok{(}\NormalTok{fgets}\OperatorTok{(}\NormalTok{line}\OperatorTok{,} \KeywordTok{sizeof}\NormalTok{ line}\OperatorTok{,}\NormalTok{ stdin}\OperatorTok{))} \OperatorTok{\{}
        \CommentTok{// strip trailing newline (if present)}
\NormalTok{        line}\OperatorTok{[}\NormalTok{strcspn}\OperatorTok{(}\NormalTok{line}\OperatorTok{,} \StringTok{"}\SpecialCharTok{\textbackslash{}n}\StringTok{"}\OperatorTok{)]} \OperatorTok{=} \CharTok{\textquotesingle{}}\SpecialCharTok{\textbackslash{}0}\CharTok{\textquotesingle{}}\OperatorTok{;}
\NormalTok{        printf}\OperatorTok{(}\StringTok{"}\SpecialCharTok{\%03d}\StringTok{: }\SpecialCharTok{\%s\textbackslash{}n}\StringTok{"}\OperatorTok{,}\NormalTok{ n}\OperatorTok{++,}\NormalTok{ line}\OperatorTok{);}
    \OperatorTok{\}}
    \ControlFlowTok{if} \OperatorTok{(}\NormalTok{ferror}\OperatorTok{(}\NormalTok{stdin}\OperatorTok{))} \OperatorTok{\{}\NormalTok{ perror}\OperatorTok{(}\StringTok{"stdin error"}\OperatorTok{);} \OperatorTok{\}}
    \ControlFlowTok{return} \DecValTok{0}\OperatorTok{;}
\OperatorTok{\}}
\end{Highlighting}
\end{Shaded}

\subsubsection{Common pitfalls}\label{common-pitfalls-1}

\begin{itemize}
\tightlist
\item
  Forgetting \texttt{\&} in \texttt{scanf} arguments (except for
  \texttt{\%s}, which already decays to pointer).
\item
  Using \texttt{scanf("\%s",\ buf)} without a width: use
  \texttt{"\%127s"} to avoid overflow.
\item
  Mixing \texttt{scanf} and \texttt{fgets} without handling leftover
  newlines.
\item
  Not checking return values of input functions.
\end{itemize}

\subsubsection{Why this matters}\label{why-this-matters-1}

Mastering \texttt{stdio.h} lets you talk to the outside world: read user
input, print results, log errors, and write tools that compose with
shells via redirection and pipes.

\subsubsection{Exercises}\label{exercises-55}

\begin{enumerate}
\def\labelenumi{\arabic{enumi}.}
\tightlist
\item
  Read two integers and a double from one line using \texttt{fgets} +
  \texttt{sscanf}. Print them back with labels and formatting.
\item
  Write a tiny REPL: read a line; if it's \texttt{"quit"}
  (case-sensitive), exit; otherwise print the length of the line
  (excluding newline).
\item
  Print a table of \texttt{i}, \texttt{i-i}, \texttt{i-i-i} for
  \texttt{i=1..12}, aligned in columns using width specifiers.
\item
  Read a word with \texttt{scanf("\%127s",\ buf)}, then read a full line
  with \texttt{fgets}. Demonstrate how to handle the leftover newline so
  the \texttt{fgets} works as intended.
\item
  Print a pointer value with \texttt{\%p}, and a \texttt{size\_t} with
  \texttt{\%zu}. Explain why \texttt{\%d} would be incorrect for
  \texttt{size\_t}.
\end{enumerate}

\subsection{\texorpdfstring{12.2 Math Functions
(\texttt{math.h})}{12.2 Math Functions (math.h)}}\label{math-functions-math.h}

The \texttt{\textless{}math.h\textgreater{}} header provides standard
mathematical functions for real numbers. They operate primarily on
\texttt{double} (with variants for \texttt{float} and
\texttt{long\ double}) and return results in the same type.

\subsubsection{Common Functions}\label{common-functions}

\begin{itemize}
\item
  Powers and roots

\begin{Shaded}
\begin{Highlighting}[]
\DataTypeTok{double}\NormalTok{ pow}\OperatorTok{(}\DataTypeTok{double}\NormalTok{ x}\OperatorTok{,} \DataTypeTok{double}\NormalTok{ y}\OperatorTok{);}   \CommentTok{// x\^{}y}
\DataTypeTok{double}\NormalTok{ sqrt}\OperatorTok{(}\DataTypeTok{double}\NormalTok{ x}\OperatorTok{);}            \CommentTok{// √x}
\DataTypeTok{double}\NormalTok{ cbrt}\OperatorTok{(}\DataTypeTok{double}\NormalTok{ x}\OperatorTok{);}            \CommentTok{// ∛x}
\end{Highlighting}
\end{Shaded}
\item
  Trigonometry (radians)

\begin{Shaded}
\begin{Highlighting}[]
\DataTypeTok{double}\NormalTok{ sin}\OperatorTok{(}\DataTypeTok{double}\NormalTok{ x}\OperatorTok{);}
\DataTypeTok{double}\NormalTok{ cos}\OperatorTok{(}\DataTypeTok{double}\NormalTok{ x}\OperatorTok{);}
\DataTypeTok{double}\NormalTok{ tan}\OperatorTok{(}\DataTypeTok{double}\NormalTok{ x}\OperatorTok{);}
\DataTypeTok{double}\NormalTok{ asin}\OperatorTok{(}\DataTypeTok{double}\NormalTok{ x}\OperatorTok{);}
\DataTypeTok{double}\NormalTok{ acos}\OperatorTok{(}\DataTypeTok{double}\NormalTok{ x}\OperatorTok{);}
\DataTypeTok{double}\NormalTok{ atan}\OperatorTok{(}\DataTypeTok{double}\NormalTok{ x}\OperatorTok{);}
\DataTypeTok{double}\NormalTok{ atan2}\OperatorTok{(}\DataTypeTok{double}\NormalTok{ y}\OperatorTok{,} \DataTypeTok{double}\NormalTok{ x}\OperatorTok{);} \CommentTok{// angle from (x,y)}
\end{Highlighting}
\end{Shaded}
\item
  Exponential and logarithms

\begin{Shaded}
\begin{Highlighting}[]
\DataTypeTok{double}\NormalTok{ exp}\OperatorTok{(}\DataTypeTok{double}\NormalTok{ x}\OperatorTok{);}     \CommentTok{// e\^{}x}
\DataTypeTok{double}\NormalTok{ log}\OperatorTok{(}\DataTypeTok{double}\NormalTok{ x}\OperatorTok{);}     \CommentTok{// natural log}
\DataTypeTok{double}\NormalTok{ log10}\OperatorTok{(}\DataTypeTok{double}\NormalTok{ x}\OperatorTok{);}   \CommentTok{// base{-}10 log}
\end{Highlighting}
\end{Shaded}
\item
  Rounding and absolute values

\begin{Shaded}
\begin{Highlighting}[]
\DataTypeTok{double}\NormalTok{ fabs}\OperatorTok{(}\DataTypeTok{double}\NormalTok{ x}\OperatorTok{);}     \CommentTok{// absolute value}
\DataTypeTok{double}\NormalTok{ ceil}\OperatorTok{(}\DataTypeTok{double}\NormalTok{ x}\OperatorTok{);}     \CommentTok{// round up}
\DataTypeTok{double}\NormalTok{ floor}\OperatorTok{(}\DataTypeTok{double}\NormalTok{ x}\OperatorTok{);}    \CommentTok{// round down}
\DataTypeTok{double}\NormalTok{ round}\OperatorTok{(}\DataTypeTok{double}\NormalTok{ x}\OperatorTok{);}    \CommentTok{// round to nearest}
\DataTypeTok{double}\NormalTok{ trunc}\OperatorTok{(}\DataTypeTok{double}\NormalTok{ x}\OperatorTok{);}    \CommentTok{// drop fractional part}
\end{Highlighting}
\end{Shaded}
\item
  Hypotenuse and distance

\begin{Shaded}
\begin{Highlighting}[]
\DataTypeTok{double}\NormalTok{ hypot}\OperatorTok{(}\DataTypeTok{double}\NormalTok{ x}\OperatorTok{,} \DataTypeTok{double}\NormalTok{ y}\OperatorTok{);} \CommentTok{// sqrt(x\^{}2 + y\^{}2) safely}
\end{Highlighting}
\end{Shaded}
\end{itemize}

\subsubsection{Type-Specific Variants}\label{type-specific-variants}

C provides suffixes to match argument types:

\begin{itemize}
\tightlist
\item
  \texttt{sinf}, \texttt{cosf}, \texttt{sqrtf} → work with
  \texttt{float}
\item
  \texttt{sinl}, \texttt{cosl}, \texttt{sqrtl} → work with
  \texttt{long\ double}
\end{itemize}

\begin{Shaded}
\begin{Highlighting}[]
\DataTypeTok{float}\NormalTok{ xf }\OperatorTok{=} \FloatTok{0.5}\BuiltInTok{f}\OperatorTok{;}
\NormalTok{printf}\OperatorTok{(}\StringTok{"}\SpecialCharTok{\%f\textbackslash{}n}\StringTok{"}\OperatorTok{,}\NormalTok{ sinf}\OperatorTok{(}\NormalTok{xf}\OperatorTok{));}
\end{Highlighting}
\end{Shaded}

\subsubsection{Constants}\label{constants}

\texttt{\textless{}math.h\textgreater{}} (since C99) provides:

\begin{itemize}
\tightlist
\item
  \texttt{M\_PI} (not standard everywhere, but common: π ≈ 3.14159)
\item
  In C23, constants are part of \texttt{\textless{}math.h\textgreater{}}
  under \texttt{\#define} when available.
\end{itemize}

You can also define your own:

\begin{Shaded}
\begin{Highlighting}[]
\DataTypeTok{const} \DataTypeTok{double}\NormalTok{ PI }\OperatorTok{=} \FloatTok{3.141592653589793}\OperatorTok{;}
\end{Highlighting}
\end{Shaded}

\subsubsection{Example: Right Triangle}\label{example-right-triangle}

\begin{Shaded}
\begin{Highlighting}[]
\PreprocessorTok{\#include }\ImportTok{\textless{}stdio.h\textgreater{}}
\PreprocessorTok{\#include }\ImportTok{\textless{}math.h\textgreater{}}

\DataTypeTok{int}\NormalTok{ main}\OperatorTok{(}\DataTypeTok{void}\OperatorTok{)} \OperatorTok{\{}
    \DataTypeTok{double}\NormalTok{ a }\OperatorTok{=} \FloatTok{3.0}\OperatorTok{,}\NormalTok{ b }\OperatorTok{=} \FloatTok{4.0}\OperatorTok{;}
    \DataTypeTok{double}\NormalTok{ c }\OperatorTok{=}\NormalTok{ hypot}\OperatorTok{(}\NormalTok{a}\OperatorTok{,}\NormalTok{ b}\OperatorTok{);}   \CommentTok{// safer than sqrt(a*a + b{-}b)}
    \DataTypeTok{double}\NormalTok{ angle }\OperatorTok{=}\NormalTok{ atan2}\OperatorTok{(}\NormalTok{b}\OperatorTok{,}\NormalTok{ a}\OperatorTok{)} \OperatorTok{{-}} \FloatTok{180.0} \OperatorTok{/}\NormalTok{ M\_PI}\OperatorTok{;} \CommentTok{// in degrees}
\NormalTok{    printf}\OperatorTok{(}\StringTok{"c = }\SpecialCharTok{\%.2f}\StringTok{, angle = }\SpecialCharTok{\%.1f}\StringTok{°}\SpecialCharTok{\textbackslash{}n}\StringTok{"}\OperatorTok{,}\NormalTok{ c}\OperatorTok{,}\NormalTok{ angle}\OperatorTok{);}
    \ControlFlowTok{return} \DecValTok{0}\OperatorTok{;}
\OperatorTok{\}}
\end{Highlighting}
\end{Shaded}

Output:

\begin{verbatim}
c = 5.00, angle = 53.1°
\end{verbatim}

\subsubsection{Error Handling}\label{error-handling}

\begin{itemize}
\item
  Many functions return NaN (not*a-number) or \texttt{HUGE\_VAL} on
  errors.
\item
  Use \texttt{\textless{}math.h\textgreater{}} macros:

\begin{Shaded}
\begin{Highlighting}[]
\NormalTok{isnan}\OperatorTok{(}\NormalTok{x}\OperatorTok{),}\NormalTok{ isinf}\OperatorTok{(}\NormalTok{x}\OperatorTok{),}\NormalTok{ isfinite}\OperatorTok{(}\NormalTok{x}\OperatorTok{)}
\end{Highlighting}
\end{Shaded}
\end{itemize}

Example:

\begin{Shaded}
\begin{Highlighting}[]
\PreprocessorTok{\#include }\ImportTok{\textless{}math.h\textgreater{}}
\PreprocessorTok{\#include }\ImportTok{\textless{}stdio.h\textgreater{}}

\DataTypeTok{int}\NormalTok{ main}\OperatorTok{(}\DataTypeTok{void}\OperatorTok{)} \OperatorTok{\{}
    \DataTypeTok{double}\NormalTok{ x }\OperatorTok{=}\NormalTok{ sqrt}\OperatorTok{({-}}\FloatTok{1.0}\OperatorTok{);}
    \ControlFlowTok{if} \OperatorTok{(}\NormalTok{isnan}\OperatorTok{(}\NormalTok{x}\OperatorTok{))}\NormalTok{ puts}\OperatorTok{(}\StringTok{"x is NaN"}\OperatorTok{);}
    \ControlFlowTok{return} \DecValTok{0}\OperatorTok{;}
\OperatorTok{\}}
\end{Highlighting}
\end{Shaded}

\subsubsection{Why It Matters}\label{why-it-matters-51}

Mathematics is at the core of many programs: graphics, simulations,
engineering, finance. Using \texttt{\textless{}math.h\textgreater{}}
gives you reliable, efficient implementations across platforms.

\subsubsection{Exercises}\label{exercises-56}

\begin{enumerate}
\def\labelenumi{\arabic{enumi}.}
\tightlist
\item
  Compute the area of a circle given radius \texttt{r} using
  \texttt{PI-r-r} and \texttt{M\_PI}.
\item
  Convert degrees to radians and compute \texttt{sin}, \texttt{cos}, and
  \texttt{tan} of 30°, 45°, and 60°.
\item
  Read two sides of a right triangle, compute hypotenuse with
  \texttt{hypot} and the angle with \texttt{atan2}.
\item
  Demonstrate \texttt{ceil}, \texttt{floor}, \texttt{round}, and
  \texttt{trunc} on values \texttt{2.7} and \texttt{-2.7}.
\item
  Compute compound interest: given \texttt{P}, annual rate \texttt{r},
  and years \texttt{n}, compute \texttt{P\ -\ pow(1+r,\ n)}.
\end{enumerate}

\subsection{\texorpdfstring{12.3 Time and Date
(\texttt{time.h})}{12.3 Time and Date (time.h)}}\label{time-and-date-time.h}

The \texttt{\textless{}time.h\textgreater{}} header provides facilities
for working with time, dates, and durations. It's useful for logging,
measuring execution, and scheduling.

\subsubsection{Core Types}\label{core-types}

\begin{itemize}
\item
  \texttt{time\_t} Represents a point in time (seconds since the epoch,
  usually Jan 1, 1970 UTC).
\item
  \texttt{struct\ tm} Broken-down calendar time. Fields:

\begin{Shaded}
\begin{Highlighting}[]
\KeywordTok{struct}\NormalTok{ tm }\OperatorTok{\{}
    \DataTypeTok{int}\NormalTok{ tm\_sec}\OperatorTok{;}   \CommentTok{// 0–60}
    \DataTypeTok{int}\NormalTok{ tm\_min}\OperatorTok{;}   \CommentTok{// 0–59}
    \DataTypeTok{int}\NormalTok{ tm\_hour}\OperatorTok{;}  \CommentTok{// 0–23}
    \DataTypeTok{int}\NormalTok{ tm\_mday}\OperatorTok{;}  \CommentTok{// 1–31 (day of month)}
    \DataTypeTok{int}\NormalTok{ tm\_mon}\OperatorTok{;}   \CommentTok{// 0–11 (months since January)}
    \DataTypeTok{int}\NormalTok{ tm\_year}\OperatorTok{;}  \CommentTok{// years since 1900}
    \DataTypeTok{int}\NormalTok{ tm\_wday}\OperatorTok{;}  \CommentTok{// 0–6 (Sunday = 0)}
    \DataTypeTok{int}\NormalTok{ tm\_yday}\OperatorTok{;}  \CommentTok{// 0–365 (days since Jan 1)}
    \DataTypeTok{int}\NormalTok{ tm\_isdst}\OperatorTok{;} \CommentTok{// daylight saving time flag}
\OperatorTok{\};}
\end{Highlighting}
\end{Shaded}
\end{itemize}

\subsubsection{Getting the Current Time}\label{getting-the-current-time}

\begin{Shaded}
\begin{Highlighting}[]
\PreprocessorTok{\#include }\ImportTok{\textless{}stdio.h\textgreater{}}
\PreprocessorTok{\#include }\ImportTok{\textless{}time.h\textgreater{}}

\DataTypeTok{int}\NormalTok{ main}\OperatorTok{(}\DataTypeTok{void}\OperatorTok{)} \OperatorTok{\{}
    \DataTypeTok{time\_t}\NormalTok{ now }\OperatorTok{=}\NormalTok{ time}\OperatorTok{(}\NormalTok{NULL}\OperatorTok{);}
\NormalTok{    printf}\OperatorTok{(}\StringTok{"Epoch time: }\SpecialCharTok{\%lld\textbackslash{}n}\StringTok{"}\OperatorTok{,} \OperatorTok{(}\DataTypeTok{long} \DataTypeTok{long}\OperatorTok{)}\NormalTok{now}\OperatorTok{);}

    \KeywordTok{struct}\NormalTok{ tm }\OperatorTok{{-}}\NormalTok{local }\OperatorTok{=}\NormalTok{ localtime}\OperatorTok{(\&}\NormalTok{now}\OperatorTok{);}
\NormalTok{    printf}\OperatorTok{(}\StringTok{"Local: }\SpecialCharTok{\%d}\StringTok{{-}}\SpecialCharTok{\%02d}\StringTok{{-}}\SpecialCharTok{\%02d}\StringTok{ }\SpecialCharTok{\%02d}\StringTok{:}\SpecialCharTok{\%02d}\StringTok{:}\SpecialCharTok{\%02d\textbackslash{}n}\StringTok{"}\OperatorTok{,}
\NormalTok{        local}\OperatorTok{{-}\textgreater{}}\NormalTok{tm\_year }\OperatorTok{+} \DecValTok{1900}\OperatorTok{,}
\NormalTok{        local}\OperatorTok{{-}\textgreater{}}\NormalTok{tm\_mon }\OperatorTok{+} \DecValTok{1}\OperatorTok{,}
\NormalTok{        local}\OperatorTok{{-}\textgreater{}}\NormalTok{tm\_mday}\OperatorTok{,}
\NormalTok{        local}\OperatorTok{{-}\textgreater{}}\NormalTok{tm\_hour}\OperatorTok{,}
\NormalTok{        local}\OperatorTok{{-}\textgreater{}}\NormalTok{tm\_min}\OperatorTok{,}
\NormalTok{        local}\OperatorTok{{-}\textgreater{}}\NormalTok{tm\_sec}
    \OperatorTok{);}
    \ControlFlowTok{return} \DecValTok{0}\OperatorTok{;}
\OperatorTok{\}}
\end{Highlighting}
\end{Shaded}

\subsubsection{\texorpdfstring{Formatting Time
(\texttt{strftime})}{Formatting Time (strftime)}}\label{formatting-time-strftime}

Convert \texttt{struct\ tm} into a formatted string:

\begin{Shaded}
\begin{Highlighting}[]
\DataTypeTok{char}\NormalTok{ buf}\OperatorTok{[}\DecValTok{64}\OperatorTok{];}
\NormalTok{strftime}\OperatorTok{(}\NormalTok{buf}\OperatorTok{,} \KeywordTok{sizeof}\NormalTok{ buf}\OperatorTok{,} \StringTok{"\%Y{-}\%m{-}}\SpecialCharTok{\%d}\StringTok{ \%H:\%M:\%S"}\OperatorTok{,}\NormalTok{ local}\OperatorTok{);}
\NormalTok{puts}\OperatorTok{(}\NormalTok{buf}\OperatorTok{);}
\end{Highlighting}
\end{Shaded}

Common format specifiers:

\begin{itemize}
\tightlist
\item
  \texttt{\%Y} = year (2025), \texttt{\%m} = month, \texttt{\%d} = day
\item
  \texttt{\%H} = hour, \texttt{\%M} = minute, \texttt{\%S} = second
\item
  \texttt{\%a} = weekday name, \texttt{\%b} = month name
\end{itemize}

\subsubsection{Measuring Durations}\label{measuring-durations}

Use \texttt{clock()} to measure CPU time:

\begin{Shaded}
\begin{Highlighting}[]
\PreprocessorTok{\#include }\ImportTok{\textless{}stdio.h\textgreater{}}
\PreprocessorTok{\#include }\ImportTok{\textless{}time.h\textgreater{}}

\DataTypeTok{int}\NormalTok{ main}\OperatorTok{(}\DataTypeTok{void}\OperatorTok{)} \OperatorTok{\{}
\NormalTok{    clock\_t start }\OperatorTok{=}\NormalTok{ clock}\OperatorTok{();}
    \ControlFlowTok{for} \OperatorTok{(}\DataTypeTok{volatile} \DataTypeTok{long}\NormalTok{ i}\OperatorTok{=}\DecValTok{0}\OperatorTok{;}\NormalTok{ i}\OperatorTok{\textless{}}\DecValTok{100000000}\OperatorTok{;}\NormalTok{ i}\OperatorTok{++);} \CommentTok{// busy loop}
\NormalTok{    clock\_t end }\OperatorTok{=}\NormalTok{ clock}\OperatorTok{();}
    \DataTypeTok{double}\NormalTok{ secs }\OperatorTok{=} \OperatorTok{(}\DataTypeTok{double}\OperatorTok{)(}\NormalTok{end }\OperatorTok{{-}}\NormalTok{ start}\OperatorTok{)} \OperatorTok{/}\NormalTok{ CLOCKS\_PER\_SEC}\OperatorTok{;}
\NormalTok{    printf}\OperatorTok{(}\StringTok{"Elapsed CPU time: }\SpecialCharTok{\%.3f}\StringTok{ seconds}\SpecialCharTok{\textbackslash{}n}\StringTok{"}\OperatorTok{,}\NormalTok{ secs}\OperatorTok{);}
    \ControlFlowTok{return} \DecValTok{0}\OperatorTok{;}
\OperatorTok{\}}
\end{Highlighting}
\end{Shaded}

For wall-clock time differences:

\begin{Shaded}
\begin{Highlighting}[]
\DataTypeTok{time\_t}\NormalTok{ t1 }\OperatorTok{=}\NormalTok{ time}\OperatorTok{(}\NormalTok{NULL}\OperatorTok{);}
\CommentTok{// ... work ...}
\DataTypeTok{time\_t}\NormalTok{ t2 }\OperatorTok{=}\NormalTok{ time}\OperatorTok{(}\NormalTok{NULL}\OperatorTok{);}
\NormalTok{printf}\OperatorTok{(}\StringTok{"Elapsed wall time: }\SpecialCharTok{\%ld}\StringTok{ seconds}\SpecialCharTok{\textbackslash{}n}\StringTok{"}\OperatorTok{,} \OperatorTok{(}\DataTypeTok{long}\OperatorTok{)(}\NormalTok{t2 }\OperatorTok{{-}}\NormalTok{ t1}\OperatorTok{));}
\end{Highlighting}
\end{Shaded}

\subsubsection{Converting Between
Representations}\label{converting-between-representations}

\begin{itemize}
\tightlist
\item
  \texttt{localtime(\&t)} → \texttt{struct\ tm} in local time
\item
  \texttt{gmtime(\&t)} → \texttt{struct\ tm} in UTC
\item
  \texttt{mktime(\&tm)} → convert \texttt{struct\ tm} back to
  \texttt{time\_t}
\end{itemize}

\begin{Shaded}
\begin{Highlighting}[]
\KeywordTok{struct}\NormalTok{ tm t }\OperatorTok{=} \OperatorTok{\{}\DecValTok{0}\OperatorTok{\};}
\NormalTok{t}\OperatorTok{.}\NormalTok{tm\_year }\OperatorTok{=} \DecValTok{2025} \OperatorTok{{-}} \DecValTok{1900}\OperatorTok{;}
\NormalTok{t}\OperatorTok{.}\NormalTok{tm\_mon }\OperatorTok{=} \DecValTok{11}\OperatorTok{;} \CommentTok{// December}
\NormalTok{t}\OperatorTok{.}\NormalTok{tm\_mday }\OperatorTok{=} \DecValTok{25}\OperatorTok{;}
\DataTypeTok{time\_t}\NormalTok{ xmas }\OperatorTok{=}\NormalTok{ mktime}\OperatorTok{(\&}\NormalTok{t}\OperatorTok{);}
\NormalTok{printf}\OperatorTok{(}\StringTok{"Christmas 2025 epoch = }\SpecialCharTok{\%lld\textbackslash{}n}\StringTok{"}\OperatorTok{,} \OperatorTok{(}\DataTypeTok{long} \DataTypeTok{long}\OperatorTok{)}\NormalTok{xmas}\OperatorTok{);}
\end{Highlighting}
\end{Shaded}

\subsubsection{Why It Matters}\label{why-it-matters-52}

\begin{itemize}
\tightlist
\item
  Time stamps are needed in logs, files, transactions.
\item
  Duration measurement is essential for profiling and benchmarks.
\item
  Portable date/time handling avoids platform-specific hacks.
\end{itemize}

\subsubsection{Exercises}\label{exercises-57}

\begin{enumerate}
\def\labelenumi{\arabic{enumi}.}
\tightlist
\item
  Print the current time in both UTC and local time.
\item
  Use \texttt{strftime} to format today's date as
  \texttt{Saturday,\ September\ 6,\ 2025}.
\item
  Measure how long it takes to compute the sum of numbers 1--1e7.
\item
  Ask the user for a year, month, and day, build a \texttt{struct\ tm},
  convert with \texttt{mktime}, and print the day of the week.
\item
  Write a function that prints a timestamp
  \texttt{{[}YYYY-MM-DD\ HH:MM:SS{]}} for use in log messages.
\end{enumerate}

\subsection{\texorpdfstring{12.4 Random Numbers
(\texttt{stdlib.h})}{12.4 Random Numbers (stdlib.h)}}\label{random-numbers-stdlib.h}

C provides basic random number generation via
\texttt{\textless{}stdlib.h\textgreater{}}. While not cryptographically
secure, it's useful for simulations, games, and simple randomized
behavior.

\subsubsection{\texorpdfstring{\texttt{rand()} and
\texttt{srand()}}{rand() and srand()}}\label{rand-and-srand}

\begin{itemize}
\item
  \texttt{int\ rand(void);} Returns a pseudo-random integer between
  \texttt{0} and \texttt{RAND\_MAX} (at least 32767).
\item
  \texttt{void\ srand(unsigned\ int\ seed);} Seeds the random number
  generator. Same seed → same sequence.
\end{itemize}

\begin{Shaded}
\begin{Highlighting}[]
\PreprocessorTok{\#include }\ImportTok{\textless{}stdio.h\textgreater{}}
\PreprocessorTok{\#include }\ImportTok{\textless{}stdlib.h\textgreater{}}
\PreprocessorTok{\#include }\ImportTok{\textless{}time.h\textgreater{}}

\DataTypeTok{int}\NormalTok{ main}\OperatorTok{(}\DataTypeTok{void}\OperatorTok{)} \OperatorTok{\{}
\NormalTok{    srand}\OperatorTok{((}\DataTypeTok{unsigned}\OperatorTok{)}\NormalTok{time}\OperatorTok{(}\NormalTok{NULL}\OperatorTok{));}  \CommentTok{// seed with current time}
    \ControlFlowTok{for} \OperatorTok{(}\DataTypeTok{int}\NormalTok{ i}\OperatorTok{=}\DecValTok{0}\OperatorTok{;}\NormalTok{ i}\OperatorTok{\textless{}}\DecValTok{5}\OperatorTok{;}\NormalTok{ i}\OperatorTok{++)} \OperatorTok{\{}
\NormalTok{        printf}\OperatorTok{(}\StringTok{"}\SpecialCharTok{\%d\textbackslash{}n}\StringTok{"}\OperatorTok{,}\NormalTok{ rand}\OperatorTok{());}
    \OperatorTok{\}}
    \ControlFlowTok{return} \DecValTok{0}\OperatorTok{;}
\OperatorTok{\}}
\end{Highlighting}
\end{Shaded}

\subsubsection{Scaling to a Range}\label{scaling-to-a-range}

Generate numbers in \texttt{{[}0,\ n)}:

\begin{Shaded}
\begin{Highlighting}[]
\DataTypeTok{int}\NormalTok{ r }\OperatorTok{=}\NormalTok{ rand}\OperatorTok{()} \OperatorTok{\%}\NormalTok{ n}\OperatorTok{;}   \CommentTok{// biased if n doesn’t divide RAND\_MAX+1}
\end{Highlighting}
\end{Shaded}

Better scaling (avoids bias):

\begin{Shaded}
\begin{Highlighting}[]
\DataTypeTok{int}\NormalTok{ rand\_range}\OperatorTok{(}\DataTypeTok{int}\NormalTok{ min}\OperatorTok{,} \DataTypeTok{int}\NormalTok{ max}\OperatorTok{)} \OperatorTok{\{}
    \ControlFlowTok{return}\NormalTok{ min }\OperatorTok{+}\NormalTok{ rand}\OperatorTok{()} \OperatorTok{\%} \OperatorTok{(}\NormalTok{max }\OperatorTok{{-}}\NormalTok{ min }\OperatorTok{+} \DecValTok{1}\OperatorTok{);}
\OperatorTok{\}}
\end{Highlighting}
\end{Shaded}

Example:

\begin{Shaded}
\begin{Highlighting}[]
\NormalTok{printf}\OperatorTok{(}\StringTok{"dice: }\SpecialCharTok{\%d\textbackslash{}n}\StringTok{"}\OperatorTok{,}\NormalTok{ rand\_range}\OperatorTok{(}\DecValTok{1}\OperatorTok{,}\DecValTok{6}\OperatorTok{));}
\end{Highlighting}
\end{Shaded}

\subsubsection{Random Floating-Point
Values}\label{random-floating-point-values}

Scale to \texttt{{[}0,1)}:

\begin{Shaded}
\begin{Highlighting}[]
\DataTypeTok{double}\NormalTok{ r }\OperatorTok{=} \OperatorTok{(}\DataTypeTok{double}\OperatorTok{)}\NormalTok{rand}\OperatorTok{()} \OperatorTok{/} \OperatorTok{(}\NormalTok{RAND\_MAX }\OperatorTok{+} \FloatTok{1.0}\OperatorTok{);}
\end{Highlighting}
\end{Shaded}

To \texttt{{[}a,b)}:

\begin{Shaded}
\begin{Highlighting}[]
\DataTypeTok{double}\NormalTok{ rand\_double}\OperatorTok{(}\DataTypeTok{double}\NormalTok{ a}\OperatorTok{,} \DataTypeTok{double}\NormalTok{ b}\OperatorTok{)} \OperatorTok{\{}
    \ControlFlowTok{return}\NormalTok{ a }\OperatorTok{+} \OperatorTok{(}\NormalTok{b }\OperatorTok{{-}}\NormalTok{ a}\OperatorTok{)} \OperatorTok{{-}} \OperatorTok{((}\DataTypeTok{double}\OperatorTok{)}\NormalTok{rand}\OperatorTok{()} \OperatorTok{/} \OperatorTok{(}\NormalTok{RAND\_MAX }\OperatorTok{+} \FloatTok{1.0}\OperatorTok{));}
\OperatorTok{\}}
\end{Highlighting}
\end{Shaded}

\subsubsection{Deterministic Sequences}\label{deterministic-sequences}

\begin{Shaded}
\begin{Highlighting}[]
\NormalTok{srand}\OperatorTok{(}\DecValTok{1234}\OperatorTok{);}  \CommentTok{// fixed seed}
\NormalTok{printf}\OperatorTok{(}\StringTok{"}\SpecialCharTok{\%d}\StringTok{ }\SpecialCharTok{\%d}\StringTok{ }\SpecialCharTok{\%d\textbackslash{}n}\StringTok{"}\OperatorTok{,}\NormalTok{ rand}\OperatorTok{(),}\NormalTok{ rand}\OperatorTok{(),}\NormalTok{ rand}\OperatorTok{());}
\end{Highlighting}
\end{Shaded}

This always produces the same sequence - useful for debugging.

\subsubsection{Example: Coin Toss}\label{example-coin-toss}

\begin{Shaded}
\begin{Highlighting}[]
\PreprocessorTok{\#include }\ImportTok{\textless{}stdio.h\textgreater{}}
\PreprocessorTok{\#include }\ImportTok{\textless{}stdlib.h\textgreater{}}
\PreprocessorTok{\#include }\ImportTok{\textless{}time.h\textgreater{}}

\DataTypeTok{int}\NormalTok{ main}\OperatorTok{(}\DataTypeTok{void}\OperatorTok{)} \OperatorTok{\{}
\NormalTok{    srand}\OperatorTok{((}\DataTypeTok{unsigned}\OperatorTok{)}\NormalTok{time}\OperatorTok{(}\NormalTok{NULL}\OperatorTok{));}
    \ControlFlowTok{for} \OperatorTok{(}\DataTypeTok{int}\NormalTok{ i}\OperatorTok{=}\DecValTok{0}\OperatorTok{;}\NormalTok{ i}\OperatorTok{\textless{}}\DecValTok{10}\OperatorTok{;}\NormalTok{ i}\OperatorTok{++)} \OperatorTok{\{}
\NormalTok{        puts}\OperatorTok{(}\NormalTok{rand}\OperatorTok{()} \OperatorTok{\%} \DecValTok{2} \OperatorTok{?} \StringTok{"Heads"} \OperatorTok{:} \StringTok{"Tails"}\OperatorTok{);}
    \OperatorTok{\}}
    \ControlFlowTok{return} \DecValTok{0}\OperatorTok{;}
\OperatorTok{\}}
\end{Highlighting}
\end{Shaded}

\subsubsection{Limitations}\label{limitations}

\begin{itemize}
\tightlist
\item
  Not secure: predictable with seed.
\item
  Period may be short depending on implementation.
\item
  For secure randomness (e.g., crypto), use platform APIs
  (\texttt{arc4random}, \texttt{/dev/urandom}, \texttt{getrandom})
  instead.
\end{itemize}

\subsubsection{Why It Matters}\label{why-it-matters-53}

Randomness is key for:

\begin{itemize}
\tightlist
\item
  Games (dice rolls, shuffles)
\item
  Simulations (Monte Carlo)
\item
  Testing (randomized inputs)
\end{itemize}

Understanding limitations helps avoid misuse in security-critical code.

\subsubsection{Exercises}\label{exercises-58}

\begin{enumerate}
\def\labelenumi{\arabic{enumi}.}
\tightlist
\item
  Write a \texttt{rand\_range(int\ min,\ int\ max)} function and test it
  by generating 20 random integers between 5 and 15.
\item
  Generate 100 random doubles in \texttt{{[}0,1)} and compute their
  average.
\item
  Simulate rolling two dice 1000 times. Count how often the sum is 7.
\item
  Seed the generator twice with the same value. Verify the outputs
  match.
\item
  Modify the coin toss program to run until you get 3 heads in a row.
  Print the number of tosses required.
\end{enumerate}

\subsection{\texorpdfstring{12.5 Strings Revisited
(\texttt{string.h})}{12.5 Strings Revisited (string.h)}}\label{strings-revisited-string.h}

Strings in C are just arrays of \texttt{char} ending with a null
terminator
(\texttt{\textquotesingle{}\textbackslash{}0\textquotesingle{}}). The
\texttt{\textless{}string.h\textgreater{}} library provides many
functions to manipulate them safely and efficiently.

\subsubsection{Measuring and Copying}\label{measuring-and-copying}

\begin{Shaded}
\begin{Highlighting}[]
\PreprocessorTok{\#include }\ImportTok{\textless{}string.h\textgreater{}}
\PreprocessorTok{\#include }\ImportTok{\textless{}stdio.h\textgreater{}}

\DataTypeTok{int}\NormalTok{ main}\OperatorTok{(}\DataTypeTok{void}\OperatorTok{)} \OperatorTok{\{}
    \DataTypeTok{char}\NormalTok{ s}\OperatorTok{[]} \OperatorTok{=} \StringTok{"hello"}\OperatorTok{;}
\NormalTok{    printf}\OperatorTok{(}\StringTok{"len=}\SpecialCharTok{\%zu\textbackslash{}n}\StringTok{"}\OperatorTok{,}\NormalTok{ strlen}\OperatorTok{(}\NormalTok{s}\OperatorTok{));}   \CommentTok{// 5}

    \DataTypeTok{char}\NormalTok{ buf}\OperatorTok{[}\DecValTok{20}\OperatorTok{];}
\NormalTok{    strcpy}\OperatorTok{(}\NormalTok{buf}\OperatorTok{,}\NormalTok{ s}\OperatorTok{);}                   \CommentTok{// copies entire string}
\NormalTok{    printf}\OperatorTok{(}\StringTok{"}\SpecialCharTok{\%s\textbackslash{}n}\StringTok{"}\OperatorTok{,}\NormalTok{ buf}\OperatorTok{);}

\NormalTok{    strncpy}\OperatorTok{(}\NormalTok{buf}\OperatorTok{,} \StringTok{"worldwide"}\OperatorTok{,} \KeywordTok{sizeof}\NormalTok{ buf}\OperatorTok{);} \CommentTok{// copy with limit}
\NormalTok{    buf}\OperatorTok{[}\KeywordTok{sizeof}\NormalTok{ buf }\OperatorTok{{-}} \DecValTok{1}\OperatorTok{]} \OperatorTok{=} \CharTok{\textquotesingle{}}\SpecialCharTok{\textbackslash{}0}\CharTok{\textquotesingle{}}\OperatorTok{;}            \CommentTok{// ensure termination}
\NormalTok{    printf}\OperatorTok{(}\StringTok{"}\SpecialCharTok{\%s\textbackslash{}n}\StringTok{"}\OperatorTok{,}\NormalTok{ buf}\OperatorTok{);}
    \ControlFlowTok{return} \DecValTok{0}\OperatorTok{;}
\OperatorTok{\}}
\end{Highlighting}
\end{Shaded}

\begin{itemize}
\tightlist
\item
  \texttt{strlen(s)} → number of characters before
  \texttt{\textquotesingle{}\textbackslash{}0\textquotesingle{}}
\item
  \texttt{strcpy(dest,\ src)} → copy until
  \texttt{\textquotesingle{}\textbackslash{}0\textquotesingle{}} (unsafe
  if dest is too small)
\item
  \texttt{strncpy(dest,\ src,\ n)} → safer copy, but may leave string
  unterminated
\end{itemize}

\subsubsection{Concatenation}\label{concatenation}

\begin{Shaded}
\begin{Highlighting}[]
\DataTypeTok{char}\NormalTok{ buf}\OperatorTok{[}\DecValTok{32}\OperatorTok{]} \OperatorTok{=} \StringTok{"Hello, "}\OperatorTok{;}
\NormalTok{strcat}\OperatorTok{(}\NormalTok{buf}\OperatorTok{,} \StringTok{"World!"}\OperatorTok{);}
\NormalTok{printf}\OperatorTok{(}\StringTok{"}\SpecialCharTok{\%s\textbackslash{}n}\StringTok{"}\OperatorTok{,}\NormalTok{ buf}\OperatorTok{);}  \CommentTok{// "Hello, World!"}
\end{Highlighting}
\end{Shaded}

Safer:
\texttt{strncat(buf,\ "World!",\ sizeof\ buf\ -\ strlen(buf)\ -\ 1);}

\subsubsection{Comparison}\label{comparison}

\begin{Shaded}
\begin{Highlighting}[]
\NormalTok{strcmp}\OperatorTok{(}\StringTok{"abc"}\OperatorTok{,}\StringTok{"abc"}\OperatorTok{)}   \CommentTok{// 0 (equal)}
\NormalTok{strcmp}\OperatorTok{(}\StringTok{"abc"}\OperatorTok{,}\StringTok{"abd"}\OperatorTok{)}   \CommentTok{// \textless{}0}
\NormalTok{strcmp}\OperatorTok{(}\StringTok{"abd"}\OperatorTok{,}\StringTok{"abc"}\OperatorTok{)}   \CommentTok{// \textgreater{}0}
\NormalTok{strncmp}\OperatorTok{(}\StringTok{"abc"}\OperatorTok{,}\StringTok{"abd"}\OperatorTok{,}\DecValTok{2}\OperatorTok{)} \CommentTok{// 0 (first 2 chars equal)}
\end{Highlighting}
\end{Shaded}

Use \texttt{==} only for pointers, not contents.

\subsubsection{Searching}\label{searching}

\begin{Shaded}
\begin{Highlighting}[]
\DataTypeTok{char} \OperatorTok{{-}}\NormalTok{s }\OperatorTok{=} \StringTok{"the quick brown fox"}\OperatorTok{;}
\NormalTok{printf}\OperatorTok{(}\StringTok{"}\SpecialCharTok{\%s\textbackslash{}n}\StringTok{"}\OperatorTok{,}\NormalTok{ strchr}\OperatorTok{(}\NormalTok{s}\OperatorTok{,}\CharTok{\textquotesingle{}q\textquotesingle{}}\OperatorTok{));}   \CommentTok{// "quick brown fox"}
\NormalTok{printf}\OperatorTok{(}\StringTok{"}\SpecialCharTok{\%s\textbackslash{}n}\StringTok{"}\OperatorTok{,}\NormalTok{ strstr}\OperatorTok{(}\NormalTok{s}\OperatorTok{,}\StringTok{"brown"}\OperatorTok{));} \CommentTok{// "brown fox"}
\end{Highlighting}
\end{Shaded}

Functions:

\begin{itemize}
\tightlist
\item
  \texttt{strchr(s,c)} → first occurrence of \texttt{c}
\item
  \texttt{strrchr(s,c)} → last occurrence of \texttt{c}
\item
  \texttt{strstr(s,sub)} → substring search
\end{itemize}

\subsubsection{Tokenizing}\label{tokenizing}

Break a string into tokens:

\begin{Shaded}
\begin{Highlighting}[]
\PreprocessorTok{\#include }\ImportTok{\textless{}string.h\textgreater{}}
\PreprocessorTok{\#include }\ImportTok{\textless{}stdio.h\textgreater{}}

\DataTypeTok{int}\NormalTok{ main}\OperatorTok{(}\DataTypeTok{void}\OperatorTok{)} \OperatorTok{\{}
    \DataTypeTok{char}\NormalTok{ text}\OperatorTok{[]} \OperatorTok{=} \StringTok{"red,green,blue"}\OperatorTok{;}
    \DataTypeTok{char} \OperatorTok{{-}}\NormalTok{tok }\OperatorTok{=}\NormalTok{ strtok}\OperatorTok{(}\NormalTok{text}\OperatorTok{,} \StringTok{","}\OperatorTok{);}
    \ControlFlowTok{while} \OperatorTok{(}\NormalTok{tok}\OperatorTok{)} \OperatorTok{\{}
\NormalTok{        puts}\OperatorTok{(}\NormalTok{tok}\OperatorTok{);}
\NormalTok{        tok }\OperatorTok{=}\NormalTok{ strtok}\OperatorTok{(}\NormalTok{NULL}\OperatorTok{,} \StringTok{","}\OperatorTok{);}
    \OperatorTok{\}}
    \ControlFlowTok{return} \DecValTok{0}\OperatorTok{;}
\OperatorTok{\}}
\end{Highlighting}
\end{Shaded}

⚠️ \texttt{strtok} modifies the input string and is not thread-safe.
Modern alternatives: \texttt{strtok\_r} (POSIX) or manual parsing.

\subsubsection{Memory Operations}\label{memory-operations}

\begin{itemize}
\tightlist
\item
  \texttt{memcpy(dest,\ src,\ n)} → copy raw bytes
\item
  \texttt{memmove(dest,\ src,\ n)} → like \texttt{memcpy}, but safe for
  overlapping regions
\item
  \texttt{memset(ptr,\ val,\ n)} → fill memory with value
\end{itemize}

Example:

\begin{Shaded}
\begin{Highlighting}[]
\DataTypeTok{int}\NormalTok{ a}\OperatorTok{[}\DecValTok{5}\OperatorTok{]} \OperatorTok{=} \OperatorTok{\{}\DecValTok{1}\OperatorTok{,}\DecValTok{2}\OperatorTok{,}\DecValTok{3}\OperatorTok{,}\DecValTok{4}\OperatorTok{,}\DecValTok{5}\OperatorTok{\};}
\DataTypeTok{int}\NormalTok{ b}\OperatorTok{[}\DecValTok{5}\OperatorTok{];}
\NormalTok{memcpy}\OperatorTok{(}\NormalTok{b}\OperatorTok{,}\NormalTok{ a}\OperatorTok{,} \KeywordTok{sizeof}\NormalTok{ a}\OperatorTok{);}  \CommentTok{// copy entire array}
\end{Highlighting}
\end{Shaded}

\subsubsection{Example: Reversing a
String}\label{example-reversing-a-string}

\begin{Shaded}
\begin{Highlighting}[]
\PreprocessorTok{\#include }\ImportTok{\textless{}stdio.h\textgreater{}}
\PreprocessorTok{\#include }\ImportTok{\textless{}string.h\textgreater{}}

\DataTypeTok{void}\NormalTok{ reverse}\OperatorTok{(}\DataTypeTok{char} \OperatorTok{{-}}\NormalTok{s}\OperatorTok{)} \OperatorTok{\{}
    \DataTypeTok{size\_t}\NormalTok{ n }\OperatorTok{=}\NormalTok{ strlen}\OperatorTok{(}\NormalTok{s}\OperatorTok{);}
    \ControlFlowTok{for} \OperatorTok{(}\DataTypeTok{size\_t}\NormalTok{ i}\OperatorTok{=}\DecValTok{0}\OperatorTok{;}\NormalTok{ i}\OperatorTok{\textless{}}\NormalTok{n}\OperatorTok{/}\DecValTok{2}\OperatorTok{;}\NormalTok{ i}\OperatorTok{++)} \OperatorTok{\{}
        \DataTypeTok{char}\NormalTok{ tmp }\OperatorTok{=}\NormalTok{ s}\OperatorTok{[}\NormalTok{i}\OperatorTok{];}
\NormalTok{        s}\OperatorTok{[}\NormalTok{i}\OperatorTok{]} \OperatorTok{=}\NormalTok{ s}\OperatorTok{[}\NormalTok{n}\OperatorTok{{-}}\DecValTok{1}\OperatorTok{{-}}\NormalTok{i}\OperatorTok{];}
\NormalTok{        s}\OperatorTok{[}\NormalTok{n}\OperatorTok{{-}}\DecValTok{1}\OperatorTok{{-}}\NormalTok{i}\OperatorTok{]} \OperatorTok{=}\NormalTok{ tmp}\OperatorTok{;}
    \OperatorTok{\}}
\OperatorTok{\}}

\DataTypeTok{int}\NormalTok{ main}\OperatorTok{(}\DataTypeTok{void}\OperatorTok{)} \OperatorTok{\{}
    \DataTypeTok{char}\NormalTok{ str}\OperatorTok{[]} \OperatorTok{=} \StringTok{"abcdef"}\OperatorTok{;}
\NormalTok{    reverse}\OperatorTok{(}\NormalTok{str}\OperatorTok{);}
\NormalTok{    puts}\OperatorTok{(}\NormalTok{str}\OperatorTok{);}  \CommentTok{// "fedcba"}
    \ControlFlowTok{return} \DecValTok{0}\OperatorTok{;}
\OperatorTok{\}}
\end{Highlighting}
\end{Shaded}

\subsubsection{Why It Matters}\label{why-it-matters-54}

Strings are everywhere: input, output, file names, protocols. Mastering
\texttt{\textless{}string.h\textgreater{}} is essential to handle them
safely and correctly in C.

\subsubsection{Exercises}\label{exercises-59}

\begin{enumerate}
\def\labelenumi{\arabic{enumi}.}
\tightlist
\item
  Write a function
  \texttt{safe\_copy(char\ -dst,\ size\_t\ n,\ const\ char\ -src)} that
  copies with \texttt{strncpy} and ensures null termination.
\item
  Concatenate \texttt{"Hello"} and \texttt{"World"} with a space into a
  buffer and print it.
\item
  Write a program that counts how many times \texttt{"cat"} appears in
  \texttt{"catapult\ scatter\ catalog"}.
\item
  Tokenize a string \texttt{"one\ two\ three"} on spaces and print each
  token.
\item
  Implement a function \texttt{is\_palindrome(const\ char\ -s)} that
  returns 1 if a string reads the same backward and forward, ignoring
  case.
\end{enumerate}

Here's a comprehensive problem set for Chapter 12 (Standard Library
Essentials). This covers \texttt{stdio.h}, \texttt{math.h},
\texttt{time.h}, \texttt{stdlib.h}, and \texttt{string.h} with hands-on
exercises for beginners.

\subsection{Problems}\label{problems-10}

\subsubsection{\texorpdfstring{Input/Output
(\texttt{stdio.h})}{Input/Output (stdio.h)}}\label{inputoutput-stdio.h-1}

\begin{enumerate}
\def\labelenumi{\arabic{enumi}.}
\tightlist
\item
  Write a program that reads two integers from the user using
  \texttt{scanf} and prints their sum.
\item
  Use \texttt{fgets} to read a full line of text, then print the line
  and its length.
\item
  Print a table of numbers 1--10 alongside their squares and cubes,
  aligned in neat columns.
\item
  Redirect program output to a file
  (\texttt{./prog\ \textgreater{}\ out.txt}). Modify the program to
  write errors to \texttt{stderr} instead of \texttt{stdout}.
\item
  Write a program that reads words until EOF and prints them numbered
  (\texttt{1:\ word}, \texttt{2:\ word}, \ldots).
\end{enumerate}

\subsubsection{\texorpdfstring{Math Functions
(\texttt{math.h})}{Math Functions (math.h)}}\label{math-functions-math.h-1}

\begin{enumerate}
\def\labelenumi{\arabic{enumi}.}
\setcounter{enumi}{5}
\tightlist
\item
  Compute the area and circumference of a circle given its radius. Use
  \texttt{M\_PI} if available.
\item
  Convert 45 degrees into radians and print its sine, cosine, and
  tangent.
\item
  Write a program that asks for two sides of a right triangle and prints
  its hypotenuse using \texttt{hypot}.
\item
  Demonstrate \texttt{ceil}, \texttt{floor}, \texttt{round}, and
  \texttt{trunc} on \texttt{-2.7} and \texttt{2.7}.
\item
  Implement compound interest: given \texttt{P}, annual rate \texttt{r},
  and years \texttt{n}, compute \texttt{P\ -\ pow(1+r,\ n)}.
\end{enumerate}

\subsubsection{\texorpdfstring{Time and Date
(\texttt{time.h})}{Time and Date (time.h)}}\label{time-and-date-time.h-1}

\begin{enumerate}
\def\labelenumi{\arabic{enumi}.}
\setcounter{enumi}{10}
\tightlist
\item
  Print the current time in local time and UTC.
\item
  Format today's date as \texttt{Saturday,\ September\ 6,\ 2025} using
  \texttt{strftime}.
\item
  Measure how long it takes to sum integers 1--100 million.
\item
  Ask the user for a year, month, and day, then print what day of the
  week it falls on.
\item
  Write a function \texttt{timestamp()} that prints
  \texttt{{[}YYYY-MM-DD\ HH:MM:SS{]}} for logging.
\end{enumerate}

\subsection{\texorpdfstring{Random Numbers
(\texttt{stdlib.h})}{Random Numbers (stdlib.h)}}\label{random-numbers-stdlib.h-1}

\begin{enumerate}
\def\labelenumi{\arabic{enumi}.}
\setcounter{enumi}{15}
\tightlist
\item
  Write a \texttt{rand\_range(int\ min,\ int\ max)} function. Use it to
  simulate rolling a dice 10 times.
\item
  Generate 100 random doubles in \texttt{{[}0,1)} and compute their
  average.
\item
  Simulate rolling two dice 1000 times. Count how many times the sum is
  7.
\item
  Demonstrate that seeding with the same value gives the same random
  sequence.
\item
  Simulate a coin toss until you get 3 heads in a row. Print the number
  of tosses needed.
\end{enumerate}

\subsection{\texorpdfstring{Strings
(\texttt{string.h})}{Strings (string.h)}}\label{strings-string.h}

\begin{enumerate}
\def\labelenumi{\arabic{enumi}.}
\setcounter{enumi}{20}
\tightlist
\item
  Write a
  \texttt{safe\_copy(char\ -dst,\ size\_t\ n,\ const\ char\ -src)}
  function using \texttt{strncpy} and test it.
\item
  Concatenate \texttt{"Hello"} and \texttt{"World"} with a space in
  between.
\item
  Count how many times \texttt{"cat"} appears in
  \texttt{"catapult\ scatter\ catalog"}.
\item
  Tokenize the string \texttt{"red,green,blue"} with \texttt{strtok} and
  print each token.
\item
  Implement \texttt{is\_palindrome(const\ char\ -s)} that ignores case.
  Test with \texttt{"Radar"} and \texttt{"level"}.
\end{enumerate}




\end{document}
